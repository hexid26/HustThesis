% TODO

% [ ] 语句通畅度修改
% [ ] 公式是否增加推导过程
% [ ] D2D 通信增加几篇文献并做简要说明

\chapter{基于边缘计算的群智感知任务调度机制}

如今,人类社会已经进入万物互联时代。
无处不在的移动设备让移动群智感知的覆盖区域更广阔、采集任务更丰富、数据收集更全面。
为了实现高效率、大覆盖的移动群智感知应用,边缘计算已经替代云计算为移动群智感知提供基础支撑。
通过在边缘服务器上部署相关服务,可以增加群智感知应用的执行效率。
% 利用边缘服务器分发感知任务,可以大大提高感知任务的空间覆盖率。
同时,借助低成本的 D2D 通信发送感知任务、收集感知数据,助于招募更多的参与者,还能减少主干网络的流量负载。
% 同时,低成本的 D2D 通信也有助于更多的移动设备参与到群智感知过程中。
% 尽管大量的研究工作利用模拟或系统实验证明了边缘计算和D2D通信对移动群智感知的积极作用,但是这些研究工作并没有考虑移动设备自身的运动特征对移动群智感知造成的影响。
考虑移动设备运动状态对 D2D 通信造成的影响,以及网络边缘各种资源和群智感知应用执行效率之间的联系,本章将移动群智感知的场景进行参数量化,针对移动群智感知的执行过程建立随机过程分析模型。
然后对模型进行分析,讨论了移动群智感知执行过程中各参数对感知效率的影响机理,并对感知效率进行预测。
最后根据数学模型,分析了不同边缘资源调度的收益。

\section{本章引言}

如文献~\citen{DBLP:journals/fgcs/AntonicMPZ16,DBLP:conf/ccnc/MessaoudRG16}中所讨论,当使用云计算模型支撑移动群智感知应用时,所有的感知数据都会被收集到云端。
随着移动设备的增加和感知数据的膨胀,上传至云端的感知数据会急剧消耗主干网络的带宽资源。
边缘计算的提出,将云端服务迁移到位于网络边缘侧的边缘服务器(例如蜂窝网络中的基站或者大型无线网络接入点)上,使计算资源、存储资源、应用服务更加贴近数据产生源,以此缓解群智感知应用对云端服务的依赖。
由于边缘网络中移动设备多、地域分布广,研究者们普遍认为边缘计算可以显著提升网络应用的感知效率~\cite{DBLP:conf/sigcomm/BonomiMZA12}。
近年来,学术界和工业界都在不断挖掘边缘计算的巨大潜力,致力于将边缘计算应用到不同领域的开创性工作中~\cite{DBLP:journals/access/MarjanovicAZ18,DBLP:journals/iotj/ChiangZ16}。


% 目前,移动计算领域与边缘计算日益紧密的结合在一起。
鉴于边缘计算的分布式特性以及可利用资源更靠近数据源,移动计算应用可以利用边缘计算实现高带宽、低时延的实时服务。
移动群智感知作为移动计算领域中的一种典型应用,利用无线网络和移动设备中内置的传感器来完成对真实世界的数字感知工作。
不同于传统的专用型无线传感器网络,移动群智感知可以更方便快捷地收集与人类生产、生活、自然环境等息息相关的各种数字信息~\cite{DBLP:journals/cm/GuoCZYC16}。
利用边缘计算支撑的移动群智感知应用,已经广泛应用于其它不同领域,例如无线网络性能测量~\cite{DBLP:journals/cm/RosenLLCMB14}、天气预报~\cite{DBLP:journals/tpds/ZhaoMTL15}、空气质量监测~\cite{DBLP:conf/huc/ZhangXWC14}、城市噪声监测~\cite{DBLP:conf/huc/ZhengLWZLC14}和城市智能交通建设~\cite{DBLP:conf/icdcs/ZhouJL15}等。

% 除了边缘计算能够帮助移动群智感知提高感知效率,
同时,研究者还发现使用成本更低的D2D通信,也可以达到这一目的~\cite{DBLP:journals/puc/WangLL17}。
利用移动设备之间的D2D通信,使得不具备蜂窝网络通信能力的移动设备也能够快速加入群智感知,例如可穿戴设备,监控设备等。
同时,在使用成本方面,D2D通信的能耗成本更低且不需要服务成本,因此更适合应对群智感知中大量零散感知数据的交换。
另一方面,网络边缘侧的 D2D 网络和主干网络相互隔离,在机会式通信网络中对感知数据进行收集、去冗、压缩,也可以有效缓解海量感知数据对主干网络的资源占用。
除此之外,D2D通信作为5G网络中的重要的通信手段,将在物联网场景下成为核心通信手段之一。

% 边缘计算支撑移动群智感知应用,
% 由于产生感知数据的移动设备(例如智能手机、智能手表)和位于数据中心的云端服务相距甚远,感知数据在传输过程中会遇到较高的网络时延或者不可预测的网络抖动,导致传输时延增加。
% 另一方面,将海量的感知数据汇总到云平台,会消耗主干网中大量的带宽资源。
% 通过边缘计算的支援,移动群智感知应用可以依靠边缘服务器中的可利用资源,辅助任务转发和数据收集工作,扩大移动群智感知应用的适用范围,降低感知成本,提高感知效率。
% 另一方面,介于蜂窝通信的成本较高,减退了许多潜在移动群智感知参与者的热情。
% 为了解决这一问题,许多研究人员设计了激励机制来吸引更多的群智感知参与者,也有部分研究人员发现利用成本更低的D2D通信,同样可以缓解这一问题~\cite{DBLP:journals/puc/WangLL17}。

\begin{figure}[!b]
  \centering
  \vspace{-1em}
  \includegraphics[width=240pt]{./figures/Sec_UIC/移动群智感知应用场景.pdf}
  \vspace{-0.5em}
  \caption{边缘计算支撑下的移动群智感知应用场景}
  \vspace{-1em}
  \label{Figure_UIC_MCS}
\end{figure}

基于上述原因,本章的研究工作针对图~\ref{Figure_UIC_MCS} 所描述的边缘计算支撑下的移动群智感知应用场景。
首先,目标区域内的基站从云端服务器获取感知任务,利用蜂窝网络或 D2D 通信的方式,将感知任务分发给区域内的移动设备。
其次,接收到感知任务的移动设备,在执行任务的同时,借助 D2D 通信将感知任务分发给相邻的移动设备。
待感知任务执行完毕,参与感知任务的移动设备利用蜂窝网络或者 D2D 通信,再将感知数据反馈给基站。
最后,基站对收集到的感知数据执行去冗、合并、压缩等操作后,再交付给云端服务器执行最终的运算处理。

% 在本文中,提出边缘计算授权MCS与D2D通信,其中整个MCS过程有两个主要阶段。在第一阶段(任务分发)中,边缘节点作为源节点执行以将任务分发到移动设备。在第二阶段(数据收集)中,移动设备成为源节点,边缘节点成为目标节点。任务和传感数据都以机会方式通过D2D通信传输。在这种情况下,第一个自然问题是服务部署如何影响感知效率(例如,覆盖)。直观地知道通过部署相应的服务来利用更多的边缘节点,将实现更高的传感质量。但是,在边缘节点中部署服务并不是免费的。因此,量化这种对服务部署决策的影响是非常重要的。现有的研究,例如[13]  -  [15],已经分析了D2D机会网络中的消息传递时延。这些研究表明,消息传递性能受很多因素的影响很大,例如移动设备的数量和移动设备的遭遇率。正如注意到的,他们通常假设有一个数据源节点和一个目标节点。它们都不能用于分析边缘计算授权MCS的性能,因为可能有多个边缘节点分别在任务分发阶段和数据收集阶段中作为源节点和目的节点执行。

在该场景下,重点考虑边缘网络中感知任务的分发效率和感知数据的收集效率,以提高移动群智感知应用的覆盖范围和感知效率。
在任务分发过程中,已部署任务分发服务的基站作为任务分发的源头,不断地将移动群智感知任务分发至移动设备。
在数据收集过程中,移动设备作为感知数据的源头,将数据回传至已部署数据收集服务的基站。
任务的传播和感知数据的收集不仅可以使用蜂窝网络,也可以使用成本更低的 D2D 通信。
现有研究~\cite{DBLP:conf/wcnc/QinF13,DBLP:journals/twc/LiW14,DBLP:journals/winet/ZhaoMLT18}已经分析了D2D机会式通信网络中的信息传递时延。
这些研究工作表明,D2D通信机制中消息传递时延的影响因素主要有移动设备的数量和移动设备的相遇率。
然而,这些研究工作在边缘网络中只考虑了单源节点对单目标节点、单源节点对多目标节点以及多源节点对单目标节点的工作场景。
% 选取了一个边缘服务器同时作为数据分发源和数据收集目的地。
在真实场景下,可以存在多个源节点和多目标节点。
因此,这些研究工作并不适用于分析边缘计算下的移动群智感知过程。
为了提高移动群智感知的感知效率,必须用新的方法量化边缘网络中服务资源部署对移动群智感知的影响。
在该述求下,第一个挑战就是探索边缘网络中的资源部署对移动群智感知应用覆盖范围和感知效率的影响机理。
% 另外,移动群智感知应用中的设备随着时间的推移会改变自身的位置。
第二个挑战是探索移动设备之间 D2D 通信对移动群智感知应用感知效率的影响。

为了解决这两个问题,本章结合边缘计算模型下的移动群智感知与D2D通信,对群智感知执行过程进行理论分析、性能分析。本章的主要贡献如下:

1)利用常微分方程组,描述边缘计算环境下的移动群智感知中的任务分发阶段和数据收集阶段。
通过对常微分方程组进行分析和求解,量化移动群智感知应用场景中的边缘服务以及感知设备的移动模型,并推导出量化参数和感知效率之间的关系。


2)基于移动群智感知应用的生命周期,本章设计了一种时间划分算法,以找出任务分发阶段和数据收集阶段的最佳时间分配,以提高移动群智感知应用的感知效率。

3)在边缘计算场景下对移动群智感知的执行效率进行预测,并分析了不同边缘资源调度对感知效率的影响。

% 3)经过模拟实验分析,验证了模型和算法的正确性和准确性。

\section{移动群智感知过程分析}

% 本章中,重点研究了与论文[22]中描述的MCS过程。 在第一个任务分发阶段,任务分发者将MCS任务分发到移动设备。 此后,在第二数据收集阶段,已经接收到任务的移动设备进行感测并将感测数据报告回数据收集器。 进一步使雾计算参与上述程序。 在本节中,首先介绍一些预备,然后定义随机过程分析的系统模型。

% \textbf{『把 MCS 场景再说清楚一点。』}
本章采用了和文献~\citen{DBLP:journals/tpds/ZhaoMTL15}相似的移动群智感知应用范式,主要研究对象为移动群智感知应用中的两个重要过程:任务分发过程和数据收集过程。
其中,任务分发过程是指已部署任务分发服务的基站将任务分发至边缘网络中的移动设备,数据收集过程是指已经接受到任务的移动设备将感知数据反馈给已部署数据收集服务的基站。
考虑到移动群智感知参与者本身具备移动性,本章也将移动设备的运动特征纳入考虑范围。
本小节重点介绍分析模型中的参数定义和模型建立。

\subsection{场景定义}
\label{UIC:Scenario}

% 当结合雾计算时,任务分发者和数据收集器都被部署为雾节点中的服务。在不失一般性的情况下,假设基站是本文中的雾节点。图2描述了具有D2D通信的雾计算授权MCS的工作过程。首先,MCS服务提供商选择适当的基站来部署MCS服务。其次,在任务分发阶段,这些基站开始向移动设备传播MCS任务。为了避免蜂窝通信成本,基站仅将任务传递到进入其通信范围的移动设备。然后,已经完成任务的移动设备开始将任务转发到在其移动期间通过D2D通信遇到的其他移动设备。传播的时间消耗是T传播。在时间段T感测中,移动设备获得感测数据并将它们发送回数据收集器。设T集合表示数据收集的时间消耗。在整个阶段,假设所有移动设备都是MCS服务的志愿者。请注意,具有该任务的任何移动设备都可以帮助传播任务并通过D2D通信收集传感数据。显然,这形成了与流行病学路由的机会网络。

在边缘计算场景下,移动群智感知中的任务分发服务和数据收集服务都可以部署在边缘服务器中。
在不失普适性的情况下,本章假设蜂窝网络的基站作为能够承载这些服务的边缘服务器。
此时移动群智感知应用的执行过程大体分为四步:
1)选择合适的基站部署任务分发服务和数据收集服务;
2)在任务分发阶段,部署有任务分发服务的基站利用蜂窝网络和D2D通信将感知任务发送到附近的移动设备上;
3)移动设备在收到感知任务之后,一边处理感知任务,一边利用D2D通信将任务广播给附近的其它移动设备;
4)已经完成任务的移动设备,可以利用蜂窝网络或者D2D通信,将感知数据反馈给已经部署数据收集服务的基站。
由于激励机制并不是本章的主要研究内容,因此在移动群智感知应用执行期间,本章假设目标区域内的移动设备都是移动群智感知应用的志愿者。

\begin{figure}[!h]
  \centering
  % \vspace{-1em}
  \includegraphics[width=240pt]{./figures/Sec_UIC/移动群智感知应用的时间分布.pdf}
  \vspace{-0.5em}
  \caption{移动群智感知应用的时间分布}
  % \vspace{-1em}
  \label{Figure_MCS_Delay}
\end{figure}

图~\ref{Figure_MCS_Delay} 描述了边缘计算场景下,移动群智感知应用中各阶段的时间开销。
当基站中部署好任务分发服务和数据收集服务之后,
$T_{dissemination}$ 表示移动群智感知任务从基站部署到移动设备的时间开销;
$T_{sensing}$ 表示移动设备执行感知任务的时间开销;
$T_{collection}$ 表示移动设备将感知数据反馈回基站的时间开销;
$T_{total\_time}$ 是移动群智感知应用在边缘设备和端设备上的总执行时间。

% 图3总结了雾计算授权MCS与D2D通信的任务分发的两种主要通信模式。 一旦移动设备完成任务,除了传播任务外,它还进行协作和机会感知以在移动期间获取感测数据。 然后,在数据收集阶段,移动设备开始将其感测数据发送回数据收集器,即,利用MCS服务部署的雾节点。 与任务分发阶段类似,传感数据可以直接传输到基站或通过基于D2D的流行路由到达基站,如图4所示。

\begin{figure}[!b]
  \centering
  % \vspace{-1em}
  \includegraphics[width=280pt]{./figures/Sec_UIC/任务分发过程.pdf}
  \vspace{-0.5em}
  \caption{感知任务分发途径}
  % \vspace{-1em}
  \label{Figure_PropagationProcedure}
\end{figure}

图~\ref{Figure_PropagationProcedure} 展示了边缘网络中感知任务的分发方法。
首先,部署有任务分发服务的基站利用蜂窝网络以固定的速率将任务部署到可通信的移动设备上。
因此在感知任务分发的前期,主要依赖蜂窝网络传播分发感知任务。
同时,已经收到感知任务的移动设备在其运动过程中,可能会遇到其它还未收到感知任务的移动设备。
在这种情况下,相遇的移动设备之间可以利用 D2D 通信进行感知任务的分发。
随着移动设备的不断相遇,收到感知任务的移动设备数量会不断扩大,极大提高感知任务的分发效率。
由于 D2D 通信需要两个移动设备的距离满足一定条件,因此使用 D2D 通信传播感知任务会受到移动设备运动状态的影响。

图~\ref{Figure_FeedbacksCollection} 展示了边缘网络中感知数据的收集方法。
利用 D2D 通信,相邻的两个移动设备可以交换自己的感知数据。
一方面,通过一系列数据交换操作,远端移动设备的感知数据有机会传送到负责收集数据的基站。
另一方面,部署有数据收集服务的基站也可以利用蜂窝通信网络,直接与覆盖范围内移动设备通信,获取该设备上的感知数据。
同时,与基站通信的移动设备,可能和其它移动设备已经交换过感知数据,因此基站在一个移动设备上可以收集到来自不同设备的感知数据。

% 计算下的移动群智感知服务利用蜂窝网络和D2D通信完成任务分发。
% 已经收到任务的移动设备,在其移动过程中,会有一定的概率遇到没有接收到任务的移动设备。
% 此时,已经收到任务的设备利用D2D通信,将任务分发至还没有收到任务的设备。
% 待任务执行完成后,移动设备可以利用蜂窝网络将感知数据传送给基站,也可以利用D2D通信将结果委托给在移动过程中相遇的其它设备,并传送给基站。
% 这一过程,如图~\ref{Figure_FeedbacksCollection} 所示。

\begin{figure}[!h]
  \centering
  % \vspace{-1em}
  \includegraphics[width=350pt]{./figures/Sec_UIC/数据收集过程.pdf}
  \vspace{-0.5em}
  \caption{感知数据收集途径}
  \vspace{-1.5em}
  \label{Figure_FeedbacksCollection}
\end{figure}

\subsection{移动模型}
% 按照上述工作流程,在本文中,认为在MCS服务中部署了N f个雾节点。它们既可以作为任务分发者,也可以作为数据收集者。在雾中,有N个移动设备随机移动并且可以自愿参与MCS过程。由于移动,移动设备可以机会性地与另一个移动设备相遇并且在接触持续时间期间与遇到的移动设备通信。假设接触持续时间足够长以完成任务或传感数据的传输。

在场景定义中,参与移动群智感知的移动设备主要有两类:负责感知任务分发和数据收集的基站,以及执行感知任务的移动设备。
对于基站而言,一般为固定式部署,不存在移动特征。
但是对于移动设备而言,每个移动设备都有自己的运动轨迹,这些设备遵循自身的轨迹移动,在相遇时利用 D2D 通信进行数据交互。
为了抽象描述移动设备因相遇而产生的 D2D 通信行为,这里使用移动设备之间的相遇率来表示 D2D 通信发生的频率。
因此,定义 $\lambda_n$ 表示移动设备之间的相遇率。

在 Groenevelt 等人~\cite{DBLP:journals/pe/GroeneveltNK05}的研究中,给出了机会式通信网络中节点相遇率的计算方法。
借助该理论, $\lambda_n$ 可以采用式~\eqref{Formula_EncounterRate} 进行计算得出。
其中,$d$ 表示移动设备 D2D 通信的覆盖范围(单位:米),
$\mathbb{E}[V^*]$ 指的是特定区域内所有移动设备的平均速度(单位:米/秒),
$A$ 表示 D2D 通信网络中所期望的覆盖区域面积(单位:平方米),
$w$ 表示移动设备的运动模型对应的常量系数,例如随机路径运动模型所对应的系数为 1.3683~\cite{DBLP:journals/pe/GroeneveltNK05}。
基于这一理论,可以将群智感知的覆盖范围、以及移动设备的运动轨迹转化为移动设备的相遇概率。

\vspace{-1em}
\begin{equation}
  % \vspace{-1em}
  \label{Formula_EncounterRate}
  \begin{gathered}
  \lambda = \frac{2 w d \mathbb{E}[V^*]}{A}
  \end{gathered}
  % \vspace{-0.5em}
\end{equation}

除此之外,基站可以利用蜂窝网络进行感知任务的传播和感知数据的收集,基站与移动设备之间的通信概率以 $\lambda_f$ 表示。
由于基站不存在移动属性且蜂窝网络的通信覆盖半径远远大于 D2D 通信,因此 $\lambda_f$ 和移动设备的运动特征没有直接关系,仅表示基站和移动设备因群智感知发生通信的概率。
但是在日常行为中,用户会使用移动设备拨打电话、发送短信、查看社交网络信息。
这些操作,都需要移动设备和基站进行数据交互,因此也有研究人员将感知任务信息压缩至这类频繁的数据交互中,以加快感知任务的执行。
基于此,本章将 $\lambda_f$ 作为可变量来对待。

% 基站也会以一定的速率将感知任务发送至移动设备,或者从移动设备收集感知数据。
% 将基站和移动设备因感知任务而交互的速率定义为基站和移动设备之间的相遇率 $\lambda_f$。
% 由于基站和移动设备使用蜂窝网络进行数据通信,所以 $\lambda_f$ 和移动设备的运动轨迹没有直接联系。
% 所以,$\lambda_f$ 也可以理解为基站和移动设备因群智感知发生通信的概率。

\section{随机过程分析及感知效率优化算法}
% 在本节中,给出了任务传播阶段,数据收集阶段的随机过程分析,以及在给定参数(如N,N f,λn,λ)的覆盖度量中可实现的感知效率的推导。 F 。 为方便读者,表I总结了本文中使用的主要符号。
本小节主要阐述任务分发过程和数据收集过程的理论分析方法。
表 ~\ref{table_notations_UIC} 给出了本章中使用的符号定义。

\begin{table}[h]
  % \vspace{-0.5em}
  \caption{数学符号及定义}
  \vspace{-0.5em}
  \centering
  \label{table_notations_UIC}
  % \centering
  \begin{tabular}{|c|p{9.5cm}|}
  \hline
  \textbf{符号} & \textbf{定义}\\
  \hline
  $N$ & 目标区域内移动设备的数量\\\hline
  $N_f$ & 目标区域内基站的数量\\\hline
  $\lambda_n$ & 移动设备之间的相遇概率\\\hline
  $\lambda_f$ & 基站和移动设备的通信概率\\\hline
  $I(t)$ & $t$ 时刻已经收到任务的移动设备的数量\\\hline
  $I'(t)$ & $I(t)$ 的导数\\\hline
  $C_T{t}$ & 生命周期为 $T$ 的感知应用在 $t$ 时刻的感知任务覆盖率\\\hline
  $P_{rcv}(t)$ & $t$ 时刻基站收到感知数据的概率\\\hline
  $P_{nD2D}(t)$ & 感知数据使用 D2D 通信未送达的概率\\\hline
  $P_{nDirect}(t)$ & 感知数据使用蜂窝网络未送达的概率\\\hline
  $S(t)$ & 感知数据在网络中的拷贝数量\\\hline
  % $\boldsymbol{G}$ & 将目标区域划分为网格 $g(i)$ 的集合($g(i) \in \boldsymbol{G}$)\\\hline
  % $L_{ij}(t)$ & $t$ 时刻移动设备$j$是否在网格 $g(i)$中\\\hline
  % $C_{ij}(t)$ & 设备 $j$ 在 $g(i)$ 中的停留时间\\\hline
  % $C(t)$ & $t$ 时刻,移动群质感是的区域覆盖质量\\\hline
  $D(t)$ & 任务分发时间为 $t$ 时,感知数据的收集总量\\\hline
\end{tabular}
  % \vspace{-1em}
\end{table}

\subsection{任务分发过程分析}

在感知任务分发过程中,使用蜂窝网络分发感知任务的速度和区域中基站数量 $N_f$,以及基站与移动设备的通信概率 $\lambda_f$ 相关。
对于已经收到感知任务的移动设备而言,在其移动过程中利用D2D通信对任务进行二次分发。
这种基于 D2D 通信的感知任务分发和人群中传染病的扩散原理类似。
因此可以借助 SIR(Susceptible Infective Removal)模型来描述移动设备之间的任务分发过程。
令 $t$ 时刻,已经收到感知任务的移动设备数量为 $I(t)$。
则此时 $I(t)$ 的增量可以分为两个部分来表示,一部分通过蜂窝网络收到感知任务,另一部分则通过 D2D 网络收到感知任务。
所以,对于 $t$ 时刻而言,已经收到感知任务的移动设备增量 $I'(t)$ 可以用式~\eqref{Formula_SIR_with_fog} 进行计算。

% 当移动设备从基站收到感知任务之后,
% 因此可以借助 SIR(Susceptible Infective Removal)模型来描述移动设备之间的任务分发过程。
% 但是和传统 SIR 模型不同,基站还可以利用蜂窝网络直接将感知任务分发至移动设备。
% 因此感知任务的分发速度还和基站数量 $N_f$ 、基站与移动设备的通信概率 $\lambda_f$ 相关。
% 借助 SIR 模型,将已经收到感知任务的移动设备数量记为 $I(t)$。
% 由于这些设备会以 D2D 通信的方式继续分发感知任务,其数量也会影响感知任务的分发速度。
% 考虑到感知任务可以同时借助 D2D 网络和蜂窝网络进行传播,因此对于 $t$ 时刻而言,已经收到感知任务的移动设备的增量 $I'(t)$ 可以用式~\eqref{Formula_SIR_with_fog} 进行计算。

\begin{equation}
  \label{Formula_SIR_with_fog}
  \begin{aligned}
    I'(t) &= N_f \lambda_f (N-I(t)) + I(t) \lambda_n (N-I(t)) \\
    &= N_f \lambda_f N + ( N \lambda_n - N_f \lambda_f) I(t) - \lambda_n I^2(t) 
  \end{aligned}
\end{equation}

式~\eqref{Formula_SIR_with_fog} 是一个典型的黎卡提(Riccati)常微分方程形式。
因此,其求解过程可以转化为如式~\eqref{Formula_Riccati} 所示的通项公式求解。

\vspace{-1em}
\begin{equation}
  \label{Formula_Riccati}
  \left\{
    \begin{aligned}
      &I'(t) = A \times I^2(t) + B \times I(t) + C \\
      &A = -\lambda_n \\
      &B = N \lambda_n - N_f \lambda_f \\
      &C = N_f \lambda_f N \\
      &I(t) = \frac{C(B+\sqrt{B^2 - 4AC})(e^{\sqrt{B^2 - 4AC} t} - 1)}{-2ACe^{\sqrt{B^2 - 4AC} t} + B^2 -2AC + B \sqrt{B^2 - 4AC}} + D, D 为常数
    \end{aligned}
  \right.
\end{equation}

考虑到初始条件 $I(0)=0$,即在移动群智感知开始时刻($t=0$)没有移动设备收到感知任务,则 $t$ 时刻收到感知任务的移动设备数量可由式~\eqref{Formula_It_with_fog} 表示。
该式中,存在限制条件 $0 < \lambda_f \times N_f < 1$,其中 $0 < \lambda_f \times N_f$ 代表基站一定能和移动设备发生通信;$\lambda_f \times N_f < 1$ 代表基站不会在极短的时间内和所有移动设备通信。
% 当 $\lambda_f \times N_f = 1$ 时,也说明

% \vspace{-1em}
\begin{equation}
\label{Formula_It_with_fog}
\begin{aligned}
I(t) = \frac{N (e^{(\lambda_n N + \lambda_f N_f) t} - 1)}{e^{(\lambda_n N + \lambda_f N_f) t} + \frac{\lambda_n N}{\lambda_f N_f}}, \forall\ \lambda_n \in [0, 1),\ 0 < \lambda_f \times N_f < 1
\end{aligned}
\end{equation}

当移动设备接收到移动群智感知任务之后,根据任务内容执行感知操作并将感知数据存储在本地缓存中,然后等待机会将数据传送回基站。
为了描述移动群智感知任务的覆盖率,这里使用已接受感知任务的移动设备数量与感知区域内移动设备总数的比例作为标准。
用 $C_T(t)$ ($t < T$)表示生命周期为 $T$ 的群智感知应用在执行 $t$ 时间段后的感知任务覆盖率,$C_T(t)$ 的计算方法如式~\eqref{Formula_Coverage_of_Task}。
当 $t=0$ 时,$C_T(t)$ 值为 0 表示没有移动节点收到感知任务。
当 $C_T(t) = 1$ 时,表示目标区域内的所有移动设备均已接收到感知任务。

\begin{equation}
  \label{Formula_Coverage_of_Task}
  C_T(t) = \frac{I(t)}{N} = \frac{e^{(\lambda_n N + \lambda_f N_f) t} - 1}{e^{(\lambda_n N + \lambda_f N_f) t} + \frac{\lambda_n N}{\lambda_f N_f}}
\end{equation}

\subsection{数据收集过程分析}

% 感知设备在收到感知任务之后,需要在群智感知应用的生命周期内将感知数据反馈给基站,通过基站对大量的感知数据进行去冗、加工、压缩之后,在交付给云端资源进行后续计算。
% 为了确保感知数据的质量,移动群智感知应用的发起者多期望能够收集到充足的感知数据,并且让感知数据覆盖范围广阔。
% 这也意味着在规定的时间范围内,基站必须尽可能收集所有移动设备上的感知数据。
% 由于 D2D 通信的引入,移动设备间的通信和设备的运动状态有关。
% 因此,必须考虑移动设备
% 然而蜂窝网络的使用成本较高,而 D2D 通信的频率又和移动设备的运动状态有关。
% 因此,在时域上分析数据收集过程也是非常重要的工作。

已经收到任务的移动设备按照要求获取感知数据后,需要及时地将感知数据传回基站。
而移动群智感知应用在其生命周期内,收集到的数据越多且数据覆盖范围越大,则移动群智感知服务所能提供的服务质量就越高。
这也意味着在规定的时间范围内,基站必须尽可能收集来自所有移动设备上的感知数据。
因此,在时域上分析数据收集过程也是非常重要的工作。

和任务分发时的途径一样,感知数据在收集过程中,移动设备也可以使用蜂窝网络或者 D2D 网络发送感知数据。
在这一过程中,任何部署有感知数据收集服务的基站都可以作为感知数据的目的地。
在使用 D2D 通信时,移动设备可以相互交换感知数据,并将收到的感知数据进行打包处理并为下一次转发做好准备。
当感知数据包到达任何一个基站,则该数据包中的所有感知数据被认为已经收集完成。
令一份感知数据从开始发送到经历时间段 $t$ 后,该数据被基站接收的概率记为 $P_{rcv}(t)$。
在边缘网络中,移动设备可以分别使用蜂窝通信或者 D2D 通信传输感知数据,并且这两种传播方法可视为相互独立事件。
因此,$P_{rcv}(t)$ 的计算方法如式~\eqref{Formula_ProbaRcvT}。

\vspace{-1em}
\begin{equation}
  \label{Formula_ProbaRcvT}
  \begin{aligned}
    P_{rcv}(t) = 1 - P_{nD2D}(t) P_{nDirect}(t)
  \end{aligned}
\end{equation}

上式中,$P_{nD2D}(t)$ 是指该感知数据包未能通过 D2D 通信方式送达基站的概率,$P_{nDirect}(t)$ 是指数据包未能通过蜂窝网络送达基站的概率。
基于章节~\ref{UIC:Scenario} 中的场景定义,感知数据每利用 D2D 网络传播一次,边缘网络中该数据的拷贝便增加一份,用 $S(t)$ 表示一份感知数据在边缘网络中的拷贝数量。
由于移动设备利用一次 D2D 通信将感知数据发送给非基站的概率为 $N/(N + N_f)$。
因此,$P_{nD2D}(t)$ 可以通过计算感知数据的数量比例以求出其被基站收集的概率。
其计算方法如式~\eqref{Formula_ProbaNP} 所描述。

\begin{equation}
  \label{Formula_ProbaNP}
  \begin{aligned}
  P_{nD2D}(t) = (\frac{N}{N + N_f})^{S(t) - 1}, S(t) \geq 1
  \end{aligned}
\end{equation}

当移动设备使用蜂窝网络传输感知数据时,由于该移动设备可能与其它移动设备利用 D2D 通信交换过感知数据,因此一个移动设备用蜂窝网络可能会上传多个感知任务。
由于一份感知数据在边缘网络中存在 $S(t)$ 份拷贝,所以不含有该数据的 $N - S(t)$ 个移动设备与基站通信时,该数据不会被上传至基站。
因此,$P_{nDirect}(t)$ 可以利用式~\eqref{Formula_ProbaNA}进行计算得出。

\begin{equation}
  \label{Formula_ProbaNA}
  \begin{aligned}
  P_{nDirect}(t) = (\frac{N-S(t)}{N})^{\lambda_f N_f t}, \forall \ \lambda_f N_f \in [0,1)
  \end{aligned}
\end{equation}

在上述两式中,$S(t)$ 代表感知数据因 D2D 传输产生的拷贝数量。
这一过程可以理解为单传播源的 SIR 模型。
因此,其计算方法如式~\eqref{Formula_St}。

\begin{equation}
\label{Formula_St}
\begin{aligned}
    S(t) = \frac{N e^{\lambda_n N t}}{e^{\lambda_n N t} + N -1}, \forall \ \lambda_n \in [0,1)
  \end{aligned}
\end{equation}

将式~\eqref{Formula_ProbaNP}、式~\eqref{Formula_ProbaNA} 和式~\eqref{Formula_St} 带入式~\eqref{Formula_ProbaRcvT},可以得到 $P_{rcv}(t)$ 的最终计算方法,如式~\eqref{Formula_ProbaRcvTF}。
式中,$T_{total\_time}$ 代表移动群智感知应用的生命周期,$t$ 代表感知任务分发的执行时间,则感知数据收集的执行时间可以用 $T_{total\_time} - t$ 来表示。
% 因此移动群智感知应用的生命周期 $T_{total\_time}$ 为 $t$ 和 $t'$ 之和。

\begin{equation}
\vspace{-2em}
\label{Formula_ProbaRcvTF}
  \begin{gathered}
    P_{rcv}(t) = 1 -  (\frac{N}{N + N_f})^{\frac{N e^{\lambda_n N t}}{e^{\lambda_n N t} + N -1} - 1} \times (\frac{N-\frac{N e^{\lambda_n N (T_{total\_time} - t)}}{e^{\lambda_n N (T_{total\_time} - t)} + N -1}}{N})^{\lambda_f N_f (T_{total\_time} - t)}, \\
    \forall \  t > 0, t < T_{total\_time}
  \end{gathered}
\end{equation}

% \subsection{感知覆盖范围分析}

% 在移动群智感知应用中,往往需要感知数据的产生位置能够广泛的地域范围。
% 传感数据所覆盖的地域范围越广,则移动群智感知服务的普适性越强。
% 例如,一个十字路口的交通状况需要通过附近所有道路上的车辆数量来预测。
% 为了计算移动群智感知应用的覆盖率,将群智感知应用的目标区域 $\boldsymbol{G}$ 切割成一组网格单元,如图~\ref{Figure_CoverageofArea} 所示,$\boldsymbol{G} = \{g_1,g_2,g_3,\ldots,g_m\}$。
% 网格单元的大小意味着地域空间上的感知密度,该参数由移动群智感知应用的开发者所决定。

% \begin{figure}[!h]
%   \centering
%   % \vspace{-1.5em}
%   \includegraphics[width=200pt]{./figures/Sec_UIC/移动群智感知的空间分布.pdf}
%   \vspace{-0.5em}
%   \caption{移动群智感知的空间分布}
%   \vspace{-0.5em}
%   \label{Figure_CoverageofArea}
% \end{figure}

% 在分析感知覆盖时,用二进制变量 $L_{ij}(t)$ 表示时刻 $t$ 时,移动设备 $j$ 是否在网格$g(i)$中。
% 其定义如式~\eqref{Formula_LocationIJ} 。

% \begin{equation}
%   \label{Formula_LocationIJ}
%   L_{ij}(t) = \left \{
%   \begin{aligned}
%   & 1,\ \text{$t$时刻时移动设备 $j$ 位于网格 $g(i)$ 中}\\
%   & 0,\ \text{$t$时刻时移动设备 $j$ 不在网格 $g(i)$ 中}
%   \end{aligned}
%   \right.
% \end{equation}

% 利用 $L_{ij}(t)$,移动设备 $j$ 在网格 $g(i)$ 中的总停留时间 $C_{ij}$ 可以用式~\eqref{Formula_UserTrace} 表示。

% \vspace{-1.5em}
% \begin{equation}
%   \label{Formula_UserTrace}
%   C_{ij}(t) = \int_{0}^{t}L_{ij}(\varepsilon )d\varepsilon , \forall \ i \in [0,m], j \in [0,N]
% \end{equation}

% 因此,在 $t$ 时刻,移动群智感知服务的感知效率 $C(t)$ 可以利用来式~\eqref{Formula_UserCoverageGrip} 来计算。

% \vspace{-1.5em}
% \begin{equation}
%   \label{Formula_UserCoverageGrip}
%   \begin{gathered}
%   C(t) = \sum^{m}_{i=0}\sum^{N}_{j=0} g_i \times C_{ij}(t), \forall \ t>0
%   \end{gathered}
% \end{equation}

\subsection{优化目标}

对于群智感知应用的发起者而言,在其生命周期内收集到的感知数据总量是衡量感知效率的重要指标之一。
通过前文对群智感知执行过程的分析,可以得到群智感知应用部署率和执行时间之间的关系 $C_T(t)$,以及感知数据的收集概率 $P_{rcv}(t)$。
令图~\ref{Figure_MCS_Delay} 中的 $T_{total\_time}$ 表示群智感知应用的生命周期,并令 $t$ 表示感知任务分发的执行时间,则 $T_{total\_time} - t$ 代表感知数据收集的执行时间。
为了方便度量单个移动设备收到感知任务并成功反馈感知数据的总时长,在此假设每个移动设备收到感知任务之后只会产生一组感知数据。
因此在感知应用中产生的感知数据总数量为 $N \times C_T(t)$,这些数据传回基站的概率为 $P_{rcv}(t)$。
用 $D(t)$ 表示群智感知应用在生命周期内收集到的感知数据总量,其计算方法如式~\eqref{Formula_DataAmount}。

\vspace{-2.5em}
\begin{equation}
  \label{Formula_DataAmount}
  \begin{gathered}
    \begin{aligned}
      D(t) = &N \cdot C_T(t) \cdot P_{rcv}(T_{total\_time}-t)\\
      = &N \times \frac{e^{(\lambda_n N + \lambda_f N_f) t} - 1}{e^{(\lambda_n N + \lambda_f N_f) t} + \frac{\lambda_n N}{\lambda_f N_f}} \\
      &\times (1 -  (\frac{N}{N + N_f})^{\frac{N e^{\lambda_n N t}}{e^{\lambda_n N t} + N -1} - 1} \times (\frac{N-\frac{N e^{\lambda_n N (T_{total\_time} - t)}}{e^{\lambda_n N (T_{total\_time} - t)} + N -1}}{N})^{\lambda_f N_f (T_{total\_time} - t)}),
    \end{aligned}\\
    \forall \ t \in [0, T_{total\_time}]
  \end{gathered}
\end{equation}

在该式中,移动设备数量 $N$、基站数量 $N_f$、移动设备之间的相遇率 $\lambda_n$、以及移动设备和基站之间的通信概率 $\lambda_f$ 可以看做场景参数,而感知任务分发的时间 $t$ 则是式中唯一的自变量。
当 $t = 0$ 和 $t = T_{total\_time}$ 时,分别表示感知任务的分发时间为 0 和感知数据的收集时间为 0,在这两种情况下,$D(t)$ 的计算结果为0,意味着没有收到任何感知数据。
因此,需要找到群智感知在实行过程中合适的时间配额划分,让 $D(t)$ 取值最大是本章的优化目的。
由于 $t = 0$ 和 $t = T_{total\_time}$ 时,$D(t)$ 恒等于 $0$,而通过观察 $D(t)$ 的函数式,可以发现 $D(t)$ 在 $ t \in [0, T_{total\_time}] $ 时是连续函数。
因此,一定存在一个 $t$ 的取值,使 $D(t)$ 最大。
而本章节的最终目标,就是在现有条件下,让移动群智感知应用能够收集到更多的感知数据,即使的 $D(t)$ 达到最大值。

\subsection{基于感知任务生命周期的感知任务执行划分算法}
\label{UIC:algo}

% 在本节中,在最后一节中介绍了一个基于随机过程分析的应用用例。尽管具有D2D通信的MCSwith应该是时延容忍的,但是一些应用也受到截止时间的限制,因为感测数据和截止时间变得非常宝贵。由于MCS过程有两个阶段,因此研究如何在这两个阶段分配时间以最大限度地提高平均水平具有重要意义。根据上面的分析,假设任务分发阶段持续,即分配给任务分发的时间,可以得出deadlineTtotaltimeas接收到的数据包总数

前文已经阐述移动群智感知应用中,足够多的参与设备和大量的反馈数据才能保证移动群智感知应用的感知效率。
在上一小节中已经提到,当移动群智感知的目标场景已经确定时,感知任务分发时长和感知数据收集时长,都会影响感知应用在其生命周期内能收集的感知数据总量,进而影响感知效率。
所以在有限的生命周期内,需要合理分配感知任务分发过程和感知数据收集过程的时间配额,以提高移动群智感知应用的感知效率。
当任务分发过程的时间配额为 $t$ 时,在生命周期 $T_{total\_time}$ 中群智感知应用可以收集到的感知数据总数可以利用式~\eqref{Formula_DataAmount} 计算。
% 因此,必须基于式~\eqref{Formula_DataAmount} 给出合理的群智感知执行时间划分,令 $D(t)$ 最大。


\begin{algorithm}[!b]
  \setstretch{\algostretch}
  \KwIn{$N$ 移动设备的总数量\\$\quad\quad\quad N_f$ 基站的数量\\$\quad\quad\quad \lambda_n$ 移动设备之间的相遇率\\$\quad\quad\quad \lambda_f$  基站和移动设备之间的通信概率\\$\quad\quad\quad T_{total\_time}$  移动群智感知服务的生命周期}
  \KwOut{$max_D ($D(t)$ 的最大值), t_{best}$ (对应 $t$ 的值)}
  % \KwIn{$N_f$ 基站的数量}
  % \KwIn{$\lambda_n$ 移动设备之间的相遇率}
  % \KwIn{$\lambda_f$  基站和移动设备之间的通信概率}
  % \KwIn{$T_{total\_time}$  移动群智感知服务的生命周期}
  % \KwData{『输入数据』}
  % \KwData{$Occ[1:T]$: 每个程序当前的NCP大小,初始为0,随着缓存路的分配增加,最终为每个程序在整个缓存上的NCP}
  令 $\Delta t = T_{total\_time}/N_{slots}$\\
  % 令 $t = \Delta t, 2 \Delta t, 3 \Delta t, 4 \Delta t,\ldots,N_{slots} \Delta t$\\
  定义数组 $res[N_{slots}]$ \\
  令 $index = 0;\ max_D = 0;\ t_{best} = 0$\\
  \For{$index \in [0, N_{slots}]$}{
    $res[index] = D(index \times \Delta t)$ \quad (式~\eqref{Formula_DataAmount})\\
    \If {$D(index \times \Delta t) > max_D$}{
      $max_D = D(index \times \Delta t)$ \quad 替换 $D(t)$ 的最大值\\
      $t_{best} = index \times \Delta t$ \quad 替换令 $D(t)$ 取最大值的 $t_{best}$
    }
    % $index$ ++
  }
  % return $maximum, t_{best}$\\
  \caption{利用枚举找出令 $D(t)$ 最大的 $t$ 值}
  \label{algo_findbestt}
\end{algorithm}

由于 $D(t)$ 的导数计算过于复杂,无法通过导数推导来求得 $D(t)$ 取最大值时 $t$ 的解析解。
另一方面, $D(t)$ 可以视为单调递增函数与单调递减函数的乘积,其单调性可能存在多个拐点,导致无法直接确定 $D(t)$ 取最大值时 $t$ 的取值范围。
因此,这里使用枚举算法来找出 $t$ 的最优解。
% 由于 $D(t)$ 的表达式中,仅存在 $t$ 一个参数确定的函数式,

% 因此,不同阶段的时间配额会直接影响到移动群智感知应用的感知效率。
% 当 $t=0$ 或 $t=t_{total\_time}$ 时,$D(t)$ 恒为零。
% 且当 $0 \leq t \leq t_{total\_time}$ 时,$C_T(t)\geq 0$ 且 $C_T(t)$ 为单调递增函数;$C_F(T_{total\_time}-t) \geq 0$ 且 $C_F(T_{total\_time}-t)$ 为单调递减函数。
% 因此,当$0 \leq t \leq t_{total\_time}$时,$D(t) \geq 0$ 且对于 $t$ 必定存在一个值使得 $D(t)$ 取到最大值,即移动群智感知应用的感知效率达到最优。
% 为了找到最佳的时间划分,本文设计了一个枚举算法来找出 $t$ 的最优解。
该算法伪代码如算法~\ref{algo_findbestt} 所示。
在该算法中,将移动群智感知应用中的生命周期分成 $N_{slots}$ 份。
以 $\Delta t = T_{total\_time} / N_{slots}$ 为步进,对移动群智感知应用中的感知任务分发时间 $t$ 进行枚举( $t = \Delta t, 2 \Delta t, 3 \Delta t, \ldots , N_{slots} \Delta t$ )。
根据 $t$ 的 $N_{slots}$ 种取值,求出对应的 $D(t)$。
最后,找出 $D(t)$ 的最大值以及其所对应的 $t$。
该算法的时间复杂度为 $O(N_{slots})$。


\section{实验结果及分析}

% 为了验证分析的正确性和准确性,在本节中报告了基于模拟的结果。 此外,的算法在寻找最优时间分配方面的效率得到了证明。
为了验证本章分析方法的可行性,本节将随机过程分析模型的分析结果与 ONE 模拟器~\cite{DBLP:conf/simutools/OK09}对移动群智感知应用的仿真结果进行对比。
通过对比理论分析结果和模拟结果,验证随机过程分析方法的正确性以及感知应用执行时间划分算法的效果。
最后,根据理论分析结果和模拟结果,提出了面向群智感知效率优化的边缘资源调度策略。

\subsection{ONE 模拟器设置}
% 进行跟踪驱动模拟,以广泛评估对雾计算中MCS应用的基于ODE的分析。遵循ONE模拟器[ 24 ]的原理,捕获雾环境中的网络行为,并获得参数,如N;国家森林机构;andf. n通过追踪数据包路径和所有移动设备的状态,可以获得有效MCS参与者的数量、雾节点接收到的数据包数量,以及在一定时间内达到的覆盖率。在的随机过程分析中,这些信息可以由场景参数sN导出;国家森林机构;andf. n将分析结果和仿真结果进行比较,能够验证分析的正确性和准确性。为了彻底验证随机过程分析的准确性,首先验证对每个阶段的分析,然后对整体分析进行比较。
ONE~(Opportunistic Network Environment)模拟器~\footnote{The ONE simulator https://akeranen.github.io/the-one/} 是一个开源的机会式通信网络环境模拟器。
利用该模型器,可以构建具有如下特征的机会式通信网络:
1)移动设备可以单独配置其移动模型或移动轨迹;
2)可以在数据转发过程中部署不同的路由算法;
3)实时记录模拟网络环境中的移动设备运动状态和数据流状态;
4)利用地图信息和真实世界中移动设备的轨迹数据,模拟真实世界中的网络场景。

\begin{table}[h]
  % \vspace{-0.5em}
  \caption{ONE模拟器中的场景参数及其定义}
  \vspace{-0.5em}
  \centering
  \label{table_notations_ONE}
  % \centering
  \begin{tabular}{|c|p{7cm}|}
  \hline
  \textbf{场景参数} & \textbf{定义}\\
  \hline
  $N$ & 移动设备的数量\\\hline
  $N_f$ & 基站的数量\\\hline
  $\boldsymbol{L}[N]$ & 移动设备的位置\\\hline
  $\boldsymbol{V}$ & 移动设备的轨迹和速度\\\hline
  $\boldsymbol{L}[N_f]$ & 基站的位置\\\hline
  $f_{task}$ & 基站发送感知任务的频率\\\hline
  $A$ & 目标区域的面积\\\hline
  $d$ & D2D 通信的覆盖半径\\\hline
  % $\lambda_n$ & 移动设备之间的相遇概率\\\hline
  % $\lambda_f$ & 基站和移动设备的通信概率\\\hline
  % $I(t)$ & $t$ 时刻已经收到任务的移动设备的数量\\\hline
  % $I'(t)$ & $t$ 时刻 $I(t)$ 的增量\\\hline
  % $P_{rcv}(t)$ & $t$ 时刻基站收到感知数据的概率\\\hline
  \end{tabular}
  % \vspace{-1em}
\end{table}

% 在测试过程中,本章使用 ONE 模拟器对边缘计算模型下的移动群智感知应用进行模拟仿真。
在 ONE 模拟器中,可以设置的主要场景参数如表~\ref{table_notations_ONE} 所示。
模拟参数设置完成之后,ONE 模拟器会根据自定义的群智感知应用逻辑和数据转发路由算法,模拟移动设备的移动过程、设备之间的通信过程。
仿真完成后,模拟网路中的所有操作会以日志的方式进行保存。
通过对日志文件的分析,可以还原网络中的数据包流向,进而得到群智感知应用执行过程中,感知任务的部署数量、感知数据的收集数量、以及感知数据的覆盖范围。


% 在仿真实验中,主要的输入参数有:$N$、$N_f$、$\lambda_n$、$\lambda_f$,其具体含义如表~\ref{table_notations_ONE} 所示。
% 对仿真过程产生的日志信息进行处理,可以获得参与感知任务的移动设备数量以及每个数据包所经过的具体路径。
% 结合模拟器中移动设备的运动轨迹以及日志的时间戳进一步分析,还可以获得边缘服务器在任意时刻收到的感知数据数量以及已收集的感知数据所覆盖的地域范围。

利用模拟器中设置的$\boldsymbol{V}$、$A$ 和 $d$ 参数,可以利用式~\eqref{Formula_EncounterRate} 计算出模拟场景下移动设备之间的相遇率。
而基站发送感知任务的频率,可以转化为基站和移动设备的通信概率 $\lambda_f$。
结合 $N$ 和 $N_f$ 作为已知参数,带入式~\eqref{Formula_DataAmount} 中计算感知数据收集总量。
将模拟仿真和随机过程分析得到的结果进行对比,可以验证本章模型的正确性。
在对比过程中,先分别验证任务分发过程和数据收集过程的覆盖率,再针对移动群智感知全过程,验证感知效率。
最后分析边缘网络中各场景参数对感知效率的影响并提出调度策略。

% 首先通过使用不同的参数asN来验证对任务传播阶段的分析;国家森林机构;andf,which n定义在表一中。结果在图6中报告,包括模拟结果和通过求解等式获得的分析结果。( 4 )。从这个图中,可以观察到获得了高精度,因为分析结果都非常接近于模拟结果。在5000组实验中,在任务传播阶段,平均误差为5.7 %。还注意到,有效的MCS参与者( EMP )的数量随着时间的增加而增加。最初,增长率低的原因是雾节点,很少有“受感染”的移动设备能够使其他“易受感染”的移动设备生效。后来,有效的MCS参与者的数量急剧增加,因为已经有很多有效的MCS参与者。然而,这种增长最终会收敛,因为大多数移动电话已经被“感染”。这种现象符合的常识。

\subsection{群智感知随机模型分析}

首先通过使用不同的场景参数 $N$(移动设备数量)、$N_f$(基站数量)、$\lambda_n$(移动设备之间的相遇率)和 $\lambda_f$(移动设备和基站的通信概率)来验证本章对任务分发过程的分析。
图~\ref{Figure_PropagationTest} 反映了不同参数取值对移动群智感知效率的影响。
图中正下方的数值分别为场景参数 $N$、$N_f$、$\lambda_n$ 和 $\lambda_f$ 的取值。
图中横坐标为时间轴,纵坐标为目标区域内收到感知任务的移动设备数量,其结果可通过式~\eqref{Formula_Coverage_of_Task} 计算得到。
通过曲线对比分析结果和模拟结果,两者的变化趋势是非常相近的。
在5000组不同参数的实验对比中,任务分发数量的平均误差为 5.7\%。
对图像进行观察可以发现,移动群智感知应用执行初期,感知服务部署速度偏低,这是因为大部分感知任务只能通过基站分发。
在获得感知任务的移动设备增多之后,设备之间的 D2D 通信让区域内的任务分发源也越来越多,因此能够执行感知任务的移动设备数量呈指数趋势上升。
当感知任务的覆盖接近饱和时,未接受感知任务的移动设备数量较少,因此感知任务的部署速度逐渐降低,获得感知任务的移动设备总数逐步趋近于目标区域内的移动设备总数。

% \begin{figure}[!h]
%   \centering
%   % \vspace{-2em}
%   {\includegraphics[width=214pt]{./figures/Sec_UIC/Propagation/F3-1.pdf}}
%   {\includegraphics[width=214pt]{./figures/Sec_UIC/Propagation/F3-2.pdf}}\\
%   {\includegraphics[width=214pt]{./figures/Sec_UIC/Propagation/F3-3.pdf}}
%   {\includegraphics[width=214pt]{./figures/Sec_UIC/Propagation/F3-4.pdf}}
%   \vspace{-0.5em}
%   \caption{任务分发过程中分析结果和模拟结果对比}
%   \vspace{-0.5em}
%   \label{Figure_PropagationTest}
% \end{figure}

\begin{figure}[!h]
  \centering
  \begin{subfigure}[h]{0.48\linewidth}
    \centering
    \includegraphics[width=214pt]{./figures/Sec_UIC/Propagation/F3-1.pdf}
    \label{Figure_PropagationTestA}
    \vspace{-1.5em}
    \caption{$N=10000, N_f=10, \\\lambda_n=0.00004,\lambda_f=0.00003$}
  \end{subfigure}
  % \vspace{1em}
  \begin{subfigure}[h]{0.48\linewidth}
    \centering
    \includegraphics[width=214pt]{./figures/Sec_UIC/Propagation/F3-2.pdf}
    \label{Figure_PropagationTestB}
    \vspace{-1.5em}
    \caption{$N=2000, N_f=15, \\\lambda_n=0.0005,\lambda_f=0.0003$}
  \end{subfigure}\\
  \begin{subfigure}[h]{0.48\linewidth}
    \centering
    \includegraphics[width=214pt]{./figures/Sec_UIC/Propagation/F3-3.pdf}
    \label{Figure_PropagationTestC}
    \vspace{-1.5em}
    \caption{$N=40000, N_f=1, \\\lambda_n=0.000003,\lambda_f=0.000002$}
  \end{subfigure}
  \begin{subfigure}[h]{0.48\linewidth}
    \centering
    \includegraphics[width=214pt]{./figures/Sec_UIC/Propagation/F3-4.pdf}
    \label{Figure_PropagationTestD}
    \vspace{-1.5em}
    \caption{$N=40000, N_f=2, \\\lambda_n=0.000003,\lambda_f=0.000004$}
  \end{subfigure}
  \vspace{-0.5em}
  \caption{任务分发过程中分析结果和模拟结果对比}
% \vspace{-1.2em}
\label{Figure_PropagationTest}
\end{figure}

然后,用同样的方法来验证数据收集过程的分析。
假设在数据收集过程开始时($t=0$),区域内的所有移动设备都已经收到感知任务且采集到一份感知数据。
随着时间的推移,移动设备需要借助蜂窝网络和 D2D 通信将感知数据发送给 $N_f$ 个基站中的任意之一。
其中,移动设备和基站的蜂窝通信依然遵循通信概率 $\lambda_f$。
因此,可以利用式~\eqref{Formula_ProbaRcvTF} 计算被成功接收数据包的理论数量。
在 ONE 模拟器中,也可以利用日志信息获得每个基站收集到的感知数据数量。
图~\ref{Figure_CollectionTest} 展示了模拟试验和理论分析的对比结果。
图中,两者的变化趋势依然保持一致。
在5000组不同参数的实验对比中,数据收集理论分析的平均误差为 9.6\%。
且变化趋势和实际情况相符。
同时,通过对比图~\ref{Figure_PropagationTest} 和图~\ref{Figure_CollectionTest} ,可以发现在感知数据收集末期,理论结果和模拟结果有明显逼近趋势。
在模拟实验中,感知数据收集总量在某些时刻突然大幅增加,造成这种现象的原因是因为某一移动设备在和基站通信之前经由 D2D 通信得到了其它移动设备的感知数据。

% \begin{figure}[!h]
%   \vspace{-0.5em}
%   \centering
%   {\includegraphics[width=214pt]{./figures/Sec_UIC/Collection/R1-0.pdf}}
%   {\includegraphics[width=214pt]{./figures/Sec_UIC/Collection/R1-1.pdf}}\\
%   {\includegraphics[width=214pt]{./figures/Sec_UIC/Collection/R1-2.pdf}}
%   {\includegraphics[width=214pt]{./figures/Sec_UIC/Collection/R1-3.pdf}}
%   \vspace{-0.5em}
%   \caption{数据收集过程中分析结果和模拟结果对比}
%   \vspace{-1em}
%   \label{Figure_CollectionTest}
% \end{figure}

\begin{figure}[!h]
  \centering
  \begin{subfigure}[h]{0.48\linewidth}
    \centering
    \includegraphics[width=214pt]{./figures/Sec_UIC/Collection/R1-0.pdf}
    \label{Figure_CollectionTestA}
    \vspace{-1.5em}
    \caption{$N=10000, N_f=10, \\\lambda_n=0.000004,\lambda_f=0.000006$}
  \end{subfigure}
  % \vspace{1em}
  \begin{subfigure}[h]{0.48\linewidth}
    \centering
    \includegraphics[width=214pt]{./figures/Sec_UIC/Collection/R1-1.pdf}
    \label{Figure_CollectionTestB}
    \vspace{-1.5em}
    \caption{$N=10000, N_f=100, \\\lambda_n=0.000004,\lambda_f=0.000009$}
  \end{subfigure}\\
  \begin{subfigure}[h]{0.48\linewidth}
    \centering
    \includegraphics[width=214pt]{./figures/Sec_UIC/Collection/R1-2.pdf}
    \label{Figure_CollectionTestC}
    \vspace{-1.5em}
    \caption{$N=3000, N_f=2, \\\lambda_n=0.00001,\lambda_f=0.00002$}
  \end{subfigure}
  \begin{subfigure}[h]{0.48\linewidth}
    \centering
    \includegraphics[width=214pt]{./figures/Sec_UIC/Collection/R1-3.pdf}
    \label{Figure_CollectionTestD}
    \vspace{-1.5em}
    \caption{$N=3000, N_f=18, \\\lambda_n=0.00002,\lambda_f=0.00009$}
  \end{subfigure}
  \vspace{-0.5em}
  \caption{数据收集过程中分析结果和模拟结果对比}
% \vspace{-1.2em}
\label{Figure_CollectionTest}
\end{figure}


% \begin{figure}[!h]
%   \centering
%   % \vspace{-1em}
%   {\includegraphics[width=214pt]{./figures/Sec_UIC/RcvsSim/0.pdf}}
%   {\includegraphics[width=214pt]{./figures/Sec_UIC/RcvsSim/1.pdf}}\\
%   {\includegraphics[width=214pt]{./figures/Sec_UIC/RcvsSim/2.pdf}}
%   {\includegraphics[width=214pt]{./figures/Sec_UIC/RcvsSim/3.pdf}}
%   \vspace{-0.5em}
%   \caption{不同时间配额下的分析结果和模拟结果}
%   \vspace{-1em}
%   \label{Figure_EntireTest}
% \end{figure}

\begin{figure}[!b]
  \centering
  \begin{subfigure}[h]{0.48\linewidth}
    \centering
    \includegraphics[width=214pt]{./figures/Sec_UIC/RcvsSim/0.pdf}
    \label{Figure_EntireTestA}
    \vspace{-1.5em}
    \caption{$N=5000, N_f=2, \\\lambda_n=0.00003,\lambda_f=0.0004$}
  \end{subfigure}
  % \vspace{1em}
  \begin{subfigure}[h]{0.48\linewidth}
    \centering
    \includegraphics[width=214pt]{./figures/Sec_UIC/RcvsSim/1.pdf}
    \label{Figure_EntireTestB}
    \vspace{-1.5em}
    \caption{$N=5000, N_f=6, \\\lambda_n=0.00003,\lambda_f=0.008$}
  \end{subfigure}\\
  \begin{subfigure}[h]{0.48\linewidth}
    \centering
    \includegraphics[width=214pt]{./figures/Sec_UIC/RcvsSim/2.pdf}
    \label{Figure_EntireTestC}
    \vspace{-1.5em}
    \caption{$N=6000, N_f=2, \\\lambda_n=0.00005,\lambda_f=0.0012$}
  \end{subfigure}
  \begin{subfigure}[h]{0.48\linewidth}
    \centering
    \includegraphics[width=214pt]{./figures/Sec_UIC/RcvsSim/3.pdf}
    \label{Figure_EntireTestD}
    \vspace{-1.5em}
    \caption{$N=6000, N_f=6, \\\lambda_n=0.00005,\lambda_f=0.0016$}
  \end{subfigure}
  \vspace{-0.5em}
  \caption{不同时间配额下的分析结果和模拟结果}
% \vspace{-1.2em}
\label{Figure_EntireTest}
\end{figure}

最后,结合群智感知任务分发和感知任务收集,对边缘网络中群智感知应用执行过程进行模拟和分析。
在 ONE 中,设置不同的参数以模拟不同场景下的群智感知应用执行状态。
通过设置不同的任务分发执行时间 $t$,观察在预设生命周期 $T_{total\_time}$ 内基站可以收到的感知数据数量。
同时,利用式~\eqref{Formula_DataAmount} 可以通过理论分析计算出感知数据的收集总量。
该实验使用了四组不同的场景参数来观察不同的感知应用执行时间划分对最终感知数据收集总量的影响。
在所有的对比实验中,移动群智感知应用的生命周期 $T_{total\_time}$ 被设置为100秒。
令感知任务分发的执行时长为 $t$,则从 $t$ 时刻开始,所有移动设备和基站不再继续分发感知任务,而是开始收集感知数据。
由于感知应用的生命周期为 100 秒时,若感知任务分发的执行时间为0秒时,代表没有感知任务进行分发,因此收集到的感知数据总数为0;同样,当感知任务分发的执行时间为100秒时,代表没有进行感知数据的收集,此时收集到的感知数据总数也为0。
模拟结果和分析结果的对比如图~\ref{Figure_EntireTest} 所示。
横坐标 $t$ 表示感知应用执行过程的时间划分,即感知任务分发过程所占用的时间。
由于感知数据可以直接从传感器读取,时间占用极少,
因此 $T_{total\_time} - t$  表示了感知数据收集的时长。
图示结果直观的反映出本章的理论分析方法和模拟实验结果非常吻合,在不同时间配额划分方案下,分析结果总是接近模拟结果。
这也验证了本章对任务分发和数据收集两个阶段的分析也是准确无误的。
此外,在图中还能观察到,接收到的感知数据总量首先随着任务分发的时间配额的增加而增加,在达到最大值之后,可收集的感知数据总量开始变少。
由于在固定生命周期内,需要同时兼顾感知任务的分发和感知数据的收集,因此,当感知应用执行时间的划分不合理时,会导致感知数据的收集总量减少,降低感知服务质量。
这也意味着存在一个最佳时间分配,可以最大限度地提高感知效率。


% \vspace{-1em}
% \begin{figure}[!ht]
%   \centering
%   \vspace{-1.5em}
%   {\includegraphics[width=214pt]{./figures/Sec_UIC/BestT/0.pdf}}
%   {\includegraphics[width=214pt]{./figures/Sec_UIC/BestT/1.pdf}}\\
%   {\includegraphics[width=214pt]{./figures/Sec_UIC/BestT/2.pdf}}
%   {\includegraphics[width=214pt]{./figures/Sec_UIC/BestT/3.pdf}}
%   \vspace{-0.5em}
%   \caption{不同时间划分的对感知效率的影响}
%   \vspace{-1.5em}
%   \label{Figure_BestT}
% \end{figure}

\begin{figure}[!b]
  \centering
  \begin{subfigure}[h]{0.48\linewidth}
    \centering
    \includegraphics[width=214pt]{./figures/Sec_UIC/BestT/0.pdf}
    \label{Figure_BestTA}
    \vspace{-1.5em}
    \caption{$t=25$s}
  \end{subfigure}
  % \vspace{1em}
  \begin{subfigure}[h]{0.48\linewidth}
    \centering
    \includegraphics[width=214pt]{./figures/Sec_UIC/BestT/1.pdf}
    \label{Figure_BestTB}
    \vspace{-1.5em}
    \caption{$t=50$s}
  \end{subfigure}\\
  \begin{subfigure}[h]{0.48\linewidth}
    \centering
    \includegraphics[width=214pt]{./figures/Sec_UIC/BestT/2.pdf}
    \label{Figure_BestTC}
    \vspace{-1.5em}
    \caption{$t=70$s}
  \end{subfigure}
  \begin{subfigure}[h]{0.48\linewidth}
    \centering
    \includegraphics[width=214pt]{./figures/Sec_UIC/BestT/3.pdf}
    \label{Figure_BestTD}
    \vspace{-1.5em}
    \caption{$t=59.5$s}
  \end{subfigure}
  \vspace{-0.5em}
  \caption{不同时间划分的对感知效率的影响}
% \vspace{-1.2em}
\label{Figure_BestT}
\end{figure}

为了验证时间配额划分算法的准确性,测试中使用参数 $N=6000$、 $N_f=4$、 $\lambda_n = 0.000027$、 $\lambda_f=0.00018$ 定义的模拟场景。
移动群智感知应用的生命周期 $T_{total\_time}$ 设置为 100 秒。
在目标区域内,一共有 $N$ 个移动设备,由于每个移动设备生成一组感知数据,因此可以利用收集到的感知数据总数与 $N$ 的比值来表示感知数据的收集比例,这一比例也反映出感知应用的执行效率。
在不同时间配额之前的时间段内,由于只执行了感知任务的分发,没有对感知数据进行收集工作,所以收集到的感知数据数量为 $0$。
图~\ref{Figure_BestT} 展示了任务分发过程的时间配额分别为25秒、50秒、75秒以及 59.5 秒时移动群智感知应用的执行效率。
其中,59.5 秒是利用感知任务执行划分算法找出的最佳任务分发时间配额。
可以发现,当感知任务分发的时间配额为59.5秒时,所收集到的感知数据总量远远高于其它三种时间划分方法。
通过图中四种时间划分的结果对比可以发现,当任务分发时间分配过少时(25秒、50秒),只有少量的移动设备收到感知任务,所以收集到的感知数据也较少。
当任务分发时间分配过多时(70秒),所有的感知数据无法及时被发送至基站,导致在生命周期内无法收集齐所有的感知数据。
当时间划分为 59.5 秒时,感知任务分发过程和感知数据收集过程达到平衡,此时可以收集到最多的感知数据。
在算法执行中,$N\_slots$ (章节~\ref{UIC:algo})被设置为 $10^7$,算法执行时间不超过 2500 毫秒。

\subsection{边缘资源对感知效率的影响}

在之前的测试中,以 $N$(移动设备数量)、$N_f$(基站数量)、$\lambda_n$(移动设备之间的相遇率)和 $\lambda_f$(移动设备和基站的通信概率)作为参数来描述边缘网络中的不同资源部署场景。
通过之前的测试结果反映出不同的场景参数对群智感知应用的执行效率有不同的影响效果。
基于此发现,本节单独讨论各参数对感知应用执行效率的影响,以说明边缘网络中不同资源对群智感知应用的作用。
在该实验中,将 $N=5000$、$N_f=2$、$\lambda_n = 0.00003$、$\lambda_f=0.0004$ 作为基准场景,然后单独变化各参数的取值,来观察感知数据的接收总量变化。
实验结果如图~\ref{Figure_SingleValTest} 所示。
群智感知应用的生命周期依然设置为 100 秒。
通过仿真数据和理论数据的对比,得知本章的理论分析模型和模拟过程基本吻合。
在实验中,$N_f$ 以倍数关系增长,$N$、$\lambda_n$ 和$\lambda_f$ 均以 10\% 步进等比例增长。
在对比中可以观察到,在边缘计算支撑下的移动群智感知应用中,
移动设备的数量$N$对感知效率的影响最大,其次是移动设备的相遇率 $\lambda_n$。
而基站数量 $N_f$ 和基站与移动设备的通信概率 $\lambda_f$ 的影响作用,相对较小。

\begin{figure}[!h]
  \centering
	% \vspace{-1em}
	{\includegraphics[width=210pt]{./figures/Sec_UIC/SingleVar/1n.pdf}}
	{\includegraphics[width=210pt]{./figures/Sec_UIC/SingleVar/1ln.pdf}}\\
	{\includegraphics[width=210pt]{./figures/Sec_UIC/SingleVar/1nf.pdf}}
	{\includegraphics[width=210pt]{./figures/Sec_UIC/SingleVar/1lf.pdf}}
	\vspace{-1em}
	\caption{各参数对感知效率的影响}
	\vspace{-1.5em}
	\label{Figure_SingleValTest}
\end{figure}

\subsection{模型效率分析}

对于群智感知应用的发起者而言,可以通过 ONE 模拟器来获知群智感知应用的执行效率。
尽管使用模拟器可以得到较真实时间数十倍以上的加速比,但是当模拟场景中节点数量增加时,模拟器的工作时长也会成倍增加。
该测试的运行环境使用了 Intel XEON E5-2696 CPU(@3GHz)、64G 内存,ONE 模拟器中设置的区域大小为1000000平方米,基站均匀分布在该区域内。
移动设备在该区域中,个体速度处于 $1m/s$ 至 $10m/s$ 之间,以随机游走的方式移动。
通过表~\ref{tab_one} 可以发现,不同的节点数量,让 ONE 模拟器的工作时长急剧增加。
而使用本章提出的群智感知执行过程分析模型,预测感知应用执行效率的时间复杂度为 $O(N_{slots})$。
其中,$N_{slots}$ 是感知应用生命周期的划分数量。
当 $N_{slots} = 10^6$ 时,同样测试环境下的计算时间为 0.487 秒。
因此,使用本章提出的分析模型,可以更快的预测群智感知应用的执行效率。

% Please add the following required packages to your document preamble:
% \usepackage{multirow}
\begin{table}[!h]
\centering
\caption{ ONE 模拟器在不同场景下的工作时间}
\label{tab_one}
\begin{tabular}{|c|c|c|c|c|c|c|}
\hline
\multicolumn{2}{|c|}{\multirow{2}{*}{\textbf{\begin{tabular}[c]{@{}c@{}}模拟耗时\\ 单位:秒\end{tabular}}}} & \multicolumn{5}{c|}{\textbf{移动设备数量}} \\ \cline{3-7} 
\multicolumn{2}{|c|}{} & \textbf{1000} & \textbf{2000} & \textbf{3000} & \textbf{4000} & \textbf{5000} \\ \hline
\multirow{7}{*}{\textbf{\begin{tabular}[c]{@{}c@{}}基\\ 站\\ 数\\ 量\end{tabular}}} & \textbf{4} & 14.48 & 56.70 & 147.12 & 266.24 & 426.21 \\ \cline{2-7} 
  & \textbf{5} & 16.00 & 71.03 & 160.79 & 326.92 & 498.94 \\ \cline{2-7} 
  & \textbf{6} & 20.91 & 83.83 & 200.72 & 380.97 & 637.36 \\ \cline{2-7} 
  & \textbf{7} & 23.06 & 97.93 & 229.49 & 458.05 & 694.22 \\ \cline{2-7} 
  & \textbf{8} & 26.94 & 106.89 & 251.90 & 480.30 & 788.00 \\ \cline{2-7} 
  & \textbf{9} & 29.26 & 117.59 & 284.35 & 533.33 & 866.55 \\ \cline{2-7} 
  & \textbf{10} & 33.14 & 131.58 & 305.30 & 573.59 & 925.05 \\ \hline
\end{tabular}
\end{table}

\subsection{不同边缘资源调度的收益分析}
\label{ISPA:结论}

% \begin{figure}[!h]
%   \centering
%   % \vspace{-1em}
% 	{\includegraphics[width=210pt]{./figures/Sec_UIC/args/1n.pdf}}
% 	{\includegraphics[width=210pt]{./figures/Sec_UIC/args/1ln.pdf}}\\
% 	{\includegraphics[width=210pt]{./figures/Sec_UIC/args/1nf.pdf}}
% 	{\includegraphics[width=210pt]{./figures/Sec_UIC/args/1lf.pdf}}
% 	% \vspace{-1em}
% 	\caption{各参数对感知效率的收益比例}
% 	% \vspace{-4em}
% 	\label{Figure_SingleVal}
% \end{figure}

为了更直观比较不同边缘资源对移动群智感知效率的影响,在此将 $N=5000$、$N_f=2$、$\lambda_n = 0.00003$、$\lambda_f=0.0004$ 做为图~\ref{Figure_SingleVal} 的基准测试场景(图中 base 实线)。
同样,感知应用的生命周期依然为100秒。
通过对四种不同参数进行变化,可以看出图中曲线变化趋势有着非常明显的不同。
在这四幅图中,横坐标代表移动群智感知应用的执行时间。
纵坐标代表当前时间点基站能收到感知数据数量。
因此,该图可以直观的反映出群智感知应用在执行过程中,哪一阶段能够收获最多的感知数据。
同时,图中曲线和横轴构成的图形的面积,也就是对应场景参数下感知数据收集的总数量。
图~\ref{Figure_SingleVal}(a) 表述的是群智感知中移动设备数量的变化对感知数据和感知效率的影响。
相较于其他三幅图,可以发现群智感知志愿者数量的基数对感知结果以及感知效率都有着巨大的影响。
这也从另一方面说明了大多数研究为何要关注于参与者的招募工作和激励工作。
通过对比图~\ref{Figure_SingleVal}(c) 和图~\ref{Figure_SingleVal}(d),可以发现基站的数量虽然会对感知数据的总量和感知效率产生影响,但是与其租用更多的基站作为边缘服务节点,提高基站和移动设备的通信频率显然更加划算。
当基站与移动设备的通信频率达到一定瓶颈时,此时再租用更多的基站作为边缘服务器,性价比更高。
另一方面,租用更多的基站,意味着蜂窝通信网络覆盖的范围更大。
随着目标区域的扩大,增加参与群智感知的移动设备数量可以显著提高群智感知的执行效率。
但是通过图~\ref{Figure_SingleVal}(b)和图~\ref{Figure_SingleVal}(c)的对比,可以发现是否需要增加基站来作为群智感知应用的边缘服务节点,还需要根据移动设备的运动状态进行调整。

\begin{figure}[!h]
  \centering
  % \vspace{-1em}
  \begin{subfigure}[t]{0.48\linewidth}
    \includegraphics[width=210pt]{./figures/Sec_UIC/args/1n.pdf}
    \vspace{-1.5em}
    \caption{移动设备数量对感知效率的影响}
  \end{subfigure}
  \begin{subfigure}[t]{0.48\linewidth}
    \includegraphics[width=210pt]{./figures/Sec_UIC/args/1ln.pdf}
    \vspace{-1.5em}
    \caption{移动设备相遇率对感知效率的影响}
  \end{subfigure}\\
  \begin{subfigure}[b]{0.48\linewidth}
    \includegraphics[width=210pt]{./figures/Sec_UIC/args/1nf.pdf}
    \vspace{-1.5em}
    \caption{边缘服务数量对感知效率的影响}
  \end{subfigure}
  \begin{subfigure}[b]{0.48\linewidth}
    \includegraphics[width=210pt]{./figures/Sec_UIC/args/1lf.pdf}
    \vspace{-1.5em}
    \caption{边缘设备通信概率对感知效率的影响}
  \end{subfigure}
	% \vspace{-1em}
	\caption{各参数对感知效率的收益影响}
	% \vspace{-4em}
	\label{Figure_SingleVal}
\end{figure}

\section{本章小结}

本章研究了边缘计算支撑下的典型移动群智感知应用,利用边缘服务进行感知任务传播和感知数据收集。
同时,采用 D2D 通信构建机会式通信网络来加速任务分发和感知数据收集。
借助随机过程分析方法,针对感知应用的执行过程建立模型,通过常微分方程组来描述群智感知的执行过程。
通过求解方程组,找到了边缘服务数量、移动设备数量、移动设备运动状态、群智感知应用覆盖率以及感知数据收集率的关系。
通过任务传播过程和数据收集过程的量化分析,在感知应用生命周期固定的情况下,研究了感知效率影响因素优先级和改进方法,并提出任务分发和数据收集两个过程的时间划分方法。
最大限度地提高移动群智感知应用的感知效率。
并通过模拟实验,验证了模型的正确性和优化机制的有效性。
利用该模型,可以根据群智感知应用场景更快地预测感知任务在生命周期内可以收集的感知数据数量。同时,也可用来快速预测移动设备的运动状态以及边缘服务部署对感知效率的影响,以辅助调整群智感知应用在边缘环境下的执行优化策略。
