@article{IGNACIO20081982,
title = "Lower and upper bounds for a two-level hierarchical location problem in computer networks",
journal = "Computers & Operations Research",
volume = "35",
number = "6",
pages = "1982 - 1998",
year = "2008",
note = "Part Special Issue: OR Applications in the Military and in Counter-Terrorism",
issn = "0305-0548",
doi = "https://doi.org/10.1016/j.cor.2006.10.005",
url = "http://www.sciencedirect.com/science/article/pii/S0305054806002541",
author = "Aníbal Alberto Vilcapoma Ignacio and Virgílio José Martins Ferreira Filho and Roberto Diéguez Galvão",
keywords = "Hierarchical location problems, Design of computer networks, Lagrangean relaxation, Tabu search",
abstract = "In this paper a two-level hierarchical model for the location of concentrators and routers in computers networks is presented. Given a set of candidate locations and the capacities of concentrators and routers the problem is to determine how many concentrators and routers should be used, where they should be located and which users to assign to each concentrator and which concentrators to allocate to each router. A Lagrangean relaxation is used to provide lower bounds for this problem, which are tentatively strengthened by incomplete optimization through the use of CPLEX. A tabu search meta-heuristic is then developed. Computational experiments are performed using a set of randomly generated problems."
}