\documentclass[format=draft,language=chinese,degree=phd,table,usenames,dvipsnames]{hustthesis}

\usepackage{graphicx}
\usepackage{color}
\usepackage{xcolor}
\usepackage{subcaption}
\usepackage{setspace}
\usepackage{algorithm2e, setspace}
% \usepackage[caption=false,font=footnotesize]{subfig}
\usepackage{etoolbox}
% 取消连字符
% \usepackage[none]{hyphenat}

\newcommand{\algostretch}{1.25}
%\onehalfspacing
\setstretch{1.66666}
%\setlength{\tabcolsep}{1.2}
\renewcommand{\arraystretch}{1.2}

\newcommand{\red}[1]{{\color{red}{#1}}}
\def\fp{\textit{fp}}
\def\ft{\textit{ft}}
\def\vfp{\textit{vfp}}
\def\vft{\textit{vft}}
\def\mr{\textit{mr}}
\def\vmr{\textit{vmr}}
\def\sz{\textit{sz}}
\def\bw{\textit{bw}}
\def\trfc{\textit{trfc}}
\def\fr{\textit{fr}}
\def\dram{\textit{dram}}
\def\pcm{\textit{pcm}}
\def\miss{\textit{miss}}
\def\wss{\textit{wss}}

%\def\FC{\textit{FractionCache}}
\def\FC{\text{FractionCache}}

\def\AR{\textit{AR}}
\def\MR{\textit{MR}}
\def\SFMR{\textit{SFMR}}
\let\origcftsecpagefont\cftsecpagefont
\let\origcftsecafterpnum\cftsecafterpnum
\renewcommand{\cftsecpagefont}{(\origcftsecpagefont}
\renewcommand{\cftsecafterpnum}{\origcftsecafterpnum)}

\newcommand*{\citen}[1]{%
  \begingroup
    \romannumeral-`\x % remove space at the beginning of \setcitestyle
    \setcitestyle{numbers}%
    \cite{#1}%
  \endgroup
}

\stuno{D201177568}
\schoolcode{10487}
\title{基于边缘计算的\\移动群智感知执行优化机制}{Execution Optimization Mechanisms for \protect{\\} Edge Computing empowered Mobile Crowdsensing}
\author{侯海翔}{Haixiang Hou}
\major{计算机系统结构}{Computer Architecture}
\supervisor{金海\hspace{1em}教授}{Prof. Hai Jin}
\date{2019}{6}{1}

\zhabstract{
% 论文针对移动群智感知中感知、通信、计算三个重要环节

% 随着云计算技术的成熟与完善,移动群智感知通过无处不在的移动设备收集感知数据,并利用云数据中心存储、处理感知数据,将物理世界数字化,以提供更加专业、精准的服务。
% % 然而,仅靠云端融合,并不足以支撑感知任务的快速部署,以及海量感知数据的传输、存储和处理需求。
% 然而,随着移动群智感知的规模不断扩大,其产生的海量感知数据也为数据的传输、存储以及处理带来了全新的挑战。
% 借助边缘计算,使用边缘计算和端到端(Device-to-Device,简称D2D)通信等技术,不仅能在网络的边缘侧对感知数据进行处理,还可以减少移动群智感知对主干网络资源的占用,为感知任务的快速部署、感知数据的处理与传输,提供了新的优化契机。
% 如何有机融合边缘设备和端设备实现大范围、高效率、低成本的移动群智感知应用,已成为目前亟待解决的重要挑战。

% 移动群智感知将大量的移动设备作为基本感知单元,收集感知数据,将物理世界数字化,提供更加专业、精准的服务。
移动群智感知将大量的移动设备作为基本感知单元,以大规模协作的方式收集感知数据,实现物理世界的数字化。
然而,随着移动群智感知规模的不断扩大,海量感知数据的传输、存储以及处理面临着全新的挑战。
边缘计算和端到端(Device-to-Device,简称D2D)通信等技术的出现,为感知任务的快速部署、感知数据的传输与处理,提供了新的契机。
借助边缘计算,不仅可以在网络边缘侧实现感知数据的快速处理,还能减少对主干网络资源的占用。
因此,为实现大范围、高效率、低成本的移动群智感知,如何有机融合边缘设备与端设备,已成为移动群智感知研究的重要任务之一。

% 移动群智感知利用众包的方式将数据采集任务分发给无处不在的智能移动设备,并通过设备内集成的各种传感器收集感知数据。
% 边缘计算作为与云计算模型互补的分布式计算模型,具备分布广、设备多以及近用户等特性,为移动群智感知质量优化和效率优化提供了新的契机。
% 另一方面,利用5G 通信的潜在支撑技术之一端到端(Device-to-Device,简称D2D)通信,不仅有助于移动设备间的通信效率,还可以减少移动群智感知对主干网络资源的占用。
% 如何有机结合边缘计算和 D2D 通信实现大范围、高效率、低成本的移动群智感知应用,已成为目前亟待解决的重要挑战。

% 感知设备 移动 不确定性
% 移动群智感知执行效率 难以预测 模拟耗时长

针对边缘计算场景,国内外学者目前已经使用边缘计算和 D2D 通信来优化移动群智感知的执行效率、降低执行成本和能耗。
其中,部署边缘计算服务可以加快感知任务的分发和感知数据的收集;而通过在网络边缘使用 D2D 通信技术,则可以构建一个范在通信网络环境,有效减少传统蜂窝通信流量。
%针对边缘计算场景,国内外学者目前已经使用边缘计算和 D2D 通信来优化移动群智感知的执行效率、降低执行成本和能耗。
%其中,部署边缘计算服务可以加快感知任务的分发和感知数据的收集;而通过在网络边缘使用 D2D 通信技术,则可以构建一个范在通信网络环境,有效减少传统蜂窝通信流量。
% 可以减少感知数据传输对主干网络带宽资源的占用并减少数据传输成本。
由于群智感知中参与设备众多、运用场景多变,仍有许多影响其执行效率的因素需要深入讨论研究。
目前,对于移动群智感知执行优化仍存在以下三个问题:
1) 端设备的运动状态会改变 D2D 通信的机率,使移动群智感知执行效率难以分析、预测与优化;
%2)边缘计算场景中同时涉及面向海量群智感知数据流,网络资源与计算资源,计网协同,网络通信与计算任务卸载协同优化
2) 独立的网络资源或计算资源调度优化策略,均无法充分发挥边缘计算的资源协同管理优势,导致移动群智感知数据处理效率低、代价高;
%边缘计算场景中同时涉及面向海量群智感知数据流,网络资源与计算资源,计网协同,网络通信与计算任务卸载协同优化,使、导致、引发边缘计算效率低、成本高;
%2) 移动设备向边缘计算平台卸载感知数据处理任务时会产生大量的数据流,而数据流调度不当会因网络资源限制对感知数据的处理效率产生影响;
3) 大规模移动群智感知中,由于对海量感知数据的分布特征和迁移规律考虑不足,导致边缘服务的利用率欠佳。
% 收集工作,欠缺低成本的高效收集方法。
针对以上问题,提出多个针对性的优化机制,对边缘计算场景下的移动群智感知执行过程进行相应优化。

% 在移动群智感知中,由于大量感知设备的移动状态具有不确定性,导致 D2D 通信构成的机会式通信网络让移动群智感知的执行效率难以预测、评估。
% 同时,在网络边缘利用边缘设备处理感知数据,需要对大量的群智感知数据流进行管理和调度。
% 另一方面,离散分布的海量感知数据,

% 在网络的边缘侧,使用成本低廉的 D2D 通信以及边缘计算技术可以对移动群智感知应用产生许多积极的影响作用,但是对于移动群智感知执行优化仍存在以下三个问题:
% 1) 端设备的运动状态对移动群智感知执行效率的影响。由于 D2D 网络的机会式通信特性,使用 D2D 通信分发感知任务、收集感知数据时,移动设备的运动状态会影响移动群智感知的执行效率。
% 2) 借助计算卸载处理感知数据时的网络流调度问题。对于计算能力较弱的感知设备,需要将部分计算任务卸载到边缘设备中。而感知设备数量庞大,需要在网络边缘侧对大量数据流进行合理的调度,以保证感知数据的处理效率。
% 3) 大规模移动群智感知中海量感知数据的收集问题。在大规模移动群智感知应用中,例如智慧城市感知,如何利用边缘计算服务收集物理世界中离散分布的海量感知数据,也是提高感知数据收集效率、减少主干网络资源占用的关键问题。

% 尽管使用成本低廉的 D2D 通信以及边缘计算技术可以对移动群智感知应用产生许多积极的影响作用,但是对于移动群智感知执行优化仍存在以下不足之处:
% 第一,在感知过程中,由于 D2D 网络的机会式通信特性,使用 D2D 通信传播感知任务,收集感知数据时,移动设备的运动状态会影响群智感知应用的执行,亟待深入分析该影响,并提升群智感知执行效率;
% 第二,在通信过程中,因感知设备资源不足,需将部分计算任务卸载到其他设备(或云中心)上,所产生的大量网络通信将影响应用执行效率。因此,如何利用高效便捷的网络资源管理技术(例如软件定义网络)调度边缘网络资源,保障网络通信需求,减少感知数据传输时延和处理时延,也是优化群智感知执行效率的关键工作之一;
% 第三,在数据处理过程中,针对大规模移动群智感知应用(例如智慧城市感知),感知数据的处理需要边缘计算服务的支撑,还需研究探讨边缘计算服务的部署对感知数据收集和处理效率的影响。

针对端设备的运动状态对移动群智感知执行效率的影响,提出了基于边缘计算的群智感知任务调度机制,以
% 优化移动群智感知的执行效率。
提高感知任务在多任务传播源、多数据接收端场景下的执行效率。
通过在网络的边缘侧同时使用蜂窝通信和 D2D 通信,提高群智感知任务在多发射源、多接收端场景下的执行效率。
同时,借助随机过程理论分析了感知任务的执行过程,
挖掘出影响群智感知执行效率的主要因素,例如移动设备的移动特征、边缘服务的部署、群智感知生命周期等。
%基于理论模型,设计了感知任务执行时间的最佳划分方法,找出感知任务执行的最佳时间划分,并分析了移动特征、边缘计算服务部署等因素对感知任务执行效率的影响效果。
基于理论模型,设计了最大化感知任务执行效率的时间划分方法,并分析了移动特征、边缘服务部署等因素对感知任务执行效率的影响。

% 针对感知过程中移动设备运动状态对 D2D 通信造成的影响
% ,提出了基于感知效率的任务调度机制,以提高感知任务在多任务传播源、多数据接收端场景下的执行效率。
% 目前的分析方法,无法准确描述移动群智感知应用的完整执行过程。
% 因此,首先借助随机过程分析感知任务的执行过程,
% 挖掘出影响群智感知执行效率的主要因素,例如移动设备的移动特征、边缘计算服务的部署、群智感知生命周期等。
% 基于理论模型,设计了感知任务执行时间的划分方法,在感知任务生命周期内提高移动群智感知的感知效率。

%其次,针对利用计算卸载处理感知数据时的网络流调度问题,基于计网协同原则,提出了面向计算任务卸载的群智感知网络流调度机制,以提高感知数据处理效率并减少能耗开销。
针对群智感知数据处理时计算资源与网络资源相互牵制的问题,基于计网协同原则,提出了面向计算任务卸载的群智感知网络流调度机制,以提高感知数据处理效率并减少能耗开销。
为解决使用计算卸载时所导致感知设备和边缘设备之间存在的大量数据流问题,
% 使用计算卸载技术,会导致感知设备和边缘设备之间产生大量的数据流。
通过软件定义网络(Software-Defined Networking,简称SDN)技术对这些端到边的数据流进行管理和调度。
该方案综合考虑了带宽、时延的约束,以及 SDN 交换机流表容量约束所带来的影响。
首先,利用排队论模型分别描述了端设备和边缘设备上的计算任务处理过程。
%然后,建立链路资源约束条件、计算卸载决策以及数据处理能耗之间的整数线性规划模型。
然后,综合考虑数据处理时延要求、网络资源以及流表容量限制等约束条件,建立计算卸载决策与网络流调度协同优化的整型线性规划模型。
并通过对该模型进行分析,设计了低计算复杂度的计网协同调度算法,取得了最佳调度方案87.4\%的节能效益。

% 针对通信过程中的网络资源调度问题,提出了面向计算任务卸载的网络流调度机制。
% 移动群智感知中,需要大量的可靠链接来保障性能较弱的感知设备与边缘计算服务之间进行计算任务卸载。
% 因此,可通过软件定义网络(Software-Defined Networking,简称SDN)技术对大量端到端、端到边网络链接进行管理和调度。
% 此时不仅需要考虑带宽、时延的约束,还需要考虑大量链接对 SDN 交换机流表容量产生的影响。
% 基于此现状,首先利用排队论模型描述了端设备和边缘设备上的计算任务处理过程,建立计算卸载决策、能耗以及链路资源之间的整数线性规划模型。
% 通过对该模型的分析求解,设计了保障感知数据平均处理时延的高能效任务卸载算法。

针对大规模移动群智感知中海量感知数据的收集问题,提出了基于城市居民移动特征的感知数据收集方法,以提高 D2D 通信利用率,从而达到节约成本的目的。
%并合理部署边缘计算服务,以更快捷地收集感知数据,。
%利用城市中已有的公交网络和公共交通中的乘客流动特征,建立了考虑乘客轨迹、公交路线的面向边缘计算服务部署决策的整型线性规划模型。
%利用城市中已有的公交网络和公共交通中的乘客流动特征,建立了
%通过对该模型的分析,
首先,将考虑公交网络拓扑以及乘客流动特征的感知数据收集过程描述为多商品流问题,并由此构建了面向边缘服务部署决策的整型线性规划模型。
接着,通过分析城市居民的移动规律,判断移动设备聚集的时机和区域,设计了边缘服务部署决策算法,借助D2D通信减少大规模移动群智感知的数据收集成本。
与基于整型线性规划模型的最优部署方案相比,所提算法总收集成本仅高出3.02\%,优于现有部署方案。


% 针对群智感知中的数据处理问题,提出了基于城市居民移动特征的边缘计算服务部署策略。
% 通过分析城市居民的移动特征,判断移动设备聚集的时机和区域并合理部署边缘计算服务,以更快捷地收集、处理感知数据,提高 D2D 通信的利用率。
% 借助城市中已有的公交网络和公共交通中的乘客流动特征,建立了乘客轨迹、公交路线和边缘计算服务部署决策的整数线性规划模型。
% 通过对该模型的分析求解,设计了可有效降低智慧城市感知应用中边缘计算服务部署成本的启发式算法。

综上所述,针对移动群智感知中任务的执行、数据的处理和收集,借助边缘计算和 D2D 通信,优化了群智感知应用的执行效率、降低感知数据的处理能耗、并减少感知数据的收集成本。
% 资源调度优化机制可以减少数据传输、数据处理的时间,在保障感知质量的前提下合理降低能耗成本和服务成本。
% 同时,与D2D通信的深度结合,为移动群智感知技术在 5G 时代中的运用打下坚实的基础。
}

\zhkeywords{边缘计算,移动群智感知,端到端通信,资源调度}

\enabstract{

Mobile crowdsensing distributes sensing tasks to ubiquitous mobile devices and collects sensing data from various sensors integrated on these devices. 
%However, it is challenging to achieve wide-coverage mobile crowdsensing with low latency. 
It is widely agreed that processing the sensing data at the centralized cloud suffers from low latency and high cost problems.
To address such problems, edge computing is considered as a complementary, or even alternative, computing paradigm to cloud.
%It provides new opportunities for mobile crowdsensing, thanks to the advantages from wide coverage, prevalent devices, and the support of heterogeneous network infrastructure.
Thanks to the close proximity to the data sources, edge computing not only enables fast sensing data processing, but is also promising in exploring various newly emerged communication technologies like \emph{Device-to-Device} (D2D) communication.
%D2D has become an efficient way to reduce the cost of data exchange during the sensing phase among mobile devices.
Therefore, there have been increasing research interests in integrating edge computing for mobile crowdsensing.

Deploying the crowdsensing services at the network edge can fasten the data collection and processing. Meanwhile, exploring D2D communications at the network edge, and between the mobile devices, can lower the reliance on cellular communications to lower the communication cost and promote the willingness of mobile crowdsensing participation. However, with the consideration of crowd participation, high mobility, and versatile crowdsensing scenarios, there still exist many issues affecting the crowdsensing efficiency to be addressed: 1) Due to mobility of the participated mobile devices, it is hard to accurately capture the possibilities of D2D communications, as well as to analyze and optimize the crowdsensing efficiency. 2) Solely focusing on either communication optimization or computation optimization cannot fully explore the advantages of resource convergence in edge computing empowered crowdsensing, leading to low efficiency and high cost. 3) Without the consideration of sensing data distribution and mobility characteristics, the edge service utility is limited.

%In general, mobile crowdsensing application contains three phases: sensing, communication, and data processing.
%In the sensing phase, the initiator of mobile crowdsensing task generally employs edge servers, as well as D2D communication, to conduct the dissemination of sensing tasks and the collection of sensing data.
%One of the key issues of enhancing the quality-of-sensing is how to orchestrate edge servers and D2D communication for prompting the efficiency of the sensing task and enlarge the sensing coverage.
%In the communication phase, data exchanges in mobile crowdsensing mainly adopts D2D communication supplemented by cellular communication, in order to reduce the communication overhead and the occupation of the backbone network resource.
%Challenged by massive amounts of sensing data transmission, it is also important to work on the \emph{Software-Defined Networking}~(SDN) based scheme to schedule the network resource and plan the transmission paths efficiently.
%In the data processing phase, edge computing model is widely used in mobile crowdsensing to process massive sensing data by edge services.
%We propose several resource schedule and optimization technologies to enhance the quality of sense, and to solve problems above in a conquer-and-divide manner.


In D2D based crowdsensing in edge computing, the mobile devices and edge servers actually form an opportunistic network with multiple sources and multiple destinations.
To optimize the crowdsensing efficiency, it is critical to accurately describe the sensing task dissemination and data collection behaviors with the consideration of device mobilities and edge server distributions. 
To this end, we first conduct stochastic analysis and build \emph{Ordinary Differential Equations}~(ODEs) to describe the opportunistic task dissemination and data collection behaviors. Our proposed framework is able to find out the major factors that affect the crowdsensing efficiency.
Then, based on our stochastic analysis framework, we design a time allocation scheme for lifetime limited crowdsensing applications, and analyze how the mobility pattern and edge server distribution affect the crowdsensing efficiency via extensive experiments.

Due to the limited computation capabilities on the mobile devices, some sensing data processing tasks must be offloaded to the edge servers. In this case, the offloading decision and the sensing data transmissions shall be jointly optimized to promote the task processing efficiency and reduce the unnecessary energy consumption. \emph{Software-Defined Networking}~(SDN)  has been regarded as an effective technology to manage the data flows at the network edges. In this case, besides the constraints from the network bandwidth, the quality-of-sensing requirements, the edge server processing capabilities, the flow table size limitations of the SDN switches are also un-ignorable. As a result, we first derive queuing based models to describe the sensing task processing on both mobile devices and edge servers, and then accordingly form the offloading decision and data flow scheduling for overall energy minimization into an \emph{Integer Linear Programming}~(ILP) problem. To tackle the high computation complexity of solving ILP, a low complexity heuristic algorithm is proposed, achieving 87.4\% energy efficiency against to the optimal solution.

%Second, in the communication phase, we propose a network resource scheduling mechanism based on task offloading with the SDN technology to optimize the communication efficiency of mobile devices in edge networks.
%For mobile devices of limited capabilities, the original sensing data can be directly handed over to the edge server for processing.
%However, the volume of raw sensing data, such as video and image, is huge, which easily introduces network congestion.
%Regarding this case, we build an \emph{integer linear programming}~(ILP) model between offload decision, energy consumption, and network resource which is established.
%By analysising and sloving the model, we archieve the load balance of links in the edge network and maximize the benefits of computing offloading.


It is desirable to maximization the utilization of D2D communications by deploying large number of mobile devices to lower the crowdsensing operational cost. But this may result in high capital cost at the same time. To this end, a service deployment strategy that balances the operational cost and capital cost by exploring the public transportation network and citizen mobility characteristics is proposed. The sensing data collection at the edge services is described as a multi-commodity flow problem, based on which the cost minimization oriented edge service deployment problem is formulated into an ILP form. By analyzing the citizen mobility characteristics to identify the D2D based data collection opportunities, a edge service deployment strategy is invented. Compared with the optimal solution, the invented strategy only requires 3.02\% higher cost, significantly outperforms existing strategies.

%Third, in the data processing phase, we proposes an edge service scheduling mechanism based on the urban social network to solve the deployment problem of edge service  for mobile crowdsensing in cities. 
%Under this scenario, edge services are usually deployed with the help of the social network characteristics of urban residents to improve the sensing efficiency of mobile crowdsensing.
%Regarding this case, we take passenger trajectoris, bus lines, and edge service deployment decisions into account, and build an ILP model using the existing bus network and passenger flow characteristics in public transportation.
%By analysising and sloving the model, the deployment cost of edge services in urban mobile crowdsensing applications is reduced.

In summary, we systematically analyze and optimize the main phases in edge computing empowered mobile crowdsensing, with a special emphasis on the exploration of D2D communications. The proposed strategies can promote the quality-of-sensing, improve the energy efficiency, and reduce the crowdsensing cost.
%In summary, focusing on the three main phases of mobile crowdsensing, we use edge resource scheduling optimization technologies to reduce the time of data transmission and data processing, while reasonably reducing the energy consumption and service cost, to ensure the sensing efficiency. 
%In addition, it also makes mobile crowdsensing more suiectable for 5G-enabled applications in the tight context of D2D communication.
}
\enkeywords{Edge Computing, Mobile Crowdsensing, D2D Communication, Resource Scheduling}

\begin{document}

\frontmatter
\maketitle
\makeabstract
\tableofcontents
\listoffigures
\listoftables
\mainmatter
% \def\A{\forall}
\def\Tusresponse{{T}_{us\_response}}
\def\Tus{{T}_{us}}
\def\Tsu{{T}_{su}}
\def\Tscloud{{T}_{s\_server}}
\def\Tuscloud{{T}_{us\_server}}
\def\Tulocal{{T}_{u\_local}}
\def\Tuslocal{{T}_{us\_local}}
\def\au{{\alpha}_{u}}
\def\as{{\alpha}_{s}}
\def\aus{{\alpha}_{us}}
\def\xus{{x}_{us}}
\def\xu{{x}_{u}}
\def\us{{\mu}_{s}}
\def\uu{{\mu}_{u}}
\def\lambdau{{\lambda}_{u}}
\def\lambdaus{{\lambda}_{us}}
\def\Tl{{T}_{l}}
\def\lus{l_{us}}
\def\Lus{\boldsymbol{L}_{us}}
\def\xlus{x_{\lus}}
\def\ylus{y_{\lus}}
\def\Tlus{T_{\lus}}
\def\Blus{B_{\lus}}
\def\Bu{B^{u}_u}
\def\Bd{B^{d}_u}
\def\xrl{x^{\lus}_{r}}
\def\eu{e_u}
\def\es{e_s}


% TODO

% [ ] 潘老师的修改意见
% [ ] 泽哥的修改意见
% [ ] D2D 通信需要增加说明情况

\chapter{绪论}

% \textbf{柔和一些,尝试使用这个逻辑:移动群智感知很重要,是补充,甚至是替代传统无线传感网的有效途径。同时,云计算是对传感数据进行处理的重要支撑平台。然而,……乏力。边缘计算,……能够有效地弥补云计算的不足。然而,如何……仍旧是亟待解决的问题。本文将就此从感知、通信以及计算三个角度展开分析讨论。本章……}

随着 5G 通信标准的确立,物联网应用进入了新的篇章。
% 在无所不在的移动设备中,大部分都同时装备了多种传感器,并具有一定的计算能力和网络通信能力。
面对生活中无处不在的移动设备和其内部集成的传感芯片,研究者们使用技术手段将它们联合起来,构建成一个比传统无线传感网络效率更高、适用范围更广、成本更低廉的新型应用范式——移动群智感知。
% 利用群质感知,可以在真实世界中提取数字信息以达到环境测量、监控等目的,并创造出更多有社会价值的应用。
然而,移动群智感知应用中大量的参与设备和海量分散的感知数据,对感知任务的调度、感知数据的传输和处理,都提出了极高的要求。
云计算作为集中式服务模型,在支撑覆盖区域广、时效性强的移动群智感知应用时已渐显乏力。
边缘计算作为一种与云计算互补的分布式计算模型,能够有效地弥补云计算的不足。
然而,如何利用其特性为移动群智感知提供基础支撑,是一个亟待解决的研究课题。
本文将针对移动群智中的感知、通信以及数据处理三个环节展开分析讨论。
本章首先介绍移动群智感知场景下边缘计算中的资源调度问题;其次简要介绍当前国内外研究现状和现有工作的不足;然后阐述本文研究的目的与意义;最后介绍论文的组织结构和层次关系。
% 移动群智感知中,参与设备数量庞大,感知数据种类繁多、结构各异,导致传感数据量也达到了新的数量级。
% 为了快速处理庞大的感知数据集,通常使用云计算作为移动群智感知的重要支撑平台。
% 然而,云计算作为集中式服务模型,在支撑覆盖区域广、时效性强的移动群智感知应用时已渐显乏力。
% 边缘计算作为一种与云计算互补的分布式计算模型,能够有效地弥补云计算的不足。
% 如何利用其特性为移动群智感知服务提供基础支撑,是一个亟待解决的研究课题。
% 本文将就此从感知、通信以及计算三个角度展开分析讨论。
% 本章首先介绍移动群智感知场景下边缘计算中的资源调度问题;其次简要介绍当前国内外研究现状和现有工作的不足;再次阐述了本文研究的目的与意义;最后介绍论文的组织结构和层次关系。

\section{研究背景}

\subsection{边端融合}
边缘计算(Edge Computing)~\cite{DBLP:journals/cacm/ArmbrustFGJKKLPRSZ10}是与云计算(Cloud Computing)相辅相成的一种新型计算模式。
云计算作为集中式处理模式,需要将用户数据收集至云数据中心~\cite{DBLP:conf/icdcs/Montresor16},才能有效地服务于用户,即“数据入云”。
但在面临 万物互联的新型网络空间时,云计算服务将面临以下挑战:

1)\textbf{实时性不足}。
在物联网场景中,存在着许多强实时性的应用。在云计算中,数据需要不断在客户端和云端之间往返,导致响应时延增加、用户体验下降。
例如在智能驾驶应用中,云计算无法达到毫秒级的数据处理时延~\cite{DBLP:conf/cvpr/GeigerLU12}。
移动终端上的虚拟现实(Virtual reality,简称VR)框架 Furion~\cite{DBLP:conf/mobicom/LaiHCSD17}在探索过程中发现,仅靠云计算服务无法帮助移动终端获得高质量的实时 VR 服务。
而利用边缘计算将渲染服务卸载到边缘服务器中,MUVR~\cite{DBLP:conf/edge/LiG18}成功实现了一个低通信且稳定的多用户 VR 框架。

2)\textbf{带宽不足}。
在网络空间中,将边缘设备产生的数据传送至云计算中心,需要消耗大量的带宽资源。
Intel 在 2016 年的报告~\cite{DBLP:journals/micro/KatoTINTH15}指出,一辆智能驾驶汽车工作一天可以产生 4 太字节(Terabyte,简称TB) 的数据。
而一架波音787飞机在飞行途中,其数据产生速率达到了 5 吉字节(Gigabyte,简称GB)每秒~\cite{JCRD/shi17}。
如此巨大的数据量,利用云服务进行保存和处理,不仅需要耗费大量的带宽,且网络传输产生的时延也可能导致计算服务耗时过高。

3)\textbf{能耗过高}。
云计算进入实践以来,研究者们更多关注数据中心中的能耗问题。
在美国,能源部对全美数据中心能耗的分析报告~\footnote{2016年美国数据中心能耗报告 https://eta.lbl.gov/publications/united-states-data-center-energy}显示,2006年度数据中心总能耗为610亿千瓦时,2014年为700亿千瓦时,预计2020年,在节能技术的帮助下,能够将年度总能耗控制在730亿千瓦时左右。
据《中国数据中心能效研究报告》~\cite{whitebooks/cn15}显示,我国2015年度数据中心能源消耗已经超越1000亿千瓦时。
报告同时指出,国家发改委已经将“绿色数据中心”列为“十二五”规划中的十大重点工程之一。
尽管学术界与工业界已经在绿色能源领域获得显著的成效,但是随着用户、应用数据的增加,全球数据中心的能耗仍在进一步上升。

4)\textbf{用户数据安全和隐私}。
在万物互联的网络空间中,边缘设备已经进入用户的私人生活空间。
例如家用摄像头、便携式移动设备、智能网联汽车,这些设备中都拥有大量的私人隐私数据。
一旦这些数据被上传至云端,用户隐私的泄露风险也会成倍增加。
目前,欧盟已经强制实施“通用数据保护条例”(General data protection regulation,简称GDPR)来保护用户的隐私。
对于云计算公司,如何保护用户的数据安全和隐私安全,也更为重要。

为了避免云计算在物联网场景下的不足之处,2013年西北太平洋国家实验室的 Ryan LaMothe 提出一种新型计算模型概念——边缘计算。
2015年,欧洲电信标准协会~\footnote{ETSI https://www.etsi.org}(European telecommunications standards institute,简称ETSI)在白皮书~\cite{hu2015mobile}中正式定义了边缘计算的概念。
边缘计算将应用任务在靠近数据源(例如移动设备、传感器、最终用户等)的设备资源上(如基站)进行处理。
% “边缘”主要相对于云计算中心而言,指代数据产生源到附近的任意计算、存储、网络资源。
同年,思科也推出了雾计算白皮书~\cite{computing2015internet}。
雾计算(Fog Computing)这一概念最早于2011年由Bonomi首次提出~\cite{bonomi2011connected}。
雾计算通过虚拟化架构将远端的云端服务迁移至本地节点,让高高在上的云服务更贴近用户,从而提高应用服务的访问效率和服务质量。
边缘计算和雾计算在思路上有相似之处,都是将云端服务落地到更接近用户的设备上。
% 但是雾计算更多探讨的是实际应用的落地,边缘计算更倾向于研究边缘计算体系结构中的问题。
因此,可以看出云边端融合的发展趋势已经凸显,也将必然成为未来信息系统的发展方向。
% 所以学术界更愿意探讨边缘计算下的本质问题。

\begin{figure}[!h]
  \centering
  % \vspace{-1em}
  \includegraphics[width=430pt]{./figures/Sec_Intro/边缘计算基础架构.pdf}
  \vspace{-0.5em}
  \caption{边缘计算基础场景}
  \vspace{-1em}
  \label{Figure_EC_Architecture}
\end{figure}

图~\ref{Figure_EC_Architecture} 描绘了边缘计算的基础场景。
在云中心,拥有大量的计算资源、存储资源、服务资源。
然而边缘设备必须通过主干网络,才能使用云上的各种资源和服务。
但是,如果将云上的服务迁移至边缘网络中的边缘服务器(基站),将服务本地化后,又可以催生出新的应用范式。
例如,将车辆信息、交通信号信息、道路监控信息结合定位服务、导航服务,可以实现简单的车联网应用和智能驾驶服务。
通过融合蜂窝网络、WiFi、蓝牙网络、端到端通信(Device-to-device,简称D2D),可以增加移动群智感知的数据获取量以满足更多应用需求。
相较于云计算模型,边缘计算具有3个明显的优势:

\textbf{1)大量临时数据不需要上传至云端服务器}。
在边缘计算中,应用程序可以利用边缘节点中的资源完成数据的存储、计算工作。
避免用户数据上传至云端服务器,为主干网络节省了大量的带宽资源。

\textbf{2)计算任务不再总是需要云计算中心的响应}。
在边缘计算中,云端服务可以卸载至资源丰富的边缘节点上。
更短的网络路径让应用服务的响应延时大大减少,不仅增加了响应能力,也提高了用户的使用体验。

\textbf{3)隐私数据保存在边缘设备,无需上传}。
由于用户隐私数据可以存储在边缘设备中而不是云端服务器,减少了隐私数据的传输路径,在一定程度上规避了隐私泄露的风险。

得益于以上优势,近年来边缘计算得到了学术界和工业界的肯定,并获得了突飞猛进的发展。
利用边缘计算,不仅可以在广袤的无线网络接入范围内提供更好、更快、更准确的信息技术服务和云计算能力;还可以在数据的边缘,利用富余的资源快速完成应用服务。
目前,边缘计算已经被采用在以下典型应用场景中:公共安全中的实时数据处理、智能网联车和自动驾驶、虚拟现实、工业物联网、智能家居以及智慧城市。
这些应用场景与人类的生产、生活、娱乐息息相关。
在这些应用场景中,边缘计算支撑着物联网中海量的数据传输和处理工作,其中包括固定部署的无线传感网络产生的数据,移动群智感知网络产生的数据等。

% 并且这些应用还有一个共同的特点,它们的需要从大量的边缘设备中收集大规模的传感数据作为『数据分析』的基础。这种应用范式,也叫移动群智感知。


% \textbf{不对,不是从边缘设备中收集。这里不妨直说,边缘计算支撑物联网数据处理,包括固定部署的无线传感网络产生的数据,移动群智感知网络产生的数据等。}

\subsection{移动群智感知}

% \textbf{无线传感网络+为什么要有移动群智感知+D2D通信}

移动群智感知(Mobile crowdsensing,简称MCS)是一种基于物联网的“以人为本”的感知模式。
其构想起源于2006年《连线》(Wired)杂志提出的众包(Crowdsensing)一词,旨在利用分布式解决方案将感知工作分配出去来共同完成应用任务或提供服务。
利用这一特性,移动群智感知可以完成个体无法实现的复杂环境下大规模动态社会感知任务,例如交通拥堵状态、城市空气质量监测等。
在早期,这些社会感知任务可以利用无线传感网络~\cite{DBLP:journals/cn/AkyildizSSC02}完成。
由于不同类型的感知任务需要部署不同功能的传感器甚至不同架构的传感网络,随着社会感知任务需求的增多和变更,静态的传感网络在部署和维护上都会耗费大量的人力成本和物力成本~\cite{CNKI/2006/WSNRen}。

移动群智感知利用众包的方式,将感知任务众包给用户的移动设备(例如手机、平板、智能手表、智能汽车等)作为基本感知设备。
这些移动设备集成了丰富的传感器,可以获取大量与设备所处环境相关的数字信息,例如环境光(光学传感器)、噪声(麦克风)、地理位置(GPS 传感器)、移动状态(陀螺仪、加速计)等。
除此之外,这些移动设备还可以利用自身优秀的通信能力快速交付数据,甚至利用本地计算资源进行数据处理。
相对于无线传感网络而言,移动群智感知应用的部署成本低、灵活性高,更适合复杂网络环境下的大规模动态社会感知任务。

除了移动设备内置了大量传感器之外,大量的现代交通工具中,雷达、摄像头、GPS、陀螺仪、加速计等设备也已经成为保障安全驾驶的必备传感器。
而物联网的飞速普适,让更多的传感设备具备了网络连接功能和简单数据处理逻辑。
因此,移动群智感知可以收集到种类更多、信息更全的感知数据,从而实现更多创新型研究与应用。
西北工业大学利用校内学生的智能手机,在校区内实现了基于移动群智感知的噪声监测系统~\cite{CNKI/2014/CSNSYu}。
通过众包方式收集不同位置和不同时间的噪声污染数据,该系统利用离散数据重建出高精度的城市噪声时空分布图,为城市噪声治理提供了可视化的监控平台。
文献~\citen{DBLP:conf/wcnc/AliAEJH12}利用车内传感器和路上行人手机内的传感器,将多个信息源混合在一起。
在不借助其它传感器和通信网络的前提下,搭建了智能交通拥堵检测系统。
Waze~\footnote{www.waze.com}是一款基于移动群智感知的导航服务。
在传统的允许用户帮助编辑地图资源的基础上,Waze 开创性的引入用户的 Facebook 和 Twitter 消息,实时更新周边商户信息和道路交通状态。
通过整合用户数据,Waze 比传统的导航服务内容更加丰富、信息也更精准。

\begin{figure}[!h]
  \centering
  % \vspace{-1em}
  \includegraphics[width=400pt]{./figures/Sec_Intro/移动群智感知应用场景.pdf}
  \vspace{-0.5em}
  \caption{移动群智感知应用场景}
  \vspace{-1em}
  \label{Figure_MCS_Application}
\end{figure}

图~\ref{Figure_MCS_Application} 展示了移动群智感知的典型应用场景。
用户提交任务请求后,移动群智感知服务(例如交通服务、健康医疗服务、社会服务)从目标区域内的传感器、移动设备、或车辆等具备感知功能和网络功能的设备上收集相关数据。
待数据汇总后,根据用户的需求,对感知数据进行处理和运算,最后将推断结果反馈给用户。

在物联网环境下,人类社会和网络空间中已经包含了海量的、具备多元化传感功能的智能设备。
众多的潜在参与者、广阔的空间覆盖、便利的数字信息提取,让移动群智感知成为了环境研究~\cite{DBLP:conf/sensys/DuttaAKMMWW09}、人类社会关系研究~\cite{DBLP:conf/globecom/AslIAM13}、智慧城市建设研究~\cite{DBLP:journals/cm/WangZWCHM16}的基础工具。
% 相较于无线传感网络,移动群智感知不仅能够灵活部署,还能够节省成本,成为了许多科技公司青睐的数据收集方案。
% 同时,这些以人为本的海量感知数据,能够帮助应用程序提供更高品质的服务。

\subsection{边端融合场景下的移动群智感知研究及挑战}

% \textbf{边缘计算在移动群智感知中的作用}

% \textbf{举例:MCS+Edge Computing}

早期,移动群智感知所收集的传感数据内容单一,结构简单(例如天气数据、噪声数据)。
但随着应用范围的不断扩展,感知数据已经逐步走向多样化、异构化的发展趋势。
尤其是在社会公共安全领域,视频数据已成为必不可少的感知数据。
例如武汉市的“雪亮工程”,于2019年中期将在全市部署150万个公共安全视频监控。
这样庞大监控网络无论是在数据备份还是在实时数据分析上,都对现有的计算模型提出了巨大的挑战。
2018年因安全事故被推上焦点的滴滴出行,也在新版客户端内加入了服务时段录音、录像功能。
目前,滴滴出行仅在广州市的日均服务次数已超过30万次,乘客和司机的手机每天所收集的感知数据容量,已经达到TB级别~\cite{DBLP:conf/aaai/Yao0KTJLGYL18}。
对于如此大量的录音、录像数据的实时处理,边缘计算提供了新的处理方法。

文献~\citen{DBLP:conf/edge/SunLS17}基于边缘计算,在前端或者靠近视频源的位置,对视频内容进行预处理,以检测视频传感器是否故障,并根据内容特征对视频的质量进行动态调整。
基于视频的前置分析处理,文献~\citen{DBLP:journals/iotj/ZhangZSZ18}对视频分析服务进行自动化部署,并让边缘设备之间可以协同处理图像数据,实现了基于视频监控的自动报警系统。
同时,通过与周边摄像头联动,该系统可以在部署范围内对指定目标实时追踪。
针对滴滴出行的安全问题,韦恩州立大学的研究者利用乘客手机中的传感器,监控车辆行驶状态和声音信号,对车辆及人身安全做出评估与预测~\cite{DBLP:conf/edge/LiuZQS18}。
这些研究工作,都充分挖掘了边缘网络中的可利用资源,利用移动群智感知和边缘计算,创造出和安全息息相关的强实时性服务。

% \textbf{资源管理分配与任务调度}
\begin{figure}[!ht]
  \centering
  \vspace{-1em}
  \includegraphics[width=310pt]{./figures/Sec_Intro/边缘计算的移动群智感知.pdf}
  \vspace{-0.5em}
  \caption{边缘计算支撑的移动群智感知}
  \vspace{-1em}
  \label{Figure_MCS_with_EC}
\end{figure}

如图~\ref{Figure_MCS_with_EC} 所示,由边缘计算支撑的移动群智感知任务的执行可以简单划分为7步。
1)移动群智感知用户提交任务需求到云端;
2)云端服务器将移动群智感知任务拆分后,利用主干网络将任务信息发送至边缘服务器;
3)边缘服务器利用蜂窝网络将任务分发至感知设备,除此之外,边缘设备也可以利用 WiFi、蓝牙、D2D通信等一切可利用的通信手段将任务分发出去;
4)边缘设备完成感知任务后,利用可利用的网络资源将感知数据发送回部署有感知服务的边缘节点;
5)边缘服务器汇集感知数据后,根据自身资源对数据进行预处理,例如数据的解码、去冗余、分类、整合等;
6)边缘服务器在移动群智感知任务达到要求或任务生命周期结束时,将结果数据上传回云端服务器;
7)云端服务器收到所有边缘服务器的返回数据后,计算、整理成最终结果,并将最终结果交付给提交任务的用户。

在边缘计算中,边缘服务器代替云端服务器分发任务,利用边缘网络中的各种通信手段,可以召集更多的数据产生者。
同时,边缘服务器利用自身的计算资源,执行数据去冗余、特征提取、数据打包和简单运算等预处理工作,减少了云端服务器的工作负载 ,降低了主干网络上感知数据造成的流量压力,间接地提高了感知数据处理效率。

% 在边缘计算模型中,为了保障移动群智感知服务的稳定性和服务质量,必须针对移动群智感知的工作阶段对边缘网络中的各种资源进行调度和管理。
% 移动群智感知应用的工作流程可以简单划分为四步~\cite{zh_cn:shi}:任务分发,传感器感知,数据上传和数据处理。
% 其中,任务分发是指根据移动群智感知任务的覆盖区域、执行任务的用户数量、以及任务的持续时间,快速将感知任务发送到合适的移动设备上。
% 移动设备收到感知任务后,通知传感器工作并产生感知数据。
% 然后,移动设备利用自身的通信能力,将传感数据上传到等待数据处理的地方(例如云服务器)。
% 最后,对这些数据进行去重、映射、分析,从而推断出符合共同利益的结果~\cite{DBLP:journals/cm/GantiYL11}。

目前,智慧城市建设已经成为当前信息领域的前沿研究热点。
智慧城市的构建,需要对城市主体和物理世界对象进行大量的数字信息挖掘,在数据处理和挖掘之后,从中获得人类活动和城市运作规律之间的潜在联系,并将其应用到各种服务和创新中。
在智慧城市场景下,移动群智感知已经成为数据信息收集的主要手段。
2016年,阿里云提出了“城市大脑”~\footnote{阿里云城市大脑 https://et.aliyun.com/brain/city},经过两年的发展,如今的“城市大脑2.0”已经在杭州市实现了智能交通管制和实时车辆查找。
2017年,Google 的 Sidewalk 实验室也启动了 Quayside~\footnote{Sidewalk Toronto https://sidewalktoronto.ca} 高科技新区项目。
随着城市的发展,网络空间中的感知设备正处在多样化、异构化的发展历程,所以利用边缘计算将数据在网络边缘进行处理是智慧城市的优先解决方案,也是移动群智感知应用的重要支撑。

\subsection{面向移动群智感知的边缘计算研究及其挑战}


边缘计算利用自身的优势弥补了云计算模型中海量数据传输延时高,隐私安全处理敏感等问题,使得边缘计算更适用于新兴的物联网应用场景~\cite{DBLP:journals/cm/SunA16}。
将云计算能力扩展至距离终端最近的边缘侧,以满足“多连接,低延时,大带宽”的新需求。
近年来,业界和学术界已经着手将边缘计算结合到实际生产与应用中。
这些应用,大多数与智慧城市建设密切相关。
例如,文献~\citen{DBLP:journals/cm/GantiYL11} 在车辆上部署 GPS 和加速度计,可以在车辆行驶过程中定位城市中的坑洼路面。
科特迪瓦共和国的一个非政府环境组织在非洲阿比让市电信公司的帮助下,在阿比让市利用25个蜂窝基站和移动群智感知应用~\cite{DBLP:conf/huc/ZhangXWC14},向阿比让市民提供即时空气质量监测结果。
文献~\citen{DBLP:conf/hotweb/WuDZS17}通过消防车上部署边缘服务器,实时采集消防员的红外摄像头数据及其它传感数据(例如定位信息、温度信息、空气成分信息等),在第一时间构建可视化的火场信息图并传送给远程消防控制中心,以实现科学救灾并保障消防员人生安全。

这些研究工作与实际应用,与移动群智感知及边缘计算都密不可分,揭示了面向移动群智感知的边缘计算模型在未来信息系统中的重要地位。
在边缘计算模型中,为了保障移动群智感知服务的稳定性和服务质量,必须针对移动群智感知的工作阶段对边缘网络中的各种资源进行调度和管理。
移动群智感知应用的工作流程可以简单划分为四步~\cite{zh_cn:shi}:任务分发、传感器感知、数据上传和数据处理。
如何针对移动群智感知应用所处的不同阶段,将边缘设备有机地组织起来,快速调度边缘网络中的计算资源、网络资源,合理分配各类资源,为边缘计算与群智感知的融合发展与实际应用带来了新的挑战。
目前,面向移动群智感知的边缘计算中主要研究挑战有以下三点。

第一个挑战是\textbf{边缘计算中面向移动群智感知的感知任务调度}。
在移动群智感知应用中,需要大量的移动设备参与者和边缘设备为移动群智感知服务提供输入数据。
然而,边缘设备自身并不具备移动群智感知服务的数据发现能力。
因此,移动群智感知服务需要将数据采集任务利用网络分发到合适的边缘设备上,雇佣这些智能设备采集传感数据,并将传感数据收集至边缘计算设备进行统一处理。
在早期的移动群智感知研究工作中,移动群智感知的参与设备多使用蜂窝通信~\cite{DBLP:conf/globecom/ZhangJLLC16,DBLP:conf/icdcs/XiaoWHHH16}上传感知数据。
随后,Karaliopoulos 等人的研究~\cite{DBLP:conf/infocom/KaraliopoulosTK15}发现,在传感数据上传过程中利用D2D通信,可以减少边缘设备的能耗成本的网络通信成本。
尽管在边缘网络中使用成本更低的 D2D 通信可以提高感知任务的执行效率,但是由于边缘网络缺少集中式管理策略,移动群智感知任务的发起者无法直接获得任务部署的覆盖范围和执行数量,只能依靠感知数据的收集状态来判断感知应用的执行进度和效率。
所以,如何利用边缘计算让更多的边缘设备参与群智感知应用、并对感知应用执行效率的影响因素进行深入研究,是提高移动群智感知应用执行效率和服务质量的关键因素。


% \textbf{『此段落重写』}
第二个挑战是\textbf{针对边缘计算场景下感知数据处理的网络资源调度}。
% 近年来,边缘设备的性能越来越强大,其可使用的网络接入技术也逐渐多样化。
% 在早期的移动群智感知研究工作中,移动群智感知的参与设备多使用蜂窝通信~\cite{DBLP:conf/globecom/ZhangJLLC16,DBLP:conf/icdcs/XiaoWHHH16}来上传感知数据。
% 随后,Karaliopoulos 等人的研究~\cite{DBLP:conf/infocom/KaraliopoulosTK15}发现,在传感数据上传过程中利用D2D通信,可以减少边缘设备的能耗成本的网络通信成本,并以此来为移动群智感知服务雇佣更多的参与者。
由于边缘计算的引入,研究人员开始借助计算卸载技术~\cite{DBLP:journals/network/MaZZWP13}将感知数据的处理工作交给靠近感知数据源的边缘计算设备。
由于感知设备的计算能力各不相同,一方面不同设备使用计算卸载时所需要的的任务卸载比例不同,另一方面感知设备和计算设备的数据交互也会大大增加网络中的数据流总量。
尽管软件定义网络~(Software-Defined Network,简称SDN)这一未来网络技术能够对大量数据流进行更为有效的管理,但是其在网络调度时也引入了传统网络调度所没有的流表容量限制。
因此,如何分配感知设备的计算卸载比例并调度大量的数据流,是在边缘计算支撑下的群智感知应用中,平衡网络负载和能耗成本的重要研究内容。

% 文献~\citen{DBLP:journals/tpds/ZhaoMTL15}通过研究机会式通信网络中的数据包转发规律,发现在边缘网络中利用数据包融合技术可以有效减少数据传输延时。
% 在边缘计算的帮助下,群智感知应用不仅可以利用 D2D 通信上传感知数据,还可以利用靠近感知设备的边缘服务器加快感知数据的处理效率。
% 虽然利用边缘计算可以在网络边缘侧帮助群智感知完成大量的传输和计算工作,但是也会让边缘网络遍布海量的数据流。
% 因此,如何利用有限的网络资源来承载海量的数据流,也成为
% 同时,
% 随着 5G 和物联网技术的飞速发展,边缘设备可以利用蓝牙、WiFi 、近场通信、D2D通信等技术,在小范围内快速交换数据信息。
% 在复杂网络环境中,优化移动群智感知数据的传输路径、减少通信时延,也是提高移动群智感知服务质量的关键因素。

第三个挑战是\textbf{边缘网络中面向移动群智感知的边缘服务调度}。
边缘计算中,服务端从云端转变到网络边缘侧,如何让这些边缘设备快速获取自身周边存在的服务,是边缘服务在部署时的一个重要问题。
在边缘计算场景中,用户和边缘设备的参与方式均为动态过程,例如车联网、移动群智感知应用中的参与者等。
这些设备和参与者随着地理位置和网络环境的改变,都会在网络中引起拓扑结构的变化。如何针对变化拓扑部署并迁移边缘服务,也是边缘计算在网络层需要考虑的难题。
再者,边缘设备会产生大量的传感数据,如何平衡边缘网络中的数据负载,合理调度边缘网络中的数据流量,尽快将数据交于边缘服务节点,也是提高移动群智感知服务质量时需要解决的问题。
所以,在动态场景下,边缘服务的部署和迁移策略也成为优化移动群智感知服务的关键点。



% 针对移动群智感知的任务分发阶段,需要借助边缘服务节点将任务快速部署到

% 针对移动群智感知的数据采集、传输、以及处理,需要对边缘网络中的网络资源、计算资源、存储资源进行合理的规划和调度。
% 但是,物联网网络空间中网络的『复杂度』,终端设备的多样化、异构化,和终端设备的移动特征,都为边缘计算造成了诸多挑战。

% \textbf{挑战三:资源的动态调度。}

% 第一,边缘设备

% 数据可分布(任务分发+数据采集)\\
% 网络可分布(SDN 中网络管理和路径规划)\\
% 资源可分布(边缘服务节点弹性部署)\\

% \textbf{挑战二:编程模型的支撑。}
% 在云计算场景中,应用程序可以在目标平台上编写和调试,最终在云平台上执行。
% 所有的基础设施对于用户而言都是透明化管理。
% 例如亚马逊的 Lambda 计算服务\footnote{aws.amazon.com/cn/lambda},可实现任务应用代码的自适应部署和运行。
% 使用虚拟化手段,例如虚拟机~\cite{}、容器~\cite{}(Docker),依然可以做到应用服务的快速部署和迁移。
% 然而,在边缘计算中,应用或服务需要在进行更小粒度的切分后,再部署至合适的资源节点上。
% \textbf{这里加一点文献讲 code offloading}
% 代码卸载(code offloading)\\




\section{国内外研究现状}

面向移动群智感知的边缘计算资源调度研究有很多。
基于前文所提到的主要研究挑战,借助边缘计算提升移动群智感知服务质量的研究方法可以分为三类。
1)网络资源调度优化。%通过合理分配无线网络资源、优化链路以减少主干网络资源占用并减少感知数据传输网络传输时延。
利用新型的网络资源管理技术提高通信效率,例如 SDN 和网络功能虚拟化(Network Function Virturlization,简称NFV);
引入新型通信模型,例如D2D通信,将移动设备构建成灵活的自组织网络,减少对核心网络的依赖。
2)计算任务卸载调度优化。利用计算任务卸载技术减少服务时延和能耗需求,提高移动群智感知的用户体验并减少成本。
3)边缘服务部署策略优化。在地域范围内合理部署边缘服务器以支撑大面积、低延时的移动群智感知应用并保障感知质量。

\subsection{网络资源调度优化}

在网络边缘侧,大量的移动设备通常借助无线网络接入网络服务,达到数据互通的目的。
借助SDN和NFV,不仅能够快速部署并管理网络通信资源,还能针对数据流实时更新数据处理逻辑以提高网络吞吐率。
% 也能够在边缘网络中更加便捷的管理网络资源、进行资源分配。
另一方便,在物联网环境下,移动设备可以使用蜂窝网络、蓝牙、WiFi、D2D通信、车载网络等多种无线通信手段进行数据交互。随着5G 通信技术的日益完善,D2D通信模型已经成为减少核心网络负载的重要手段。

\textbf{(1)面向边端融合的网络资源管理优化}

SDN 将网络功能抽象化,实现了弹性和动态的网络资源管理~\cite{DBLP:journals/jzusc/WangWYG18}。
利用软件可编程性,真正做到了网络功能的可控制、可测量。
同时,SDN 控制层和数据层分离的特性,使得不同的网络应用可以做到数据互通、业务隔离。
另一方面,SDN 具有极高的自由度和普适性,也使其成为大规模异构网络管理利器。
种种优势,让 SDN 成为边缘计算中的关键网络技术之一。

% \textbf{『下面一段话要加入』}
% SDN通常部署在数据中心。Lantz等人~\cite{Lantz:2010:NLR:1868447.1868466}为笔记本电脑中的SDN构建一个快速原型。SDN还可用于中间件的有效政策执行~\cite{Qazi:2013:SMP:2486001.2486022}。在APIs的支持下,SDN可以部署和管理统一的~\cite{Ferguson:2013:PNA:2486001.2486003}。为了广泛提供SDN,研究人员将重点放在SDN [10]的安全性和可扩展性上。2014年,SDN扩展到互联网,并创造了软件定义的互联网交换节点概念~\cite{Gupta:2014:SSD:2619239.2626300}。现在,SDN用于在数据来自物联网~\cite{Morabito:2017:FBS:3094405.3094413}的情况下管理服务功能链。

在移动群智感知应用中,原始数据主要来源于边缘设备的传感器。
在该场景下,用户和边缘设备多处于运动状态中,例如智能车联网中的车辆、移动群智感知应用中的参与者~\cite{DBLP:journals/jsac/LyuNTLWGP17}。
利用这种特性,文献~\citen{DBLP:conf/itsc/PiaoA17}通过安装在车辆上的智能手机中嵌入的运动传感器来监控城市路面损坏状况。
文献~\citen{DBLP:journals/tase/PengGXGY18}借助移动群智感知利用移动用户提供的传感数据自动更新城市数字地图。
在这类应用中,设备和参与者随着地理位置和所处网络环境的改变,都会在网络中引起网络拓扑结构的变化。
然而,移动群智感知服务往往以连续的方式收集参与者的信息并提供反馈结果。
所以在边缘计算实际应用中,随着参与者或边缘设备状态的改变,边缘网络需要时刻对边缘设备的网络状态进行维护,并合理调度边缘网络中的数据流。

Bhaskar等人~\cite{DBLP:conf/infocom/RimalVM16}的工作表明,使用边缘计算和云计算相结合的应用场景中,使用 SDN 管理网络资源,可以在不影响网络性能的情况下减少数据包时延。
Liu等人~\cite{DBLP:conf/edge/LiuWB16}利用网络边缘中的计算资源和存储资源,让 WiFi 接入点可以获取并处理接入设备的上下文信息,在任务卸载的同时,为连入终端提供预设的自定义服务。



\textbf{(2)D2D通信在群智感知中的应用研究}

在 5G 通信网络中,D2D 通信根据使用场景,可以划分为5类:
1)运营商控制 的 D2D 通信;%(D2D communication controlled by operator,DC-OC);
2)运营商控制的 D2D 中继;%(D2D relaying controlled by operator,DR-OC);
3)服务器控制的 D2D 通信;%(D2D communication controlled by server,DC-SC)
4)服务器控制的 D2D 中继;%(D2D relaying controlled by server,DR-SC,);
5)终端控制的 D2D 通信。%(D2D communication controlled by device,DC-DC)。
由于终端控制的 D2D 通信并不依赖于核心网络功能、且自由度高,因此受到了广大研究者的青睐,

在早期移动群智感知研究中,大多数研究者假设移动群智感知的参与者使用蜂窝通信上传感知数据。
但是一些研究人员发现,蜂窝网络通信所产生的费用和能耗较高,可能会使移动用户不愿意自愿作为感知数据收集者~\cite{DBLP:conf/globecom/ZhangJLLC16,DBLP:conf/icdcs/XiaoWHHH16}。
于是研究者开始讨论怎样利用移动设备的D2D通信来减少蜂窝网络的使用以实现更低的能耗和成本。
文献~\citen{DBLP:conf/infocom/KaraliopoulosTK15}研究了基于D2D通信网络中的移动群智感知志愿者的招募问题。
文献~\citen{DBLP:journals/puc/WangLL17}验证了D2D通信方式从移动设备收集感知数据的可行性,并讨论如何招募移动用户作为志愿者。 
文献~\citen{DBLP:journals/tpds/ZhaoMTL15}分析了D2D机会式通信网络中数据包转发的传播规律,并通过融合这些数据包来减少传输时延,提高移动群智感知的数据收集效率。
文献~\citen{DBLP:journals/cm/BastugBD14}利用在基站和移动设备上主动缓存用户数据,结合D2D通信和用户的社交网络关系图,在完成数据交付的同时并成功减少了主干网络上22\%的数据流量。

基于这一趋势,研究者开始关注移动群智感知应用中D2D通信所形成的机会式通信网络中数据转发的效率问题。
文献~\citen{DBLP:conf/wcnc/QinF13}分析了机会网络中基于网络编码的数据传递性能。
文献~\citen{DBLP:journals/twc/LiW14}研究D2D通信中,远程节点发送消息的时间开销问题,并提出了D2D通信中消息传播的空间范围和时间范围限制。
文献~\citen{DBLP:journals/winet/ZhaoMLT18}基于D2D通信网络,构建了机会式通信下感知数据收集的时延分析框架。
文献~\citen{DBLP:conf/mwcn/OrsiniBL15}将D2D通信引入边缘计算的基础设施组件,为边缘设备提供更多的网络资源,并设计出一种安全可靠、适应性强的基础框架来满足移动应用的弹性网络带宽需求。

\subsection{面向边端融合的计算卸载调度优化}

计算卸载~\cite{DBLP:journals/network/MaZZWP13}是指将计算任务迁移到计算能力充沛的计算设备中,例如网格、集群或者云服务器。
使用计算卸载技术,性能较弱的移动设备也可以将计算指令迁移~\cite{DBLP:journals/monet/KumarLLB13}到云端服务器并快速获得任务结果。
% 将计算任务卸载到资源更丰富的服务器,可以有效弥补移动设备能力不足的缺陷。
通常,计算卸载需要借助虚拟化、容器等技术在云端服务器构建与移动设备一致的运行时环境,然后将移动设备上的应用或者计算指令卸载到服务器上。
针对边缘计算场景,移动设备的计算任务通常被卸载到边缘服务器上。
但是在网络的边缘侧,边缘服务器的数量和计算能力都不及云端服务器。
因此,在边缘计算中实施计算卸载主要面临两方面的问题:移动设备的计算任务调度和边缘服务器上的计算服务调度。

\textbf{(1)移动设备的计算任务调度}

近年来,智能移动设备的性能大幅增强,甚至能够运行影像编辑、VR 等“重量级”应用。
然而这类计算密集型应用,包括3D游戏等,会持续占用计算资源并导致大量的能耗开销。
移动设备受到能耗、性能、散热等各方面的制约,无法维持这类应用的长时间稳定运行。
然而,移动设备通过牺牲有限的能耗储备和部分网络资源,将计算任务卸载到满足应用需求的服务器上~\cite{DBLP:conf/ispa/KovachevYK12},不仅能够更快获取计算结果,还能避免重载任务导致的能耗开销。
例如,移动设备上的视频剪辑软件需要大量的计算资源,图像渲染时间长持续运行会导致设备过热甚至卡顿,无法达到流畅的用户体验。
为了让移动设备能够在处理计算密集型应用时更快、更节能,许多研究工作验证了计算任务从移动设备卸载到云平台的可行性~\cite{DBLP:conf/secon/HassanXWC15,DBLP:journals/pervasive/SatyanarayananBCD09,DBLP:journals/computer/KumarL10}。
% 通过将重载任务卸载到服务端,不仅能够缓解移动设备的压力,还能带来更好的服务质量。

移动设备的计算任务卸载方法分为两种。
一种是在边缘服务器上构建完整的同构运行时环境,让应用能够直接部署到边缘服务器上;另一种是代码卸载~\cite{DBLP:conf/mobisys/CuervoBCWSCB10},即以执行代码的形式将更细粒度的计算任务卸载到服务器端。
但是,无论使用哪种方法,要达到实用意义必须满足两个先决条件~\cite{Flores:2017bv}: 1)移动设备与服务端的传输时延较低;2)代码卸载到服务端所需要的成本应小于移动设备处理任务所需的成本。
因此,代码卸载执行之前必须经过合理的决策判断,文献~\citen{Kosta:2012cj}发现,不同的决策方法会产生不同的收益,在个别情况下甚至会产生负面影响。
文献~\citen{Chen:2016bd,Baier:2012hb,Han:2012dl}分别从能耗、延时和成本三个层面讨论了移动设备将计算任务卸载到云服务器上的主要问题。

Hu 等人~\cite{DBLP:conf/apsys/HuGHWACPS16}在WiFi和蜂窝网络中,通过边缘网络中的边缘服务器进行计算任务卸载,可以大大降低计算密集型应用的时延,例如增强现实类应用和认知辅助类应用。
Takahashi 等人~\cite{DBLP:conf/mobilecloud/TakahashiTK15}将任务卸载应用于Web浏览加速服务。通过将 Web 应用程序卸载到边缘服务器上,边缘服务器收集大量的 Web 内容并根据用户的请求,快速将 Web 内容渲染成最终结果。移动设备可以直接获取渲染结果,迅速显示在屏幕上,节省渲染的时间和能耗。对于 Web 上的音视频数据,依然通过客户端硬件进行解码播放。
Sardellitti等人~\cite{DBLP:journals/tsipn/SardellittiSB15}首先研究了单移动用户将计算任务卸载到多个边缘服务器上的任务调度问题,并建立了非凸优化模型。当移动用户增加时,作者将调度方法改进为逐次图逼近算法,以获取原始非凸优化的最优联合解。该算法能够在保障每个用户服务质量的前提下,实现所有移动用户的总能耗最小。

\textbf{(2)边缘服务器上的计算服务调度}

在边缘计算中,为了降低传输延时和服务响应延时,移动设备的计算任务通常被卸载到边缘服务器上。
因此,边缘服务器上不仅需要部署和移动设备对应的运行时环境,还需要根据应用场景拉取云端服务器上的业务逻辑。
不同于云计算环境下的计算任务卸载,边缘服务器的数量有限、且在物理位置上呈现分布式结构。
因此,在边缘计算场景下,边缘服务器并不能支撑所有移动设备的计算任务卸载。
尤其是在移动群智感知应用中,参与感知任务的移动设备数量庞大,边缘服务器无法支撑移动群智感知应用的完整业务逻辑。
所以,针对移动群智感知应用,边缘服务器通常部署简单的数据处理服务,在网络边缘侧进行数据的收集、去冗余、合并、压缩等简单操作,生成群智感知分析的中间结果,再交由云端服务器进行最终的数据处理和分析。

文献~\citen{gao2015cloudlets}尝试将云上服务卸载到边缘服务器上,当客户端使用Wi-Fi或4G网络时,相较于云计算而言,实验结果表明在边缘服务器进行计算任务卸载后可以降低50\%以上的服务响应时延,同时还能帮助移动设备节省最多40\%的能耗。
文献~\citen{DBLP:conf/saso/MehtaTKTKE16}在边缘网络中部署小型的数据中心,能够更好地支撑带宽要求较高的应用,相对于在云上数据中心租用服务的方法,在网络边缘层部署服务能够节省60\%左右的成本。
Nastic等人~\cite{DBLP:conf/edge/NasticTD16}在物联网中加入中间件以提供多层次的服务,将中间件作为轻量级资源部署在边缘设备上,使中间件可以执行指定的应用,并通过 API 的方式对任务卸载调度进行动态管理。
Farris 等人~\cite{DBLP:journals/fgcs/FarrisMNAI17}构建了移动物联网即服务的边缘计算模型,将边缘服务器和移动设备耦合成动态的物联网边缘云。通过和公有云服务商合作,建立边缘计算服务联盟。同时将服务供应商的分组问题抽象为合作博弈的纳什均衡问题,将博弈问题的解作为任务卸载的依据。

\subsection{边缘服务部署策略及调度优化}

物联网能够让成千上万的设备进行交互和通信,产生并交换真实世界中各种对象的数字信息。
% 这些设备产生并交换现实世界中的事物产生的数据。
利用信息处理和抽象技术~\cite{DBLP:journals/iotj/GanzPBC15},物联网已经整合了网络世界和物理世界。
% 真实世界的信息化,提高了人类生活的智能程度。
目前物联网网络空间中的设备包含大量用户数据,使得物联网应用有了宽泛的应用场景,涉及人类生产、生活的各个方面~\cite{DBLP:journals/cm/WangYXJD17}。

为了弥补物联网设备资源受限的缺陷,通常引入云计算来进行数据处理。
最近的研究~\cite{DBLP:journals/iotj/RazzaqueMPC16}指出了云计算在物联网应用中存在较大的时延和不可预测的网络抖动干扰。
边缘计算作为一种分布式计算模型,旨在将存储资源、计算资源、网络资源以及应用服务部署到任何离终端设备或数据源最近的地方,以提高服务质量~\cite{AI201877}。
为了向海量的终端用户提供高效安全的服务,边缘计算成为未来的物联网建设的首选支撑技术~\cite{DBLP:journals/iotj/LinYZYZZ17}。
Brogi 等人~\cite{DBLP:Journals/IOTj/BroGif17}调查后发现边缘计算可以增强物联网的开发能力,并支持物联网应用的弹性资源需求。
文献~\citen{DBLP:Journals/IOTj/BasuDanlS17}在公共汽车上部署边缘服务器,通过车辆和乘客的移动设备来感知并监控道路状况。
% 同时,车辆可以与边缘服务一起使用,以实现这些车辆通信和计算资源的最佳利用~\cite{DBLP:journals/tvt/HouLCWJC16}。
针对紧急情况和医疗保健,文献~\citen{7466912}在上传过程中使用边缘计算对传感数据进行预处理,以达到时延敏感和隐私敏感的应用要求。
文献~\citen{7835115}构建了一个基于边缘计算的人脸识别框架,以解决安全验证服务中的部分安全和隐私问题。
由于边缘服务的部署灵活,边缘计算可以自由定义数据感知模式和类型,配合数据挖掘技术,为物联网应用创造了快速响应、实时决策、安全改进等诸多商业价值。
% 总之,边缘计算为受限网络中的物联网应用提供了实时决策、低时延、安全改进的商业价值。

在边缘服务部署和调度方面,
Satria 等人~\cite{DBLP:journals/fgcs/SatriaPJ17}研究了边缘服务器如何利用交叉覆盖和动态中继方法,来避免边缘服务器过载或故障时导致的服务失效。
T{\"{a}}rneberg 等人~\cite{DBLP:journals/fgcs/TarnebergMWTEKE17}对移动网络空间中的数据中心、网络、应用和用户进行建模,抽象出移动网络中的资源管理模型和应用部署模型,用以评估应用部署的成本,并针对最小部署成本做出合适的资源调度分配。
Chen 等人~\cite{DBLP:journals/ton/ChenJLF16}借助分布式的博弈论方法,设计了一种高效的计算卸载模型。由于无线网络中的信道干扰问题,多移动用户使用相同的无线信道卸载计算任务会引起带宽资源失衡。作者首先证明了无线网络中多用户计算卸载时的资源竞争是一个 NP(Non-Deterministic Polynomial)问题,然后借助纳什均衡和收敛时间的上限,实现了高效的分布式计算卸载算法。
Zhang 等人~\cite{DBLP:conf/rndm/ZhangMLV016}将边缘计算应用到车载网络中,设计了一种车联网中的计算卸载框架,利用边缘服务器帮助智能车完成大型计算任务。在该框架中,使用了契约理论方法进行任务卸载决策。通过优先匹配最优契约,最大化边缘服务提供商的收益,同时针对边缘服务器的有限资源进行计算资源分配,提高边缘服务器上的资源利用率。
Habak 等人~\cite{DBLP:conf/IEEEcloud/HabakAHZ15}设计了一种“微云”系统,将相互通信范围内的移动设备构建成一个可自配置的动态移动云。“微云”中的移动设备共享自身的容量、能耗等信息,动态协调并分配可以使用的计算能力、存储能力、网络带宽等。利用这种机制,可以减少动态行为所带给系统的波动。通过将计算任务卸载到云端,平摊移动云中的计算负载,最终实现“微云”系统中的群体利益最大化。


% 1)整合网络
% 首先,本文总结边缘计算的体系结构建立。
% 其次,结合移动群智感知,介绍目前边缘网络中的网络与通信管理手段。
% 然后,总结了移动群智感知下,边缘服务的卸载方法。
% 首先,整合边缘网络中各种无线通信手段,增加移动群智感知应用的参与者。
% 其次,利用边缘计算的服务卸载功能,减少系统时延,优化移动群智感知应用的用户体验。
% 最后,利用边缘计算的分布式特征,针对移动群智感知应用的特殊需求进行合理的资源分配。
% 本节主要从这3个方面来介绍边缘计算中的资源调度研究现状。

% \textbf{要引用的文章}
% \cite{DBLP:journals/jsac/LyuNTLWGP17}

\subsection{现有研究的不足}

% \textbf{总括:面向移动群智感知的边缘计算资源调度,现有研究有哪些不足(感知、通信、数据处理)}

上述国内外的主要研究工作,主要围绕边缘计算展开,并让其能够更好的支撑移动群智感知应用。
但是,针对移动群智感知的三个重要过程:感知、通信和数据处理,仍然有一些关键问题没有得到深入的研究。

% 综合上述国内外的主要研究工作,面向移动群智感知的边缘计算资源调度研究还存在如下问题。

\textbf{第一,移动设备的运动状态以及 D2D 通信对群智感知任务的影响。}
对于移动群智感知的感知过程,任务的发起者更关注的是感知任务的覆盖范围、最终收集的感知数据量和执行任务所需要的成本。
% 为了进一步扩大移动群智感知的覆盖范围并提高数据采集量,科研人员尝试在边缘网络中部署更多的边缘服务器来加快任务的传播;引入D2D通信来帮助感知任务的扩散;改进激励机制吸引更多的参与者。
文献~\citen{DBLP:journals/tpds/ZhaoMTL15,DBLP:conf/infocom/RimalVM16}发现在边缘网络中利用边缘服务器分发感知任务,也能加快任务部署并提高任务覆盖率;文献~\citen{DBLP:conf/mwcn/OrsiniBL15,gao2015cloudlets}已经验证终端设备之间的D2D通信可以提高感知任务的部署效率和感知数据的收集效率,并且在相同成本下可以增加感知任务的部署量。
尽管这些研究都将移动设备作为数据采集的基础单位,但是它们并没有深入考虑这些移动设备的运动特征对移动群智感知应用带来的影响。
尤其是在边缘网络中,交叉覆盖着不同结构的无线网络,移动设备会随着位置的变化而改变网络接入点,导致网络状态的改变。
另一方面,鉴于D2D通信的引入,终端设备的移动行为让机会式通信网络中的数据交互更加的频繁,并且不同的运动模式对感知任务部署效率和感知数据收集效率都有不同的影响。
因此,探讨终端设备的移动特征和D2D通信对移动群智感知中任务覆盖及数据收集的影响,是移动群智感知优化的关键问题之一。
% 因此,本文针对边缘计算支撑下的移动群智感知应用,研究终端设备的移动特征对移动群智感知中任务覆盖和数据收集的影响,并以数学建模的方式来解释终端设备移动特征的对移动群智感知的作用原理,并提出相关的优化方案。


% \textbf{第一,边缘计算场景下的感知质量缺少系统性研究。}
% 目前现有的研究工作主要利用边缘计算中的网络资源调度、计算任务卸载,以达到减少移动群智感知应用网络时延、增加移动群智感知参与者的目的。
% 虽然这些研究都能在不同层面对移动群智感知应用做出优化,但是并没有可用的参照体系来比较不同方法的优化质量。
% 只能单一的比较时延增减、或者能耗增减。
% 因此,需要对移动群智感知的服务质量进行合理的定义,并纳入不同资源进行定量分析,作为不同优化策略好坏的评判标准,并指出不同场景下感知质量优化的关键资源类型。

\textbf{第二,计算任务卸载对移动群智感知的影响。}
利用边缘计算,文献~\citen{DBLP:conf/mobilecloud/TakahashiTK15,DBLP:conf/edge/LiuWB16}采用计算卸载技术并借助边缘服务来进行感知数据的处理和收集工作。
然而计算卸载的运用也极大增加了边缘网络中的网络链接数量。
当使用未来网络管理技术 SDN 对大量的数据流进行调度时,需要考虑链路的延时、带宽对计算卸载决策的影响。
同时,由于 SDN 路由器中存储转发规则的三态内容可寻址存储器(Ternary Content Addressable Memory,简称TCAM)成本昂贵,SDN 规则空间~\cite{Katta:2014es}也为群智感知的网络调度带来新的挑战。
所以,在边缘计算中运用 SDN 虽然能够更加方便的管理边缘网络中的通信资源,但是在网络调度时需要同时考虑链路的延时、带宽,SDN 路由器的流表容量以及感知数据的计算卸载决策。
因此,网络资源调度与未来网络技术的结合,也是边缘计算更好支撑移动群智感知的关键问题之一。

% SDN 技术的不断积累和快速发展,移动群智感知可以在边缘网络中更加方便的管理移动设备的网络资源并将感知数据的计算卸载到边缘服务器上。
% 但是,由于SDN路由器中存储转发规则的三态内容可寻址存储器(Ternary content addressable memory,简称TCAM)成本昂贵,因此无法避免规则空间的限制~\cite{Katta:2014es}。
% 所以,利用SDN管理大量的终端设备本身具有一定的难度。
% 所以,在边缘计算中运用SDN虽然能够更加方便的管理移动设备的网络结构,但是在调度网络资源时也应当考虑链路调度对移动群智感知中的计算任务卸载带来影响。
% 如何利用SDN分配给移动设备分配网络资源并规划计算任务卸载,也是优化移动群智感知效率的重要研究内容。

% \textbf{第二,受限的网络资源会影响任务卸载决策。}
% 利用边缘计算支撑移动群智感知应用,可以将数据预处理服务部署在移动节点上,以此将移动设备上的数据处理工作卸载到边缘服务器上,从而节省移动设备的能耗,降低移动用户的成本。
% 无论是以移动设备的全体能耗最小化为目标,还是以边缘服务提供商的利益最大化为目标,这些任务卸载决策算法,都忽略了边缘网络中的时延限制和带宽限制,并且只考虑了单一的通信手段。
% 尽管很多研究将感知数据的预处理卸载到边缘服务器上,并以移动设备的全体能耗最小化或边缘服务提供商的利益最大化为目标进行任务卸载决策。
% 除此之外,移动设备可以使用多种网络通信方式进行数据传输。
% 所以任务卸载算法不仅考虑到收益问题,还要规划好通信方式并选择合适的通信链路。

% 无论是移动群智感知还是边缘计算场景中,都遍布着大量的移动用户。
% 这也意味着大量的智能设备都处于移动状态。
% 然而大部分的研究在考虑资源调度时,并没有考虑设备的运动特征所带来的影响。
% 例如移动设备从某个移动网络中进入另一个移动网络,网络抖动或者链路变化所带来的影响。
% 这些连锁反应也会影响任务卸载的收益。
% 另一方面,在移动群智感知应用中,设备的移动亦会改变感知数据的覆盖范围,影响感知质量。
% 这些都是现有研究缺乏的内容。

% 在软件定义网络( SDN )中,规则空间是由三进制内容可寻址存储器( TCAM )引起的不可避免的限制。当负载变得更受欢迎时,负载数量的急剧增加会影响链路调度和客户的服务质量( QoS )。因此,SDN中的链路调度必须考虑规则空间、带宽和QoS的约束。为了优化移动计算中的能耗,构建了一个模型来理解负载决策、时延和能耗之间的关系。实现了一个二阶段算法,通过使用交换机中的规则空间和链路中的带宽来调度链路。通过对比评估,验证了算法的可行性。当SDN网络中规则空间的使用不充分时,与最佳解决方案相比,该算法可以将超过90 \%的能量效率存档。

\textbf{第三,边缘服务部署对移动群智感知的影响。}
在边缘计算中,通常租用基站作为边缘服务器来部署相应的数据处理服务来加速感知数据处理~\cite{DBLP:journals/fgcs/TarnebergMWTEKE17,DBLP:journals/ton/ChenJLF16,DBLP:conf/IEEEcloud/HabakAHZ15},以提高移动群智感知的效率并减少主干网络中的感知数据流量。
在物理世界中,虽然已有大量的基站可以作为边缘服务器,但是对于涉及智慧城市应用的大规模移动群智感知而言,部署的边缘服务器过多,不仅会造成覆盖区域的重复,还会大大提高移动群智感知应用的执行成本。
因此,在大范围移动群智感知应用中,通常会利用人类的社交网络特性来选择边缘服务器。
例如文献~\citen{Cnki:Yu2018}通过研究移动社交网络,利用社交亲密度和关系网络,选择亲密度较高的用户进行任务分发。
文献~\citen{DBLP:conf/wasa/Yan0WWW17}利用D2D通信构建多维社会网络,根据感知用户的权重在适当的位置部署边缘服务器,用来维护网络局部稳定性和全局网络的中心度。
% \textbf{举例}。
然而,人类的社交网络通常会根据参与人自身行为发生变动,导致部署方案的效率发生变化。
因此,对于智慧城市类型的移动群智感知,迫切地需要一种稳定的边缘服务部署和调度方法,在不同需求下保障移动群智感知应用的效率,并减少移动群智感知的执行成本。

% 包含『关键技术』和『应用举例』

\section{研究意义与目的}

针对上述国内外现有研究的不足,本文围绕移动群智感知的感知、通信和计算卸载三个层面展开边缘计算中的资源调度研究。如图~\ref{Figure_Re_Part} ,本文主要创新点如下:

\begin{figure}[!h]
  \centering
  % \vspace{-1em}
  \includegraphics[width=420pt]{./figures/Sec_Intro/研究内容和各部分之间的联系.pdf}
  % \vspace{-0.5em}
  \caption{研究内容之间的联系}
  \vspace{-1em}
  \label{Figure_Re_Part}
\end{figure}

首先,针对感知过程中移动设备运动状态对 D2D 通信造成的影响,提出了基于群智感知效率的任务调度机制,以优化感知任务的执行效率。
在大多数移动群智感知研究中,仅有少数研究者考虑设备运动状态对 D2D 通信的影响。
不同于以往单发射源单接收端~\cite{DBLP:journals/twc/LiW14}、单发射源多接收端~\cite{DBLP:conf/wasa/Yan0WWW17}、多发射源单接收端~\cite{DBLP:journals/tpds/ZhaoMTL15}的研究,
本文创新地考虑了边缘计算支撑的移动群智感知场景中,往往存在多个发射源和多个接收端。
通过量化边缘计算中的资源分配,本文建立了移动群智感知的执行效率分析模型,以反映群智感知应用在其生命周期内所收集的感知数据数量与各类资源分配的关系。
基于该模型,可以通过随机过程分析方法来描述移动群智感知应用中的两个关键过程:任务分发过程和数据收集过程。
同时该模型可以反映出边缘计算中不同类型资源对群智感知执行效率的影响程度,并且分析了不同种类资源调度的优化收益。

其次,针对边缘网络中的感知数据计算卸载,提出了考虑流表容量限制的网络资源调度算法,以提高感知数据的处理效率。
在边缘网络支撑下的群智感知应用中,对于智能设备(例如智能手机)而言,感知数据可以在本地进行处理后再上传至边缘服务器。
但是对于摄像头或者可穿戴设备而言,往往需要将计算任务卸载至边缘服务器以减少数据处理时延。
面对感知数据计算卸载,一方面,需要考虑边缘设备与移动设备数据处理能力与能耗上的差异;另一方面,还需要考虑边缘网络中海量数据流的调度和管理。
特别是当使用 SDN 对海量数据流进行调度时,需要额外注意 SDN 交换机的流表容量限制。
因此,本文在考虑SDN 流表容量限制的同时,提出了计算任务卸载和网络链路的联合调度算法。
该算法弥补了传统链路调度算法缺少流表容量的空白,同时保障的感知数据的处理效率。

最后,针对城市级别群智感知应用中感知数据收集成本过高的问题,提出了基于用户移动特征的边缘服务部署机制,以减少感知数据的收集成本。
在智慧城市建设中,群智感知扮演着举足轻重的角色。
通过收集城市中各类传感器的感知数据,可以利用信息化技术在不同维度反映出城市的各种特征,辅助市政建设。
但是在城市范围内收集感知数据,往往需要耗费大量的时间成本和物力成本。
为了解决这两大难题,本文考虑公共交通行为中乘客的流动特征,结合 D2D 通信技术,旨在研究城市移动群智感知中边缘服务的部署决策和移动群智感知数据收集成本之间的关系。
通过该研究,提出了基于社会网络的边缘服务的部署决策和调度算法。
该算法在保障感知质量最大化的前提下,求解成本最小的服务部署策略。

% 为了让移动群智感知实现更大的范围覆盖并收集更多的传感数据,可以利用边缘计算将移动群智感知任务分发给更多、更远的边缘设备。
% 本文通过在边缘服务器上部署任务分发服务和数据收集服务,并引入D2D通信来提高任务分发和数据收集的效率。
% 通过量化边缘计算中的资源分配,本文建立了移动群智感知的感知质量分析模型。
% 基于这一模型,可以利用数学的方式来描述移动群智感知应用中的两个关键过程:任务分发过程和数据收集过程。
% 同时该模型可以反映出边缘计算中不同类型资源对感知质量的影响系数。
% 基于此研究,本文针对移动群智感知任务的执行过程提出了基于感知质量的资源调度机制。

% \textbf{(2)面向任务卸载的边缘网络调度}

% 移动群智感知应用中,移动设备可以将感知数据的处理任务卸载到已部署相关服务的边缘服务器上。
% 对于智能设备(例如智能手机)而言,感知数据可以在本地进行处理后再上传至边缘服务器。
% 但是对于摄像头或者可穿戴设备而言,往往需要将计算任务卸载至边缘服务器以减少数据处理时延。
% 但是在边缘计算中,边缘设备所处的网络环境往往处于异构且不稳定的状态,因此在计算任务卸载决策时,还需要考虑网络资源的限制和额外的能耗开销。
% 因此,本文研究了边缘网络中移动群智感知计算任务卸载和网络状态的关系,提出了计算任务卸载和网络链路的联合调度算法。
% 在最大化计算任务卸载收益的基础上,保障网络中各链路的负载均衡。

% \textbf{(3)基于用户移动特征的边缘服务调度}

% % 在万物互联的场景下,移动群智感知和信息物理系统(Cyber Physical Systems,CPS)的结合越来越紧密。
% 不少移动群智感知应用已经结合了移动用户的社交网络来优化移动群智感知的感知质量。
% 契合移动群智感知在智慧城市建设中的重要作用,本文在城市级别范围内利用社会网络和D2D通信来优化大规模移动群智感知的感知质量。
% 本文考虑公共交通行为中乘客的流动特征,结合D2D通信技术,旨在研究城市移动群智感知中边缘服务的部署决策和移动群智感知执行成本之间的关系。
% 基于该研究,提出了基于社会网络的边缘服务的部署决策和调度算法,在感知质量最大化的前提下,求解成本最小的服务部署策略。

\section{论文组织结构}

% 论文组织结构如图~\ref{}所示。

% 本文主要研究内容为在边缘计算中利用资源分配和调度机制优化移动群智感知应用的感知质量,本文的内容组织结构如下。

本文主要研究内容为在边缘计算中利用资源分配和调度机制优化移动群智感知应用的执行效率,本文的内容组织结构如图~\ref{Figure_Re_Instruction} 所示。

\begin{figure}[!h]
\vspace{-0.8em}
\centering
  \includegraphics[width=440pt]{./figures/Sec_Intro/论文组织结构.pdf}
  % \vspace{-0.5em}
  \caption{论文组织结构}
  \vspace{-1em}
  \label{Figure_Re_Instruction}
\end{figure}

第一章为绪论。首先给出面向移动群智感知的边缘计算技术的的研究背景和面临的挑战,然后介绍现阶段国内外研究现状,最后给出研究内容和主要贡献以及论文组织结构。

第二章为基于随机过程的群智感知执行效率理论分析及应用。
首先基于随机移动模型建立移动群智感知应用执行过程、感知设备移动特征、移动边缘网络各类资源以及时间的关系,并依据传染病模型对移动群智感知任务分发的过程建立数学模型。
接着分析移动群智感知任务的持续时间和边缘节点移动模型对感知任务分发过程以及数据回收过程的影响,以研究群智感知执行效率与边缘网络中各类资源的关系。
然后通过理论分析,探讨了提高感知任务执行效率的途径和方法。
接着提出优化感知效率的任务执行时间划分算法,并结合仿真模拟验证该算法对移动群智感知任务执行效率的优化效果。
最后,分析了不同种类资源调度的优化收益。

第三章为考虑流表容量限制的能耗最小化感知数据计算卸载调度。
由于移动设备计算能力参差不齐,在边缘计算模型中,移动群智感知的数据预处理服务(例如压缩、去冗余、校验)可以依靠边缘服务进行加速。
虽然卸载数据处理任务能够在一定程度上减少移动设备的能耗,但是大量的移动设备使用计算卸载技术会在边缘网络中产生海量的数据流。
为此,当使用 SDN 技术对边缘网络进行管理时,不仅需要协调移动设备的卸载决策,还需要考虑大量数据流对 SDN 交换机流表容量带来的冲击。
在保证感知数据处理效率的前提下,本文针对计算卸载决策和网络调度建立理论分析模型,平衡通信能耗与计算能耗,并提出了让移动设备群体能耗最小的计算卸载决策和网络链路分配算法。

第四章为基于用户移动特征的成本最小化边缘服务部署。
为了在广阔的城市范围内实现群智感知数据收集,仅依靠蜂窝网络的通信成本过高。
但是借助 5G 通信中的 D2D 通信技术,可以大大减少感知数据的收集成本。
本文基于城市公交路线分析城市居民在公共交通中的移动轨迹,建立移动群智感知数据收集成本和边缘服务部署策略的关系模型。
基于该模型,在保障感知数据收集效率的前提下,提出基于公交路线的成本最小化边缘服务部署决策算法。
最后采用仿真模拟验证算法效率以及成本节省比例。

第五章总结全文,概括文中的主要贡献并展望未来的研究工作。


% TODO

% [ ] 语句通畅度修改
% [ ] 公式是否增加推导过程
% [ ] D2D 通信增加几篇文献并做简要说明

\chapter{面向群智感知质量优化的边缘资源调度}

如今,人类社会已经进入万物互联时代。
无处不在的移动设备让移动群智感知的覆盖区域更开阔、采集任务更丰富、数据收集更全面。
为了实现高效率、大覆盖的移动群智感知应用,边缘计算已经替代云计算为移动群智感知提供基础支撑。
通过在边缘服务器上部署相关服务,可以增加群智感知应用的执行效率。
% 利用边缘服务器分发感知任务,可以大大提高感知任务的空间覆盖率。
利用 D2D 通信发送感知任务收集感知数据,不仅能减少主干网络的流量负载,还能减少感知数据的传输时延。
% 尽管大量的研究工作利用模拟或系统实验证明了边缘计算和D2D通信对移动群智感知的积极作用,但是这些研究工作并没有考虑移动设备自身的运动特征对移动群智感知造成的影响。
考虑移动设备运动状态对 D2D 通信造成的影响,以及边缘服务部署和群智感知应用执行效率之间的联系,本章将移动群智感知的场景进行参数量化,针对移动群智感知的执行过程建立随机过程分析模型。
然后对模型进行分析,讨论了移动群智感知执行过程中各参数对感知质量的影响机理。
最后根据分析结果,提出了面向感知质量优化的边缘资源调度方法。

\section{研究背景}

如文献~\cite{DBLP:journals/fgcs/AntonicMPZ16,DBLP:conf/ccnc/MessaoudRG16}中所讨论,当使用云计算模型支撑移动群智感知应用时,所有的感知数据都会被收集到云端。
随着移动设备的增加和感知数据的膨胀,上传至云端的感知数据会急剧消耗主干网络的带宽资源。
边缘计算的提出,将云端服务迁移到位于网络边缘侧的边缘服务器(例如蜂窝网络中的基站或者大型无线网络接入点)上,使计算资源、存储资源、应用服务更加贴近数据产生源,以此缓解群智感知应用对云端服务的依赖。
由于边缘网络中移动设备多、地域分布广,研究者们普遍认为边缘计算可以显著提升网络应用的服务质量~\cite{DBLP:conf/sigcomm/BonomiMZA12}。
近年来,学术界和工业界都在不断挖掘边缘计算的巨大潜力,致力于将边缘计算应用到不同领域的开创性工作中~\cite{DBLP:journals/access/MarjanovicAZ18,DBLP:journals/iotj/ChiangZ16}。


% 目前,移动计算领域与边缘计算日益紧密的结合在一起。
鉴于边缘计算的分布式特性以及可利用资源更靠近数据源,移动计算应用可以利用边缘计算实现高带宽、低延时的实时服务。
移动群智感知作为移动计算领域中的一种典型应用,利用无线网络和移动设备中内置的传感器来完成对真实世界的数字感知工作。
不同于传统的专用型无线传感器网络,移动群智感知可以更方便快捷地收集与人类生活、自然环境息息相关的各种数字信息~\cite{DBLP:journals/cm/GuoCZYC16}。
利用边缘计算支撑的移动群智感知应用,已经广泛应用于其它不同领域,例如无线网络性能测量~\cite{DBLP:journals/cm/RosenLLCMB14}、天气预报~\cite{DBLP:journals/tpds/ZhaoMTL15}、空气质量监测~\cite{DBLP:conf/huc/ZhangXWC14}、城市噪声监测~\cite{DBLP:conf/huc/ZhengLWZLC14}和城市智能交通建设~\cite{DBLP:conf/icdcs/ZhouJL15}等。

除了边缘计算能够帮助移动群智感知提高感知质量,研究者还发现使用成本更低的D2D通信,也可以达到这一目的~\cite{DBLP:journals/puc/WangLL17}。
利用移动设备之间的D2D通信,使得不具备蜂窝网络通信能力的移动设备也能够快速加入群智感知,例如可穿戴设备,监控设备等。
同时,在使用成本方面,D2D通信的能耗成本更低且不需要服务成本,因此更适合应对群智感知中的海量数据交换。
另一方面,在 D2D 通信构建的机会式网络和主干网络相互隔离,在机会式通信网络中对感知数据进行收集、去冗、压缩,也可以有效缓解海量感知数据对主干网络的资源占用。
除此之外,D2D通信作为5G网络中的重要的通信手段,将在物联网场景下成为核心通信手段之一。

% 边缘计算支撑移动群智感知应用,
% 由于产生感知数据的移动设备(例如智能手机、智能手表)和位于数据中心的云端服务相距甚远,感知数据在传输过程中会遇到较高的网络延时或者不可预测的网络抖动,导致传输延时增加。
% 另一方面,将海量的感知数据汇总到云平台,会消耗主干网中大量的带宽资源。
% 通过边缘计算的支援,移动群智感知应用可以依靠边缘服务器中的可利用资源,辅助任务转发和数据收集工作,扩大移动群智感知应用的适用范围,降低感知成本,提高感知质量。
% 另一方面,介于蜂窝通信的成本较高,减退了许多潜在移动群智感知参与者的热情。
% 为了解决这一问题,许多研究人员设计了激励机制来吸引更多的群智感知参与者,也有部分研究人员发现利用成本更低的D2D通信,同样可以缓解这一问题~\cite{DBLP:journals/puc/WangLL17}。

\begin{figure}[!b]
  \centering
  \vspace{-1em}
  \includegraphics[width=220pt]{./figures/Sec_UIC/移动群智感知应用场景.pdf}
  \vspace{-0.5em}
  \caption{边缘计算支撑下的移动群智感知应用场景}
  \vspace{-1em}
  \label{Figure_UIC_MCS}
\end{figure}

基于上述原因,本章的研究工作针对图~\ref{Figure_UIC_MCS} 所描述的边缘计算支撑下的移动群智感知应用场景。
首先,目标区域内的基站从云端服务器获取感知任务,利用蜂窝网络和 D2D 通信的方式,将感知任务分发给区域内的移动设备。
其次,接收到感知任务的移动设备,在执行任务的同时,借助 D2D 通信将感知任务分发给相邻的移动设备。
待感知任务执行完毕,参与感知任务的移动设备利用蜂窝网络或者 D2D 通信,再将感知数据反馈给基站。
最后,基站对收集到的感知数据执行去冗、合并、压缩等操作后,再交付给云端服务器执行最终的运算处理。

% 在本文中,提出边缘计算授权MCS与D2D通信,其中整个MCS过程有两个主要阶段。在第一阶段(任务分发)中,边缘节点作为源节点执行以将任务分发到移动设备。在第二阶段(数据收集)中,移动设备成为源节点,边缘节点成为目标节点。任务和传感数据都以机会方式通过D2D通信传输。在这种情况下,第一个自然问题是服务部署如何影响感知质量(例如,覆盖)。直观地知道通过部署相应的服务来利用更多的边缘节点,将实现更高的传感质量。但是,在边缘节点中部署服务并不是免费的。因此,量化这种对服务部署决策的影响是非常重要的。现有的研究,例如[13]  -  [15],已经分析了D2D机会网络中的消息传递延迟。这些研究表明,消息传递性能受很多因素的影响很大,例如移动设备的数量和移动设备的遭遇率。正如注意到的,他们通常假设有一个数据源节点和一个目标节点。它们都不能用于分析边缘计算授权MCS的性能,因为可能有多个边缘节点分别在任务分发阶段和数据收集阶段中作为源节点和目的节点执行。

在该场景下,重点考虑边缘网络中感知任务的分发效率和感知数据的收集效率,以提高移动群智感知应用的覆盖范围和感知质量。
在任务分发过程中,已部署任务分发服务的基站作为任务分发的源头,不断地将移动群智感知任务分发至移动设备。
在数据收集过程中,移动设备作为感知数据的源头,将数据回传至已经部署数据收集服务的基站。
任务的传播和感知数据的转发不仅可以通过蜂窝数据进行发送,也可以通过D2D通信的方式传输。
现有研究~\cite{DBLP:conf/wcnc/QinF13,DBLP:journals/twc/LiW14,DBLP:journals/winet/ZhaoMLT18}已经分析了D2D机会式通信网络中的信息传递延时。
这些研究工作表明,D2D通信机制中消息传递延时的影响因素主要有移动设备的数量和移动设备的相遇率。
然而,这些研究工作在边缘网络中只选取了一个边缘服务器同时作为数据分发源和数据收集目的地。
在真实场景下,可以存在多个源节点和目标节点。
因此,这些研究工作并不适用于分析边缘计算下的移动群智感知过程。
为了提高移动群智感知的感知质量,必须用新的方法量化边缘网络中服务资源部署对移动群智感知的影响。
在该述求下,第一个挑战就是探索边缘网络中的资源部署对移动群智感知应用覆盖范围和服务质量的影响机理。
% 另外,移动群智感知应用中的设备随着时间的推移会改变自身的位置。
第二个挑战是探索移动设备的运动模型对移动群智感知应用服务质量的影响。

为了解决这两个问题,本章在边缘计算下的移动群智感知中结合D2D通信,并对数据传输过程进行理论分析、性能分析。本章的主要贡献如下:

1)利用常微分方程组,描述边缘计算环境下的移动群智感知中的任务分发阶段和数据收集阶段。
通过对常微分方程组进行分析和求解,量化移动群智感知应用场景中的边缘服务以及感知设备的移动模型,并推导出量化参数和感知质量之间的关系。


2)基于移动群智感知应用的生命周期,本章设计了一种时间划分算法,以找出任务分发阶段和数据收集阶段的最佳时间分配,帮助移动群智感知应用获得更好的感知质量。

% 3)经过模拟实验分析,验证了模型和算法的正确性和准确性。

\section{移动群智感知过程分析}

% 本章中,重点研究了与论文[22]中描述的MCS过程。 在第一个任务分发阶段,任务分发者将MCS任务分发到移动设备。 此后,在第二数据收集阶段,已经接收到任务的移动设备进行感测并将感测数据报告回数据收集器。 进一步使雾计算参与上述程序。 在本节中,首先介绍一些预备,然后定义随机过程分析的系统模型。

\textbf{『把 MCS 应用再说清楚一点。』}
本章采用了和文献~\cite{DBLP:journals/tpds/ZhaoMTL15}相似的移动群智感知应用范式,主要研究对象为移动群智感知应用中的两个重要过程:任务分发过程和数据收集过程。
其中,任务分发过程是指已部署任务分发服务的基站将任务分发至边缘网络中的移动设备,数据收集过程是指已经接受到任务的移动设备将感知数据反馈给已部署数据收集服务的基站。
考虑到移动群智感知参与者本身具备移动性,本章也将移动设备的运动特征纳入考虑范围。
本小节重点介绍分析模型中的参数定义和模型建立。

\subsection{场景定义}
\label{UIC:Scenario}

% 当结合雾计算时,任务分发者和数据收集器都被部署为雾节点中的服务。在不失一般性的情况下,假设基站是本文中的雾节点。图2描述了具有D2D通信的雾计算授权MCS的工作过程。首先,MCS服务提供商选择适当的基站来部署MCS服务。其次,在任务分发阶段,这些基站开始向移动设备传播MCS任务。为了避免蜂窝通信成本,基站仅将任务传递到进入其通信范围的移动设备。然后,已经完成任务的移动设备开始将任务转发到在其移动期间通过D2D通信遇到的其他移动设备。传播的时间消耗是T传播。在时间段T感测中,移动设备获得感测数据并将它们发送回数据收集器。设T集合表示数据收集的时间消耗。在整个阶段,假设所有移动设备都是MCS服务的志愿者。请注意,具有该任务的任何移动设备都可以帮助传播任务并通过D2D通信收集传感数据。显然,这形成了与流行病学路由的机会网络。

在边缘计算场景下,移动群智感知中的任务分发服务和数据收集服务都可以部署在边缘服务器中。
在不失普适性的情况下,本章假设蜂窝网络的基站作为能够承载这些服务的边缘服务器。
此时移动群智感知应用的执行过程大体分为四步:
1)选择合适的基站部署任务分发服务和数据收集服务;
2)在任务分发阶段,部署有任务分发服务的基站利用蜂窝网络和D2D通信将感知任务发送到附近的移动设备上;
3)移动设备在收到感知任务之后,一边处理感知任务,一边利用D2D通信将任务广播给附近的其它移动设备;
4)已经完成任务的移动设备,可以利用蜂窝网络或者D2D通信,将感知数据反馈给已经部署数据收集服务的基站。
由于激励机制并不是本章的主要研究内容,因此在移动群智感知应用执行期间,本章假设目标区域内的移动设备都是移动群智感知应用的志愿者。

\begin{figure}[!h]
  \centering
  \vspace{-1em}
  \includegraphics[width=240pt]{./figures/Sec_UIC/移动群智感知应用的时间分布.pdf}
  \vspace{-0.5em}
  \caption{移动群智感知应用的时间分布}
  \vspace{-1em}
  \label{Figure_MCS_Delay}
\end{figure}

图~\ref{Figure_MCS_Delay} 描述了边缘计算环境中,移动群智感知应用中各阶段的时间开销。
在基站中部署好任务分发服务和数据收集服务之后,
$T_{dissemination}$ 表示移动群智感知任务从基站发送到移动设备的时间开销;
$T_{sensing}$ 表示移动设备执行感知任务的时间开销;
$T_{collection}$ 表示移动设备将感知数据反馈回基站的时间开销;
$T_{total\_time}$ 是前三者之和。

% 图3总结了雾计算授权MCS与D2D通信的任务分发的两种主要通信模式。 一旦移动设备完成任务,除了传播任务外,它还进行协作和机会感知以在移动期间获取感测数据。 然后,在数据收集阶段,移动设备开始将其感测数据发送回数据收集器,即,利用MCS服务部署的雾节点。 与任务分发阶段类似,传感数据可以直接传输到基站或通过基于D2D的流行路由到达基站,如图4所示。

图~\ref{Figure_PropagationProcedure} 展示了边缘网络中感知任务的分发方法。
首先,部署有任务分发服务的基站利用蜂窝网络以固定的速率将任务部署到可通信的移动设备上。
因此在感知任务分发的前期,主要依赖蜂窝网络传播分发感知任务。
同时,已经收到感知任务的移动设备在其运动过程中,可能会遇见其它还未收到感知任务的移动设备。
在这种情况下,相遇的移动设备之间可以利用 D2D 通信进行感知任务的分发。
随着移动设备的不断相遇,收到感知任务的移动设备数量会不断扩大,极大提高感知任务的分发效率。
由于 D2D 通信需要两个移动设备的距离满足一定条件,因此使用 D2D 通信传播感知任务会受到移动设备运动状态的影响。

\begin{figure}[!h]
  \centering
  % \vspace{-1em}
  \includegraphics[width=280pt]{./figures/Sec_UIC/任务分发过程.pdf}
  \vspace{-0.5em}
  \caption{感知任务分发途径}
  \vspace{-1em}
  \label{Figure_PropagationProcedure}
\end{figure}

图~\ref{Figure_FeedbacksCollection} 展示了边缘网络中感知数据的收集方法。
利用 D2D 通信,相邻的两个移动设备可以交换自己的感知数据。
通过一系列数据交换操作,远端移动设备的感知数据有机会传送到负责收集数据的基站。
另一方面,部署有数据收集服务的基站也可以利用蜂窝通信网络,直接与覆盖范围内移动设备通信,获取该设备上的感知数据。
同时,与基站通信的移动设备,可能和其它移动设备已经交换过感知数据,因此基站在一个移动设备上可以收集到来自不同设备的感知数据。

% 计算下的移动群智感知服务利用蜂窝网络和D2D通信完成任务分发。
% 已经收到任务的移动设备,在其移动过程中,会有一定的概率遇到没有接收到任务的移动设备。
% 此时,已经收到任务的设备利用D2D通信,将任务分发至还没有收到任务的设备。
% 待任务执行完成后,移动设备可以利用蜂窝网络将感知数据传送给基站,也可以利用D2D通信将结果委托给在移动过程中相遇的其它设备,并传送给基站。
% 这一过程,如图~\ref{Figure_FeedbacksCollection} 所示。

\begin{figure}[!h]
  \centering
  % \vspace{-1em}
  \includegraphics[width=350pt]{./figures/Sec_UIC/数据收集过程.pdf}
  \vspace{-0.5em}
  \caption{感知数据收集途径}
  \vspace{-1.5em}
  \label{Figure_FeedbacksCollection}
\end{figure}

\subsection{移动模型}
% 按照上述工作流程,在本文中,认为在MCS服务中部署了N f个雾节点。它们既可以作为任务分发者,也可以作为数据收集者。在雾中,有N个移动设备随机移动并且可以自愿参与MCS过程。由于移动,移动设备可以机会性地与另一个移动设备相遇并且在接触持续时间期间与遇到的移动设备通信。假设接触持续时间足够长以完成任务或传感数据的传输。

在场景定义中,参与移动群智感知的移动设备主要有两类:负责感知任务分发和数据收集的基站,以及执行感知任务的移动设备。
对于基站而言,一般为固定式部署,不存在移动特征。
但是对于移动设备而言,每个移动设备都有自己的运动轨迹,这些设备遵循自身的轨迹移动,在相遇时利用 D2D 通信进行数据交互。
为了抽象描述移动设备因相遇而产生的 D2D 通信行为,这里使用移动设备之间的相遇率来表示 D2D 通信发生的频率。
因此,定义 $\lambda_n$ 表示移动设备之间的相遇率。

在 Groenevelt 等人~\cite{DBLP:journals/pe/GroeneveltNK05}的研究中,给出了机会式通信网络中节点相遇率的计算方法。
借助该理论, $\lambda_n$ 可以采用式~\eqref{Formula_EncounterRate} 进行计算得出。
其中,$d$ 表示移动设备 D2D 通信的覆盖范围(单位:米),
$\mathbb{E}[V^*]$ 指的是特定区域内所有移动设备的平均速度(单位:米/秒),
$A$ 表示 D2D 通信网络中所期望的覆盖区域面积(单位:平方米),
$w$ 表示移动设备的运动模型对应的常量系数,例如随机路径运动模型所对应的系数为 1.3683。
基于这一理论,可以将群智感知的覆盖范围、以及移动设备的运动轨迹转化为移动设备的相遇概率。

\vspace{-1em}
\begin{equation}
  % \vspace{-1em}
  \label{Formula_EncounterRate}
  \begin{gathered}
  \lambda = \frac{2 w d \mathbb{E}[V^*]}{A}
  \end{gathered}
  % \vspace{-0.5em}
\end{equation}

除此之外,基站可以利用蜂窝网络进行感知任务的传播和感知数据的收集,基站与移动设备之间的通信概率以 $\lambda_f$ 表示。
由于基站不存在移动属性且蜂窝网络的通信覆盖半径远远大于 D2D 通信,因此 $\lambda_f$ 和移动设备的运动特征没有直接关系,仅表示基站和移动设备因群智感知发生通信的概率。
但是在日常行为中,用户会使用移动设备拨打电话、发送短信、查看社交网络信息。
这些操作,都需要移动设备和基站进行数据交互,因此也有研究人员将感知任务信息压缩至这类频繁的数据交互中,以加快感知任务的执行。
基于此,本章将 $\lambda_f$ 作为可变量来对待。

% 基站也会以一定的速率将感知任务发送至移动设备,或者从移动设备收集感知数据。
% 将基站和移动设备因感知任务而交互的速率定义为基站和移动设备之间的相遇率 $\lambda_f$。
% 由于基站和移动设备使用蜂窝网络进行数据通信,所以 $\lambda_f$ 和移动设备的运动轨迹没有直接联系。
% 所以,$\lambda_f$ 也可以理解为基站和移动设备因群智感知发生通信的概率。

\section{随机过程分析模型及感知质量优化算法}
% 在本节中,给出了任务传播阶段,数据收集阶段的随机过程分析,以及在给定参数(如N,N f,λn,λ)的覆盖度量中可实现的感知质量的推导。 F 。 为方便读者,表I总结了本文中使用的主要符号。
本小节主要阐述任务分发过程和数据收集过程的理论分析方法。
表 ~\ref{table_notations_UIC} 给出了本章中使用的符号定义。

\begin{table}[h]
  % \vspace{-0.5em}
  \caption{数学符号及定义}
  \vspace{-0.5em}
  \centering
  \label{table_notations_UIC}
  % \centering
  \begin{tabular}{|c|p{9.5cm}|}
  \hline
  \textbf{符号} & \textbf{定义}\\
  \hline
  $N$ & 目标区域内移动设备的数量\\\hline
  $N_f$ & 目标区域内基站的数量\\\hline
  $\lambda_n$ & 移动设备之间的相遇概率\\\hline
  $\lambda_f$ & 基站和移动设备的通信概率\\\hline
  $I(t)$ & $t$ 时刻已经收到任务的移动设备的数量\\\hline
  $I'(t)$ & $I(t)$ 的导数\\\hline
  $C_T{t}$ & 生命周期为 $T$ 的感知应用在 $t$ 时刻的感知任务覆盖率\\\hline
  $P_{rcv}(t)$ & $t$ 时刻基站收到感知数据的概率\\\hline
  $P_{nD2D}(t)$ & 感知数据使用 D2D 通信未送达的概率\\\hline
  $P_{nDirect}(t)$ & 感知数据使用蜂窝网络未送达的概率\\\hline
  $S(t)$ & 感知数据在网络中的拷贝数量\\\hline
  % $\boldsymbol{G}$ & 将目标区域划分为网格 $g(i)$ 的集合($g(i) \in \boldsymbol{G}$)\\\hline
  % $L_{ij}(t)$ & $t$ 时刻移动设备$j$是否在网格 $g(i)$中\\\hline
  % $C_{ij}(t)$ & 设备 $j$ 在 $g(i)$ 中的停留时间\\\hline
  % $C(t)$ & $t$ 时刻,移动群质感是的区域覆盖质量\\\hline
  $D(t)$ & 任务分发时间为 $t$ 时,感知数据的收集总量\\\hline
\end{tabular}
  % \vspace{-1em}
\end{table}

\subsection{任务分发过程分析}

在感知任务分发过程中,使用蜂窝网络分发感知任务的速度和区域中基站数量 $N_f$,以及基站与移动设备的通信概率 $\lambda_f$ 相关。
对于已经收到感知任务的移动设备而言,在其移动过程中利用D2D通信对任务进行二次分发。
这种基于 D2D 通信的感知任务分发和人群中传染病的扩散原理类似。
因此可以借助 SIR(Susceptible Infective Removal)模型来描述移动设备之间的任务分发过程。
令 $t$ 时刻,已经收到感知任务的移动设别数量为 $I(t)$。
则此时 $I(t)$ 的增量可以分为两个部分来表示,一部分通过蜂窝网络收到感知任务,另一部分则通过 D2D 网络收到感知任务。
所以,对于 $t$ 时刻而言,已经收到感知任务的移动设备增量 $I'(t)$ 可以用式~\eqref{Formula_SIR_with_fog} 进行计算。

% 当移动设备从基站收到感知任务之后,
% 因此可以借助 SIR(Susceptible Infective Removal)模型来描述移动设备之间的任务分发过程。
% 但是和传统 SIR 模型不同,基站还可以利用蜂窝网络直接将感知任务分发至移动设备。
% 因此感知任务的分发速度还和基站数量 $N_f$ 、基站与移动设备的通信概率 $\lambda_f$ 相关。
% 借助 SIR 模型,将已经收到感知任务的移动设备数量记为 $I(t)$。
% 由于这些设备会以 D2D 通信的方式继续分发感知任务,其数量也会影响感知任务的分发速度。
% 考虑到感知任务可以同时借助 D2D 网络和蜂窝网络进行传播,因此对于 $t$ 时刻而言,已经收到感知任务的移动设备的增量 $I'(t)$ 可以用式~\eqref{Formula_SIR_with_fog} 进行计算。

\begin{equation}
  \label{Formula_SIR_with_fog}
  \begin{aligned}
    I'(t) &= N_f \lambda_f (N-I(t)) + I(t) \lambda_n (N-I(t)) \\
    &= N_f \lambda_f N + ( N \lambda_n - N_f \lambda_f) I(t) - \lambda_n I^2(t) 
  \end{aligned}
\end{equation}

式~\eqref{Formula_SIR_with_fog} 是一个典型的黎卡提(Riccati)常微分方程形式。
因此,其求解过程可以转化为如式~\eqref{Formula_Riccati} 所示的通项公式求解。

\vspace{-1em}
\begin{equation}
  \label{Formula_Riccati}
  \left\{
    \begin{aligned}
      &I'(t) = A \times I^2(t) + B \times I(t) + C \\
      &A = -\lambda_n \\
      &B = N \lambda_n - N_f \lambda_f \\
      &C = N_f \lambda_f N \\
      &I(t) = \frac{C(B+\sqrt{B^2 - 4AC})(e^{\sqrt{B^2 - 4AC} t} - 1)}{-2ACe^{\sqrt{B^2 - 4AC} t} + B^2 -2AC + B \sqrt{B^2 - 4AC}} + D, D 为常数
    \end{aligned}
  \right.
\end{equation}

考虑到初始条件 $I(0)=0$,即在移动群智感知开始时刻($t=0$)没有移动设备收到感知任务,则 $t$ 时刻收到感知任务的移动设备数量可由式~\eqref{Formula_It_with_fog} 表示。
该式中,存在限制条件 $0 < \lambda_f \times N_f < 1$,其中 $0 < \lambda_f \times N_f$ 代表基站一定能和移动设备发生通信;$\lambda_f \times N_f < 1$ 代表基站不会在极短的时间内和所有移动设备通信。
% 当 $\lambda_f \times N_f = 1$ 时,也说明

% \vspace{-1em}
\begin{equation}
\label{Formula_It_with_fog}
\begin{aligned}
I(t) = \frac{N (e^{(\lambda_n N + \lambda_f N_f) t} - 1)}{e^{(\lambda_n N + \lambda_f N_f) t} + \frac{\lambda_n N}{\lambda_f N_f}}, \forall\ \lambda_n \in [0, 1),\ 0 < \lambda_f \times N_f < 1
\end{aligned}
\end{equation}

当移动设备接收到移动群智感知任务之后,根据任务内容执行感知操作并将感知数据存储在本地缓存中,然后等待机会将数据传送回基站。
为了描述移动群智感知任务的覆盖率,这里使用已接受感知任务的移动设备数量与感知区域内移动设备总数的比例作为标准。
用 $C_T(t)$ ($t < T$)表示生命周期为 $T$ 的群智感知应用在执行 $t$ 时间段后的感知任务覆盖率,$C_T(t)$ 的计算方法如式~\eqref{Formula_Coverage_of_Task}。
当 $t=0$ 时,$C_T(t)$ 值为 0 表示没有移动节点收到感知任务。
当 $C_T(t) = 1$ 时,表示目标区域内的所有移动设备均已接收到感知任务。

\begin{equation}
  \label{Formula_Coverage_of_Task}
  C_T(t) = \frac{I(t)}{N} = \frac{e^{(\lambda_n N + \lambda_f N_f) t} - 1}{e^{(\lambda_n N + \lambda_f N_f) t} + \frac{\lambda_n N}{\lambda_f N_f}}
\end{equation}

\subsection{数据收集过程分析}

% 感知设备在收到感知任务之后,需要在群智感知应用的生命周期内将感知数据反馈给基站,通过基站对大量的感知数据进行去冗、加工、压缩之后,在交付给云端资源进行后续计算。
% 为了确保感知数据的质量,移动群智感知应用的发起者多期望能够收集到充足的感知数据,并且让感知数据覆盖范围广阔。
% 这也意味着在规定的时间范围内,基站必须尽可能收集所有移动设备上的感知数据。
% 由于 D2D 通信的引入,移动设备间的通信和设备的运动状态有关。
% 因此,必须考虑移动设备
% 然而蜂窝网络的使用成本较高,而 D2D 通信的频率又和移动设备的运动状态有关。
% 因此,在时域上分析数据收集过程也是非常重要的工作。

已经收到任务的移动设备按照要求获取感知数据后,需要及时地将感知数据传回基站。
而移动群智感知应用在其生命周期内,收集到的数据越多n数据覆盖范围越大,则移动群智感知服务所能提供的服务质量就越高。
这也意味着在规定的时间范围内,基站必须尽可能收集来自所有移动设备上的感知数据。
因此,在时域上分析数据收集过程也是非常重要的工作。

和任务分发时的途径一样,感知数据在收集过程中,移动设备也可以使用蜂窝网络或者 D2D 网络发送感知数据。
在这一过程中,任何部署有感知数据收集服务的基站都可以作为感知数据的目的地。
在使用 D2D 通信时,移动设备可以相互交换感知数据,并将收到的感知数据进行打包处理并为下一次转发做好准备。
当感知数据包到达任何一个基站,则该数据包中的所有感知数据被认为已经收集完成。
令一份感知数据从开始发送到经历时间段 $t$ 后,该数据被基站接收的概率记为 $P_{rcv}(t)$。
在边缘网络中,移动设备可以分别使用蜂窝通信或者 D2D 通信传输感知数据,并且这两种传播方法可视为相互独立事件。
因此,$P_{rcv}(t)$ 的计算方法如式~\eqref{Formula_ProbaRcvT}。

\vspace{-1em}
\begin{equation}
  \label{Formula_ProbaRcvT}
  \begin{aligned}
    P_{rcv}(t) = 1 - P_{nD2D}(t) P_{nDirect}(t)
  \end{aligned}
\end{equation}

上式中,$P_{nD2D}(t)$ 是指该感知数据包未能通过 D2D 通信方式送达基站的概率,$P_{nDirect}(t)$ 是指数据包未能通过蜂窝网络送达基站的概率。
基于章节~\ref{UIC:Scenario} 中的场景定义,感知数据每利用 D2D 网络传播一次,边缘网络中该数据的拷贝便增加一份,用 $S(t)$ 表示一份感知数据在边缘网络中的拷贝数量。
由于移动设备利用一次 D2D 通信将感知数据发送给非基站的概率为 $N/(N + N_f)$。
因此,$P_{nD2D}(t)$ 可以通过计算感知数据的数量比例以求出其被基站收集的概率。
其计算方法如式~\eqref{Formula_ProbaNP} 所描述。

\begin{equation}
  \label{Formula_ProbaNP}
  \begin{aligned}
  P_{nD2D}(t) = (\frac{N}{N + N_f})^{S(t) - 1}, S(t) \geq 1
  \end{aligned}
\end{equation}

当移动设备使用蜂窝网络传输感知数据时,由于该移动设备可能与其它移动设备利用 D2D 通信交换过感知数据,因此一个移动设备用蜂窝网络可能会上传多个感知任务。
由于一份感知数据在边缘网络中存在 $S(t)$ 份拷贝,所以不含有该数据的 $N - S(t)$ 个移动设备与基站通信时,该数据不会被上传至基站。
因此,$P_{nDirect}(t)$ 可以利用式~\eqref{Formula_ProbaNP}进行计算得出。

\begin{equation}
  \label{Formula_ProbaNA}
  \begin{aligned}
  P_{nDirect}(t) = (\frac{N-S(t)}{N})^{\lambda_f N_f t}, \forall \ \lambda_f N_f \in [0,1)
  \end{aligned}
\end{equation}

在上述两式中,$S(t)$ 代表感知数据因 D2D 传输产生的拷贝数量。
这一过程可以理解为单传播源的 SIR 模型。
因此,其计算方法如式~\eqref{Formula_St}。

\begin{equation}
\label{Formula_St}
  \begin{aligned}
    S(t) = \frac{N e^{\lambda_n N t}}{e^{\lambda_n N t} + N -1}, \forall \ \lambda_n \in [0,1)
  \end{aligned}
\end{equation}

将式~\eqref{Formula_ProbaNP}、式~\eqref{Formula_ProbaNA} 和式~\eqref{Formula_St} 带入式~\eqref{Formula_ProbaRcvT},可以得到 $P_{rcv}(t)$ 的最终计算方法,如式~\eqref{Formula_ProbaRcvTF}。
式中,$t$ 代表感知任务分发的执行时间,$t'$ 代表感知数据收集的执行时间。
因此移动群智感知应用的生命周期 $T_{total\_time}$ 为 $t$ 和 $t'$ 之和。

\begin{equation}
  \label{Formula_ProbaRcvTF}
  \begin{gathered}
    P_{rcv}(t, t') = 1 -  (\frac{N}{N + N_f})^{\frac{N e^{\lambda_n N t}}{e^{\lambda_n N t} + N -1} - 1} \times (\frac{N-\frac{N e^{\lambda_n N t'}}{e^{\lambda_n N t'} + N -1}}{N})^{\lambda_f N_f t'}, \\
    \forall \  t > 0, t' >0
  \end{gathered}
\end{equation}

% \subsection{感知覆盖范围分析}

% 在移动群智感知应用中,往往需要感知数据的产生位置能够广泛的地域范围。
% 传感数据所覆盖的地域范围越广,则移动群智感知服务的普适性越强。
% 例如,一个十字路口的交通状况需要通过附近所有道路上的车辆数量来预测。
% 为了计算移动群智感知应用的覆盖率,将群智感知应用的目标区域 $\boldsymbol{G}$ 切割成一组网格单元,如图~\ref{Figure_CoverageofArea} 所示,$\boldsymbol{G} = \{g_1,g_2,g_3,\ldots,g_m\}$。
% 网格单元的大小意味着地域空间上的感知密度,该参数由移动群智感知应用的开发者所决定。

% \begin{figure}[!h]
%   \centering
%   % \vspace{-1.5em}
%   \includegraphics[width=200pt]{./figures/Sec_UIC/移动群智感知的空间分布.pdf}
%   \vspace{-0.5em}
%   \caption{移动群智感知的空间分布}
%   \vspace{-0.5em}
%   \label{Figure_CoverageofArea}
% \end{figure}

% 在分析感知覆盖时,用二进制变量 $L_{ij}(t)$ 表示时刻 $t$ 时,移动设备 $j$ 是否在网格$g(i)$中。
% 其定义如式~\eqref{Formula_LocationIJ} 。

% \begin{equation}
%   \label{Formula_LocationIJ}
%   L_{ij}(t) = \left \{
%   \begin{aligned}
%   & 1,\ \text{$t$时刻时移动设备 $j$ 位于网格 $g(i)$ 中}\\
%   & 0,\ \text{$t$时刻时移动设备 $j$ 不在网格 $g(i)$ 中}
%   \end{aligned}
%   \right.
% \end{equation}

% 利用 $L_{ij}(t)$,移动设备 $j$ 在网格 $g(i)$ 中的总停留时间 $C_{ij}$ 可以用式~\eqref{Formula_UserTrace} 表示。

% \vspace{-1.5em}
% \begin{equation}
%   \label{Formula_UserTrace}
%   C_{ij}(t) = \int_{0}^{t}L_{ij}(\varepsilon )d\varepsilon , \forall \ i \in [0,m], j \in [0,N]
% \end{equation}

% 因此,在 $t$ 时刻,移动群智感知服务的感知质量 $C(t)$ 可以利用来式~\eqref{Formula_UserCoverageGrip} 来计算。

% \vspace{-1.5em}
% \begin{equation}
%   \label{Formula_UserCoverageGrip}
%   \begin{gathered}
%   C(t) = \sum^{m}_{i=0}\sum^{N}_{j=0} g_i \times C_{ij}(t), \forall \ t>0
%   \end{gathered}
% \end{equation}

\subsection{优化目标}

对于群智感知应用的发起者而言,在其生命周期内收集到的感知数据总量是衡量感知质量的重要指标之一。
通过前文对群智感知执行过程的分析,可以得到群智感知应用部署率和执行时间之间的关系 $C_T(t)$,以及感知数据的收集概率 $P_{rcv}(t)$。
令图~\ref{Figure_MCS_Delay} 中的 $T_{total\_time}$ 表示群智感知应用的生命周期,并令 $t$ 表示感知任务分发的执行时间,则 $T_{total\_time} - t$ 代表感知数据收集的执行时间。
为了方便度量单个移动设备收到感知任务并成功反馈感知数据的总时长,在此假设每个移动设备收到感知任务收到感知任务之后只会产生一组感知数据。
因此在感知应用中产生的感知数据总数量为 $N \times C_T(t)$,这些数据传回基站的概率为 $P_{rcv}(t)$。
用 $D(t)$ 表示群智感知应用在生命周期内收集到的感知数据总量,其计算方法如式~\eqref{Formula_DataAmount}。

\vspace{-2.5em}
\begin{equation}
  \label{Formula_DataAmount}
  \begin{gathered}
    \begin{aligned}
      D(t) = &N \cdot C_T(t) \cdot P_{rcv}(T_{total\_time}-t)\\
      = &N \times \frac{e^{(\lambda_n N + \lambda_f N_f) t} - 1}{e^{(\lambda_n N + \lambda_f N_f) t} + \frac{\lambda_n N}{\lambda_f N_f}} \\
      &\times (1 -  (\frac{N}{N + N_f})^{\frac{N e^{\lambda_n N t}}{e^{\lambda_n N t} + N -1} - 1} \times (\frac{N-\frac{N e^{\lambda_n N (T_{total\_time} - t)}}{e^{\lambda_n N (T_{total\_time} - t)} + N -1}}{N})^{\lambda_f N_f (T_{total\_time} - t)}),
    \end{aligned}\\
    \forall \ t \in [0, T_{total\_time}]
  \end{gathered}
\end{equation}

在该式中,移动设备数量 $N$、基站数量 $N_f$、移动设备之间的相遇率 $\lambda_n$、以及移动设备和基站之间的通信概率 $\lambda_f$ 可以看做场景参数,而感知任务分发的时间 $t$ 则是式中唯一的自变量。
当 $t = 0$ 和 $t = T_{total\_time}$ 时,分别表示感知任务的分发时间为 0 和感知数据数据的收集时间为 0,在这两种情况下,$D(t)$ 的计算结果为0,意味着没有收到任何感知数据。
因此,需要找到群智感知在实行过程中合适的时间配额划分,让 $D(t)$ 取值最大是本章的优化目的。
由于 $t = 0$ 和 $t = T_{total\_time}$ 时,$D(t)$ 恒等于 $0$,而通过观察 $D(t)$ 的函数式,可以发现 $D(t)$ 在 $ t \in [0, T_{total\_time}] $ 时是连续函数。
因此,一定存在一个 $t$ 的取值,使 $D(t)$ 最大。

\subsection{基于感知任务生命周期的感知任务执行划分算法}
\label{UIC:algo}

% 在本节中,在最后一节中介绍了一个基于随机过程分析的应用用例。尽管具有D2D通信的MCSwith应该是延迟容忍的,但是一些应用也受到截止时间的限制,因为感测数据和截止时间变得非常宝贵。由于MCS过程有两个阶段,因此研究如何在这两个阶段分配时间以最大限度地提高平均水平具有重要意义。根据上面的分析,假设任务分发阶段持续,即分配给任务分发的时间,可以得出deadlineTtotaltimeas接收到的数据包总数

前文已经阐述移动群智感知应用中,足够多的参与设备和大量的反馈数据才能保证移动群智感知应用的感知质量。
在上一小节中已经提到,当移动群智感知的目标场景已经确定时,感知任务分发时长和感知数据收集时长,都会影响感知应用在其生命周期内能收集的感知数据总量,进而影响感知质量。
所以在有限的生命周期内,需要合理分配感知任务分发过程和感知数据收集过程的时间配额,以提高移动群智感知应用的感知质量。
当任务分发过程的时间配额为 $t$ 时,在生命周期 $T_{total\_time}$ 中群智感知应用可以收集到的感知数据总数可以利用式~\eqref{Formula_DataAmount} 计算。
因此,必须基于式~\eqref{Formula_DataAmount} 给出合理的群智感知执行时间划分,令 $D(t)$ 最大。


\begin{algorithm}[!b]
  \setstretch{\algostretch}
  \KwIn{$N$ 移动设备的总数量}
  \KwIn{$N_f$ 基站的数量}
  \KwIn{$\lambda_n$ 移动设备之间的相遇率}
  \KwIn{$\lambda_f$  基站和移动设备之间的通信概率}
  \KwIn{$T_{total\_time}$  移动群智感知服务的生命周期}
  % \KwData{『输入数据』}
  % \KwData{$Occ[1:T]$: 每个程序当前的NCP大小,初始为0,随着缓存路的分配增加,最终为每个程序在整个缓存上的NCP}
  令 $\Delta t = T_{total\_time}/N_{slots}$\\
  % 令 $t = \Delta t, 2 \Delta t, 3 \Delta t, 4 \Delta t,\ldots,N_{slots} \Delta t$\\
  定义数组 $res[N_{slots}]$ \\
  令 $index = 0;\ max_D = 0;\ t_{best} = 0$\\
  \For{$index \in [0, N_{slots}]$}{
    $res[index] = D(index \times \Delta t)$ \quad (式~\eqref{Formula_DataAmount})\\
    \If {$D(index \times \Delta t) > max_D$}{
      $max_D = D(index \times \Delta t)$ \quad 替换 $D(t)$ 的最大值\\
      $t_{best} = index \times \Delta t$ \quad 替换令 $D(t)$ 取最大值的 $t_{best}$
    }
    % $index$ ++
  }
  % return $maximum, t_{best}$\\
  \KwOut{$max_D ($D(t)$ 的最大值), t_{best}$ (对应 $t$ 的值)}
  \caption{利用枚举找出令 $D(t)$ 最大的 $t$ 值}
  \label{algo_findbestt}
\end{algorithm}

由于 $D(t)$ 的导数计算过于复杂,无法通过导数推导来求得 $D(t)$ 取最大值时 $t$ 的解析解。
另一方面, $D(t)$ 可以视为单调递增函数与单调递减函数的乘积,其单调性可能存在多个拐点,导致无法直接确定 $D(t)$ 取最大值时 $t$ 的取值范围。
因此,这里使用枚举算法来找出 $t$ 的最优解。
% 由于 $D(t)$ 的表达式中,仅存在 $t$ 一个参数确定的函数式,

% 因此,不同阶段的时间配额会直接影响到移动群智感知应用的感知质量。
% 当 $t=0$ 或 $t=t_{total\_time}$ 时,$D(t)$ 恒为零。
% 且当 $0 \leq t \leq t_{total\_time}$ 时,$C_T(t)\geq 0$ 且 $C_T(t)$ 为单调递增函数;$C_F(T_{total\_time}-t) \geq 0$ 且 $C_F(T_{total\_time}-t)$ 为单调递减函数。
% 因此,当$0 \leq t \leq t_{total\_time}$时,$D(t) \geq 0$ 且对于 $t$ 必定存在一个值使得 $D(t)$ 取到最大值,即移动群智感知应用的感知质量达到最优。
% 为了找到最佳的时间划分,本文设计了一个枚举算法来找出 $t$ 的最优解。
该算法伪代码如算法~\ref{algo_findbestt} 所示。
在该算法中,将移动群智感知应用中的生命周期分成 $N_{slots}$ 份。
以 $\Delta t = T_{total\_time} / N_{slots}$ 为步进,对移动群智感知应用中的感知任务分发时间 $t$ 进行枚举( $t = \Delta t, 2 \Delta t, 3 \Delta t, \ldots , N_{slots} \Delta t$ )。
根据 $t$ 的 $N_{slots}$ 种取值,求出对应的 $D(t)$。
最后,找出 $D(t)$ 的最大值以及其所对应的 $t$。
该算法的时间复杂度为 $O(N_{slots})$。


\section{实验结果及分析}

% 为了验证分析的正确性和准确性,在本节中报告了基于模拟的结果。 此外,的算法在寻找最优时间分配方面的效率得到了证明。
为了验证本章分析方法的可行性,本节将随机过程分析模型的分析结果与 ONE 模拟器~\cite{DBLP:conf/simutools/OK09}对移动群智感知应用的仿真结果进行对比。
通过对比理论分析结果和模拟结果,验证随机过程分析方法的正确性以及感知应用执行时间划分算法的效果。
最后,根据理论分析结果和模拟结果,提出了面向群智感知质量优化的边缘资源调度策略。

\subsection{ONE 模拟器设置}
% 进行跟踪驱动模拟,以广泛评估对雾计算中MCS应用的基于ODE的分析。遵循ONE模拟器[ 24 ]的原理,捕获雾环境中的网络行为,并获得参数,如N;国家森林机构;andf. n通过追踪数据包路径和所有移动设备的状态,可以获得有效MCS参与者的数量、雾节点接收到的数据包数量,以及在一定时间内达到的覆盖率。在的随机过程分析中,这些信息可以由场景参数sN导出;国家森林机构;andf. n将分析结果和仿真结果进行比较,能够验证分析的正确性和准确性。为了彻底验证随机过程分析的准确性,首先验证对每个阶段的分析,然后对整体分析进行比较。
ONE~(Opportunistic Network Environment)模拟器~\footnote{The ONE simulator https://akeranen.github.io/the-one/} 是一个开源的机会式通信网络环境模拟器。
利用该模型器,可以构建具有如下特征的机会式通信网络:
1)移动设备可以单独配置其移动模型或移动轨迹;
2)可以在数据转发过程中部署不同的路由算法;
3)实时记录模拟网络环境中的移动设备运动状态和数据流状态;
4)利用地图信息和真实世界中移动设备的轨迹数据,模拟真实世界中的网络场景。

\begin{table}[h]
  % \vspace{-0.5em}
  \caption{ONE模拟器中的场景参数及其定义}
  \vspace{-0.5em}
  \centering
  \label{table_notations_ONE}
  % \centering
  \begin{tabular}{|c|p{7cm}|}
  \hline
  \textbf{场景参数} & \textbf{定义}\\
  \hline
  $N$ & 移动设备的数量\\\hline
  $N_f$ & 基站的数量\\\hline
  $\boldsymbol{L}[N]$ & 移动设备的位置\\\hline
  $\boldsymbol{V}$ & 移动设备的轨迹和速度\\\hline
  $\boldsymbol{L}[N_f]$ & 基站的位置\\\hline
  $f_{task}$ & 基站发送感知任务的频率\\\hline
  $A$ & 目标区域的面积\\\hline
  $d$ & D2D 通信的覆盖半径\\\hline
  % $\lambda_n$ & 移动设备之间的相遇概率\\\hline
  % $\lambda_f$ & 基站和移动设备的通信概率\\\hline
  % $I(t)$ & $t$ 时刻已经收到任务的移动设备的数量\\\hline
  % $I'(t)$ & $t$ 时刻 $I(t)$ 的增量\\\hline
  % $P_{rcv}(t)$ & $t$ 时刻基站收到感知数据的概率\\\hline
  \end{tabular}
  % \vspace{-1em}
\end{table}

% 在测试过程中,本章使用 ONE 模拟器对边缘计算模型下的移动群智感知应用进行模拟仿真。
在 ONE 模拟器中,可以设置的主要场景参数如表~\ref{table_notations_ONE} 所示。
模拟参数设置完成之后,ONE 模拟器会根据自定义的群智感知应用逻辑和数据转发路由算法,模拟移动设备的移动过程、设备之间的通信过程。
仿真完成后,模拟网路中的所有操作会以日志的方式进行保存。
通过对日志文件的分析,可以还原网络中的数据包流向,进而得到群智感知应用执行过程中,感知任务的部署数量、感知数据的收集数量、以及感知数据的覆盖范围。


% 在仿真实验中,主要的输入参数有:$N$、$N_f$、$\lambda_n$、$\lambda_f$,其具体含义如表~\ref{table_notations_ONE} 所示。
% 对仿真过程产生的日志信息进行处理,可以获得参与感知任务的移动设备数量以及每个数据包所经过的具体路径。
% 结合模拟器中移动设备的运动轨迹以及日志的时间戳进一步分析,还可以获得边缘服务器在任意时刻收到的感知数据数量以及已收集的感知数据所覆盖的地域范围。

利用模拟器中设置的$\boldsymbol{V}$、$A$ 和 $d$ 参数,可以利用式~\eqref{Formula_EncounterRate} 计算出模拟场景下移动设备之间的相遇率。
而基站发送感知任务的频率,可以转化为基站和移动设备的通信概率 $\lambda_f$。
结合 $N$ 和 $N_f$ 作为已知参数,带入式~\eqref{Formula_DataAmount} 中计算感知数据收集总量。
将模拟仿真和随机过程分析得到的结果进行对比,可以验证本章模型的正确性。
在对比过程中,先分别验证任务分发过程和数据收集过程的覆盖率,再针对移动群智感知全过程,验证感知质量。
最后分析边缘网络中各场景参数对感知质量的影响并提出调度策略。

% 首先通过使用不同的参数asN来验证对任务传播阶段的分析;国家森林机构;andf,which n定义在表一中。结果在图6中报告,包括模拟结果和通过求解等式获得的分析结果。( 4 )。从这个图中,可以观察到获得了高精度,因为分析结果都非常接近于模拟结果。在5000组实验中,在任务传播阶段,平均误差为5.7 %。还注意到,有效的MCS参与者( EMP )的数量随着时间的增加而增加。最初,增长率低的原因是雾节点,很少有“受感染”的移动设备能够使其他“易受感染”的移动设备生效。后来,有效的MCS参与者的数量急剧增加,因为已经有很多有效的MCS参与者。然而,这种增长最终会收敛,因为大多数移动电话已经被“感染”。这种现象符合的常识。

\subsection{随机过程分析模型准确度测试}

首先通过使用不同的场景参数 $N$(移动设备数量)、$N_f$(基站数量)、$\lambda_n$(移动设备之间的相遇率)和 $\lambda_f$(移动设备和基站的通信概率)来验证本章对任务分发过程的分析。
图~\ref{Figure_PropagationTest} 反应了不同参数取值对移动群智感知效率的影响。
图中正下方的数值分别为场景参数 $N$、$N_f$、$\lambda_n$ 和 $\lambda_f$ 的取值。
图中横坐标为时间轴,纵坐标为目标区域内收到感知任务的移动设备数量,其结果可通过式~\eqref{Formula_Coverage_of_Task} 计算得到。
通过曲线对比分析结果和模拟结果,两者的变化趋势是非常相近的。
在5000组不同参数的实验对比中,任务分发数量的平均误差为 5.7\%。
对图像进行观察可以发现,移动群智感知应用执行初期,感知服务部署速度偏低,这是因为大部分感知任务只能通过基站分发。
在获得感知任务的移动设备增多之后,设备之间的 D2D 通信让区域内的任务分发源也越来越多,因此能够执行感知任务的移动设备数量呈指数趋势上升。
当感知任务的覆盖接近饱和时,未接受感知任务的移动设备数量较少,因此感知任务的部署速度逐渐降低,获得感知任务的移动设备总数逐步趋近于目标区域内的移动设备总数。

\begin{figure}[!h]
  \centering
  % \vspace{-2em}
  {\includegraphics[width=214pt]{./figures/Sec_UIC/Propagation/F3-1.pdf}}
  {\includegraphics[width=214pt]{./figures/Sec_UIC/Propagation/F3-2.pdf}}\\
  {\includegraphics[width=214pt]{./figures/Sec_UIC/Propagation/F3-3.pdf}}
  {\includegraphics[width=214pt]{./figures/Sec_UIC/Propagation/F3-4.pdf}}
  \vspace{-0.5em}
  \caption{任务分发过程中分析结果和模拟结果对比}
  \vspace{-0.5em}
  \label{Figure_PropagationTest}
\end{figure}

然后,用同样的方法来验证数据收集过程的分析。
假设在数据收集过程开始时($t=0$),区域内的所有移动设备都已经收到感知任务且采集到一份感知数据。
随着时间的推移,移动设备需要借助蜂窝网络和 D2D 通信将感知数据发送给 $N_f$ 个基站中的任意之一。
其中,移动设备和基站的蜂窝通信依然遵循通信概率 $\lambda_f$。
因此,可以利用式~\eqref{Formula_ProbaRcvTF} 计算被成功接收数据包的理论数量。
在 ONE 模拟器中,也可以利用日志信息获得每个基站收集到的感知数据数量。
图~\ref{Figure_CollectionTest} 展示了模拟试验和理论分析的对比结果。
图中,两者的变化趋势依然保持一致。
在5000组不同参数的实验对比中,数据收集理论分析的平均误差为 9.6\%。
且变化趋势和实际情况相符。
同时,通过对比图~\ref{Figure_PropagationTest} 和图~\ref{Figure_CollectionTest} ,可以发现感知数据收集的末期并感知任务分发末期具有明显的逼近趋势,并且在部分数据收集时期有突发的感知数据增幅。
造成这种现象的原因是因为某一移动设备在和基站通信之前经由 D2D 通信得到了其它移动设备的感知数据。

\begin{figure}[!h]
  \centering
  % \vspace{-0.5em}
  {\includegraphics[width=214pt]{./figures/Sec_UIC/Collection/R1-0.pdf}}
  {\includegraphics[width=214pt]{./figures/Sec_UIC/Collection/R1-1.pdf}}\\
  {\includegraphics[width=214pt]{./figures/Sec_UIC/Collection/R1-2.pdf}}
  {\includegraphics[width=214pt]{./figures/Sec_UIC/Collection/R1-3.pdf}}
  \vspace{-0.5em}
  \caption{数据收集过程中分析结果和模拟结果对比}
  \vspace{-0.5em}
  \label{Figure_CollectionTest}
\end{figure}

最后,结合群智感知任务分发和感知任务收集,对边缘网络中群智感知应用执行过程进行模拟和分析。
在 ONE 中,设置不同的参数以模拟不同场景下的群智感知应用执行状态。
通过设置不同的任务分发执行时间 $t$,观察在预设生命周期 $T_{total\_time}$ 内基站可以收到的感知数据数量。
同时,利用式~\ref{Formula_DataAmount} 可以通过理论分析计算出感知数据的收集总量。
该实验使用了四组不同的场景参数来观察不同的感知应用执行时间划分对最终感知数据收集总量的影响。
在所有的对比实验中,移动群智感知应用的生命周期 $T_{total\_time}$ 被设置为100秒。
令感知任务分发的执行时长为 $t$,则从 $t$ 时刻开始,所有移动设备和基站不再继续分发感知任务,而是开始收集感知数据。
模拟结果和分析结果的对比如图~\ref{Figure_EntireTest} 所示。
横坐标 $t$ 表示感知应用执行过程的时间划分,即感知任务分发过程所占用的时间。
由于感知数据可以直接从传感器读取,时间占用极少,
因此 $T_{total\_time} - t$  表示了感知数据收集的时长。
图示结果直观的反映出本章的理论分析方法和模拟实验结果非常吻合,在不同时间配额划分方案下,分析结果总是接近模拟结果。
这也验证了本章对任务分发和数据收集两个阶段的分析也是准确无误的。
此外,在图中还能观察到,接收到的感知数据总量首先随着任务分发的时间配额的增加而增加,在达到最大值之后,可收集的感知数据总量开始变少。
由于在固定生命周期内,需要同时兼顾感知任务的分发和感知数据的收集,因此,当感知应用执行时间的划分不合理时,会导致感知数据的收集总量减少,降低感知服务质量。
这也意味着存在一个最佳时间分配,可以最大限度地提高感知质量。

\begin{figure}[!h]
  \centering
  % \vspace{-1em}
  {\includegraphics[width=214pt]{./figures/Sec_UIC/RcvsSim/0.pdf}}
  {\includegraphics[width=214pt]{./figures/Sec_UIC/RcvsSim/1.pdf}}\\
  {\includegraphics[width=214pt]{./figures/Sec_UIC/RcvsSim/2.pdf}}
  {\includegraphics[width=214pt]{./figures/Sec_UIC/RcvsSim/3.pdf}}
  \vspace{-0.5em}
  \caption{不同时间配额下的分析结果和模拟结果}
  \vspace{-1em}
  \label{Figure_EntireTest}
\end{figure}

为了验证时间配额划分算法的准确性,测试中使用参数 $N=6000$、 $N_f=4$、 $\lambda_n = 0.000027$、 $\lambda_f=0.00018$ 定义的模拟场景。
移动群智感知应用的生命周期 $T_{total\_time}$ 设置为 100 秒。
在目标区域内,一共有 $N$ 个移动设备,由于每个移动设备生成一组感知数据,因此可以利用收集到的感知数据总数与 $N$ 的比值来表示感知数据的收集比例,这一比例也反映出感知应用的执行效率。
图~\ref{Figure_BestT} 展示了任务分发过程的时间配额分别为25秒、50秒、75秒以及 59.5 秒时移动群智感知应用的执行效率。
其中,59.5 秒是利用感知任务执行划分算法找出的最佳任务分发时间配额。
在不同时间配额之前的时间段内,由于只执行了感知任务的分发,没有对感知数据进行收集工作,所以收集到的感知数据数量为 $0$。
通过图中四种时间划分的结果对比可以发现,当任务分发时间分配过少时(25秒、50秒),只有少量的移动设备收到感知任务,所以收集到的感知数据也较少。
当任务分发时间分配过多时(70秒),所有的感知数据无法及时被发送至基站,导致在生命周期内无法收集齐所有的感知数据。
当时间划分为 59.5 秒时,感知任务分发过程和感知数据收集过程达到平衡,此时可以收集到最多的感知数据。
在算法执行中,$N\_slots$ (章节~\ref{UIC:algo})被设置为 $10^7$,算法执行时间不超过 2500 毫秒。

% \vspace{-1.5em}
\begin{figure}[!h]
  \centering
  \vspace{-1.5em}
  {\includegraphics[width=214pt]{./figures/Sec_UIC/BestT/0.pdf}}
  {\includegraphics[width=214pt]{./figures/Sec_UIC/BestT/1.pdf}}\\
  {\includegraphics[width=214pt]{./figures/Sec_UIC/BestT/2.pdf}}
  {\includegraphics[width=214pt]{./figures/Sec_UIC/BestT/3.pdf}}
  \vspace{-0.5em}
  \caption{不同时间划分的对感知质量的影响}
  \vspace{-1em}
  \label{Figure_BestT}
\end{figure}


\subsection{边缘服务部署对感知质量的影响}

\begin{figure}[!b]
  \centering
	% \vspace{-1.5em}
	{\includegraphics[width=210pt]{./figures/Sec_UIC/SingleVar/1n.pdf}}
	{\includegraphics[width=210pt]{./figures/Sec_UIC/SingleVar/1ln.pdf}}\\
	{\includegraphics[width=210pt]{./figures/Sec_UIC/SingleVar/1nf.pdf}}
	{\includegraphics[width=210pt]{./figures/Sec_UIC/SingleVar/1lf.pdf}}
	\vspace{-1em}
	\caption{各参数对感知质量的影响}
	% \vspace{-1.5em}
	\label{Figure_SingleValTest}
\end{figure}

在之前的测试中,以 $N$(移动设备数量)、$N_f$(基站数量)、$\lambda_n$(移动设备之间的相遇率)和 $\lambda_f$(移动设备和基站的通信概率)作为参数来描述边缘网络中的不同资源部署场景。
通过之前的测试结果也反映出不同的场景参数对群智感知应用的执行效率也有不同的影响效果。
基于此发现,本节单独讨论各参数对感知应用执行效率的影响,以说明边缘网络中不同资源对群智感知应用的作用。
在该实验中,将 $N=5000$、$N_f=2$、$\lambda_n = 0.00003$、$\lambda_f=0.0004$ 作为基准场景,然后单独变化各参数的取值,来观察感知数据的接收总量变化。
实验结果如图~\ref{Figure_SingleValTest} 所示。
群智感知应用的生命周期依然设置为 100 秒。
通过仿真数据和理论数据的对比,可以得知本章的理论分析模型和模拟过程基本吻合。
在实验中,$N_f$ 以倍数关系增长,$N$、$\lambda_n$ 和$\lambda_f$ 均以 10\% 步进等比例增长。
在对比中可以观察到,在边缘计算支撑下的移动群智感知应用中,
移动设备的数量$N$对感知效率的影响最大,其次是移动设备的相遇率 $\lambda_n$。
而基站数量 $N_f$ 和基站与移动设备的通信概率 $\lambda_f$ 的影响作用,都相对较小。

\subsection{边缘服务部署策略分析}
\label{ISPA:结论}

\begin{figure}[!b]
  \centering
	% \vspace{-1.5em}
	{\includegraphics[width=210pt]{./figures/Sec_UIC/args/1n.pdf}}
	{\includegraphics[width=210pt]{./figures/Sec_UIC/args/1ln.pdf}}\\
	{\includegraphics[width=210pt]{./figures/Sec_UIC/args/1nf.pdf}}
	{\includegraphics[width=210pt]{./figures/Sec_UIC/args/1lf.pdf}}
	\vspace{-1em}
	\caption{各参数对感知质量的收益比例}
	% \vspace{-1.5em}
	\label{Figure_SingleVal}
\end{figure}

为了更直观比较不同边缘服务资源对移动群智感知质量的影响,在此将 $N=5000$、$N_f=2$、$\lambda_n = 0.00003$、$\lambda_f=0.0004$ 做为图~\ref{Figure_SingleVal} 的基准测试场景(图中 base 实线)。
同样,感知应用的生命周期依然为100秒。
通过对四种不同参数进行变化,可以看出图中曲线变化趋势有着非常明显的不同。
在这四幅图中,横坐标代表移动群智感知应用的执行时间。
纵坐标代表当前时间点基站能收到感知数据数量。
因此,该图可以直观的反映出群智感知应用在执行过程中,哪一阶段能够收获最多的感知数据。
同时,图中曲线和横轴构成的图形的面积,也就是对应场景参数下感知数据收集的总数量。
第一幅图表述的是群智感知中移动设备数量的变化对感知数据和感知效率的影响。
相较于其他三幅图,可以发现群智感知志愿者数量的基数对感知结果以及感知效率都有着巨大的影响。
这也从另一方面说明了大多数研究为何要关注于参与者的招募工作和激励工作。
然而,对于边缘服务而言,可以发现基站的数量虽然会对感知数据的总量和感知效率产生影响,但是与其租用更多的基站作为边缘服务节点,提高基站和移动设备的通信频率显然更加划算。
当基站与激动设备的通信频率达到一定瓶颈时,此时再租用更多的基站作为边缘服务器,要更具有性价比。
另一方面,租用更多的基站,意味着蜂窝通信网络覆盖的范围更大。
随着目标区域的扩大,可以增加参与群智感知的移动设备数量。
但是是否需要增加基站来作为群智感知应用的边缘服务节点,还需要根据移动设备的具体分布状态进行调整。

\section{本章小结}

本章研究了边缘计算支撑下的典型移动群智感知应用,利用边缘服务进行感知任务传播和感知数据收集。
同时,采用 D2D 通信构建机会式通信网络来加速任务分发和感知数据收集。
借助随机过程分析方法,针对感知应用的执行过程建立模型,通过常微分方程组来描述群智感知的执行过程。
通过求解方程组,找到了边缘服务数量、移动设备数量、移动设备运动状态、群智感知应用覆盖率以及感知数据收集率的关系。
通过任务传播过程和数据收集过程的量化分析,在感知应用生命周期定的情况下,研究了感知质量影响因素优先级和改进方法,并提出任务分发和数据收集两个过程的时间划分方法。
最大限度地提高移动群智感知应用的感知质量。
并通过模拟实验,验证了模型的正确性和优化机制的有效性。


\chapter{面向计算任务卸载的群智感知网络流调度机制}

在移动群智感知中,由于移动设备的处理能力存在个体差异,计算能力较弱的移动设备可以将部分感知数据处理任务卸载至边缘服务器上,以减少因本地数据处理导致的时延。
该过程涉及大量数据流的管理,软件定义网络(Software-Defined Network,简称SDN)被视为网络流管理的有效未来网络技术。
然而,SDN 交换机的流表容量因三进制内容可寻址存储器(Ternary Content Addressable Memory,简称 TCAM)的容量而受到不可避免的限制。
面对大量的移动设备,SDN 在调度移动设备的网络链接时,必须考虑流表容量、带宽和任务卸载决策之间的约束。
基于这些约束条件,本章首先基于计网协同的原则构建了整型线性规划(Integer Linear Programming,简称 ILP)模型来描述负载决策、传输时延和以及感知数据处理成本之间的关系。
然后提出依据权重的链路调度算法,根据 SDN 交换机中的流表容量和链路带宽来调度移动设备和边缘服务器的通信链路,并保障任务卸载所带来的能耗收益。
最后通过对比评估,验证了算法的可行性与调度效果。经验证,该算法达到最佳调度方案87.4\%的节能效益。

\section{本章引言}

由于移动设备个体性能的差异,同一计算任务在不同移动设备上的运行时间和能耗开销也存在较大的区别。
对于性能较弱的移动设备而言,为了加快部分应用的执行速率,可以使用计算任务卸载技术,将本地难以高效执行的计算过程,迁移到计算能力充沛的云端服务器上,从而提高应用执行效率。
例如,使用移动设备执行导航应用,用户只需要上报自己的地理位置、目的地以及出行方式,云端服务器会根据用户的需求进行路线规划,再将路线结果和地图资源反馈给用户。
借助云端的算力,用户还可以在移动设备上进行复杂的图像处理工作(Adobe Creative Cloud),甚至在平板电脑上游玩画质优良的主机游戏、电脑游戏(云游戏)等。

对于移动群智感知而言,大部分研究工作将重心放在参与者激励~\cite{CNKI:JiaChaopeng, DBLP:journals/comsur/ZhangYSLTXM16, CNKI:WuMCSIncentive}和任务分发工作上~\cite{DBLP:conf/huc/LiuGWWYZ16, DBLP:conf/infocom/Xiao0HWL15, DBLP:conf/mass/LiLW15},这些工作已经取得了良好的效果。
因此,本章的研究重点围绕感知数据的收集工作展开。
在群智感知数据收集过程中,需要移动设备对感知数据进行一定的处理工作再进行上传。
对于普通的数值读取式传感器,这类传感数据体积小、处理逻辑简单,大部分的移动设备可以快速完成原始感知数据的处理工作。
然而,对于视频、音频、图像、雷达信号等类别的传感数据,由于原始感知数据信息量大、且处理算法复杂度高,如果没有专用芯片的帮助,移动设备无法对这类原始感知数据进行实时处理。
为了解决这类原始感知数据的处理工作,在移动群智感知场景中常依靠计算卸载技术借助外部算力对这些感知数据进行处理~\cite{Lee:2013fj, Linthicum:2017vv, Kumar:2013dq}。

% 而用户之间的文件共享,也可以依托云存储服务,无需消耗本地设备的网络资源和存储资源。
% 合理利用任务卸载技术,可以帮助性能较弱的终端设备完成复杂任务,降低执行成本。

% 在移动群智感知应用的执行过程中,计算任务卸载技术也被广泛应用。
在城市感知中,多使用网络摄像头等设备采集大量的视频、图像、音频数据。
在车联网环境中,智能车上的雷达设备也会生成大量的异构传感信息。
对于这些感知设备,其计算能力并不能满足对原始数据的实时处理需求,也没有足够的存储资源保存原始数据。
因此,对于这一类群智感知应用,可以将复杂的感知数据处理任务托付给负责收集感知数据的边缘服务节点,在对感知数据处理的同时,也完成了感知数据的收集工作。
因此,在部分实时性要求高、感知数据处理复杂的群智感知应用中,可以借助边缘计算在数据产生源附近部署相应的边缘服务,来实现原始感知数据的实时处理与收集。

然而,边缘计算不同于云计算模型,由于边缘服务器的分布较为独立,没有专用的内部网络支撑计算任务在不同服务器之间的调度。
因此,在进行计算任务卸载时,感知设备和边缘服务器之间的网络链接需要根据计算卸载决策进行协同调度。
基于这一目的,本章利用 SDN 对边缘网络中的网络资源进行管理,利用其控制平面和数据平面的隔离的特征~\cite{Committee:2012un},实现计算任务卸载和网络资源的协同调度。

% 网络中的设备数量和流量类型正在急剧增长。为了更有效地管理网络,新出现的网络架构SDN被发明出来。SDN的目标是分离控制平面和数据平面~\cite{Committee:2012un}。SDN采用可编程网络,并利用虚拟化技术,这与其他网络体系结构不同。同时,SDN还可以共享网络基础设施,使网络功能实现软件化。

\begin{figure}[!h]
  \centering
  \includegraphics[width=440pt]{./figures/Sec_ISPA/OffloadingSDN.pdf}
  \vspace{-1em}
  \caption{基于 SDN 的计算任务卸载和网络资源协同调度}
  \vspace{-1em}
  \label{fig_OffloadingSDN}
\end{figure}

图~\ref{fig_OffloadingSDN} 展示了移动群智感知中,计算任务卸载和边缘网络资源的协同调度场景。
在该图中,性能不同的感知设备所需要卸载的计算任务比例也不相同,而基站也有计算能力强弱之分。
例如麦克风、照相机这些纯感知设备本身并没有数据处理能力,因此需要将数据处理工作卸载至服务器上。
而音频的处理工作相较于图像处理工作要更为简单,因此麦克风的原始感知数据可以上传至性能较弱的基站上,而原始图像数据则上传至处理能力较强的基站上。
对于可穿戴设备、全栈式摄像头、手机平板、VR 设备而言,其自身已具备一定的处理能力,因此只需要卸载部分的计算任务。
而对于笔记本电脑等计算能力充沛的设备,所有的数据处理可以在本地执行完之后再上传至基站。
在该场景中,可以通过 SDN 技术对网络连接进行规划,以确保感知数据的传输时延以及链路带宽资源的合理利用。

尽管利用 SDN 可以在边缘网络中实现基于计算任务卸载的网络资源调度,但是在面对成千上万的感知设备时,SDN 能够管理的网络资源会因流表容量受到一定的约束。
在 SDN 交换机中,多路径路由规则通常使用 TCAM 芯片进行存储,而 TCAM 芯片的容量,则代表 SDN 交换机可以存放在流表~\cite{Dasgupta:2012:DMD:2400771.2401550}中的多路径转发规则的数量。
但是由于 TCAM 芯片成本过高,所以 SDN 交换机中的流表通常只能保存10000到15000条转发规则。
所以,在调度计算任务卸载的过程中,不仅要考虑感知设备与边缘服务器的链路带宽、时延是否满足需求,还需要考虑 SDN 交换机能否支撑海量感知设备场景下的多路径转发规则存储。

另一方面,不同边缘服务器的计算能力和使用成本也存在着差异。
因此在使用计算任务卸载时,每个感知设备需要选择合适的边缘服务器进行计算任务卸载,来减少边缘服务器的使用成本。
基于上述约束条件,在移动群智感知应用中,网络资源的调度必须考虑计算任务卸载决策对链路的带宽、时延的需求,以及SDN 交换机的流表容量的约束。
为此,本章提出一种新的链路调度算法对群智感知应用中的计算任务卸载和边缘网络资源进行协同调度,
以实现网络资源的负载均衡,保障计算任务卸载的服务质量,并最大限度地降低感知数据处理能耗成本。
本章的主要贡献如下:

1)建立 ILP 模型来描述计算任务卸载决策、群智感知网络流调度、和感知数据处理能耗之间的关系。

2)为避免 ILP 模型求解的高计算复杂性,提出了二阶段网络流调度算法,并通过模拟实验验证了该算法的有效性。

\section{问题描述}

\begin{figure}[!b]
  % \vspace{-1em}
  \centering
  \includegraphics[width=400pt]{./figures/Sec_ISPA/scenario.pdf}
  \vspace{-1em}
  \caption{基于感知数据计算任务卸载的边缘计算场景}
  \label{fig_scenario}
\end{figure}

为了对真实场景进行抽象描述,本节以图~\ref{fig_scenario} 为范例对本章的研究问题进行描述。
在现实场景中,通常以子基站围绕主基站的方式部署无线通信网络。
在主基站中,为了辅助通信业务和应用,会增加额外的计算资源和存储资源。
因此,主基站可以作为边缘网络中的边缘服务器,作为感知设备计算任务卸载的目的地。
为了提高无线网络的覆盖率,主基站利用子基站或者无线网络接入点(Wireless Access Point,简称 AP)将网络资源覆盖到更广阔的地理空间中。
通过这种部署,感知设备可以直接利用子基站传输数据,也可以通过子基站再经由主基站完成数据的传输工作。
然而受到地理分布的影响,不同基站之间的链接也有时延和带宽的差异。
因此,感知设备在进行计算任务卸载的同时,不仅需要考虑主基站的性能负载,还需要考虑网络资源中带宽和时延的影响。

根据图~\ref{fig_scenario} 所描述的场景,建立分析模型时需要考虑的对象分为三类:
1)感知设备,2)由 SDN 进行路径管理的子基站,3)可执行感知数据计算任务的主基站。
为了让模型专注于研究计算任务调度和网络资环的协同调度关系,本章在场景建立时做出以下假设: 

(1)感知数据处理任务可以在感知设备或主基站上独立运行;

(2)基站之间的通信链路是稳定的;

(3)所有网络连接均通过 SDN 管理。


% 如图~\ref{fig_scenario} 所示,每个用户通过固定AP访问网络资源。用户和主基站之间的所有链接都由SDN分配。SDN的主要职责是为所有用户安排链路,以提高他们的QoS。同时,SDN的工作对用户是透明的。为了满足更多用户,可以租用云计算资源,如虚拟专用服务器(VPS)、Docker等。你想要的资源越多,你就需要付出越多。

在该场景中,使用$u$来代表单一的感知设备,使用$\boldsymbol{U}$代表所有感知设备构成的集合( $ u \in \boldsymbol{U} $)。
在边缘网络中,可以执行计算任务的主基站用$s$表示,$\boldsymbol{S}$ 代表这类基站的集合( $ s \in \boldsymbol{S} $)。
对于负责数据转发的子基站或 AP,统称为 SDN 交换机,使用$r$表示,并用$\boldsymbol{R}$ 代表这类基站的集合( $ r \in \boldsymbol{R} $)。
考虑到 SDN 通过保存多路径转发规则管理链路,这里使用 $l$ 表示感知设备到主基站的一条链路,$ \boldsymbol{L}$ 代表所有感知设备与所有主基站能够建立的全部链接集合($l \in \boldsymbol{L} $)。
由于一个感知设备到一个主基站可能存在多条路径,可以利用$l$是否经过$r$来区分起始点和终点相同的路径(章节~\ref{ISPA:Model} 中有详细描述)。

当使用计算任务卸载时,感知设备可以在本地处理部分原始感知数据后再上传至主基站,也可以使用计算任务卸载将原始感知数据交付给主基站处理,待收到处理后的数据在完成数据上传工作。
这两种方法所对应的感知数据处理时延并不相同。
如图~\ref{fig_timestaps},感知设备 $u$ 在本地对 $1/4$ 的原始感知数据进行计算处理,
然后经由SDN 交换机 $r_1$ 上传 $1/4$ 的原始感知数据至主基站 $s$ 进行处理,余下的原始感知数据则通过SDN 交换机 $r_2$ 和 $r_3$ 交付给主基站 $s'$ 完成计算处理工作。
对于本地处理的感知数据部分,将其计算时延定义为 $T_{u\_local}$。
对于交付给主基站 $s$ 处理的感知数据而言,令其网络上传时延为 $T_{us}$,将主基站 $s$ 处理这些原始感知数据的计算时延定义为 $T_{us\_server}$,而返回结果的网络传输时延定义为 $T_{su}$。
因此,感知设备 $u$ 借助主基站 $s$ 处理感知数据的时延为 $T_{us}$、$T_{us\_server}$ 和 $T_{su}$ 三者之和。
同理,感知设备 $u$ 借助主基站 $s'$ 处理感知数据的时延应当为 $T_{us'}$、$T_{us'\_server}$ 和 $T_{s'u}$ 三者之和。
令 $\Tusresponse$ 表示感知设备 $u$ 借助任意主基站 $s$ ($s \in \boldsymbol{S}$)处理感知数据的总时延,则 $\Tusresponse$ 可以利用式~\eqref{formula_cloudtime} 来计算。

\begin{figure}[!t]
  \centering
  % \vspace{-1em}
  \includegraphics[width=400pt]{./figures/Sec_ISPA/Delay.pdf}
  \vspace{-0.5em}
  \caption{计算任务卸载场景下感知数据的处理时延}
  % \vspace{-1em}
  \label{fig_timestaps}
\end{figure}

\vspace{-1em}
\begin{equation}
  \label{formula_cloudtime}
  % \begin{aligned}
    \Tusresponse = T_{us} + T_{us\_server} + T_{su},\ \forall \ u \in \boldsymbol{U}, s \in \boldsymbol{S}
  % \end{aligned}
\end{equation}

基于假设条件(1):计算任务可以在本地设备或边缘服务器上独立运行,感知设备 $u$、任意主基站 $s$ 可以并行处理感知数据。
对于感知设备 $u$ 而言,其感知数据处理时延应该为 $T_{u\_local}$、$T_{us\_server}$ ($s \in \boldsymbol{S}$)中的最大值。
其中,$T_{u\_local}$ 仅和感知设备的处理速度以及感知数据计算任务量有关,而对于 $T_{us\_server}$ ,不仅与被选择的通信链路相关,还与计算任务卸载比例以及主基站的计算能力相关。
除此之外,当计算卸载到不同的主基站时,由于链路的更换使得数据传输时延也随之发生变化。
因此,在该场景中,感知设备的计算卸载比率、计算卸载目标节点的决策都会影响感知数据的实际处理时间。

另一方面,移动群智感知应用中通常会招募海量的感知设备,由于计算卸载的使用,会导致网络中的链接数量成倍增加。
当链路规划不理想时,部分支撑链路的核心SDN 交换机可能因为 SDN 交换机流表容量的限制而无法承载新的链路请求,进而导致网络资源的负载不平衡甚至网络拥塞。
因此,在该场景下,还应当对链路的选择做出适当的调度,以避免流表容量的限制。

% 由于计算任务卸载需要将感知设备将数据交付给主基站,并等待主基站完成任务给予反馈,因此计算任务卸载的发起到完成确认,需要一定的时间开销。
% 在图~\ref{fig_timestaps} 中,${T}_{us}$是从感知设备$u$到主基站$s$的时延。
% $\Tsu$代表从主基站$s$到感知设备$u$的时延。
% $\Tuscloud$是感知设备$u$的计算任务在主基站$s$中的执行时间。
% $\Tusresponse$是感知设备$u$将计算任务卸载到主基站$s$的总时间开销。
% 基于之间的假设条件(2),基站之间的通信链路都处于稳定状态。
% 因此,在感知设备和主基站之间的通信链路不改变的情况下,感知设备和主基站之间的通信时延$\Tus$和$\Tsu$可视为常数。
% 仅当 SDN 网络在资源调度时改变了感知设备和主基站之间的通信链路,$\Tus$和$\Tsu$才会跟随新的链路发生改变。
% 在此模型中,主基站的计算响应时间$\Tusresponse$可以通过时延的总和来计算,如式~\eqref{formula_responsetime} 所示。

% \vspace{-1.5em}
% \begin{equation}
%   \label{formula_responsetime}
%   \Tusresponse = \Tus + \Tsu + \Tuscloud,\ \forall \ u \in \boldsymbol{U}, s \in \boldsymbol{S}
% \end{equation}


\section{面向感知数据计算卸载的边缘网络调度模型与算法}

为了更好地使用数学语言描述面向任务卸载的边缘网络调度问题,本节使用了表~\ref{table_notations_ispa} 中定义的数学符号,其细节含义将在后文中解释。

\begin{table}[!h]
  \caption{数学符号及定义}
  \vspace{-1em}
  \label{table_notations_ispa}
  \centering
  \begin{tabular}{|c|p{8cm}|}
    \hline
    \textbf{数学符号} & \textbf{定义}\\
    \hline
    $\boldsymbol{U}$ & 由感知设备 $u$ 构成的集合\\\hline
    $\boldsymbol{S}$ & 由主基站 $s$ 构成的集合\\\hline
    $\boldsymbol{R}$ & 由 SDN 交换机 $r$ 构成的集合\\\hline
    $\boldsymbol{C_r}$ & SDN 交换机 $r$ 的流表容量\\\hline
    $\Lus$ & 感知设备$u$ 到主基站 $s$ 的链路 $\lus$ 构成的集合\\\hline
    $\xu$ & 感知设备$u$是否使用计算任务卸载\\\hline
    $\xus$ & 感知设备$u$的计算任务是否卸载到主基站$s$\\\hline
    $\aus$ & 感知设备$u$的计算任务卸载到主基站$s$的比例\\\hline
    $\xlus$ & $\lus$ 是否作为感知设备$u$ 到主基站 $s$ 的上行链路\\\hline
    $\ylus$ & $\lus$ 是否作为感知设备$u$ 到主基站 $s$ 的下行链路\\\hline
    $\Tlus$ & 链路$\lus$的传输时延\\\hline
    $\Blus$ & 链路 $\lus$的带宽资源\\\hline
    $\xrl$ & 链路 $\lus$ 是否经过SDN 交换机 $r$\\\hline
    $\Bu$ & 感知设备 $u$的上行链路带宽需求\\\hline
    $\Bd$ & 感知设备 $u$的下行链路带宽需求\\\hline
    $\eu$ & 感知设备 $u$ 执行计算任务的能耗开销\\\hline
    $\es$ & 主基站 $s$ 执行计算任务的能耗开销\\\hline
    $\lambdau$ & 感知设备$u$上的计算任务到达率\\\hline
    $\mu_u$ & 感知设备$u$的计算任务处理速度\\\hline
    $\mu_s$ & 主基站$s$的计算任务处理速度\\\hline
    % $\Tus$ & \\\hline
    % $\Tsu$ & \\\hline
    $T_{QoS}$ & 群智感知应用中计算任务远端执行时延阈值 \\\hline
  \end{tabular}
\end{table}

\subsection{边缘网络资源调度模型}
\label{ISPA:Model}

基于假设条件(1),感知数据的计算任务可以在感知设备或主基站上独立执行。
因此,可以使用二进制变量来表示感知设备是否利用主基站来实施计算任务。
对于感知设备 $u$,用 $\xus$ 表示其是否将感知数据计算任务卸载至主基站 $s$。
当$\xus = 1$ 时,感知设备 $u$ 使用主基站 $s$ 处理感知数据;当 $\xus = 0$ 时,感知设备 $u$ 不使用主基站 $s$ 。
另一方面,由于感知设备可以划分任意比例的计算任务至主基站上,因此使用 $\aus$($\aus \in [0,1]$)来代表感知数据计算任务的卸载比例。
当 $\aus = 1$ 时,感知设备$u$的计算任务全部在主基站 $s$ 上执行,当 $\aus = 0$ 时,感知设备 $u$ 不利用主基站 $s$ 处理感知数据。

基于上述定义, $\aus$ 和 $\xus$ 之间的关系可以用式~\eqref{formula_xus}来表达。
该式中,如果 $\aus = 0$,说明感知设备 $u$ 没有将感知数据的计算任务卸载到主基站 $s$,因此 $\xus$ 的值也应该为0。
如果 $\aus \in (0, 1]$,说明感知设备 $u$ 将比例为 $\aus$ 的感知数据计算任务卸载到主基站 $s$,此时 $\xus = 1$,。

\begin{equation}
  \label{formula_xus}
  \begin{gathered}
  \frac{\aus}{A} \leq \xus \leq A \cdot \aus \quad (A \to \infty) \\
  \quad \forall \ u \in \boldsymbol{U}, s \in \boldsymbol{S}, \ 0\leq \aus \leq 1
  \end{gathered}
\end{equation}

对于感知设备 $u$ 而言,令 $\xu$ 表示其是否使用计算卸载。
由于感知设备可以选择多个主基站进行计算任务的卸载,因此只要有任意 $\aus > 0 (s \in \boldsymbol{S})$,则 $\xu = 1$。
为了方便使用 ILP 模型解模方法求解,这里使用式~\eqref{formula_xu} 定义 $\xu$ 的计算方式。

% \vspace{-0.5em}
\begin{equation}
\label{formula_xu}
\begin{gathered}
\frac{1 - \sum\limits_{s \in \boldsymbol{S}}{\aus}}{A} \leq \xu \leq A \cdot (1 - \sum_{s \in \boldsymbol{S}}{\aus}) \quad (A \to \infty) \\
\forall\ u \in \boldsymbol{U}, s \in \boldsymbol{S}, \ 0\leq \sum_{s \in \boldsymbol{S}}{\aus} \leq 1
\end{gathered}
\end{equation}

针对计算时延的计算,这里使用 $\mu$ 表示不同设备的计算能力。
其中,令$\mu_s$表示主基站 $s$ 的计算能力,令$\mu_u$表示感知设备$u$的本地计算能力。
然后令用 $\lambdau$ 表示感知设备 $u$ 上采集感知数据的速率。
根据排队论原理~\cite{Queueing:systems},可以借助 $M/M/1$ 模型,计算感知设备 $u$ 在本地执行计算任务的平均时延$T_{u\_local}$。
由于感知设备 $u$ 到主基站 $s$ 的计算任务卸载比率为 $\aus$,因此本地剩余可执行的计算任务比率为 $1 - \sum_{s \in \boldsymbol{S}}\aus$。
则 $T_{u\_local}$ 的最终计算方法如式~\eqref{formula_Tulacal}。

% \vspace{-0.5em}
\begin{equation}
  \label{formula_Tulacal}
  \Tulocal = \frac{\xu}{\uu - (1 - \sum\limits_{s \in \boldsymbol{S}} \aus) \cdot \lambdau}, \quad \forall \ u \in \boldsymbol{U}
\end{equation}

对于主基站 $s$ 而言,需要处理在不同感知设备的计算卸载任务。
因此,主基站 $s$ 上的任务到达率可以用 $\sum_{u \in \boldsymbol{U}} \aus \lambdau$ 表示。
使用 $M/M/1$ 排队论模型,主基站 $s$ 中执行计算任务的平均时延 $T_{s\_server}$ 可以利用式~\eqref{formula_Tscloud} 计算。
% 和在主基站中执行计算任务的平均时间$T_{s\_server}$,可以借助 $M/M/1$ 模型计算得出。

\begin{equation}
\label{formula_Tscloud}
\Tscloud = \frac{\xus}{\us - \sum\limits_{u \in \boldsymbol{U}} \aus \cdot \lambdau}, \quad \forall \ s \in \boldsymbol{S}
\end{equation}

由于 $\Tscloud$ 是主基站 $s$ 处理计算任务的平均时延,在整体分析时,可以用 $\Tscloud$ 代替式~\eqref{formula_cloudtime} 中的 $T_{us\_server}$。
因此,借助式~\eqref{formula_cloudtime} 、式~\eqref{formula_Tulacal} 和式~\eqref{formula_Tscloud} 可以求解感知设备 $u$ 的感知数据处理时延 $\Tusresponse$。
但是在 $\Tusresponse$ 的计算中,还涉及到链路的传输时延。
基于之前的假设条件(2),基站之间的通信链路都处于稳定状态。
这意味着在感知设备和主基站之间的通信链路不改变的情况下,每条链路的传输时延可视为定值。
因此只要能够确定感知设备 $u$ 和主基站 $s$ 之间的具体通信链路,就能够得知 $\Tsu$ 和 $\Tus$ 的值。

为了精准表示复杂网络中的任一连接感知设备 $u$ 和主基站 $s$ 的具体通信路径,这里使用该路径是否经过指定SDN 交换机的方法进行描述。
令网络中的连接 $\lus$ 表示连接感知设备 $u$ 和主基站 $s$ 的一条链路,$\boldsymbol{R}$ 表示 SDN 交换机构成的集合。
通过二进制变量$\xrl$表示链路$\lus$是否通过 SDN 交换机 $r$,其定义如式~\eqref{formula_xrl} 。
对于任一指定的感知设备 $u$ 和主基站 $s$,可以利用 $\xrl$ 找出唯一的路径,从而获得 $\Tsu$ 和 $\Tus$ 的值,最终计算出感知设备 $u$ 将感知数据计算任务卸载至主基站 $s$ 的总时延。

\begin{equation}
  \label{formula_xrl}
  \begin{gathered}
  \begin{aligned}
  \xrl = \left\{\begin{aligned}
  & 0,\quad  \lus \text{ 不经过 SDN 交换机 }\, r\\
  & 1,\quad  \lus \text{ 经过 SDN 交换机 }\, r\\
  \end{aligned}
  \right.
  \end{aligned}
  \quad \forall \ u \in \boldsymbol{U}, s \in \boldsymbol{S}, \lus \in \Lus\\
  \end{gathered}
\end{equation}

通过分析感知设备的本地感知数据处理时延和计算卸载处理时延,可以帮助单一感知设备选择合适的主基站进行计算任务卸载。
但是对于海量的群智感知设备而言,感知设备的卸载决策会影响链路的带宽、主基站的资源利用率、 SDN 交换机中的流表使用率以及感知设备和主基站上的能耗开销。
因此,基于种种可利用资源的限制,在该模型中提出了相关约束条件。

\subsection{约束条件}
\label{Constraints}

由于感知设备$u$和主基站$s$通信所使用的上行链路和下行链路可能不同,
为了在模型中精确表述感知设备$u$和主基站$s$之间使用的通信链路,
这里使用二进制变量$\xlus$和$\ylus$来表示是否使用链路$\lus$作为感知设备$u$到主基站$s$的上行链路或下行链路。
由于感知设备$u$和主基站$s$通信与否与计算卸载决策$\xus$有关,所以当 $\xus = 0$ 时,感知设备$u$和主基站$s$之间不存在通信链路。
因此,$\xlus$、$\ylus$ 和 $\xus$的关系首先应当满足式~\eqref{formula_xusl}。

\begin{equation}
  \label{formula_xusl}
  \left\{
    \begin{aligned}
    0 \leq \xlus \leq \xus\\
    0 \leq \ylus \leq \xus
    \end{aligned}
  \right.
  \quad \forall \ \lus \in \Lus, u\in \boldsymbol{U}, s\in \boldsymbol{S}\\
\end{equation}

另一方面,当$\xus = 1$时,感知设备$u$和主基站$s$之间一定存在通信链路,并且对于上行链路和下行链路的数量,分别有且仅有一条。因此,$\xlus$、$\ylus$ 和 $\xus$的关系还应当满足式~\eqref{formula_xusl1}。

\begin{equation}
  \label{formula_xusl1}
  \left\{
    \begin{aligned}
    &\sum_{\lus \in \Lus}\xlus = 1\\
    &\sum_{\lus \in \Lus}\ylus = 1
    \end{aligned}
  \right.
  \quad \forall \ u\in \boldsymbol{U}, s\in \boldsymbol{S}
\end{equation}

待链路选定后,可以利用$\Tlus$来表示链路$\lus$的时延。借助$\xlus$和$\ylus$,可以通过式~\eqref{formula_Tus} 计算感知设备$u$和主基站$s$之间上行链路时延$\Tus$和下行链路时延$\Tsu$的数值。

\begin{equation}
  \label{formula_Tus}
  \left\{
    \begin{aligned}
    &\Tus = \xlus \cdot \Tlus\\
    &\Tsu = \ylus \cdot \Tlus
    \end{aligned}
  \right.
  \quad \forall \ \lus \in \Lus, u\in \boldsymbol{U}, s\in \boldsymbol{S}
\end{equation}

\textbf{计算卸载服务质量约束:}
当感知设备发起计算卸载请求时,该过程可以理解为感知设备根据任务卸载比例将感知数据发送至主基站,主基站在计算完成后返回处理结果。
对于感知设备而言,使用计算卸载的目的就是加速感知数据的处理以调高群智感知的效率。
由于群质感知的生命周期有限,计算任务在调度决策的时候,要尽可能减少计算任务在主基站上的执行时间。
所以计算任务的平均执行时间成为计算任务调度决策的重要参考指标。
借助章节~\ref{ISPA:Model} 的模型定义,可以利用式~\eqref{formula_Tusresponse} 精确获得感知设备$u$的计算任务在主基站$s$上的平均执行时间 $\Tusresponse$。

% 尽管任务可以在本地或云中执行,但是用户总是希望他们能够尽快得到反馈,对于任务在哪里执行,用户并不关心。所以,任务在调度时应尽可能让用户的等待时间减少。因此,平均服务时间是一个非常重要的服务质量指标。通过以上工作,可以利用式~\eqref{formula_Tusresponse} 精确地计算平均服务时间。

\begin{equation}
\label{formula_Tusresponse}
\begin{gathered}
\Tusresponse = \xlus \Tlus + \frac{\xus}{\us-\sum\limits_{u \in \boldsymbol{U}}\aus \cdot \lambdau} + \ylus \Tlus,
\quad \forall \ \lus \in \Lus, u\in \boldsymbol{U}, s\in \boldsymbol{S}
\end{gathered}
\end{equation}

由于主基站之间可能存在性能差异,在此将计算任务分配给执行时间最短的主基站。
因此,感知设备$u$的感知数据计算任务平均执行时间必须满足式~\eqref{formula_tscloudconstr} 所示的限制条件。
式中,$T_{QoS}$为群智感知应用中感知数据处理时延阈值。

\begin{equation}
\label{formula_tscloudconstr}
  \left\{
    \begin{aligned}
    % \frac{\xu}{\uu - (1 - \sum\limits_{s \in \boldsymbol{S}} \aus) \cdot \lambdau} \leq T_{QoS},\\
    &\xlus \Tlus + \frac{\xus}{\us-\sum\limits_{u \in \boldsymbol{U}}\aus \cdot \lambdau} + \ylus \Tlus \leq T_{QoS}\\
    &\us-\sum\limits_{u \in \boldsymbol{U}}\aus \cdot \lambdau > 0
    \end{aligned}
  \right.
  \quad \forall \lus \  \in \Lus, u\in \boldsymbol{U}, s \in \boldsymbol{S}
\end{equation}

同样,感知设备在本地处理感知数据的时延也应当不超过群智感知应用中感知数据处理时延阈值,如式~\eqref{formula_tulocalconstr} 所示。

\begin{equation}
\label{formula_tulocalconstr}
  \left\{
    \begin{aligned}
    &\frac{\xu}{\uu - (1 - \sum\limits_{s \in \boldsymbol{S}} \aus) \cdot \lambdau} \leq T_{QoS}\\
    &\uu - (1 - \sum\limits_{s \in \boldsymbol{S}}\aus) \cdot \lambdau > 0
    % \xlus \Tl + \frac{\xus}{\us-\sum\limits_{u \in \boldsymbol{U}}\aus \cdot \lambdau} + \ylus \Tl\\
    \end{aligned}
  \right.
  \quad \forall \ u\in \boldsymbol{U}, s\in \boldsymbol{S}
\end{equation}

\textbf{带宽资源约束:}
与传统网络一样,SDN中的每条链路都有自己的最大带宽限制。
假设感知设备$u$通过链路$\lus$与主基站$s$连接。
由于感知设备数据众多,中间链路存在严重的复用现象。
因此,链路$\lus$的带宽必须满足所有感知设备的通信需求。
假设链路$\lus$的最大带宽是$\Blus$,并且上行链路和下行链路的带宽需求分别为$\Bu$和$\Bd$。那么约束条件可以描述为式~\eqref{formula_Bandwidth}。

\vspace{-1em}
\begin{equation}
% \vspace{-1em}
\label{formula_Bandwidth}
\begin{gathered}
\sum\limits_{u \in \boldsymbol{U}} \sum\limits_{s\in \boldsymbol{S}}\lambdau\aus(\xlus \cdot \Bu + \ylus \cdot \Bd) \leq \Blus, \ 
\forall \ \lus\in \Lus, u\in \boldsymbol{U}, s\in \boldsymbol{S}\\
\end{gathered}
\end{equation}

\textbf{流表容量限制:}
除了带宽限制之外,SDN中的交换机的流表容量也存在上限,即每台 SDN 交换机所能支撑的链路数量存在极限。
在此,令 $C_r$ 表示 SDN 交换机$r$的流表容量。
对于任意一条链路 $\lus$ 而言,只要其经过 SDN 交换机$r$,则交换机$r$使用一条转发规则来记录链路 $\lus$。
因此,对于任意 SDN 交换机,承载的规则数目不可超过其流表容量$C_r$。
利用之前定义的二进制变量 $xlus$ ,二者的关系应当满足如式~\eqref{formula_linkconstr}。

\vspace{-1em}
\begin{equation}
\label{formula_linkconstr}
\begin{gathered}
\sum\limits_{u \in \boldsymbol{U}}\sum\limits_{s \in \boldsymbol{S}}\sum\limits_{\lus \in Lus}(\xlus + \ylus) \cdot \xrl \leq C_r,
\quad  \forall\ u \in \boldsymbol{U}, s \in \boldsymbol{S}, \lus \in \Lus, r\in R\\
\end{gathered}
\end{equation}

\subsection{优化目标}

通过上一小节建立的模型,可以获得感知设备 $u$ 所执行的感知数据计算任务量为 $\lambdau(1-\sum_{s\in \boldsymbol{S}}\aus)$,而主基站 $s$ 上所执行的感知数据计算任务量为 $\sum_{u\in \boldsymbol{U}}\aus\lambdau$。
令感知设备 $u$ 本地执行计算任务的能耗成本为$\eu$,主基站 $s$ 中执行计算任务的能耗成本定义为$\es$。
则感知数据的处理总能耗 $E$ 可以利用式~\eqref{formula_Energy} 来计算。
而本章的最终优化目标,就是在上述约束条件下,令 $E$ 取最小值。

% \vspace{-1em}
\begin{equation}
\label{formula_Energy}
  \begin{gathered}
    E = \sum\limits_{u\in \boldsymbol{U}}(\lambdau(1-\sum\limits_{s\in \boldsymbol{S}}\aus)\eu + \lambdau\sum\limits_{s\in \boldsymbol{S}}\aus\es),
    \quad \forall\ u \in \boldsymbol{U}, s \in \boldsymbol{S}
  \end{gathered}
\end{equation}

但是,约束条件中的式~\eqref{formula_tscloudconstr} 和式~\eqref{formula_tulocalconstr} 并不能构成线性规划问题,因此需要作出适当的转化。
针对式~\eqref{formula_tscloudconstr},定义新的变量 $\beta_{us}$ 和 $\delta_{us}$。
其计算方法如式~\eqref{formula_beta} 和式~\eqref{formula_delta} 。

\vspace{-1em}
\begin{equation}
\label{formula_beta}
  \begin{gathered}
    \beta_{us} = \aus \xlus,
    \quad \forall\ u \in \boldsymbol{U}, s \in \boldsymbol{S}
  \end{gathered}
\end{equation}

\vspace{-1em}
\begin{equation}
  \label{formula_delta}
  \begin{gathered}
    \delta_{us} = \aus \ylus,
    \quad \forall\ u \in \boldsymbol{U}, s \in \boldsymbol{S}
  \end{gathered}
\end{equation}

由于$\aus$ 和 $\xlus$、$\ylus$ 之间存在线性关系,所以对于变量 $\beta_{us}$ 和 $\delta_{us}$,也存在如式~\eqref{formula_beta_linear} 和式~\eqref{formula_delta_linear} 所示的线性关系。

\begin{equation}
  \label{formula_beta_linear}
  \left\{
  \begin{aligned}
    &0 \leq \beta_{us} \leq \aus \\
    &\aus + \xlus -1 \leq \beta_{us} \leq \xlus
  \end{aligned}
  \right.
  \quad \forall\ u \in \boldsymbol{U}, s \in \boldsymbol{S}
\end{equation}

\begin{equation}
  \label{formula_delta_linear}
  \left\{
  \begin{aligned}
    &0 \leq \delta_{us} \leq \aus \\
    &\aus + \ylus -1 \leq \delta_{us} \leq \ylus
  \end{aligned}
  \right.
  \quad \forall\ u \in \boldsymbol{U}, s \in \boldsymbol{S}
\end{equation}

将式~\eqref{formula_beta_linear} 和式~\eqref{formula_delta_linear} 代入式~\eqref{formula_tscloudconstr} 中,可以得到如式~\eqref{formula_cloudtime_linear} 所示的线性约束条件。

\begin{equation}
  \label{formula_cloudtime_linear}
  \begin{gathered}
    \mu_s (\xlus \Tlus + \ylus \Tlus - T_{QoS}) + \sum_{u \in \boldsymbol{U}}\lambdau(T_{QoS} - \beta_{us} - \delta_{us}) + \xus \leq 0,\\
    \quad \forall\ u \in \boldsymbol{U}, s \in \boldsymbol{S}
  \end{gathered}
\end{equation}

针对式~\eqref{formula_tulocalconstr},由于该式中 $\uu - (1 - \sum_{s \in \boldsymbol{S}}\aus) \cdot \lambdau$ 大于0。
因此该式可以在消除分母后转化为式~\eqref{formula_tulocalconstr_linear}。

\begin{equation}
  \label{formula_tulocalconstr_linear}
  \begin{gathered}
    \xu - \uu T_{QoS} + (1 - \sum_{s \in \boldsymbol{S}}\aus) T_{QoS} \leq 0 ,
    \quad \forall\ u \in \boldsymbol{U}, s \in \boldsymbol{S}
  \end{gathered}
\end{equation}

整理上述讨论中的所有约束条件,可以将感知设备的能耗成本转化为含整型线性规划模型求解问题,其描述如式~\eqref{formula_energy} 所示。

\begin{equation}
\label{formula_energy}
\begin{aligned}
\text{求最小值\:}&{:}\ E\\
\text{限制条件\:}&{:}\ 式~\eqref{formula_xusl},式~\eqref{formula_xusl1},式~\eqref{formula_Bandwidth},\\
&\ \ 式~\eqref{formula_linkconstr},式~\eqref{formula_cloudtime_linear},式~\eqref{formula_tulocalconstr_linear}
\end{aligned}
\end{equation}

\subsection{依据权重的链路调度算法}

针对本章提出的 ILP 模型,可以利用数学工具计算出计算任务卸载与网络资源协同调度的最佳决策。
但是 ILP 模型求解的高计算复杂度,让最佳决策的求解时间成本难以接受。
同时,随着网络规模的不断扩大,求解时间也成指数级趋势上升。
为此,本节设计了一种依据权重的链路调度(Link Scheduling with Weight,简称LSW)算法,来避免 ILP 模型求解的繁冗过程,实现快速获取次优协同调度决策的目的。

LSW 算法是一个两阶段启发式算法。首先,该算法确定$\sum_{s \in \boldsymbol{S}}\aus$和$\sum_{u \in \boldsymbol{U}}\aus$,创建针对感知设备$u$和主基站$s$的不同组合,每种组合创建首选链接集合。
然后,该算法针对模型中的约束条件,为感知设备$u$和服务器$s$分配最合适的链路,并调整$\aus$,减少感知设备的能源开销。

\begin{algorithm}[!b]
\setstretch{\algostretch}
\KwIn{$\lambdau$ 感知设备$u$的计算任务到达速率\\$\quad\quad\quad \uu$ 感知设备$u$的计算任务处理速度\\$\quad\quad\quad \us$ 主基站$s$的计算任务处理速度\\$\quad\quad\quad \eu$ 感知设备执行计算任务的能耗成本\\$\quad\quad\quad \es$  主基站执行计算任务的能耗成本}
% \KwIn{$\uu$ : 感知设备$u$的计算任务处理速度}
% \KwIn{$\us$: 主基站$s$的计算任务处理速度}
% \KwIn{$\eu$: 感知设备执行计算任务的能耗成本}
% \KwIn{$\es$ : 主基站执行计算任务的能耗成本}
\KwData{网络结构、每条链路的带宽、时延数据}
\KwOut{$\aus$ 感知设备 $u$ 的计算任务卸载到主基站 $s$ 的比例\\$\quad\quad\quad \au$ 感知设备 $u$ 的计算任务卸载总比例\\$\quad\quad\quad \Lus$ 感知设备 $u$ 到主基站 $s$ 的链路 $\lus$ 构成的集合}
% \KwData{$Occ[1:T]$: 每个程序当前的NCP大小,初始为0,随着缓存路的分配增加,最终为每个程序在整个缓存上的NCP}
\If {$\eu \leq \es$}{
  \For{$u \in \boldsymbol{U}$}{
    $\sum_{s \in \boldsymbol{S}}\aus = max(0, 1 - \frac{\uu}{\lambdau})$
  }
}\Else{
  \For{$s \in \boldsymbol{S}$}{
    $\sum_{u \in \boldsymbol{U}}\aus = min(1, \frac{\us}{\lambdau})$
  }
}
\For{$u \in \boldsymbol{U}$}{
  \For{$s \in \boldsymbol{S}$}{
    为感知设备$u$ 和主基站 $s$ 创建可用链路集合$\Lus$\\
    对链路集合$\Lus$ 中的 链路按时延升序排序
  }
}
$\as = \sum_{s \in \boldsymbol{S}}\aus, \  \au = \sum_{u \in \boldsymbol{U}}\aus$\\
\caption{创建感知设备$u$ 和主基站 $s$ 的可选链路集,并根据计算任务的平均执行时延约束决定卸载比例 $\aus$}
\label{algo_aus}
\end{algorithm}

\textbf{第一阶段 (算法~\ref{algo_aus})}:
在式~\eqref{formula_Energy},$\lambdau$、$\es$ 和 $\eu$ 是常数。
变量 $\aus$ 是确定感知设备总能耗 $E$ 值的唯一变量。
为了降低总能耗开销,必须在 $\es$ 和 $\eu$ 之间调整系数$\aus$。
因此,当 $\es > \eu$ 时,$\aus$ 应当使 $\sum_{s \in \boldsymbol{S}}{\aus}$ 取最大值,当 $\es \leq \eu$ 时,$\aus$ 应当使 $\sum_{s \in \boldsymbol{S}}{\aus}$ 取最小值。
因此,LSW 算法第一阶段的工作是找出$\sum_{s \in \boldsymbol{S}}\aus$的取值范围。

利用约束条件~\eqref{formula_tscloudconstr} 和~\eqref{formula_tulocalconstr},可以得到$\sum_{s \in \boldsymbol{S}}\aus$的范围和$\sum_{u \in \boldsymbol{U}}\aus$的范围:

\vspace{-1em}

\begin{equation*}
\begin{gathered}
\sum_{s \in \boldsymbol{S}}\aus \in [max(0, 1- \frac{\uu}{\lambdau}), 1], \quad \forall\ u \in \boldsymbol{U}, s \in \boldsymbol{S}\\
\sum_{u \in \boldsymbol{U}}\aus \in [0, \frac{\us}{\lambdau}], \quad \forall\ u \in \boldsymbol{U}, s \in \boldsymbol{S}
% \vspace{-1em}
\end{gathered}
\end{equation*}

在确定$\sum_{s \in \boldsymbol{S}}\aus$和$\sum_{u \in \boldsymbol{U}}\aus$的范围后,可以根据约束条件式~\eqref{formula_tulocalconstr} 找出一组从感知设备$u$到主基站$s$的链接集合。
这些信息能够作为算法~\ref{algo_findminpaths} 的输入,帮助感知设备$u$选择连接主基站$s$的最佳链路。

\textbf{第二阶段 (算法~\ref{algo_findminpaths})}:
在算法~\ref{algo_aus}中,可以得到$\sum_{s \in \boldsymbol{S}}\aus$的最优解。
在此阶段中,需要找出所有$\aus$,使得感知设备$u$ 对不同主基站$s$的卸载比例总和逼近最佳值。
同时,基于流表容量、带宽以及时延的限制,利用贪心和动态规划相结合来达到目标。

\begin{algorithm}[!h]
% \vspace{-0.5em}
\setstretch{\algostretch}
\KwIn{$\as$ 算法~\ref{algo_aus} 的输出\\$\quad\quad\quad \au$ : 算法~\ref{algo_aus} 的输出\\$\quad\quad\quad \Lus$: 算法~\ref{algo_aus} 的输出\\$\quad\quad\quad T_{QoS}$: 群智感知应用对感知数据处理时延要求的阈值}
% \KwIn{$\au$ : 算法~\ref{algo_aus} 的输出}
% \KwIn{$\Lus$: 算法~\ref{algo_aus} 的输出}
% \KwIn{$T_{QoS}$: 群智感知应用对感知数据处理时延要求的阈值}
\KwData{ SDN 交换机集合$R$以及每个路由器$r$的流表容量 $C_r$}
\KwOut{计算卸载比例 $\aus$ ;链路选择决策 $\xlus$ 和 $\ylus$}
% \KwData{$Occ[1:T]$: 每个程序当前的NCP大小,初始为0,随着缓存路的分配增加,最终为每个程序在整个缓存上的NCP}
将所有 SDN 交换机的带宽使用率和流表使用率以及$\aus$置 $0$\\
% 将所有 $\aus$ 置 $0$\\
将$\boldsymbol{U}$ 中的元素以 $\lambdau$ 为基准按升序排序\\
将$\boldsymbol{S}$ 中的元素以 $\us$ 为基准按升序排序\\
\For{$u \in \boldsymbol{U}$}{
  \If {$T_{u\_local} < T_{Qos}$}{
    $\aus = 0\  (s \in \boldsymbol{S})$ 若本地执行计算任务效率符合要求,则不进行计算卸载
  }
  \Else{
    \For {$s \in \boldsymbol{S}$}{
      \If {$1 \leq \frac{\lambdau}{\us}$ \textbf{and} $T_{s\_cloud} < T_{QoS}$}{
        若$s$ 的处理$u$的所有数据时延满足要求,则开始选择链路\\
        \For {$l \in \Lus$}{
          从算法~\ref{algo_aus} 得到的集合 $\Lus$ 中遍历链路\\
          \If{若 $l$ 被选中后,所有链路的流表占用率和带宽占用率最大差值不超过$10\%$}{
            $\aus = 1$, \ $u$ 将所有感知数据交给$s$ 处理\\
            $\xlus[u,s,l] = \ylus[u,s,l] =  1$, \ 选定上下行链路\\
            \textbf{next} $u$
          }
        }
      }
      \If {$T_{s\_cloud} < T_{QoS}$ \textbf{and} $\au < 1$}{
        若$s$ 的处理$u$的部分数据时延满足要求,则开始选择链路\\
        \For {$l \in \Lus$}{
          % 从算法~\ref{algo_aus} 得到的集合 $\Lus$ 中遍历链路\\
          \If{若 $l$ 被选中后,所有链路的流表占用率和带宽占用率最大差值不超过$10\%$}{
            $\aus = \frac{\lambdau}{\us}$,\ 确定计算任务卸载的比例\\
            $\xlus[u,s,l] = \ylus[u,s,l] = 1$ ,\ 选定上下行链路\\
          }
        }
      }
    }
  }
}
\caption{根据计算卸载决策为感知设备 $u$ 在约束条件下选择合适的链路}
\label{algo_findminpaths}
% \vspace{-0.5em}
\end{algorithm}

首先,按$\lambdau$的升序对感知设备进行排序,再按$\us$升序对主基站进行排序。
这里的策略是优先从计算任务到达速度较慢的感知设备上卸载任务。
一般来说,主基站的能耗成本和其计算性能是正相关的。
因此,主基站的最佳选择策略是在满足平均计算时延的前提条件下,优先使用能耗最小的主基站。
然后,根据第一阶段得出的链接结合$\Lus$,遵循贪心算法来选择感知设备$u$和主基站$s$之间最快的链接。 
由于带宽资源也存在约束条件,所以最快的链路不一定是最佳选择。
因此,再利用动态规划的思量来平衡边缘网络中各链路的带宽占用。
在这一步中,使用流表容量的使用率和链路带宽占用率作为链路选择的权值。

为了保障链路资源的负载均衡,需要避免高质量链路被提前消耗完的窘境。
一旦高质量的链路被全部占用,受影响的感知设备必须降低计算任务卸载比率,这将导致感知设备的能耗成本增加。
因此算法中规定所有链路的两种占用率中,最大值和最小值相差不得超过10\%。
当某个链路的权值比其他链接的权值高出 10\%时,该链接会从可选择链接中屏蔽。
这种设定可以使各链路的带宽负载以及 SDN 交换机中流表的使用率均匀增加,实现链路调度的负载均衡。



% 链路调度算法旨在使带宽使用率和链路使用率同时增长。 在规则空间和带宽限制下保障用户的QoS而不产生额外开销是链路选择的重要依据。 因此,本文的链路调度策略为每个链路设定权重,并且使链路的权重尽可能相近。 所以,两种资源的占用率的权重比例相同。 其中,所有链路的最大权重差值不能超过10\%。 当某个链接的权重比其他链接的权重高出 10\%时,该链接会从可选择链接中屏蔽。 这个策略是为了防止高 QoS 链接被快速占满。 一旦高质量的链路和服务器被占用完全,其余用户必须降低卸载比率以确保QoS,这将导致能耗成本的增加。

\section{实验结果与分析}

为了评估 LSW 算法的性能,本节将 LSW 调度算法和最短路径优先(Shortest Path First,简称 SPF)算法以及利用 Gurobi~\footnote{https://www.gurobi.com} 求得的最优解调度进行比较。
在实验中,使用了两种不同的网络拓扑结构。
第一种是小型网络拓扑,用于验证 LSW 调度算法的可行性。
第二个网络拓扑是一个有拥有100个SDN交换机的模拟网络,用来对比传统的SPF调度算法和LSW调度算法。

\subsection{算法分析}

第一个小型网络的拓扑结构如图~\ref{fig_smallNetwork} 所示。在此图中,定义了每条链路的时延和带宽。SDN交换机的流表容量依次定义为100,100,80,80,80,100,100。
在该网络中,交换机$0$和交换机$5$是接入点,交换机$1$和交换机$6$连接到不同的主基站。
连接到交换机$6$的主基站比连接到交换机$1$的主基站快10\%。
分布在交换机$0$和交换机$5$之间的感知设备数量从10开始,以10为步进增加至100,且每个感知设备的任务到达率设置为25个任务/秒至50个任务/秒,且均匀分布。
其中,感知设备执行计算任务的能耗成本 $\eu=10$,主基站执行计算任务的能耗成本$\es=100$。

\begin{figure}[!h]
  \centering
  \begin{subfigure}[b]{0.45\linewidth}
    \includegraphics[width=180pt]{./figures/Sec_ISPA/NetworkX_spring_L}
    \label{fig_smallNetworkL}
    \caption{链路时延(秒)}
  \end{subfigure} %
  \begin{subfigure}[b]{0.45\linewidth}    
    \includegraphics[width=180pt]{./figures/Sec_ISPA/NetworkX_spring_B}
    \label{fig_smallNetworkB}
    \caption{链路带宽(兆比特/秒)}
  \end{subfigure} 
  \caption{用于准确性测试的小型网络拓扑}
  \label{fig_smallNetwork}
\end{figure}

利用三种不同的调度方法,首先比较感知设备总能耗的开销。
结果如图~\ref{fig_smallE} 所示,当感知设备数量较少时,由于各类资源占用没有达到阈值,所以链路调度算法不会对感知设备的总能耗产生太大的影响。
随着感知设备数量的增加,其能源成本之间的差距急剧扩大。
当网络中有100个感知设备时,最坏情况下,LSW调度算法产生的感知设备总能耗比最优解(Gurobi计算得出)高12.6 \%;最好情况下,LSW调度算法比SPF调度算法节约26.4\%的感知设备总能耗。
另外,当感知设备达到100时,由于SDN 交换机中的流表容量已经趋于满载,此时可以发现 LSW 调度算法在这种情况情况下存在一定的局限性,导致感知设备总能耗急剧上升。
整体而言,LSW 调度算法所能达到的效果和最优解基本持平,总能耗差异在15\%以内。

\begin{figure}[!h]
  \centering
  \includegraphics[width=310pt]{./figures/Sec_ISPA/small_E}
  \vspace{-1em}
  \caption{不同调度方法对感知设备总能耗的影响}
  % \vspace{-0.5em}
  \label{fig_smallE}
\end{figure}

% 链路调度算法LSW的目标是最小化总能耗$E$。因此,比较了三种方法计算出的不同的$E$值。如图~\ref{fig_smallE} 所示,不同的链路调度策略会影响卸载场景中的总能耗。当用户数量较少时,链路调度算法不会对能耗产生太大影响。随着用户数量的增加,能源成本之间的差距急剧扩大。当网络中有100个用户时,LSW方法的总能耗比最优解(Gurobi计算得出)高12.6 \%,比SPF方法的最优解高36.4\%。同时,100名用户使SDN网络基本上处于满负荷状态。因此,当SDN网络的负载接近满负荷时,LSW方法此时存在局限性。从这个图中,当SDN网络负载不满时,LSW方法和最优解之间的差异可以减少到小于10\%。

图~\ref{fig_smallAu} 描述的是同一网络拓扑下,不同调度方法对感知设备平均计算任务卸载比率的影响。
图中,感知设备数量从20到50区间,由于感知设备任务到达率以及计算任务处理速度设置的差异,产生了计算任务平均卸载比率上扬的趋势。
但就整体而言,随着感知设备的增加,需要使用的网络资源逐渐增多,当网络资源产生竞争时,必然影响计算任务卸载的决策,使任务卸载的比率随着用户数量的增加而减少。
图中,由于 SPF 调度算法优先选择通信时延最小的链路。
当优势链路占用过多时,剩余链路资源因无法满足其他约束条件,导致计算任务的平均卸载比例大幅下降,也从侧面反映出图~\ref{fig_smallE} 中,SPF 算法下感知设备总能耗急剧上升的原因。
在该图中,从侧面反映出LSW链路调度方法维持较高的计算资源平均卸载比率。
虽然不及最优解给出的调度策略,但基本趋势维持一致。

% \vspace{-0.5em}
\begin{figure}[!h]
  \centering
  \includegraphics[width=310pt]{./figures/Sec_ISPA/small_Au}
  % \vspace{-0.5em}
  \caption{不同调度方法对平均计算任务卸载比率的影响}
  % \vspace{-0.5em}
  \label{fig_smallAu}
\end{figure}
\vspace{-0.5em}


\begin{figure}[!h]
  \centering
  \begin{subfigure}[h]{0.99\linewidth}
    \centering
    \includegraphics[width=415pt]{./figures/Sec_ISPA/small_rule_grb.pdf}
    \label{fig_smallNetworkRG}
    \vspace{-0.5em}
    \caption{使用 Gurobi 最优解时流表使用率}
  \end{subfigure}
  % \vspace{1em}
  \begin{subfigure}[h]{0.99\linewidth}
    \centering
    \includegraphics[width=415pt]{./figures/Sec_ISPA/small_rule_lsw.pdf}
    \label{fig_smallNetworkRL}
    \vspace{-0.5em}
    \caption{LSW 调度算法下的流表使用率}
  \end{subfigure}
  \begin{subfigure}[h]{0.99\linewidth}
    \centering
    \includegraphics[width=415pt]{./figures/Sec_ISPA/small_rule_spf.pdf}
    \label{fig_smallNetworkRS}
    \vspace{-0.5em}
    \caption{SPF 调度算法下的流表使用率}
  \end{subfigure}
  \vspace{-0.5em}
  \caption{不同调度算法对流表使用率的影响}
% \vspace{-1.2em}
\label{fig_smallrule}
\end{figure}
% \vspace{-1em}

图~\ref{fig_smallrule} 显示了每个交换机中流表容量的占用情况。
对于最优解(Gurobi)和 SPF 调度算法,不同感知设备总数和流表容量占用率并没有直接关系。
在 LSW 调度算法中,有意保持每条链路使用率的平衡性。
因此,随着感知设备的增加,SDN 交换机中的流表占用增长趋势相对稳定。
所以,在SDN网络资源开销增加时,仍然留有高质量的链路预留给新加入的感知设备使用。
在SPF调度算法中,可以观察到 SDN 交换机的流表使用率存在着波动性变化。
但是对于优质链路的完全占用,会在新加入感知设备时产生较多的链路调度变化,导致额外的时间开销。
% 这意味着为了满足更多的用户,链路调度的次数会有很大的变化。这也表明SDN网络中的交换机可能会频繁地改变其状态,这将导致额外的能耗成本。

\subsection{调度效果测试}

为了验证LSW算法的普适性,
除了与 SPF 算法比较之外,本节还将 LSW 算法与文献\citen{DBLP:journals/tii/NayakDR18}中基于最小链路数量的调度算法(Mini-Max,简称 MM)、以及文献\citen{DBLP:journals/wicomm/ShenYPFD18}中提出的动态融合链路调度算法(Hybrid Routing Fusion Algorithm,简称 HRFA)进行比较。
为了比较上述四种算法,本节设计了一个拥有100个 SDN 交换机的网络拓扑。
该网络中有30个交换机作为接入点为感知设备提供网络服务。
同时,有40个 SDN 交换机分别连接40个主基站。
其中,感知设备的数量从200到3000依次增加。
这个网络中每个交换机的规则空间是2000。
在该SDN网络设置中,有30\%的链路的数据传输时延不能满足群智感知对感知数据传输的时延要求,由此来模拟真实网络中的不稳定因素。

\begin{figure}[!b]
  \centering
  \includegraphics[width=290pt]{./figures/Sec_ISPA/large_Au}
  % \vspace{-1em}
  \vspace{-0.5em}
  \caption{四种调度算法对平均任务卸载效率的影响}
  \vspace{-1em}
  \label{fig_largeAu}
\end{figure}

图~\ref{fig_largeAu} 显示了四种调度算法对平均任务卸载比率的影响。
可以看出,由于 SPF 算法优先使用高质量的的链路,所以随着感知设备总数的增加,可以满足计算卸载时延的链路越来越少,导致任务卸载平均比例持续下降。
同样,HRFA 调度算法利用链路分析,会在临近的 SDN 交换机中保持数据流数量上的平衡,因此只有当感知设备数量达到一定数量时,才会大幅影响任务卸载平均比例。
对于 MM 调度算法,由于采用了使网络中整体链路数最小化的策略,因此当感知设备数量较少时,其收益与本章 LSW 调度方法相近。
当设备数量达到2000时,由于 MM 调度算法过度追求链路数量的最小化,使得平均任务卸载比率大幅度降低。
这也使得感知数据处理的总时延更长,对于 MM 调度方法而言,部分感知设备的数据处理时延已经大于限定的阈值。
同时当用户从800增加到1500时,两种调度算法的平均任务卸载比率都出现大幅下滑。
根据实验过程日志分析,由于实验网络中存在时延过高的链路,导致任务卸载决策时为避开这些链路而减小了计算任务的卸载比例。
在该测试中,四种调度算法都没有很好的避开高时延链路带来的影响。
但是,相较于其他三种调度算法,LSW算法的抗干扰能力要更好一些。

% 当用户从800增加到1500时,两种调度算法的平均任务卸载比率都出现大幅下滑。
% 根据实验过程日志分析,是由于实验网络中人为设置了时延过高的链路,导致任务卸载决策时为避开这些链路而减小了计算任务的卸载比例。
% 在该测试中,两种调度算法都没有很好的避开高时延链路带来的影响。
% 但是,相较于 SPF 调度算法,LSW算法的抗干扰能力要更好一些。

图~\ref{fig_largeE} 显示了四种调度算法对感知设备总能耗的影响。
对比四种算法,感知数据处理需要的能耗基本与感知设备的数量呈线性关系,并没有出现大幅的增长趋势。
但是当感知设备数量超过 1500 时,可以看出 LSW 调度算法明显优于其它三种调度方法,MM 调度算法次之。
当网络中包含3000个感知设备时,LSW调度算法能够比SPF算法节省 31.9\% 的能耗。
但是与 MM 算法相比较,当设备数量为2500和3000时,利用 LSW 调度方法处理感知数据反而需要更多的能耗。
这是由于 MM 调度方法减少了计算卸载技术的使用频率,减少了服务器的能耗开销所导致的。
但是此时,部分感知设备的数据处理时延已经大于限定的阈值,没有达到本节所提到的对于感知数据处理时延的限制。

\begin{figure}[!h]
  \centering
  \includegraphics[width=290pt]{./figures/Sec_ISPA/large_E}
  % \vspace{-1em}
  \vspace{-0.5em}
  \caption{四种调度算法对能量总开销的影响}
  \label{fig_largeE}
  \vspace{-1em}
\end{figure}

图~\ref{fig_largeL} 显示了四种调度算法对边缘网络中链接使用数量的影响。
由于 MM 调度算法的宗旨就是追求网络中的链路数量最小,因此在该测试中,MM 调度算法的链路使用总数都相对较小。
而 SPF 算法由于优先使用时延最短的路径,因此当感知设备数量不断增加时,低时延的路径可能因为流表容量或者带宽限制而无法被重复使用。
因此当设备数量增加时,SPF 算法不得不启用新的链路来支撑大量的数据流,导致链路使用总数最多。
而 HRFA 调度算法和 LSW 调度算法虽然都考虑了 SDN 路由器的负载,但是由于 HRFA 调度算法只考虑了相邻 SDN 路由器的负载,因此整体效果上不如 LSW 调度算法。
% 当参与群智感知的感知设备数量相同时,LSW方法中使用的链路数比SPF算法中的链路数少10\%。
对比 MM 调度算法,虽然其链路数量使用最少,但是当设备数量为2500和3000时,部分感知设备无法满足感知数据处理的时延阈值;并且当设备数量不多于2000时,MM 调度方法需要的数据处理总能耗比 LSW 调度方法要多。
因此,本章所提出的 LSW 调度算法能够更全面的处理大规模群智感知场景下的感知数据计算卸载分配、网络资源、以及能耗之间的关系,且效果要优于 HRFA 调度算法。

\begin{figure}[!h]
  \centering
  \includegraphics[width=290pt]{./figures/Sec_ISPA/large_link}
  % \vspace{-1em}
  \vspace{-0.5em}
  \caption{四种调度算法对网络链路数量的影响}
  \vspace{-2em}
  \label{fig_largeL}
\end{figure}

\section{本章小结}

本章在 SDN 管理的边缘网络,针对群智感知中感知数据计算任务卸载决策,围绕网络资源限制和感知设备的能耗,建立了量化分析模型。
为了深入理解网络资源限制和感知设备的能耗之间的关系,将该模型表述为 ILP 模型,以减少感知设备执行感知数据计算任务的总能耗。
通过对 ILP 模型的分析,本章设计了一种计算任务卸载和网络资源协同调度算法(LSW 调度算法)。
该算法综合考虑链路的权重以及计算任务在感知设备与主基站上的执行速度,避免了 ILP 模型求解的高计算复杂度,实现了接近最优解的计算任务卸载决策以及通信链路的调度。
通过对比试验,证明了算法的有效性。
通过和其它算法的对比实验,揭示了 LSW 调度算法考虑 SDN 路由器流表容量的必要和优势,同时保障的感知数据的处理效率。

% 建立了一个SDN网络公式模型,以了解SDN中总能量成本和限制之间的关系。为了最小化总能耗,将这个问题表述为线性规划问题。在对模型进行研究之后,提出了一种链路调度算法,根据每个链路的权重来选择最合适的链路。权重由交换机中规则空间的使用和链路中带宽的使用组成。通过大量实验,证明了算法的有效性。通过对比实验,该算法比SPF算法显示出显著的优势,极大地提高了最优解的性能。




\chapter{面向用户移动特征的边缘服务调度}

% 摘要
随着智慧城市研究的深入,需要移动群智感知对城市中大量物理对象进行数字信息提取以提供数据支撑。
然而,城市中的大量的移动设备通常跟随居民日常生活轨迹而移动,因此移动群智感知无法像无线传感网络进行简单的区域划分来完成感知任务。
因此,研究者通常基于人类社交网络对感知任务进行部署,但对于感知数据的收集,并无过多讨论。
本章结合城市公交网络和城市居民的出行轨迹,利用 D2D 通信和边缘服务,设计了城市范围内低成本、高效率的感知数据收集模型。
基于此混合整数线性规划(Mixed Integer Linear Programming,简称 MILP)模型,提出了使感知成本最小化的边缘服务部署算法。
通过仿真测试,该算法得出部署成本与最优部署成本接近,收益远远高于基于乘客流量和边缘服务连接度的部署方法。

\section{研究背景}

对于智慧城市应用,需要借助移动群智感知在全市范围内收集感知数据。
在面向城市范围的移动群智感知研究中,大多数研究者更倾向于提高感知任务的部署效率。
例如使用激励机制吸引用户主动参与感知任务~\cite{CNKI:JiaChaopeng, CNKI:WuMCSIncentive, jiyubianyuanjisuangongyecaiji},或者利用用户的社交网络将感知任务快速分发给大量的移动设备~\cite{Cnki:Yu2018}。
文献~\cite{DBLP:journals/tii/TangCHPWHY17}还创建一个多层级的分布式边缘网络体系结构来执行数据表示和特征提取,并研究了城市范围内实施群智感知的可行性。
这些研究,从效率、能耗、成本等不同角度,对感知任务的部署提出了宝贵的优化意见。
然而,对于感知数据的收集工作,这些研究默认移动设备使用无线网络或者蜂窝网络直接进行感知数据交付。
此时,大量的移动设备直接将感知数据传回云端,不仅城域网会面对突发的网络负载,主干网也会因为海量的感知数据传输而发生阻塞。
Wang 等人~\cite{DBLP:conf/huc/WangZX13}利用 Bluetooth、WiFi 网关进行感知数据的上传,验证了延迟容忍的通信方式可以显著减少感知数据输出时的能耗和成本。
Karaliopoulos 等人~\cite{DBLP:conf/infocom/KaraliopoulosTK15} 在感知数据收集过程利用 D2D 通信,在感知任务的参与者中利用贪心算法选择小部分设备作为转发节点。但是这种决策方法的时间复杂度过高,并不适用于大规模的移动群智感知应用。
Wang 等人~\cite{DBLP:journals/puc/WangLL17}在上述工作中虽然做出了改进并提高了算法决策效率,但是需要收集大量的用户移动轨迹,基于用户的移动行为特征选择能够胜任 D2D 转发工作的移动设备来收集感知数据。
这些工作证明了 D2D 通信能够有效帮助移动群智感知应用减少数据收集时的能耗和成本,但是面对城市级的感知应用,仅仅依靠 D2D 通信和用户的移动行为特征还远远不够。

另一方面,利用边缘计算对群智感知的覆盖区域进行划分可以缓解这一问题,但是面对城市覆盖范围,需要部署的边缘服务数量过多也会导致成本的激增。
在不少智慧城市研究中,利用居民流向,建立了城市中吸引用户的兴趣点(Points of Interest,简称 POI)。
这些兴趣点在固定时间范围内,会产生大量的人流,因此居民的随身设备也会在此聚集、经过。
文献~\cite{DBLP:journals/iotj/ZhanXZW18}基于 POI 部署边缘服务进行感知数据的收集,发现在人流量高峰期时,可以达到不错的感知数据收集率。
利用不同种类的 POI,可以构建出不同种类的城市居民移动模型~\cite{CNKI:XiongOMCS}。
不同的城市居民移动模型,对移动群智感知的感知质量,存在着较大的影响。
例如基于游乐场、公园的移动模型,虽然 POI 上有很大的城市居民流量,但是对于单个居民而言,经过此类 POI 的规律并没有固定模式。
利用这一类移动模型,无法保障感知数据分布范围的稳定。
对于基于办公室、学校的移动模型,此类 POI 的城市居民个体行为相对稳定。
这也意味利用这些 POI 可以收集到的感知数据覆盖范围固定。
另一方面,由于此类 POI 容纳居民数量的限制,能够收集到的感知数量受居民数量影响而存在上限。

为了保障群智感知应用在城市范围内感知数据的空间分布稳定,并尽可能收集更多的感知数据,本章提出一种基于用户移动特征的边缘服务部署策略。
在城市中,绝大多数居民通常利用公共交通系统出行,例如公共汽车、地铁。
在工作日,居民出行的出发地与目的地相对固定,这种出行行为已经成为一种稳定的城市移动特征。
基于这一特征,可以利用公共交通站点,部署边缘服务,收集大量且分布稳定的感知数据。
同时,由于公共交通工具会在到站时有短暂的停靠时间,大量乘客在站点聚集,因此可以借助 D2D 通信,将感知数据直接交付给部署在公共交通站点上的边缘服务器。
采用这种做法,不仅可以减少蜂窝网络的使用,降低感知数据传输成本,还能避免主干网络资源的占用。
% 『因此,部分研究者提倡将基于交通网络构建城市居民的道路移动模型,作为在城市感知中部署边缘服务的依据,把感知数据的采集和处理服务迁移到合适的边缘服务器上~\cite{DBLP:journals/iotj/ZhanXZW18}。』


\begin{figure}[!h]
\centering
% \vspace{-1.5em}
\includegraphics[width=210pt]{figures/Sec_MONET/CPS.eps}
\vspace{-0.5em}
\caption{基于城市居民移动特征的城市感知\textbf{『需要重画』}}
% \vspace{-1em}
\label{Figure_edge}
\end{figure}

图~\ref{Figure_edge} 描述了基于公交网络的感知数据收集场景。
在城市中,部署有大量的传感器,城市居民的移动设备,可以作为这为传感器感知数据的携带者。
居民在使用公共交通出行时,会使用步行的方式或者自行车等交通工具移动到最近的公共交通站点。
此时,可以借助 D2D 通信收集收集途径传感器上的感知数据。
因此,同一路径上的居民可以收集同一区域不同时间点上的感知数据。
当居民进入公交站点,如果该站点已经部署有相应的边缘服务,则居民在等车的时候可以将感知数据上传到当前站点中的边缘服务器。
如果该站点没有部署边缘服务,居民作为公交乘客,当公交车在部署有边缘服务的公交车站停靠时,利用乘客上下车的时间空隙,通过 D2D 通信的方式将感知数据交付给边缘服务器。
对于沿途都没有经过部署有边缘服务站点的乘客,在群智感知应用生命周期的末期,再使用蜂窝网络上传感知数据。
基于这种感知数据上传策略,可以保障 D2D 通信的优先使用,以减少群智感知数据上传到成本。

基于上述感知数据收集方案,本章重点考虑生命周期有限的群智感知应用在城市感知范围内的边缘服务部署决策。
但是,对于城市中的公交路线,有大量的重复路径和交叉路径,因此 D2D 通信无法避免感知数据的重复上传。
另一方面,城市公交网络中,可以部署边缘服务的 POI 众多,在重复路径上部署边缘服务,反而导致成本的浪费。
因此,在该网络中部署边缘服务,需要针对居民的移动特征、重复路径、交叉路径做出合理的部署决策,以平衡蜂窝通信成本和边缘服务部署成本。
本章的主要贡献总结如下:

1)基于整个城市中公共交通乘客的流动模式,研究了低成本、高效益的边缘服务部署问题。

2)针对该问题建立了 MILP 模型,研究并解决了交叉路径下感知数据重复收集的问题。

3)针对 MILP 模型求解的高计算复杂度,提出了高性能的动态规划算法来求解边缘服务的部署决策。

实验结果表明,该算法能够很好地逼近最优解,优于其他竞争对手。

% 本文的其余部分组织如下。第2节介绍了边缘服务放置的相关工作。在第三节中介绍了系统模型。在第四节中,给出了成本最小化问题的MILP公式,并在第五节中提出了的启发式算法。第6节报告了绩效评估结果。最后,第7节结束了这项工作。


\section{问题描述}
\label{Sec_Monet_question}

本节针对基于公交网络的感知数据收集过程建立初步模型,并详细阐述了公交网络中重复路径和交叉路径对感知数据收集的影响。

\subsection{系统模型}

为了在城市范围内执行感知操作,部署传感网络灵活性低、改造成本高。
而使用蜂窝网络发送群智感知任务和感知数据,不仅提高了感知设备的门槛要求,也无形增加了感知数据传输的成本。
但是另一方面,城市中蜂窝基站的网络信号基本实现了城市范围的无死角覆盖。
为了让群智感知任务更好地部署在城市区域内,移动群智感知通常借助社交网络、社会网络等辅助感知任务的传播,将感知任务代码嵌入在用户社交信息中,随着社交活动的进行将感知任务扩散开。
但是,在感知数据的收集过程中,并没有研究人员借助社交网络或者社会网络来辅助数据回收,而是直接使用蜂窝基站作为了感知数据回收的代理节点。

在大城市中,为了方便居民的出行,公共交通网络的建设已经满足人口密集区域的全面覆盖。
并且随着城市规模的扩大,公共交通网络也会根据城市发展不断完善以覆盖新的区域。
而公共交通工具和公共交通站点,则是城市居民聚集的载体。
无论是乘客在公共交通站点等待车辆、还是公共交通工具在公共交通站点停靠,都会有若干分钟以上的时间将大量的居民聚集在小片区域内,足够使用 D2D 通信完成感知数据的交互。
因此,借助公共交通网络,可以利用乘客在交通工具中聚集时使用 D2D 通信收集乘客移动设备中的感知数据,从而避免对蜂窝网络通信的需求。
而城市居民的分布以及出行规律在工作日期间具备一定的稳定性,因此利用公共交通网络收集的感知数据也能保障其分布特征。
% 对于城市范围覆盖的移动群智感知应用,利用乘客的随身移动设备,不仅可以将感知任务部署到城市的边缘地带,也可以在乘客在交通工具中聚集时使用 D2D 通信将感知任务扩散。
因此,本章将公共交通站点作为 POI,通过在公共交通站点中部署边缘服务来收集感知数据、实施感知数据的预处理工作。

% 在本文中,考虑了公交车乘客参与拥挤的情况,他们可能会被转移到城市的不同地方。在基于公共汽车乘客的人群中,乘客负责收集从家里到公共汽车站的道路上的监控数据。不对传感数据采集施加任何限制。乘客可以通过从道路上专门部署的传感器收集感测数据或者通过嵌入智能手机中的传感器生成数据来获取数据。为了减轻基于基础设施的通信(例如蜂窝通信)的负担,提倡参与者尽可能使用设备对设备( D2D )通信。因此,D2D通信可能发生在乘客的智能手机和传感阶段专门部署的传感器之间。为了便于处理,假设乘客在感知阶段可以获得相同的数据量,用$s$表示。很容易将其扩展到不同乘客可以获取需要上传处理的大量数据的情况。


图~\ref{Figure_Wuhan_Buses} 所展示的是武汉市的公共交通网络,在这一交通网络中,日均乘客流量已经突破700万人次\footnote{武汉交通运输局 http://jyw.wuhan.gov.cn/Item/10002123.aspx},公共交通站点超过2000个。
在如此庞大的交通网络中,如果每个站点都部署边缘服务来帮助群智感知的数据收集,其成本是难以想象的。
另一方面,不同公共交通站点的乘客流量可能存在较大的差异,因此在不同站点部署边缘服务的收益回报也不尽相同。
因此,需要筛选出合适的公共交通站点来部署边缘服务,用较少的边缘服务部署成本,尽可能利用庞大的乘客流量来收集感知数据。

\begin{figure}[!h]
  \centering
  % \vspace{-1.5em}
  \includegraphics[width=220pt]{figures/Sec_MONET/wuhan_bus.png}
  \vspace{-0.5em}
  \caption{武汉市交通网络}
  % \vspace{-1em}
  \label{Figure_Wuhan_Buses}
\end{figure}

为了更好地理解乘客在公共交通中的换成行为,在此将公共交通网络转化为有向图 $\boldsymbol{G}=<\boldsymbol{V}, \boldsymbol{E}>$。
在有向图 $\boldsymbol{G}$ 中,顶点 $v$($v \in \boldsymbol{V}$) 表示公共交通站点,$\boldsymbol{E}$ 表示公交路线集合。
由于城市交通行为特征具有一定的稳定性~\cite{TAO201490},尤其是在工作日、普通节假日时,乘客出行行为模式基本固定。
以工作日的城市交通行为特征为例,可以根据统计数据估计公交车站 $v$ 上的乘客到达率和离开率,分别表示为 $\lambda_v$ 和 $d_v$。
在没有特殊事件发生时, $\lambda_v$ 和 $d_v$ 通常趋于稳定。
假设站点 $v$ 部署了边缘服务进行感知数据的收集工作,在 $v$ 等待上车的乘客、途经的乘客(包括换成)、下车的乘客(已经到达终点,不再乘坐交通工具)均在站点有短暂的逗留时间,可以利用 D2D 通信收集这些常客移动设备里的感知数据。
因此,在站点 $v$ 部署边缘服务的收益,不仅和 $\lambda_v$ 、 $d_v$ 相关、也和途经站点 $v$ 的乘客流量相关。
但是,途经 $v$ 的乘客和在 $v$ 下车的乘客可能在之前的旅途中上传了感知数据,因此 $d_v$ 和途经站点 $v$ 的乘客流量并无法直接说明在站点 $v$ 部署边缘服务的收益。
这也是基于公交网络部署边缘服务时,边缘服务之间相互的影响关系。

% 由于在公共交通系统中,乘客可以被视为基于预定交通轨迹在站点之间移动,所以对于不同的两个车站 $u$ 和 $v$ 而言,从车站 $u$ 上车的乘客到车站 $v$ 的到达率用 $\lambda_{uv}$ 表示。

% 对于每一个公交车站 $v$ 而言,有三类乘客:等待上车的乘客、途经的乘客(包括换成)、下车的乘客(已经到达终点,不在乘坐交通工具)。
% 假设在 $v$ 部署边缘服务进行感知数据的收集工作,这三类乘客均在站点有短暂的逗留时间,则可以利用 D2D 通信收集这些常客移动设备里的感知数据。
% 若 $v$ 没有部署边缘服务,则等待上车的乘客只能在上车后,途径其它部署有边缘服务的站点时利用 D2D 通信上传感知数据。
% 对于途经站点 $v$ 的乘客,若之前没有途经部署有边缘服务的

% 注意,由于考虑了公交车线路规划和交通条件,在实际中$t_{uv}$不一定等于 $t_{vu}$。

\subsection{公交网络中边缘服务的相互影响关系}

\begin{figure}[!b]
  \centering
  % \vspace{-1em}
  \includegraphics[width=210pt]{figures/Sec_MONET/Puzzle.eps}
  % \vspace{-0.5em}
  \caption{公交统计信息示例}
  % \vspace{-1em}
  \label{Figure_puzzle}
\end{figure}

图~\ref{Figure_puzzle} 是一个简单的公交统计信息示例,用来说明公交网络中边缘服务的相互影响关系。
图中共有 5 个公交站点,每个公交站点均有上车人数和下车人数的统计信息。
同时,站与站的连接路径上也标明了改路径运输的乘客数量。
在该示例中,一共有2100名乘客参与了交通行为。
假设此时只选取一个站点部署边缘服务,显然只考虑乘客流量最多的1号站点,此时可以收集到来自1300名乘客的感知数据。
但是,如果可以选在两个站点部署边缘服务收集感知数据,该示例下最佳的选择方案是站点2和站点3。
此时,可以收集1900名乘客的感知数据,而站点2和站点1的部署方案,最多只能收集1800名乘客的感知数据,这些数据还是基于站点3到站点1的乘客中不包括来自站点2的乘客的前提。
可以看出,当站点2部署有边缘服务时,站点1作为其后序节点,其部署边缘服务的收益显著降低,而站点2到站点3的乘客较少,使此时站点3部署边缘服务的收益高于站点1。
当可以部署的边缘服务增加到三个时,该示例中最佳的部署方案又变成了站点5、站点3、和站点1的组合。
此时,2100名乘客的感知数据可以被全部收集完。

通过这个例子,可以发现当使用公交网络收集感知数据时,不同部署方案的收益各不相同。
并且,一个公交站点部署边缘服务的决策,会直接影响其后续可连通的站点的部署收益。
而这一影响的根本原因,就是乘客的路径上有多个车站部署了边缘服务,而当其经过第一个部署有边缘服务的车站时,感知数据已经上传。
在其后续旅途中,该乘客并不含有感知数据,对于后续部署有边缘服务的节点而言,此时的乘客流量并无法表示感知数据的可收集数量。
因此,当使用公交网络作为载体收集感知数据时,需要面将乘客移动特征建立边缘服务部署策略。

\section{面向乘客移动特征的边缘服务部署模型与算法}

上一小节中使用了 $\boldsymbol{G}=<\boldsymbol{V}, \boldsymbol{E}>$ 将公交网络抽象为有向图。为了在有向图 $\boldsymbol{G}$ 中选择合适的站点 $v$ 部署边缘服务来收集感知数据,需要根据乘客的轨迹来确定站点 $v$ 部署边缘服务的收益。

\subsection{面向乘客移动特征的边缘服务部署模型}

对于群智感知应用而言,一般都存在生命周期,既对感知数据的产生时间有范围要求,过早或太晚的数据,被认为是没有意义的感知数据。
因此,边缘服务的使用仅在群智感知应用的生命周期内。
由于感知质量的要求,群智感知应用需要尽可能地收集所有的感知数据。
基于这一初衷,本节假设在群智感知应用的生命周期内,边缘服务能够帮助群智感知利用 D2D 通信收集大部分的感知数据,剩余小部分的感知数据,则直接通过蜂窝网络上传。
这种策略既可以保障感知数据收集齐全,也能利用 D2D 通信减少感知数据的传输成本。
为了最大化利用 D2D 通信,意味着边缘服务的部署决策必须带来最多的收益。

\textbf{(1) 部署成本}

在此,利用二进制变量 $x_v$ 表示站点 $v$ 上是否部署有边缘服务,其定义如式~\eqref{eq:xdefine}。

\begin{equation}
\label{eq:xdefine}
x_v =\left\{
\begin{aligned}
&1, \; 在 v 上部署边缘服务 (v\in \boldsymbol{V})\\
&0, \; 未在 v 上部署边缘服务 (v\in \boldsymbol{V})
\end{aligned}
\right.
\end{equation}

假设部署一个边缘服务的单位时间成本固定为 $c_v$,边缘服务部署总成本为$C_v$,在群智感知应用的生命周期 $T$ 内,其关系如式~\eqref{eq:Cvcv} 所表达。

\begin{equation}
\label{eq:Cvcv}
C_v = \sum_{v\in \boldsymbol{V}}x_v c_v T
\end{equation}

\textbf{(2) 前置站点部署边缘服务的影响}

为了获取站点 $v$ 部署边缘服务的可得收益,必须针对经过该站点的乘客路径考虑前置站点部署有边缘服务时所带来的影响。
假设站点 $u$ 是站点 $v$ 的前置站点,将在站点 $u$ 上车、在站点 $v$ 下车的乘客的人数记为$d^u_v$,且每名乘客的移动设备上含有一份感知数据,为简化处理,大小统一记为 $s$。
当这些乘客从站点 $u$ 到达站点 $v$ 时,这些乘客所携带的感知数据总大小用 $f^u_v$ 表示,由于每份感知数据的大小为 $s$。
则 $f^u_v$ 和 $d^u_v$ 的关系可以用式~\eqref{eq:FandD}表示。
当站点 $u$ 到 站点 $v$ 之间没有部署任何边缘服务时,表达式取等。
另一方面,站点 $u$ 到站点 $v$ 可能存在多条路径,如果只是其中一条路径上有站点部署有边缘服务,只会上传部分乘客的感知数据,导致 $f^u_v$ 小于 $d^u_v s$。

\begin{equation}
\label{eq:FandD}
f^u_v \leq d^u_v s, \forall u, v\in \boldsymbol{V} (u \neq v)
\end{equation}

显然,式~\eqref{eq:FandD}并没有深入考虑站点 $u$ 到站点 $v$ 的多路径问题。
在此,假定站点 $u$ 和站点 $v$ 之间存在可连通的站点 $w$。
$\lambda^u_{wv}$ 表示站点 $u$ 的乘客经由站点 $w$ 到达站点 $v$ 的乘客数量。
同理,以 $f^u_{wv}$ 表示这些乘客在到达站点 $v$ 时的感知数据大小,其关系如式~\eqref{eq:FandLambda}。

\begin{equation}
\label{eq:FandLambda}
f^u_{wv} \leq \lambda^u_{wv} s, \forall u \in \boldsymbol{V}, v \in \boldsymbol{V}, w \in \boldsymbol{V}, u \neq v
\end{equation}

以站点 $u$ 作为起始站,对于站点 $v$ 而言,是否部署边缘服务会直接影响其后续站点的可收集感知数据数量。
当 $x_v =1$ 时,站点 $v$ 部署有边缘服务利用 D2D 通信吸收了途经的所有感知数据。
此时,以站点 $v$ 作为出发点,其后续站点来自站点 $v$ 的感知数据大小都为 0。
当 $x_v =0$ 时,进入站点 $v$ 的感知数据,被分成两个流向,一部分被在站点 $v$ 下车的乘客带走,另一部分传递给后续的站点。
式~\eqref{eq:Vimpact}描述了站点 $v$ 是否部署边缘服务的影响关系。

\begin{equation}
  \label{eq:Vimpact}
  %	d^u_v s (1-x_v)
    \sum_{x\in \boldsymbol{V}}f^u_{xv} (1-x_v) = f^u_v + \sum_{w\in \boldsymbol{V}}f^u_{vw}, \forall u\in \boldsymbol{V}, v\in \boldsymbol{V}, u \neq v
\end{equation}

同样,对于任何站点 $u$ 而言,都有乘客上车,因此是否部署边缘服务 $x_u$ 和它对后续站点的影响应当满足式~\eqref{eq:startU}。
式中,$\lambda_u$ 表示在站点 $u$ 上车的乘客数量。
当 $x_u = 1$ 时,站点 $u$ 利用边缘服务吸收了所以在该站上车的乘客所携带的感知数据,因此后续感知数据流量为0。

\begin{equation}
\label{eq:startU}
\lambda_u s (1 - x_u) = \sum_{w\in \boldsymbol{V}}f^u_{uw}, \forall u\in \boldsymbol{V}, u \neq v
\end{equation}

\textbf{(3) 成本和收益}

通过对前置站点部署边缘服务的建模与分析,可以得出边缘服务部署决策 $x_v$ 和使用 D2D 通信上传的感知数据大小之间的关系。
令 $V_{D2D}$ 表示通过 D2D 通信上传的感知数据大小, $x_v$ 和 $V_{D2D}$ 的关系如式~\eqref{Vd2d}。

\begin{equation}
\label{Vd2d}
V_{D2D} = \sum_{v\in \boldsymbol{V}}(\sum_{u\in \boldsymbol{V}}\sum_{x\in \boldsymbol{V}}f^u_{xv} x_v + \lambda_v s x_v), \forall u \neq v
\end{equation}

\subsection{优化目标}

由于边缘服务部署之后,乘客的移动设备可以使用 D2D 通信交付感知数据。
因此,在此任务 D2D 通信的成本已经涵盖在边缘服务的部署成本之内。
为了保障城市感知的传感数据完整性,本节假设没有机会通过 D2D 通信交付的感知数据,会在感知应用生命周期结束之前直接利用蜂窝网络上传。
对于每个乘客而言,利用蜂窝网络上传感知数据的成本记为$c_c$。
因此,剩余感知数据使用蜂窝网络收集的成本$C_m$可以用式~\eqref{eq:c_m} 计算。

\begin{equation}
\label{eq:c_m}
C_m = \sum_{u\in \boldsymbol{V}}c_c(\lambda_u s - \sum_{v\in \boldsymbol{V}}(\sum_{u\in \boldsymbol{V}}\sum_{x\in \boldsymbol{V}}f^u_{xv} x_v + \lambda_v s x_v))T, \forall u \neq v
\end{equation}

结合式~\eqref{eq:Cvcv},可以得出基于公交网络的感知数据收集总成本。
再基于分析得到的限制条件和途径公交站点的流量关系,边缘服务的部署决策可以转化为如下 MILP 问题的求解:

% 现在,可以将成本最小化问题表述为
% \begin{equation}
% \min: \sum_{v\in \boldsymbol{V}}x_v c_vT+\sum_{u\in \boldsymbol{V}}c_c(\lambda_u - \max_{v\in \boldsymbol{V}} x_v \lambda'_{uv} s - \sum_{v\in \boldsymbol{V}}x_v d_{uv} s)T,\forall u \neq v
% \label{eq:origin_opt}
% \end{equation}

% 接下来,将上述优化问题线性化为混合整数线性规划(MILP),该规划通过引入辅助变量$f_{u} \leq \lambda'_{uv}, \forall u\in \boldsymbol{U}, v\in \boldsymbol{V}$形成。通过这样的定义,$f_u s$可以用来代替$\max_{v\in \boldsymbol{V}} x_v \lambda'_{uv} s$,以表示D2D可为首次到达公交车站$u$的乘客上传的最大数据速率。因此, 式~\eqref{eq:origin_opt}中的优化问题 可以等效地转换成以下形式:

\begin{equation}
\begin{aligned}
取最小值: &\quad \sum_{v\in \boldsymbol{V}}x_v c_vT \ + \\&\quad \sum_{u\in \boldsymbol{V}}c_c(\lambda_u s -  \lambda_u s x_u - \sum_{x\in \boldsymbol{V}}\sum_{v\in \boldsymbol{V}}f^u_{xv} x_v ))T, \forall u \neq v\\
\text{限制条件:} &\quad  f^u_v \leq d^u_v s, \forall u, v\in \boldsymbol{V}, u\neq v\\
% &\quad  f^u_v \geq d^u_v (1-x_v) s, \forall u, v\in V, u\neq v \\
&\quad f^u_{vw} \leq \lambda^u_{vw} s, \forall u \in \boldsymbol{V}, v \in \boldsymbol{V}, w \in \boldsymbol{V}, u \neq v\\
% &\quad f^u_{vw} \geq \lambda^u_{vw} (1-x_v) s, \forall u, v, w \in V,\\
&\quad \lambda_u s (1 - x_u) = \sum_{w\in \boldsymbol{V}}f^u_{uw}, \forall u\in \boldsymbol{V}, u \neq v\\
&\quad \sum_{x\in \boldsymbol{V}}f^u_{xv} (1-x_v) = f^u_v + \sum_{w\in \boldsymbol{V}}f^u_{vw}, \forall u\in \boldsymbol{V}, v\in \boldsymbol{V},u \neq v
%&\quad x_v \in \{0, 1\}, \forall v\in V
\end{aligned}
\label{eq:MILP}
\end{equation}

\subsection{二阶段边缘服务部署算法}

对于式~\eqref{eq:MILP} 所列出的 MILP 问题,虽然利用数学工具(例如 MATLAB、Gurobi)可以求出若干组最优解,但是求解复杂度过高,随着问题规模的扩大,求解时间呈指数级增长。
为此,本节设计了一种计算复杂度较低二阶段算法,来获得效果较好的次优解。

\textbf{第一阶段 (算法~\ref{algo_IP})}:
在公共交通构成的有向图 $\boldsymbol{G}=<\boldsymbol{V}, \boldsymbol{E}>$ 中,单凭节点结合 $\boldsymbol{V}$ 和边集合 $\boldsymbol{E}$ 无法识别边缘服务的部署收益。
由于乘客在图中的轨迹在有向图 $\boldsymbol{G}$ 中以多个节点和边构成,所以需要依靠乘客的轨迹来确定站点是否合适部署边缘服务。
根据交通信息统计,可以获得每个乘客的乘车轨迹,借由这些轨迹可以确定感知数据在站点之间的移动顺序,以及众多感知数据的移动顺序是否存在重叠部分。
基于此,将路径划分并找出合适的站点部署边缘服务。

\begin{algorithm}[!b]
\setstretch{\algostretch}
% \KwIn{$\boldsymbol{P}$ : 乘客的路径集合}
\KwIn{$n$ 乘客总数}
\KwIn{$c_v$ 边缘服务部署成本}
\KwIn{$c_c$ 一份感知数据由蜂窝网络上传的成本}
\KwData{$\boldsymbol{P}$ 所有乘客的轨迹集合, $\boldsymbol{Stations}$ 所有车站集合 }
\For{$p \in P$}{
  \If {$p \not\in \boldsymbol{IP}$}{
    将路径 $p$ 加入集合$\boldsymbol{IP}$\\
    计使用 1 次
  }
  \Else{
    在 $\boldsymbol{IP}$ 中找到 p 的使用次数\\
    使用次数加 1
  }
}
\For{$s \in \boldsymbol{Stations}$}{
  \For{$p \in \boldsymbol{IP}$}{
    \If{$s \in p$}{
      在 $s$ 站点部署边缘服务收集感知数据的数量进行累加\\
      总和记为 $s.data$
    }
  }
  \If{$ s.data > c_v/c_c$}{
    把 $s$ 添加到 $\boldsymbol{ChoiceSet}$
  }
}
\KwOut{$\boldsymbol{ChoiceSet}$}
\caption{筛选可部署服务的站点集合 $\boldsymbol{ChoiceSet}$}
\label{algo_IP}
\end{algorithm}

在算法~\ref{algo_IP}中,另公共交通中的乘客总数为 $n$。
对于任一乘客 $i$ 而言,其移动的轨迹路径用 $p_i$ 表示。
因此,可以得到所有乘客的轨迹集合 $\boldsymbol{P}$($p_i \in \boldsymbol{P}$)。
由于不同的乘客可能存在同样的轨迹,可以统计出每条路径 $p_i$ 上做经过的乘客数量。
将路径和其对应的乘客数量以键值对的形式保存在集合 $\boldsymbol{IP}$ ($(p_i, count) \in \boldsymbol{IP}$)中。
在不考虑边缘服务部署之间的影响时,集合 $\boldsymbol{IP}$ 可以很快得出在某一路径上部署边缘服务的感知数据收集总量。
同时,由于 $p_i$ 中包含该路径的站点信息,因此对于任意一个站点,也能快速计算仅在该站点部署边缘服务时的感知数据收集总量。

由于边缘服务的部署成本为 $c_v$,一份感知数据借由蜂窝网络上传的成本是  $c_c$。
因此,若一个站点部署边缘服务之后利用 D2D 通信收集到的感知数据总数小于 $c_v/c_c$ 时,可以看做在该站点部署边缘服务的收益为负。
所以通过统计单个站点利用 D2D 收集的感知数据大小,来建立可选站点的集合 $\boldsymbol{ChoiceSet}$。


% 式~\eqref{eq:MILP} 是MILP形式,被广泛认为是NP难,因此不可能在多项式时间内解决。当问题规模较大时,使用通过求解式~\eqref{eq:origin_opt} 得到的解是不可行的,尽管可以通过各种MILP求解器获得最优解,例如Gurobi。为了使该算法实用,将设计一种低计算复杂度的算法,该算法能够以可接受的计算复杂度逼近最优解。为此,设计了一个两阶段算法,介绍如下。

% 阶段1 (算法1 ) :由于公共交通中的乘客数量非常大,不可能直接解决最佳边缘服务安置位置。因为公交路线是乘客的一种社交图,使用乘客的路径来确定边缘服务的位置。从交通统计数据中,可以得到每个乘客的轨迹。跟踪路径由乘客经过的车站的顺序列表表示。然后,通过交通统计数据可以找到具有相同轨迹的乘客。如果跟踪路径中有一个车站部署了边缘服务来获取数据,所有沿着该跟踪路径行驶的乘客都可以更新他们的传感数据

\begin{algorithm}[!b]
\setstretch{\algostretch}
\KwIn{$n$ : 乘客数量}
\KwIn{$\boldsymbol{IP}$ : 乘客的路径统计信息}
\KwIn{$\boldsymbol{ChoiceSet}$ : 可部署边缘服务的站点集合}
\KwIn{$c_c$: 一份感知数据由蜂窝网络上传的成本}
\KwIn{$c_v$: 边缘服务部署成本}
% \KwIn{$\es$ : 『解释说明』}
% \KwData{『输入数据』}
令 $s_{sum}$ 表示部署边缘服务的数量\\
令 $min_s$ 表示可部署边缘服务的最小数量\\
$min_s$ = $\boldsymbol{ChoiceSet}$元素个数和 $n \times c_c/c_v$ 之间的最小值\\
令 $V_{D2D}$ 表示利用边缘服务能收集到的感知数据数量\\
令 $V_v$ 表示站点 $v$ 部署边缘服务后能收集到的感知数据数量\\
令 $P_v$ 表示 $\boldsymbol{IP}$ 中经过站点 $v$ 所处的路径数量\\
对 $\boldsymbol{IP}$ 中的站点按照 $V_v$/$P_v$ 的值从大到小排序\\
取出 $\boldsymbol{ChoiceSet}$ 中第一个站点\\
更新剩下站点可收集的感知数据数量,对 $\boldsymbol{ChoiceSet}$ 排序\\
令新的站点可接受感知数据数量为 $V'_v$\\
\While{$n - V_{D2D} > (c_v / c_c)$}{
  \While{$V'_v \leq (c_v / c_c)$}{
    \If {$\boldsymbol{ChoiceSet}$ 拥有一个以上的元素}{
      删掉当前 $\boldsymbol{ChoiceSet}$ 中首元素
    }
    \Else{
      回溯,撤销上一次选择
    }
  }
  \While{$V'_v > (c_v / c_c)$} {
    取出 $\boldsymbol{ChoiceSet}$ 中第一个站点\\
    更新剩下站点可收集的感知数据数量,对 $\boldsymbol{ChoiceSet}$ 排序\\
    令新的站点可接受感知数据数量为 $V'_v$\\
  }
}
\KwOut{站点选择集合}
\caption{找出合适的站点集合}
\label{algo_choice}
\end{algorithm}

\textbf{第二阶段 (算法~\ref{algo_choice})}:
在第一阶段算法中,集合 $\boldsymbol{ChoiceSet}$ 中记录的都为收益符合要求的站点。
因此,算法~\ref{algo_choice}的最终目标就是在集合 $\boldsymbol{ChoiceSet}$ 中找出合适的站点组合,使感知数据的收集总成本最小。
在最坏的情况下,所有的感知数据均使用蜂窝网络上传,此时对应的数据收集成本为 $c_c \cdot n$。
而引入边缘服务的宗旨,就是降低感知数据收集的成本开销。
因此,边缘服务的部署数量不应超过 $n\times c_c / c_v$。

同时,为了尽可能减少边缘服务部署决策之间的相互影响,应当优先使用被较少路径穿过、但乘客流量较大的站点。
由于这类站点只有少量路径经过,因此可以避免其部署边缘服务后,对其它站点部署收益的影响。
利用动态规划的思想,每次选择一个站点做为边缘服务的部署位置时,对后续的感知数据收集情况进行迭代运算,然后按照同样的方法对站点继续排序。
如此迭代下去,可以获得一组站点选择结合,作为边缘服务的部署参考。
但是,在满足边缘服务收集感知服务的收益同时,最后依赖蜂窝网络传输的数据量可能会大于 $c_v / c_c$。
此时,算法~\ref{algo_choice}进行回溯,寻找额外的解,通过次优组合寻找是否存在让更少的感知数据使用蜂窝网络上传的边缘服务部署方法。

% :在算法1中,具有足够收入的一组站点被表示为$ChoiceSet$。算法2是找出合理的站点组合,以最小的成本收集所有感测数据。由于具有边缘服务的站的总数有限,并且只能在集合 $ChoiceSet$中选择这些站,所以选择的站的组合被限制在小范围内。在$IP$的帮助下,通过 $ChoiceSet$中选定站点获取的数据非常有效。在知道通过D2D通信上传的数据量后,可以得到不同放置解决方案的总体数据获取成本。最后,与最低成本相对应的站点组合是最佳的边缘服务放置解决方案。当选择多个站点部署边缘服务时,冗余数据也可以通过 $IP$来计算。由于蜂窝通信的成本被称为 $c_c$,在一个站部署边缘服务的成本被称为$c_v$,具有边缘服务的站的数量不超过 $n\times c_c / c_v$。同时,在数据采集量小于$c_v/c_c$的站上部署边缘服务也是不值得的。这两个条件指定了具有边缘服务的站点数量的上限。



% \textbf{『以下内容需要修改』}





% 为了确保传感数据的及时性,采集的传感数据必须及时上传和处理。通常,众包应用程序具有一定的生命周期$T$,在此周期内,必须完成整个过程,即感测、通信和处理。乘客可以通过D2D通信将感测数据上传到部署有数据处理服务的公交车站,或者通过蜂窝通信上传到云。假设在公交车站持续时间内,所有感测数据都可以通过D2D通信成功上传。通过D2D通信不会产生通信费用,在蜂窝通信的情况下,每数据单元收取$c_c$。尽管在公交车站启用数据处理时,公交车站必须部署相应的边缘计算服务,这与云中的服务相同。在网络边缘部署众包数据处理服务并不是免费的,边缘服务每单位时间收费$c_v$。

% \subsection{用于群体数据处理的边缘服务}

% 由于大多数移动设备具有多种通信手段,这些移动设备可以通过D2D无线通信技术(例如蓝牙、WiFi )或长途无线通信(例如蜂窝网络)与其他移动设备交换数据。利用D2D无线通信,数据交换和能耗的成本大大低于长途无线通信。在公交车站部署边缘服务后,乘客可以在这些车站等候时将传感数据发送给这些服务器。如果一些车站没有部署边缘服务,则从这些车站出发的乘客可以在通过带有边缘服务的车站时上传他们的感测数据。一般来说,来自一名乘客的这些感测数据量并不太大,当公交车停在有边缘服务的车站时,上传的时间就足够了。

% 另一方面,公交车站容纳了大量的人。这意味着部署在公交车站的边缘服务可以获取大量传感数据。这些站点是预处理传感数据的最佳场所,例如冗余消除、数据挖掘、数据学习。借助边缘服务,主干网和云服务的过载可以大大减少。与此同时,由于这些边缘服务在地理上更接近乘客,因此自然可以减少服务的延迟。但是,在所有公交车站部署边缘服务是不现实的。如何部署边缘服务是一个紧迫的挑战。








\section{实验结果与分析}

为了验证问题公式和两阶段算法,本节中的时延基于模拟测试。
在模拟的同时,还比较了不同边缘服务放置解决方案,以证明在智慧城市应用中,本章提出的两阶段算法是在公交网络中部署边缘服务的最佳方法。

\textbf{『还要加入补充实验,在补做实验』}

\subsection{公共交通模拟平台}


为了回放乘客等候、乘车、到达和上传数据的行为,本节建立了一个公共交通模拟平台(Bus Traffic Simulation Platform,简称 BSTP)来模拟一个城市的交通行为。BTSP的主要输入可分为两部分。

输入的第一部分是城市的公共汽车时刻表,第二部分是每个乘客的路线。从公共汽车时刻表中,可以得到所有公交汽车的过往路线、到达时间和出发间隔。此外,可以立即生成交通有向图。输入的第二部分是每个乘客的路线,其中包含所有乘客的信息,每个乘客的起点站和终点站。对于每个乘客,都有几种可用的路径来满足乘客的需求。然后,按照可用路径构建一个乘车计划列表,然后按时间成本对列表进行排序。因此乘客可以优先选择时间成本最小的乘坐计划。

通过输入的两个部分数据,BTSP知道一辆公共汽车是否停在车站。如果一辆公交车在车站,公交车上的乘客可以按照他们的乘车计划下车,车站的乘客可以决定是否上车。与此同时,如果这个站点已经部署了边缘服务,公交车内、或者站台上的乘客可以在站点利用 D2D 通信上传感知数据。利用 BTSP 模拟,可以完整回放交通过程和数据上传过程。

\subsection{不同边缘服务部署策略的成本开销}

在BTSP中,按照实际公交时间表在武汉市试验了30条公交线路。BTSP共有273个站点,241个公共汽车在60分钟内从7:00到8:00运行。公交线路图如图~\ref{Figure_test_bus_lines} 所示,所有公交线路用绿色高亮显示。在这种情况下,假设众包应用程序有60分钟的特定生命周期,边缘服务只能部署在公交车站。在车站部署边缘服务的成本是每小时500($c_v = 500$)。通过蜂窝网络上传一段数据的成本是1 ($c_c = 1$)。

\begin{figure}[!h]
  \centering
  % \vspace{-1em}
  \includegraphics[width=300pt]{figures/Sec_MONET/test_bus_lines.png}
  \vspace{-0.5em}
  \caption{模拟测试中使用的公交路线}
  % \vspace{-1em}
  \label{Figure_test_bus_lines}
\end{figure}

遵循前文描述的系统模型,使用Gurobi找出边缘服务的最佳放置解决方案。与本文中的两阶段算法相比,构建了另外两种不同的布局策略。第一种算法是根据人员流动(Flow of People)选择站点来部署边缘服务。第二种算法是根据站点的连接度(Connectivity of Station)进行并选择部署边缘服务。由于交通的繁忙程度随着时间不同,所以在BTSP中使用不同数量的乘客进行不同时段的交通模拟。

当乘客量设置为5000、10000和50000时,BTSP中的结果如图~\ref{fig_5000}、图~\ref{fig_10000} 和图~\ref{fig_50000} 所示。由于$c_c$设置为1,最大成本等于乘客人数。在图~\ref{fig_5000} (a)、图~\ref{fig_10000} (a)和图~\ref{fig_50000} (a)中,结果表明,的两阶段算法可以找到一个更好的布局解决方案,以最小化成本,这类似于Gurobi找到的最佳解决方案。然而,FoP和CoS放置策略的性能不令人满意,它们的部署成本比最佳放置策略高出20\%。比较FoP和CoS的放置策略,它们的成本非常接近。因此,按照人员流动或站点连接部署边缘服务不是节省成本的有效方法。

\begin{figure}[!h]
  \centering
  \begin{subfigure}[b]{0.45\linewidth}
    \includegraphics[width=200pt]{./figures/Sec_MONET/cost5000.pdf}
    \label{fig_cost5000}
    \vspace{-2em}
    \caption{不同部署策略的数据s收集成本}
  \end{subfigure}
  \begin{subfigure}[b]{0.45\linewidth}
    \includegraphics[width=200pt]{./figures/Sec_MONET/data5000.pdf}
    \label{fig_data5000}
    \vspace{-2em}
    \caption{不同部署策略的数据收集数量}
  \end{subfigure}
    \vspace{-0.5em}
    \caption{5000 名乘客在 BSTP 中的模拟结果}
  \label{fig_5000}
\end{figure}

\begin{figure}[!h]
  \centering
  \begin{subfigure}[b]{0.45\linewidth}
    \includegraphics[width=200pt]{./figures/Sec_MONET/cost10000.pdf}
    \label{fig_cost10000}
    \vspace{-2em}
    \caption{不同部署策略的数据收集成本}
  \end{subfigure}
  \begin{subfigure}[b]{0.45\linewidth}
    \includegraphics[width=200pt]{./figures/Sec_MONET/data10000.pdf}
    \label{fig_data10000}
    \vspace{-2em}
    \caption{不同部署策略的数据收集数量}
  \end{subfigure}
    \vspace{-0.5em}
    \caption{10000 名乘客在 BSTP 中的模拟结果}
  \label{fig_10000}
\end{figure}

\begin{figure}[!h]
  \centering
  \begin{subfigure}[b]{0.45\linewidth}
    \includegraphics[width=200pt]{./figures/Sec_MONET/cost50000.pdf}
    \label{fig_cost50000}
    \vspace{-2em}
    \caption{不同部署策略的数据收集成本}
  \end{subfigure}
  \begin{subfigure}[b]{0.45\linewidth}
    \includegraphics[width=200pt]{./figures/Sec_MONET/data50000.pdf}
    \label{fig_data50000}
    \vspace{-2em}
    \caption{不同部署策略的数据收集数量}
  \end{subfigure}
    \vspace{-0.5em}
    \caption{50000 名乘客在 BSTP 中的模拟结果}
  \label{fig_50000}
\end{figure}

在数据采集中,每条数据都有固定的体积,等于$s$。由于所有乘客都必须上传他们的传感数据,所以收集的数据总量是乘客人数乘以$s$。如图~\ref{fig_5000} (b)、图~\ref{fig_10000} (b)、图~\ref{fig_50000} (b)所示,当成本较低时,D2D通信上传的数据变得更多。当边缘服务按照FoP策略和CoS策略部署时,D2D通信上传的数据显著减少。

在评估之前,已经告知60分钟内有241辆公共汽车在行驶。公共汽车的最大载客量设定为50人。在这个模拟场景中,5000名乘客造成较轻的交通负载,10000名乘客几乎交通满载,50000名乘客造成交通过载。结果显示,当交通负载接近满时,通过D2D通信上传的数据比例变得最大。相反,当交通负载太重或太轻时,通过D2D通信上传的数据量将变得更小。当交通负载较轻时,交通流量的减少会导致不经济的布局决策。当交通负载很重时,大量乘客将滞留在起点站或中转站。如果这个车站没有部署边缘服务,严重阻碍乘客通过D2D通信上传数据。

\section{本章小结}

本章主要研究了城市范围内的移动群智感知应用中的数据收集过程。
通过在公交车站部署边缘服务,利用城市居民的社会活动来完成感知数据的收集工作,以达到充分利用 D2D 通信,降低数据收集成本的目的。
本章围绕感知应用的生命周期、感知质量要求以及边缘服务部署策略,构建了 MILP 模型。
然后通过动态规划算法决定边缘服务的部署策略。
经仿真测试对比,该算法解决了交叉路径中感知数据的重复收集问题,提高 D2D 通信利用率并减少蜂窝网络通信的使用,降低了城市感知中数据收集的总成本。



\chapter{总结与展望}

\section{全文总结}

% 本文对边缘计算和 D2D 通信支撑的移动群智感知中感知质量优化问题进行了深入研究,
% 针对移动群智感知的三个重要过程(感知、通信和数据处理)中所面临的感知任务调度问题、网络资源调度问题、边缘服务调度问题,

在移动群智感知应用中,众多的参与设备会产生大量的传感数据,并且这些感知数据分布在广泛的地理空间中。
这为感知数据的收集、查找与分析带来了极大的挑战。
如果没有高效、合理的解决方案,庞大的感知数据将对网络资源、计算资源、存储资源造成巨大的压力。
本文利用边端融合为移动群智感知的提供支撑,针对群智感知的执行过程优化,提出了三种基于边缘融合的移动群智感知执行优化机制。
% 以提高群智感知应用的覆盖范围、感知效率和感知质量。
% 然而,参与群智感知的移动设备众多、且个体能力存在差异,需要合理利用边缘网络中的各种资源来保障群智感知应用的执行效率。
% 另一方面,移动设备之间的 D2D 通信受运动状态影响会导致网络资源不稳定且难于管理。
% 为了保障群智感知应用的执行效率和感知质量,本文围绕群智感知的三个重要过程:感知、通信和数据处理,针对群智感知应用的执行过程和边缘网络中资源进行调度进行研究。
% 本文针对边缘计算和 D2D 通信支撑下的移动群智感知中
本文的主要创新成果如下:

(1)基于边端融合的群智感知任务调度机制

在移动群智感知应用的生命周期内,可以通过提高感知任务的分发效率和感知数据的收集效率,来达到感知质量优化的目的。
为了实现这一目标,本文结合边缘服务和 D2D 通信,构建了移动群智感知应用的随机过程分析模型。
首先,考虑了移动过程中感知设备基于 D2D 通信的随机特性,
分析了感知应用生命周期内,多个边缘服务器对任务分发效率以及数据接收效率的影响。
利用常微分方程组构建了移动群智感知的执行过程模型。
通过对模型的分析与求解,进一步推导出边缘网络中各类资源配额、移动设备运动特征和感知质量的量化关系。
同时,考虑群智感知应用的有限生命周期,提出了任务分发阶段和数据收集阶段的时间划分算法,优化并预测了边缘融合场景下的群智感知执行效率,并分析了执行效率的影响因素与其作用机理。

(2)面向计算任务卸载的群智感知网络流调度机制

针对群智感知应用中感知数据处理,研究了用户设备与边缘服务器之间计算任务卸载和网络资源调度问题。并特别考虑了基于 SDN 管理的边缘网络资源,构建了 SDN 流表容量限制下的高能效卸载决策与流调度 ILP 模型。
基于对模型的求解分析,确立了任务卸载决策和移动设备对网络资源需求之间的关系。
针对求解 ILP 模型的高计算复杂度,本文提出依据权重的数据流调度算法,确定任务卸载决策以及移动设备和边缘服务器的通信链路调度决策。在保障任务卸载收益最大化的同时,实现边缘网络中的流量负载均衡,并减小感知数据处理能耗成本。
经过对比试验,该算法可以达到最优解87.4\%以上的节能效率。

(3)基于用户移动特征的群智感知数据收集方法

针对移动群智感知应用在城市感知中的海量感知数据收集问题,
借助城市中的公共交通网络分析出居民的城市移动特征,
并基于这一特征部署边缘服务,以达到充分利用 D2D 通信,降低数据收集成本的目的。
本文围绕感知应用的生命周期、感知质量要求以及边缘服务部署策略,构建了 ILP 模型。
然后通过动态规划算法决定边缘服务的部署策略。
经仿真测试对比,该算法解决了交叉路径中感知数据的重复收集问题,提高 D2D 通信利用率并减少蜂窝网络通信的使用,降低了城市感知中感知数据收集的总成本。

\section{研究展望}

本文基于移动群智感知应用的执行过程,针对网络边缘的资源调度进行了研究,以提高感知效率、降低执行成本。
但是在基于边端融合的移动群智感知中,云端资源与边缘资源的协同管理问题、边缘网络的联合自治问题、以及感知数据的流式处理问题等,本文并没有逐一考虑。
因此,为进一步提高移动群智感知的感知质量,仍需更全面、深入的研究。

(1)云端资源与边缘资源的协同管理问题

移动群智感知借助边缘计算,虽然能够利用无处不在的边缘设备收集感知数据,并通过边缘服务在传输过程中进行简单的数据处理,但是最终的运算仍然需在云平台进行处理。
另一方面,边缘计算作为与云计算模型互补的分布式计算模型,现有的边缘计算平台通常与云计算平台组合成双层架构来为应用程序提供支撑。
因此,边缘计算平台和云平台也存在数据、任务的同步与调度问题。

(2)边缘网络的联合自治问题

在移动群智感知应用中,尤其是智慧城市感知,感知任务需要覆盖的范围过于庞大。
往往需要将目标区域划分构建成多个边缘云进行覆盖。
但是移动设备的运动特征,无法保障移动设备在感知任务的生命周期内一直处于同一边缘云中。
因此,面对边缘云内移动设备的加入和退出,边缘云之间需要对移动设备进行联合管理,确保资源分配的准确性。
同时,针对感知应用,边缘云还需要确保运行状态的一致性。

(3)感知数据的流式处理问题

边缘计算的引入,让感知数据得以在传输过程中借助边缘服务得到一定程度的处理。
但是,感知数据的收集具备一定的离散性,特别是当采用 D2D 通信传输感知数据时,这种机会式通信网络让边缘服务器收到感知数据的时间间隔和数据大小毫无规律可言。
因此,边缘服务器在处理感知数据时,可以借助缓存、打包等策略,提高边缘服务器上的资源利用率。
另外,面对监控类型的感知数据,由于时效性要求,边缘服务也应当根据感知任务的要求,构建合理的数据处理、传输流水线,确保感知任务能够稳定、流畅地运行。



\begin{ack}

本人于2011年进入集群与网络计算湖北省重点实验室开始博士生涯,至今已整整八年时间。八年里,我在科研的道路上遇到过许多困难和挫折,对科研课题的迷茫,对新兴领域的空白,使我在科研道路上步履蹒跚。在科研“开荒”期间,大量的文献阅读让我掌握了做研究的基本方法,通过复现、优化前人的工作,让我的动手能力得到了极大的提高,同时,大量的实践,让我逐步清晰了科研的目标和方向,并坚持在这个领域不断学习、探索、研究。执着的坚守、失败的沮丧、等待的徬徨、成功的喜悦等无数的情绪,组成了我人生中最宝贵的经历。如今我已经来到博士生涯的末端,这条道路漫长、艰辛、坎坷,能够支撑我走完八年时间,离不开生命中那些恩重如山的导师,情同手足的同学给我的指导、帮助和鼓励。

感谢我的导师金海教授。在八年博士生涯中,金老师无论是在科研、生活方面都给予了我非常重要的帮助和无微不至的关怀。金老师在学术界的站位、思考、把握,开阔了我的视野,启迪了我的思维。在我的博士生涯开始之初,金老师对科研选题严格把关,悉心指导。当选题被否定时,金老师和我反复推敲,深入探讨,在一字一句中提升了我对科研细节的洞察力以及科研工作的判断力,也为我未来的研究指明了方向。在每一次的博士生沙龙中,金老师及时指出我研究中存在的问题,并提出修改思路和方法,为我攻克科研难关提供了极大的帮助。每次论文投稿之前,金老师认真评阅,并给出详细的修改意见。有时我的科研工作达不到金老师的要求,老师依然豁达包容,耐心指导我攻克难关。金老师在科研工作上要求严厉,但在生活中却让我感到和蔼可亲和温暖。八年博士生涯中,我经历了人生的疾病、感情的波折、学业的起落,金老师总是关心我、帮助我、鼓励我,坚定了我将科研道路走下去的信心和决心。

感谢我的指导老师寥小飞教授。从进入实验室就有幸在寥老师负责的“系统软件与体系结构”课题组,先后参与国家重点基础研究发展计划:“计算系统虚拟化基础理论与方法研究”,欧盟项目“MONICA-Mobile Cloud Computing: Networks, Services and Architecture”等重点项目研究过程。通过项目的历炼,将研究、创新有机的结合到实践中,极大的提高了我的动手能力。同时,寥老师具有对学术热点与新型技术结合有着敏锐的洞察里,使我更加清晰的了解到学术研究和业界技术之间的紧密联系。

% 感谢我的指导老师廖小飞教授。
% 从刚进实验室,我就一直在廖小飞老师负责的“系统软件与体系结构”课题组。
% 在课题组中,我参与过国家重点基础研究发展计划“计算系统虚拟化基础理论与方法研究”,也参与了欧盟项目“MONICA-Mobile Cloud Computing: Networks, Services and Architecture”。
% 通过项目的历练,将研究、创新有机地结合到实战中。
% 同时,廖老师对学术热点和新兴技术的结合有着敏锐的观察力,让我更加清晰了解到学术研究和业界技术之间的紧密联系。

感谢我的师兄曾德泽教授。
在科研的低谷期,曾老师给了我兄长般的关怀和帮助。
在科研遇到难题的时候,曾老师辅导我完善系统模型的建立,教会我如何提炼问题。
同时,曾老师还教会了我许多用于分析的工具和方法,为我在科研工作中解决实际问题提供了极大的帮助。
在论文撰写工作中,曾老师布局谋篇的建议和字斟句酌的润色,为我论文的成功发表提供了坚实的基础。


感谢吴松、余辰、刘海坤、陆枫、肖江、顾琳、潘胜利、熊穆舟等老师,在我的博士攻读期间,对我的关心和帮助无时不在。感谢李丁丁、李鹤、郑龙、张宇、范学鹏、叶晨成、郭人通等同学。感谢实验室各位同仁的陪伴,陪我渡过了八年的曲折坎坷、喜怒哀乐,我的博士生涯,因为您们而变得多姿多彩,终身难忘。

感谢我最重要的父母与家人。
谢谢你们对我在科研道路上无条件的理解与支持。
你们是我最坚实的避风港、也是我最核心的前进动力。
因为你们全力的支持,让我在科研的道路上没有多余的牵绊。

最后,感谢百忙之中评审本文并提出宝贵意见的各位专家、教授,对你们致以最诚挈的谢意。


\end{ack}

\bibliography{ref}

\appendix


\begin{publications}
\item \textbf{Hou H}, Jin H, Liao X, Zeng D. Multi-path Routing for Energy Efficient Mobile Offloading in Software Defined Networks. in: Proceedings of The IEEE International Symposium on Parallel and Distributed Processing with Applications (ISPA), 2017, pp. 360-367. Guangzhou, China. Dec. 15, 2017 %DOI 10.1109/ISPA/IUCC.2017.00058
\item \textbf{Hou H}, Jin H, Liao X, Zeng D. Stochastic Analysis on Fog Computing Empowered Mobile Crowdsensing with D2D Communications. in: Proceedings of The IEEE International Conference on Ubiquitous Intelligence and Computing (UIC), 2018, pp. 656-663.  Guangzhou, China. Oct. 18, 2018 %DOI 10.1109/SmartWorld.2018.00131
\item \textbf{Hou H}, Jin H, Liao X. Cost Efficient Edge Service Placement for Crowdsensing via Bus Passengers. Mobile Network and Application, 2019, minor revision.
\item Liao X, Li H, Jin H, \textbf{Hou H}, Jiang Y, Liu H. VMStore: Distributed storage system for multiple virtual machines. Science China Information Sciences, 2011, 54(6):1104-1118. %DOI 10.1007/s11432-011-4273-0

% \item Multi-path Routing for Energy Efficient Mobile Offloading in Software Defined Networks. in Proceedings of The IEEE International Symposium on Parallel and Distributed Processing with Applications (ISPA), 2017, \textbf{第一作者}
% \item Stochastic Analysis on Fog Computing Empowered Mobile Crowdsensing with D2D Communications. in Proceedings of The IEEE International Conference on Ubiquitous Intelligence and Computing (UIC), 2018, \textbf{第一作者}
% \item Cost Efficient Edge Service Placement for Crowdsensing via Bus Passengers. Mobile Network and Application, 2019, minor revision, \textbf{第一作者}
% \item VMStore: Distributed storage system for multiple virtual machines. Science China Information Sciences, 2011, \textbf{第四作者}
\end{publications}

\end{document}

\endinput
%%
%% End of file `hustthesis-zh-example。tex'。
