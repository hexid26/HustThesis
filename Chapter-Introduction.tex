\chapter{绪论}
本章首先介绍群智感知技术和边缘计算的相关研究背景,然后简要介绍当前国内外研究现状和现有工作的不足。
通过分析群智感知应用在大规模场景下所存在的缺陷,明确了设计面向智能感知的边缘计算资源调度机制的目的和意义。
在讨论国內外针对大规模群智感知的研究现状及不足之处之后,进一步介绍论文的研究内容和主要贡献。
最后介绍论文的组织结构。

\section{研究背景}

\subsection{群智感知技术}

群智感知技术也被称为移动群智感知技术。
群智感知技术通过租用大量具备感知能力和计算能力的移动设备(例如智能手机、平板电脑、可穿戴设备、智能车辆等)共同提取、共享数据,并对这些数据进行去重、映射、分析,从而估计、推断出符合共同利益的结果~\cite{DBLP:journals/cm/GantiYL11}。
在现实世界中应用群智感知技术,需要大量的传感器在不同空间中产生传感数据,并利用网络将传感数据汇总并交付给计算资源进行处理。

如今,配有各种传感器的智能设备已经无处不在。
常见的智能手机已经集成了大量的传感器,它们不仅可以获取与设备所处环境相关的数字信息,例如环境光(光学传感器)、噪声(麦克风)、位置(全球定位系统 GPS)、移动状态(陀螺仪、加速计)等,甚至还可以获取与使用者自身相关的数字信息,例如运动状态、心跳频率等。
在大量的现代交通工具中,雷达、摄像头、GPS、陀螺仪、加速计等设备也已经成为保障安全驾驶的必备传感器。
因此在人类的生活空间中,已经遍布了大量具备多元化传感功能的智能设备,并且这些智能设备的使用者大多数为自然人。
所以,群智感知技术的潜在参与者多、空间覆盖广,让其成为了环境研究~\cite{DBLP:conf/sensys/DuttaAKMMWW09}、人类社会关系研究~\cite{DBLP:conf/globecom/AslIAM13}、智慧城市建设研究~\cite{DBLP:journals/cm/WangZWCHM16}的基础工具。
利用群智感知技术,研究者们已经完成了很多有启发性的工作。
近年来,移动智能设备的能力得到了极大的发展,对于在不进行大规模投资的情况下收集数据的企业来说,群智感知已经成为一种极具吸引力的方案。
许多技术公司都愿意使用群智感知收集大量的数据以提供额外的高品质服务。

群智感知应用的工作流程可以简单划分为四步~\cite{zh_cn:shi}:任务分发,传感器感知,数据上传,数据处理。
在任务分发过程中,需要决策群智感知任务的覆盖区域,执行任务的用户数量,以及任务的持续时间。
根据实际需求将群智感知任务部署到合适的智能设备上。
传感器感知是指智能设备在收到任务之后,通知传感器工作并产生感知数据。
数据上传则是在传感数据产生后,由智能设备利用自身的通信能力,将传感数据上传到等待数据处理的地方(例如云服务器)。
待大部分传感数据收集完成后,由云端服务器进行去重、映射等操作,并利用数据挖掘、数据分析得出可用的结论。

\begin{figure}[h]
\centering
% \vspace{-1em}
\includegraphics[width=240pt]{./figures/MCS_Scenario.pdf}
\vspace{-0.5em}
\caption{群智感知应用典型场景}
\vspace{-1em}
\label{Figure_MCS}
\end{figure}

图~\ref{Figure_MCS} 描述了典型的群智感知应用场景。
图中,智能手机、可穿戴设备、笔记本跟随自然人的位置分布在不同的地理位置。
在该区域中,还有行驶中的车辆、监控摄像头、以及其它无线传感器散布在四周。
利用网络将传感器的数据收集之后,交付给云端服务器。
云端服务器通过对传感数据的处理和分析,得到该区域中的相关的环境数据、以及区域中的自然人所相关的数据。
因此,群智感知技术在实际使用过程中,往往需要部署借助云计算服务以完成数据的后台处理工作。

\subsection{边缘计算}

近年来,物联网和 5G 通讯技术取得了突飞猛进的发展,万物互联的时代也正在加速到来。
如今,不仅网络边缘的设备数量在急剧增长,这些设备产生的数据量也呈现爆炸式增长的趋势。
据 IDC 预测~\cite{zh_cn:shi},2020年全球数据生产总量将超过 40 泽字节(zettabyte,ZB),而物联网中产生的数据占比达到了 40\% 以上。
届时,集中式处理模型,例如传统的云计算模型,将会面临带宽不足、延时较高、能耗效率低、安全隐私风险高等种种问题。
为了避免集中式处理模型所即将面临的问题,边缘计算模型应运而生。

边缘计算是一种面向大量边缘设备所产生的海量数据的计算模型~\cite{DBLP:journals/cm/SunA16}。
它是一种将数据处理和数据存储放在网络的边缘节点的分布式计算形式。
在靠近终端设备、或者数据源头的网络边缘侧,融合通信、计算、存储、应用能力而构建分布式计算平台,是边缘计算的核心思想。
在边缘计算中,可以高效结合 WiFi 网络、5G 网络、端到端通信、软件定义网络等通信手段,将云端服务本地化,提供“智慧化”服务的基础。
边缘计算利用分布在距离终端最近的基础设施,为网络边缘设备提供具有针对性的服务与算力。
这些算力可以在数据源端完成部分的数据预处理任务,『数据预处理包括但不限于』。
在数据预处理完成后,再将剩下的数据传回云端,根据不同的业务进行最终的处理。
因此,边缘计算将网络结构和云端资源进行协同配合,为个人或企业提供满足技术需求的弹性计算服务。

『缺图』

图~\ref{『』}描述了典型云计算场景和边缘计算场景的不同之处。
相较于云计算框架,边缘计算模型的优势主要有3点。
1)边缘计算中,产生于网路边缘的数据并不用全部上传至云端,这一特性能够有效减轻骨干网络中的带宽占用;
2)在靠近数据生产者处完成大部分的数据预处理工作,减少对云计算资源的依赖,不仅可以减少服务的延时,还能提高服务的响应能力;
3)用户数据不用上传至云端服务器,而是暂存在网络边缘设备上,减少了用户隐私泄露的风险。

边缘计算利用自身的优势弥补了云计算模型中海量数据传输延时高,隐私安全处理敏感等问题,使得边缘计算更适用于新兴的物联网应用场景。
因此,近年来边缘计算得到了快速的发展和完善。
将云计算能力扩展至距离终端最近的边缘侧,以满足“大连接,低延时,大带宽”的新需求。
目前,边缘计算已经逐步应用在和人类生活息息相关的各类场景中,例如『公共安全中的实时数据处理、智能车联网、自动驾驶、虚拟现实、工业互联网、智能家居和智能城市』

\subsection{面向群智感知的边缘计算研究及其挑战}

近年来,业界和学术界已经着手将边缘计算结合到实际生产与应用中。
这些应用,大多数与智慧城市建设密切相关。
例如,论文~\cite{DBLP:journals/cm/GantiYL11} 在车辆上部署 GPS 和加速度计,可以在车辆行驶过程中定位城市中的坑洼路面;
科特迪瓦的一个环境非政府组织在非洲阿比让市电信公司的帮助下,在阿比让市利用25个蜂窝基站和群智感知应用,向阿比让市民即时提供空气质量感应结果~\cite{};
『举例』。

这些研究工作与实际应用,和群智感知以及边缘计算都密不可分,这也揭示了面向群智感知的边缘计算模型在未来城市建设中的重要地位。
随着群智感知应用的应用领域和感知范围逐步扩大,如何将边缘设备有机地组织起来,合理调度边缘网络中的计算资源、网络资源,以及云端资源的合理分配,这些研究内容都为边缘计算技术实际落地带来了新的挑战。
目前,面向群智感知的边缘计算中主要研究挑战有三点。

第一个挑战是\textbf{边缘计算中面向群智感知的任务调度问题}。
在群智感知应用中,需要大量的参与者和边缘设备为群智感知服务提供输入数据。
然而,边缘设备自身并不具备群智感知服务的发现能力。
因此,群智感知服务需要将数据采集任务利用网络分发到合适的边缘设备上,雇佣这些智能设备收集必要的传感数据,并反馈给附近的边缘计算节点。
在真实场景下,群智感知的参与者多为边缘设备的拥有者,且大多数参与群智感知的边缘设备为移动设备,所以这些边缘设备通常具有一定的移动特性。

但是由于群智感知服务往往具备基本的实时性要求~\cite{},所以以上四个步骤在实际执行中是连续且不可分割的。
因此,在边缘网络中,边缘设备

近年来,参与群智感知应用的边缘设备集成度越来越高,功能越来越强大,其可使用的网络接入技术也逐渐多样化。
在早期的群智感知研究工作中,群智感知的参与设备多使用蜂窝通信~\cite{DBLP:conf/globecom/ZhangJLLC16,DBLP:conf/icdcs/XiaoWHHH16}来上传感知数据。
随后,Karaliopoulos 等人的研究~\cite{DBLP:conf/infocom/KaraliopoulosTK15}发现,在传感数据上传过程中利用端到端通信,可以减少边缘设备的能耗成本的网络通信成本,并以此来为群智感知服务雇佣更多的参与者。
同时,论文~\cite{DBLP:journals/tpds/ZhaoMTL15}通过研究机会式通信网络~\cite{}中的数据包转发规律,发现在边缘网络中利用数据包融合技术可以有效减少数据传输延时。
随着 5G 和物联网技术的飞速发展,边缘设备可以利用蓝牙技术、WiFi 技术、进场通信技术、端到端通信技术在小范围内快速交换数据信息。

在群智感知应用中,服务所需要的原始数据基本来源于边缘设备的传感数据。
在该场景下,用户和边缘设备多处于运动状态中~\cite{},例如智能车联网中的车辆、群智感知应用中的参与者。
这些设备和参与者随着地理位置和所处网络环境的改变,都会在网络中引起网络拓扑结构的变化。
然而,群智感知服务往往以连续的方式收集参与者的信息并提供反馈结果。
所以在边缘计算实际应用中,随着参与者或者边缘设备状态的改变,边缘网络需要时刻对边缘设备的网络状态进行维护并合理调度边缘网络中的数据流~\cite{}。



第二个挑战是\textbf{边缘计算中面向群智感知的服务资源资源调度问题}。

第三个挑战是\textbf{边缘网络中的『』调度问题}。


1)网络自组织结构:
在群智感知应用中,有大量的边缘设备参与其中。
首先, 边缘计算中,服务端从云端转变到网络边缘侧,如何让这些边缘设备快算获取自身周边存在的服务,是边缘计算在网络层面的一个重要问题。
其次,在边缘计算场景中,用户和边缘设备的参与方式均为动态过程,例如车联网、群智感知应用中的参与者。
这些设备和参与者随着地理位置和网络环境的改变,都会在网络中引起拓扑结构的变化。如何针对变化拓扑部署并迁移边缘服务,也是边缘计算在网络层需要考虑的难题。
再者,边缘设备会产生大量的传感数据,这些数据虽然不会全部上传至主干网络,但如何平衡边缘网络中的数据负载,合理调度边缘网络中的数据流量,也是边缘计算网络层需要解决的问题。
最后,随着无线通信技术的发展,在边缘网络中可以利用的通信技术也变得多样化。
针对不同业务场景合理使用蓝牙、WiFi、5G、或者蜂窝网络也成为边缘网络中的热门研究方向。

2)数据处理框架:
在群智感知应用中,边缘设备无时不刻提供的传感数据构成了海量数据场景。
并且,这些海量数据具备多样性、体量大、时间相关性、地域相关性、以及冗余性等特点。
如何对这些海量数据进行实时处理,对边缘计算提出了新的数据处理需求。
因此,构建一个针对边缘数据进行管理、处理、分析以及共享的框架更显得格外重要。
与此同时,随着人工智能技术~\cite{}的快速发展,很多边缘设备还需要结合自身数据运行多种智能算法。
例如自然语言的处理~\cite{}、实时语言翻译~\cite{}、智能车辆导航~\cite{}等应用,都需要机器学习相关算法的参与。
因此在边缘计算的数据处理框架中,不仅需要针对传感数据流进行管理,还需要兼容常见的的机器学习框架,例如谷歌的 TensorFlow~\cite{}、开源的 Caffe~\cite{}等。

3)安全和隐私:
由于群智感知应用发生在最靠近用户的网络边缘侧,部分数据可以避免被上传到云端,在一定程度上降低了用户隐私数据泄露的风险~\cite{DBLP:conf/wasa/YiQL15}。
然而,相较于云计算中心,边缘网络中的某些节点依然潜伏着安全隐患,例如伪造的、或者被不法人员操控的无线接入点等。
因此,参与边缘计算的设备或者用户,其隐私和安全一样受到安全隐患的威胁。
由于边缘网络中,边缘计算节点的分布式架构和异构性也让计算节点难以进行统一化管理,从而导致新的安全问题和隐私泄露隐患。
针对这一系列安全隐患,科研人员将其归类为应用安全、信息安全、网络安全和系统安全。
在传统分布式计算架构中,这些安全隐患可以利用可信执行环境~\cite{DBLP:conf/trustcom/SabtAB15,DBLP:conf/isca/NingZSS17}进行有效的规避。
例如,Intel 软件防护管理技术~\cite{DBLP:conf/isca/HoekstraLPPC13}、Intel 内存加密技术~\cite{}、AMD 平台安全防范技术~\cite{}、AMD 内存加密技术~\cite{}等,这些技术不仅可以保障执行平台和环境的安全性,还可以对存数数据进项加密防止数据泄露。
通过这些技术进行修改,适配到边缘计算中,可以保障在边缘计算节点出现隐患时,依然保障应用和数据的可靠性和安全性。

在面向群智感知的边缘计算框架中,论文针对

利用 GPS 和麦克风,可以绘制一个区域的噪声污染情况。
\cite{DBLP:journals/cm/GantiYL11} 总结出群智感知的三类应用场景:1) 环境监测;2) 基础设施监测;3) 社交状态跟踪。

通过用户的参与状态,群智感知又可以分为两类。第一类是参与式群智感知,用户作为志愿者自愿参与群智感知并提供数据信息。第二类是机会式群体感知,智能设备能够自动地感知、收集、并共享数据,整个过程并不需要用户的主动干预。


其中一些著名的公司就包括 FaceBook、Google、和 Uber。


\section{国内外研究现状}

如今,群智感知技术已经和人类的生活密不可分。
作为分布式感应数据收集的基础框架,群智感知技术已经普遍应用于智慧城市的建设当中。
智慧城市是指把新一代信息技术、网络技术充分应用在城市管理与建设过程中的城市信息化高级形态。
而城市中的各类信息收集,则是智慧城市落地的奠基石。
利用群智感知技术,能够产生大量基于自然环境、人类社会活动等多维度的海量数据。
通过对这些数据进行数据挖掘、学习和分析,不仅可以了解居民的生活习惯、城市环境状态、城市能源消耗等多维状态,更能为实现以人为本的全面可持续发展道路提供基础支撑。

% \subsection{边缘计算的资源调度}

\subsection{边缘计算中面向群智感知的任务调度机制}

\subsection{边缘计算中面向群智感知的服务资源资源调度}

\section{研究内容和意义}
ddd

\section{论文组织结构}
第一章为绪论。首先给出面向群智感知的边缘计算技术的的研究背景和面临的挑战,然后介绍现阶段国内外研究现状,最后给出研究内容和主要贡献以及论文组织结构。

第二章为『』

第三章为边缘网络中基于移动模型的群智感知边缘服务调度机制。
首先基于随机移动模型建立群智感知任务分发和时间的关系,并依据传染病模型对群智感知任务分发的过程建立数学模型。
接着分析群智感知任务的持续时间和边缘节点的移动模型对群智感知任务分发过程以及数据回收过程的影响,并提出群智感知任务的覆盖率。
再提出提高群智感知任务覆盖率的优化算法。
最后结合仿真模拟验证优化算法对群智感知任务覆盖率的效果。

第四章为边缘网络中基于社交模型的群智感知边缘服务调度机制。
首先基于城市公交路线建立群智感知边缘服务部署框架。
通过分析城市居民在公共交通中的行为,建立群智感知数据回收和成本的关系模型。
基于此种模型,提出基于公交路线的边缘服务部署决策算法。
最后采用仿真模拟验证算法效率以及成本节省比例。

第五章总结全文,概括文中的主要贡献并展望未来的研究工作。
