\chapter{绪论}

% \textbf{柔和一些,尝试使用这个逻辑:移动群智感知很重要,是补充,甚至是替代传统无线传感网的有效途径。同时,云计算是对传感数据进行处理的重要支撑平台。然而,……乏力。边缘计算,……能够有效地弥补云计算的不足。然而,如何……仍旧是亟待解决的问题。本文将就此从感知、通信以及计算三个角度展开分析讨论。本章……}

随着 5G 通信标准的确立,物联网时代正式拉开了序幕。
在无所不在的移动设备中,大部分都同时装备了多种传感器,并具有一定的计算能力和网络通信能力。
许多研究者已经利用各种技术手段将这些移动设备联合起来,并构建成一个比传统无线传感网络效率更高、适用范围更广、且使用成本更低廉的新型应用范式——移动群智感知。
% 利用群质感知,可以在真实世界中提取数字信息以达到环境测量、监控等目的,并创造出更多有社会价值的应用。
但是,移动群智感知应用中大量的参与设备和异构的感知数据,对感知任务的部署、感知数据的传输和处理,都提出了极高的要求。
云计算作为集中式服务模型,在支撑覆盖区域广、时效性强的移动群智感知应用时已渐显乏力。
边缘计算作为一种与云计算互补的分布式计算模型,能够有效地弥补云计算的不足。
然而,如何利用其特性为移动群智感知服务提供基础支撑,是一个亟待解决的研究课题。
本文将就此从感知、通信以及计算三个角度展开分析讨论。
本章首先介绍移动群智感知场景下边缘计算中的资源调度问题;其次简要介绍当前国内外研究现状和现有工作的不足;再次阐述了本文研究的目的与意义;最后介绍论文的组织结构和层次关系。
% 移动群智感知中,参与设备数量庞大,感知数据种类繁多、结构各异,导致传感数据量也达到了新的数量级。
% 为了快速处理庞大的感知数据集,通常使用云计算作为移动群智感知的重要支撑平台。
% 然而,云计算作为集中式服务模型,在支撑覆盖区域广、时效性强的移动群智感知应用时已渐显乏力。
% 边缘计算作为一种与云计算互补的分布式计算模型,能够有效地弥补云计算的不足。
% 如何利用其特性为移动群智感知服务提供基础支撑,是一个亟待解决的研究课题。
% 本文将就此从感知、通信以及计算三个角度展开分析讨论。
% 本章首先介绍移动群智感知场景下边缘计算中的资源调度问题;其次简要介绍当前国内外研究现状和现有工作的不足;再次阐述了本文研究的目的与意义;最后介绍论文的组织结构和层次关系。


% 随着 5G 通信标准的确立,物联网时代正式拉开了序幕。
% 种类繁多的移动设备、高度集成的传感器、以及万物互联的网络空间,促使移动群智感知应用对存储资源、网络资源和计算资源的需求达到了更高的境界。
% 云计算作为集中式服务模型,在支撑覆盖区域广、时效性强的移动群智感知应用时已渐显乏力。
% 边缘计算作为一种与云计算互补的分布式计算模型,如何利用其特性为移动群智感知服务提供基础支撑,是一个亟待解决的研究课题。
% 本章首先介绍移动群智感知场景下边缘计算中的资源调度问题;其次简要介绍当前国内外研究现状和现有工作的不足;再次阐述了本文研究的目的与意义;最后介绍论文的组织结构和层次关系。

%,种类繁多的非通信专用设备也可以接入互联网,并享受云计算服务带来的便利。
% 目前云计算中心的建设速度,已经无法满足网络中所有设备平等地享受云服务带来的便利。
% 因此,边缘计算作为一种新型计算模型,用来应对复杂网络空间中的海量服务需求。
% 为了保障各终端的服务质量,如何快速分配、调度网络空间中的计算资源、网络资源和『』,是一个亟待解决的研究课题。
% 这种新型计算模型不仅要保障这些终端设备的隐私、安全和服务质量,还需要平衡分配网络空间中的计算资源、存储资源和网络资源。

\section{研究背景}

\subsection{边缘计算}
边缘计算(Edge Computing)是与云计算(Cloud Computing)~\cite{DBLP:journals/cacm/ArmbrustFGJKKLPRSZ10}相辅相成的一种新型计算模型。
云计算作为集中式处理模型,需要将用户数据收集至部署有云平台的数据中心~\cite{DBLP:conf/icdcs/Montresor16},才能有效地服务于用户。
但在面临 万物互联的新型网络空间时,云计算服务将面临以下挑战:

1)\textbf{实时性不足}。
在物联网场景中,存在着许多强实时性的应用。在云计算中,数据需要不断在客户端和云端之间往返,导致响应时延增加、用户体验下降。
例如在智能驾驶应用中,云计算无法达到毫秒级的数据处理时延~\cite{DBLP:conf/cvpr/GeigerLU12}。
移动终端上的虚拟视觉(Virtual Reality,VR)框架 Furion~\cite{DBLP:conf/mobicom/LaiHCSD17}在探索过程中发现,仅靠云计算服务无法帮助移动终端获得高质量的实时 VR 服务。
而利用边缘计算将渲染服务卸载到边缘服务器中,MUVR~\cite{DBLP:conf/edge/LiG18}成功实现了一个低通信且稳定的多用户 VR 框架。

2)\textbf{带宽不足}。
在网络空间中,将边缘设备产生的数据传送至云计算中心,需要消耗大量的带宽资源。
Intel 在 2016 年的报告~\cite{DBLP:journals/micro/KatoTINTH15}指出,一辆智能驾驶汽车工作一天可以产生 4 太字节(Terabyte,TB) 的数据。
而一架波音787飞机在飞行途中,其数据产生速率达到了 5 吉字节(Gigabyte,GB)每秒~\cite{JCRD/shi17}。
如此巨大的数据量,利用云服务进行保存和处理,不仅需要耗费大量的带宽,且网络传输导致的时延也将导致计算服务毫无意义。

3)\textbf{能耗过高}。
云计算进入实践以来,研究者们更多关注数据中心中的能耗问题。
在美国,能源部对全美数据中心能耗的分析报告~\footnote{2016年美国数据中心能耗报告 https://eta.lbl.gov/publications/united-states-data-center-energy}表明,2006年度数据中心总能耗为610亿千瓦时,2014年为700亿千瓦时,而2020年,预计在节能技术的帮助下,能够将年度总能耗控制在730亿千瓦时左右。
据《中国数据中心能效研究报告》~\cite{whitebooks/cn15}显示(『白皮书引用格式』),我国2015年度数据中心能源消耗已经超越1000亿千瓦时。
报告同时指出,国管局已经将“绿色数据中心”列为“十二五”规划中的十大重点工程之一。
尽管学术界与工业界已经在绿色能源领域获得显著的成效,但是随着用户、应用数据的增加,全球数据中心的能耗仍在进一步上升。

4)\textbf{用户数据安全和隐私}。
在外物互联的网络空间中,边缘设备已经走入用户的私人生活空间。
例如家用摄像头、便携式移动设备、智能网联汽车,这些设备中都拥有大量的私人隐私数据。
一旦这些数据被上传至云端,用户隐私的泄露风险也会成倍增加。
目前,欧盟已经强制实施“通用数据保护条例”(General Data Protection Regulation,GDPR)来保护用户的隐私。
对于云计算公司,如何保护用户的数据安全和隐私安全,也更为重要。

为了避免云计算模型在物联网场景下的不足之处,2013年西北太平洋国家实验室的 Ryan LaMothe 提出一种新性计算模型概念——边缘计算。
2015年,欧洲电信标准协会~\footnote{https://www.etsi.org}(European Telecommunications Standards Institute,ETSI)在白皮书~\cite{hu2015mobile}中正式定义了边缘计算的概念。
边缘计算就是将应用任务在靠近数据源(例如移动设备、传感器、最终用户等)的资源上进行处理。
“边缘”主要相对于云计算中心而言,指代数据产生源到附近的任意计算、存储、网络资源。
同年,思科也推出了雾计算白皮书~\cite{computing2015internet}。
雾计算(Fog Computing)这一概念最早于2011年由Bonomi首次提出~\cite{bonomi2011connected}。
雾计算通过虚拟化架构将远端的云端服务迁移至本地节点,让高高在上的云服务更贴近用户,从而提高应用服务的访问效率和服务质量。
『虽然边缘计算和雾计算在指导思路上有相似处,都是将云端服务落地到更接近用户的网络边缘侧。
但是雾计算更多探讨的是实际应用的落地;边缘计算更倾向于研究边缘计算体系结构中的问题。
所以学术界更愿意探讨边缘计算下的本质问题。』

\begin{figure}[!h]
  \centering
  % \vspace{-1em}
  \includegraphics[width=440pt]{./figures/Sec_Intro/边缘计算基础架构.pdf}
  \vspace{-0.5em}
  \caption{边缘计算基础架构}
  % \vspace{-4em}
  \label{Figure_EC_Architecture}
\end{figure}

图~\ref{Figure_EC_Architecture}描绘了边缘计算的基础架构。
在云中心,拥有大量的计算资源、存储资源、服务资源。
然而边缘设备必须通过主干网络,才能使用云上的各种资源和服务。
但是,如果将云上的服务迁移至边缘网络中的边缘服务器(基站),将服务本地化后,又可以催生出新的应用范式。
例如,将车辆信息、交通信号信息、道路监控信息结合定位服务、导航服务,可以实现简单的车联网应用和智能驾驶服务。
通过融合蜂窝网络、WiFi、蓝牙网络、端到端通信,可以增加移动群智感知的数据获取量以满足更多应用需求。
相较于云计算模型,边缘计算具有3个明显的优势:

1)\textbf{大量临时数据不需要上传至云端服务器}。
在边缘计算中,应用程序可以利用边缘节点中的资源完成数据的存储、计算工作。
避免用户数据上传至云端服务器,为主干网络节省了大量的带宽。

2)\textbf{计算任务不再需要云计算中心的响应}。
在边缘计算中,云端服务可以卸载至资源丰富的边缘节点上。
更短的网络路径让应用服务的响应延时大大减少,不仅增加了响应能力,也提高了用户的使用体验。

3)\textbf{隐私数据保存在边缘设备,无需上传}。
由于用户隐私数据可以存储在边缘设备中而不是云端服务器,减少了隐私数据的传输路径,在一定程度上规避了隐私泄露的风险。

得益于这些优势,近年来边缘计算得到了学术界和工业界的肯定,并获得了突飞猛进的发展。
利用边缘计算,不仅可以在广袤的无线网络接入范围内提供更好、更快、更准确的信息技术服务和云计算能力;还可以在数据的边缘,利用富余的资源快速完成应用服务。
目前,边缘计算已经被采用在以下典型应用场景中:公共安全中的实时数据处理、智能网联车和自动驾驶、虚拟现实、工业物联网、智能家居以及智慧城市。
这些真实的应用场景与人类的生活、生产、娱乐息息相关。
在这些应用场景中,边缘计算支撑着物联网中海量的数据传输和处理工作,其中包括固定部署的无线传感网络产生的数据,移动群智感知网络产生的数据等。

% 并且这些应用还有一个共同的特点,它们的需要从大量的边缘设备中收集大规模的传感数据作为『数据分析』的基础。这种应用范式,也叫移动群智感知。


% \textbf{不对,不是从边缘设备中收集。这里不妨直说,边缘计算支撑物联网数据处理,包括固定部署的无线传感网络产生的数据,移动群智感知网络产生的数据等。}

\subsection{移动群智感知}

% \textbf{无线传感网络+为什么要有移动群智感知+端到端通信}

移动群智感知(Mobile Crowdsensing,MCS)是一种基于物联网的“以人为本”的感知模式。
其构想起源于2006年《连线》(Wired)杂志提出的众包(Crowdsensing)一词,旨在利用分布式解决方案将感知工作分配出去来共同完成应用任务或提供服务。
利用这一特性,移动群智感知可以完成个体无法实现的复杂环境下大规模动态社会感知任务,例如交通拥堵状态、城市空气质量监测等。
在早期,这些社会感知任务可以利用无线传感网络~\cite{DBLP:journals/cn/AkyildizSSC02}完成。
由于不同类型的感知任务需要部署不同功能的传感器甚至不同架构的传感网络,随着社会感知任务需求的增多和变更,静态的传感网络在部署和维护上都会耗费大量的人力成本和物力成本~\cite{CNKI/2006/WSNRen}。

图~\ref{Figure_MCS_Application}展示了移动群智感知的典型应用场景。
用户只需要提交所需要的任务需求,移动群智感知服务(例如交通服务、健康医疗服务、社会服务)会从目标区域内的传感器、移动设备、或车辆等具备感知功能和网络功能的设备上收集相关数据。
待数据汇总后,根据用户的需求,对数据进行处理和运算,最后将推断结果反馈给用户。

\begin{figure}[!h]
  \centering
  % \vspace{-1em}
  \includegraphics[width=440pt]{./figures/Sec_Intro/移动群智感知应用场景.pdf}
  \vspace{-0.5em}
  \caption{移动群智感知应用场景}
  % \vspace{-4em}
  \label{Figure_MCS_Application}
\end{figure}

移动群智感知利用众包的方式,将感知任务众包给用户的移动设备(例如手机、平板、智能手表等)作为基本感知设备。
这些移动设备集成了丰富的传感器,可以获取大量与设备所处环境相关的数字信息,例如环境光(光学传感器)、噪声(麦克风)、地理位置(GPS 传感器)、移动状态(陀螺仪、加速计)等。
除此之外,这些移动设备还可以利用自身优秀的通信能力快速交付数据,甚至利用本地计算资源进行数据处理。
将对于无线传感网络而言,移动群智感知应用的部署成本少、灵活性高,更适合复杂网络环境下的大规模动态社会感知任务。

除了移动设备内置了大量传感器之外,大量的现代交通工具中,雷达、摄像头、GPS、陀螺仪、加速计等设备也已经成为保障安全驾驶的必备传感器。
而物联网的飞速普适,也让更多的传感设备具备了网络连接功能和简单数据处理逻辑。
因此,移动群智感知可以收集到种类更多、信息更全的感知数据,从而实现更多创新型研究与应用。
西北工业大学利用校内学生的智能手机,在校区内实现了基于移动群智感知的噪声监测系统~\cite{CNKI/2014/CSNSYu}。
通过众包方式收集不同位置和不同时间的噪声污染数据,该系统利用离散数据重建出高精度的城市噪声时空分布地图,为城市噪声治理提供了可视化的监控平台。
文献~\cite{DBLP:conf/wcnc/AliAEJH12}利用车内传感器和路上行人手机内的传感器,将多个信息源混合在一起。
在不借助其它传感器和通信网络的前提下,搭建了智能交通拥堵检测系统。
Waze~\footnote{www.waze.com}是一款基于移动群智感知的导航服务。
在传统的允许用户帮助编辑图资的基础上,Waze 开创性的引入用户的 Facebook 和 Twitter 消息,实时更新周边商户信息和道路交通状态。
通过整合用户数据,Waze 比传统的导航服务内容更加丰富、信息也更精准。

在物联网环境下,人类社会和网络空间中已经包含了海量的、具备多元化传感功能的智能设备。
众多的潜在参与者、饱满的空间覆盖、便利的数字信息提取,让移动群智感知成为了环境研究~\cite{DBLP:conf/sensys/DuttaAKMMWW09}、人类社会关系研究~\cite{DBLP:conf/globecom/AslIAM13}、智慧城市建设研究~\cite{DBLP:journals/cm/WangZWCHM16}的基础工具。
% 相较于无线传感网络,移动群智感知不仅能够灵活部署,还能够节省成本,成为了许多科技公司青睐的数据收集方案。
% 同时,这些以人为本的海量感知数据,能够帮助应用程序提供更高品质的服务。

\subsection{边缘计算与移动群智感知应用}

% \textbf{边缘计算在移动群智感知中的作用}

% \textbf{举例:MCS+Edge Computing}

早期,移动群智感知所收集的传感数据内容单一,结构简单(例如天气数据、噪声数据)。
但随着应用范围的不断扩展,感知数据已经逐步走向多样化、异构化的发展趋势。
尤其是在社会公共安全领域,视频数据已成为必不可少的感知数据。
例如武汉市的“雪亮工程”,与2019年中期将在全市部署150万个公共安全视频监控。
这样庞大监控网络无论是在数据备份还是在实时数据分析上,都对现有的计算模型提出了巨大的挑战。
而2018年因安全事故被推上焦点的滴滴出行,也在新版客户端内加入了服务时段录音、录像功能。
目前,滴滴出行仅在广州市的日均服务次数已超过30万次,乘客和司机的手机每天所收集的感知数据容量,已经达到TB级别~\cite{DBLP:conf/aaai/Yao0KTJLGYL18}。
对于如此大量的录音、录像数据的实时处理,边缘计算提供了新的处理方法。

文献~\cite{DBLP:conf/edge/SunLS17}基于边缘计算,在前端或者靠近视频源的位置,对视频内容进行预处理,以检测视频传感器是否故障,并基于内容特征对视频的质量进行动态调整。
基于视频的前置分析处理,文献~\cite{DBLP:journals/iotj/ZhangZSZ18}对视频分析服务进行自动化部署,并让边缘设备之间可以协同处理图像数据,实现了基于视频监控的自动报警系统。
同时,通过与周边摄像头联动,该系统可以在部署范围内对指定目标的实时追踪。
针对滴滴出行的安全问题,韦恩州立大学的研究者利用乘客手机中的传感器,监控车辆行驶状态和声音信号,对车辆及人身安全做出评估与预测~\cite{DBLP:conf/edge/LiuZQS18}。
这些研究工作,都充分挖掘了边缘网络中的可利用资源,利用移动群智感知和边缘计算,创造出和安全息息相关的强实时性服务。

% \textbf{资源管理分配与任务调度}

\begin{figure}[!h]
  \centering
  % \vspace{-1em}
  \includegraphics[width=360pt]{./figures/Sec_Intro/边缘计算的移动群智感知.pdf}
  \vspace{-0.5em}
  \caption{边缘计算支撑的移动群智感知}
  % \vspace{-4em}
  \label{Figure_MCS_with_EC}
\end{figure}

如图~\ref{Figure_MCS_with_EC}所示,由边缘计算支撑的移动群智感知任务的执行可以简单划分为7步。
1)移动群智感知用户提交任务需求到云端;
2)云端服务器将移动群智感知任务拆分后,利用主干网络将任务信息发送至边缘服务器;
3)边缘服务器利用蜂窝网络将任务分发至感知设备,除此之外,边缘设备也可以利用 WiFi、蓝牙、端到端通信等一切可利用的通信手段将任务分发出去;
4)边缘设备完成感知任务后,利用可利用的网络资源将感知数据发送回部署有感知服务的边缘节点;
5)边缘服务器汇集感知数据后,根据自身资源对数据进行预处理,例如数据的解码、去冗余、分类、整合等;
6)边缘服务器在移动群智感知任务达到要求或任务生命周期结束时,将结果数据上传回云端服务器;
7)云端服务器收到所有边缘服务器的返回数据后,计算、整理成最终结果,并将最终结果交付给提交任务的用户。

在边缘计算中,边缘服务器代替云端服务器分发任务,利用边缘网络中的各种通信手段,可以召集更多的数据产生者。
同时,边缘服务器利用自身的计算资源,执行数据去冗余、特征提取、数据打包和简单运算等预处理工作,为云端服务器减少工作量,也为主干网络减少了感知数据造成的流量压力。

% 在边缘计算模型中,为了保障移动群智感知服务的稳定性和服务质量,必须针对移动群智感知的工作阶段对边缘网络中的各种资源进行调度和管理。
% 移动群智感知应用的工作流程可以简单划分为四步~\cite{zh_cn:shi}:任务分发,传感器感知,数据上传和数据处理。
% 其中,任务分发是指根据移动群智感知任务的覆盖区域、执行任务的用户数量、以及任务的持续时间,快速将感知任务发送到合适的移动设备上。
% 移动设备收到感知任务后,通知传感器工作并产生感知数据。
% 然后,移动设备利用自身的通信能力,将传感数据上传到等待数据处理的地方(例如云服务器)。
% 最后,对这些数据进行去重、映射、分析,从而推断出符合共同利益的结果~\cite{DBLP:journals/cm/GantiYL11}。

目前,智慧城市建设已经成为当前信息领域的前沿研究热点。
智慧城市的构建,需要对城市主体和物理世界对象进行大量的数字信息挖掘,在数据处理和学习之后,从中获得人类活动和城市运作规律之间的潜在联系,并将其应用到各种服务和创新中。
在智慧城市场景下,移动群智感知已经成为数据信息收集的主要手段。
2016年,阿里云提出了“城市大脑”~\footnote{阿里云城市大脑 https://et.aliyun.com/brain/city},经过两年的发展,如今的“城市大脑2.0”已经在杭州市实现了智能交通管制和实时车辆查找。
2017年,Google 的 Sidewalk 实验室也启动了 Quayside~\footnote{Sidewalk Toronto https://sidewalktoronto.ca} 高科技新区项目。
随着城市的发展,网络空间中的感知设备正处在多样化、异构化的发展历程,所以利用边缘计算将数据在网络边缘进行处理是智慧城市的优先解决方案,也是移动群智感知应用的重要支撑基础。

\subsection{面向移动群智感知的边缘计算研究及其挑战}


边缘计算利用自身的优势弥补了云计算模型中海量数据传输延时高,隐私安全处理敏感等问题,使得边缘计算更适用于新兴的物联网应用场景~\cite{DBLP:journals/cm/SunA16}。
将云计算能力扩展至距离终端最近的边缘侧,以满足“大连接,低延时,大带宽”的新需求。
近年来,业界和学术界已经着手将边缘计算结合到实际生产与应用中。
这些应用,大多数与智慧城市建设密切相关。
例如,文献~\cite{DBLP:journals/cm/GantiYL11} 在车辆上部署 GPS 和加速度计,可以在车辆行驶过程中定位城市中的坑洼路面。
科特迪瓦共和国的一个非政府环境组织在非洲阿比让市电信公司的帮助下,在阿比让市利用25个蜂窝基站和移动群智感知应用~\cite{DBLP:conf/huc/ZhangXWC14},向阿比让市民即时提供空气质量感应结果。
\textbf{还有一个消防的例子}

这些研究工作与实际应用,和移动群智感知以及边缘计算都密不可分,这也揭示了面向移动群智感知的边缘计算模型在未来城市建设中的重要地位。
在边缘计算模型中,为了保障移动群智感知服务的稳定性和服务质量,必须针对移动群智感知的工作阶段对边缘网络中的各种资源进行调度和管理。
移动群智感知应用的工作流程可以简单划分为四步~\cite{zh_cn:shi}:任务分发、传感器感知、数据上传和数据处理。
如何针对移动群智感知应用所处的不同阶段,将边缘设备有机地组织起来,合理地调度边缘网络中的计算资源、网络资源、并合理分配云端资源,都为边缘计算技术实际落地带来了新的挑战。
目前,面向移动群智感知的边缘计算中主要研究挑战有三点。

第一个挑战是\textbf{边缘计算中面向移动群智感知的任务调度}。
在移动群智感知应用中,需要大量的参与者和边缘设备为移动群智感知服务提供输入数据。
然而,边缘设备自身并不具备移动群智感知服务的发现能力。
因此,移动群智感知服务需要将数据采集任务利用网络分发到合适的边缘设备上,雇佣这些智能设备采集传感数据,并将传感数据收集后进行统一处理。
在任务分发阶段,由于缺少集中式管理策略,移动群智感知任务的发起者无法直观获得任务部署的覆盖范围和执行数量,只能依靠感知数据的收集状态来判断感知应用的执行进度和效率。
所以,如何利用边缘网络让更多的边缘设备参与移动群智感知并对感知应用的执行状态进行预测,是提高移动群智感知应用覆盖率和服务质量的关键因素之一。

第二个挑战是\textbf{边缘计算中面向移动群智感知的网络资源调度}。
近年来,边缘设备的性能越来越强大,其可使用的网络接入技术也逐渐多样化。
在早期的移动群智感知研究工作中,移动群智感知的参与设备多使用蜂窝通信~\cite{DBLP:conf/globecom/ZhangJLLC16,DBLP:conf/icdcs/XiaoWHHH16}来上传感知数据。
随后,Karaliopoulos 等人的研究~\cite{DBLP:conf/infocom/KaraliopoulosTK15}发现,在传感数据上传过程中利用端到端通信,可以减少边缘设备的能耗成本的网络通信成本,并以此来为移动群智感知服务雇佣更多的参与者。
同时,文献~\cite{DBLP:journals/tpds/ZhaoMTL15}通过研究机会式通信网络中的数据包转发规律,发现在边缘网络中利用数据包融合技术可以有效减少数据传输延时。
随着 5G 和物联网技术的飞速发展,边缘设备可以利用蓝牙、WiFi 、近场通信、端到端通信等技术,在小范围内快速交换数据信息。
在复杂网络环境中,优化移动群智感知数据的传输路径、减少通信时延,也是提高移动群智感知服务质量的关键因素。

第三个挑战是\textbf{边缘计算中面向移动群智感知的边缘服务调度}。
边缘计算中,服务端从云端转变到网络边缘侧,如何让这些边缘设备快速获取自身周边存在的服务,是边缘服务在部署时的一个重要问题。
在边缘计算场景中,用户和边缘设备的参与方式均为动态过程,例如车联网、移动群智感知应用中的参与者等。
这些设备和参与者随着地理位置和网络环境的改变,都会在网络中引起拓扑结构的变化。如何针对变化拓扑部署并迁移边缘服务,也是边缘计算在网络层需要考虑的难题。
再者,边缘设备会产生大量的传感数据,如何平衡边缘网络中的数据负载,合理调度边缘网络中的数据流量,尽快将数据交于边缘服务节点,也是提高移动群智感知服务质量时需要解决的问题。
所以,在动态场景下,边缘服务的部署和迁移策略也成为优化移动群智感知服务的关键点。



% 针对移动群智感知的任务分发阶段,需要借助边缘服务节点将任务快速部署到

% 针对移动群智感知的数据采集、传输、以及处理,需要对边缘网络中的网络资源、计算资源、存储资源进行合理的规划和调度。
% 但是,物联网网络空间中网络的『复杂度』,终端设备的多样化、异构化,和终端设备的移动特征,都为边缘计算造成了诸多挑战。

% \textbf{挑战三:资源的动态调度。}

% 第一,边缘设备

% 数据可分布(任务分发+数据采集)\\
% 网络可分布(SDN 中网络管理和路径规划)\\
% 资源可分布(边缘服务节点弹性部署)\\

% \textbf{挑战二:编程模型的支撑。}
% 在云计算场景中,应用程序可以在目标平台上编写和调试,最终在云平台上执行。
% 所有的基础设施对于用户而言都是透明化管理。
% 例如亚马逊的 Lambda 计算服务\footnote{aws.amazon.com/cn/lambda},可实现任务应用代码的自适应部署和运行。
% 使用虚拟化手段,例如虚拟机~\cite{}、容器~\cite{}(Docker),依然可以做到应用服务的快速部署和迁移。
% 然而,在边缘计算中,应用或服务需要在进行更小粒度的切分后,再部署至合适的资源节点上。
% \textbf{这里加一点文献讲 code offloading}
% 代码卸载(code offloading)\\




\section{国内外研究现状}

面向移动群智感知的边缘计算资源调度研究有很多。
基于前文所提到的主要研究挑战,借助边缘计算提升移动群智感知服务质量的研究方法可以分为三类。
1)网络资源调度优化。通过合理分配无线网络资源、优化链路以减少主干网络资源占用并减少感知数据传输网络传输时延。
2)任务卸载调度优化。利用计算任务卸载技术减少服务时延和能耗需求,提高移动群智感知的用户体验并减少成本。
3)边缘服务部署策略优化。在光域范围内合理部署边缘服务器以支撑城市范围内的移动群智感知应用并保障感知质量。

\subsection{网络资源调度优化}

在网络边缘侧,大量的移动设备通常借助无线网络接入网络服务,达到数据互通的目的。
在物联网环境下,边缘设备可以使用蜂窝网络、蓝牙、WiFi、端到端通信、车载网络等多种无线通信手段进行数据交互。
另一方面,软件定义网络(Software-Defined Network,SDN)和网络功能虚拟化(Network Function Virturlization,NFV)技术的兴起,也能够在边缘网络中更加便捷的管理网络资源、进行资源分配。

文献~\cite{DBLP:conf/infocom/RimalVM16}的工作表明,使用边缘计算和云计算相结合的解决方案,可以在不影响网络性能的情况下减少数据包延迟。
文献~\cite{DBLP:conf/apsys/HuGHWACPS16}在WiFi和LTE网络中,通过边缘计算平台进行加载可以大大提高计算密集型应用的延迟,例如增强现实类应用和认知辅助类应用。
文献~\cite{gao2015cloudlets}尝试将云上服务卸载到边缘服务器上,当客户端使用Wi-Fi或4G网络时,相较于云计算而言,实验结果表明服务卸载后可以降低50\%以上的应用响应时延,同时还能帮助移动设备节省最多40\%的能耗。
文献~\cite{DBLP:conf/saso/MehtaTKTKE16}在边缘网络中部署小型的数据中心,能够更好地支撑带宽要求较高的应用,相对于在云上数据中心租用服务的方法,在网络边缘层部署服务能够节省60\%左右的成本。
文献~\cite{DBLP:journals/cm/BastugBD14}利用在基站和移动设备上主动缓存用户数据,结合端到端通信和用户的社交网络关系图,在完成数据交付的同时并成功减少了主干网络上22\%的数据流量。
文献~\cite{DBLP:conf/mwcn/OrsiniBL15}将端到端通信引入边缘计算的基础设施组件,为边缘设备提供更多的网络资源,并设计出一种安全可靠、适应性强的基础框架来满足移动应用的弹性网络带宽需求。

在移动群智感知应用中,原始数据主要来源于边缘设备的传感器。
在该场景下,用户和边缘设备多处于运动状态中~\cite{DBLP:journals/jsac/LyuNTLWGP17},例如智能车联网中的车辆、移动群智感知应用中的参与者。
利用这种特性,文献~\cite{DBLP:conf/itsc/PiaoA17},通过安装在车辆上的智能手机中嵌入的运动传感器来监控城市路面损坏状况。
文献~\cite{DBLP:journals/tase/PengGXGY18}借助移动群智感知利用移动用户提供的传感数据自动更新城市数字地图。
在这类应用中,设备和参与者随着地理位置和所处网络环境的改变,都会在网络中引起网络拓扑结构的变化。
然而,移动群智感知服务往往以连续的方式收集参与者的信息并提供反馈结果。
所以在边缘计算实际应用中,随着参与者或者边缘设备状态的改变,边缘网络需要时刻对边缘设备的网络状态进行维护,并合理调度边缘网络中的数据流。

在早期移动群智感知研究中,大多数研究者假设移动群智感知的参与者使用蜂窝通信上传感知数据。
但是一些研究人员发现,蜂窝网络通信所产生的费用和能耗较高,可能会使移动用户不愿意自愿作为数据采集者~\cite{DBLP:conf/globecom/ZhangJLLC16,DBLP:conf/icdcs/XiaoWHHH16}。
于是研究者开始讨论怎样利用移动设备的端到端通信来避免蜂窝网络通信造成的损失。
文献~\cite{DBLP:conf/infocom/KaraliopoulosTK15}研究了基于端到端通信网络中的移动群智感知志愿者的招募问题。
文献~\cite{DBLP:journals/puc/WangLL17}验证了端到端通信方式从移动设备收集感知数据的可行性,并讨论如何招募移动用户作为志愿者。 
文献~\cite{DBLP:journals/tpds/ZhaoMTL15}分析了端到端机会式通信网络中数据包转发的传播规律,并通过融合这些数据包来减少传输延迟,提高移动群智感知的数据收集效率。

基于这一趋势,研究者开始关注移动群智感知应用中端到端通信所形成的机会式通信网络中数据转发的效率问题。
文献~\cite{DBLP:conf/wcnc/QinF13}分析了机会网络中基于网络编码的数据传递性能。
文献~\cite{DBLP:journals/twc/LiW14}研究端到端通信中,远程节点发送消息的时间开销问题,并提出了端到端通信中消息传播的空间范围和时间范围限制。
文献~\cite{DBLP:journals/winet/ZhaoMLT18}基于端到端通信网络,构建了机会式通信下感知数据收集的时延分析框架。

\subsection{任务卸载调度优化}

在计算机科学中,计算卸载~\cite{DBLP:journals/network/MaZZWP13}是指将计算任务迁移到计算能力充沛的外部设备中,例如网格、集群或者云服务器。
将计算任务迁移到资源更丰富的服务器,可以有效弥补移动设备能力不足的缺陷。
使用计算卸载技术,性能较弱的移动设备也可以将计算指令迁移~\cite{DBLP:journals/monet/KumarLLB13}到云端服务器并快速获得任务结果。

计算卸载的主要手段分为两种。
一种是在云端构建完整的同构运行时环境,让应用能够直接迁移到云上;另一种是代码卸载,可以将计算任务更细粒度的迁移至异构平台,减少本地的任务执行时间和能耗。
但是,代码卸载要达到实用意义有两个先决条件~\cite{Flores:2017bv}: 1)设备与远端的传输时延较低;2)代码卸载到远端所需要的成本应小于本地处理任务所需的成本。
因此,代码卸载执行之前必须经过合理的决策判断,文献~\cite{Kosta:2012cj}发现,不同的决策方法会产生不同的收益,在个别情况下甚至会产生负面影响。
『文献~\cite{Chen:2016bd,Baier:2012hb,Han:2012dl}分别从三个层面研究了代码卸载决策算法』。

计算密集型应用的出现,如3D游戏等,会持续使用大量的计算资源。
移动设备收到能耗和性能的制约,已经无法满足日益增长的应用需求。
因此,移动设备不得不牺牲有限的网络资源和能耗储备,将计算任务卸载到满足应用需求的远端设备上~\cite{DBLP:conf/ispa/KovachevYK12}。
例如,在移动电话上编辑视频剪辑需要大量的计算能力与能耗,无法达到台式电脑上流畅的用户体验。
为了应对这些限制,许多研究人员研究如何将计算卸载到资源丰富的云计算平台~\cite{DBLP:conf/secon/HassanXWC15,DBLP:journals/pervasive/SatyanarayananBCD09,DBLP:journals/computer/KumarL10}。

Farris 等人~\cite{DBLP:journals/fgcs/FarrisMNAI17}构建了移动物联网即服务的边缘计算模型,将边缘服务器和移动设备耦合成动态的物联网边缘云。通过和公有云服务商合作,建立边缘计算服务联盟。同时将服务供应商的分组问题抽象为合作博弈的纳什均衡问题,将博弈问题的解作为任务卸载的依据。
Sardellitti等人~\cite{DBLP:journals/tsipn/SardellittiSB15}首先研究了单移动用户将计算任务卸载到多个边缘服务器上的任务调度问题,并建立了非凸优化模型。当移动用户增加时,作者将调度方法改进为逐次图逼近算法,以获取原始非凸优化的最优联合解。该算法能够在保障每个用户服务质量的前提下,实现所有移动用户的总能耗最小。
Nastic等人~\cite{DBLP:conf/edge/NasticTD16}在物联网中加入中间件以提供多层次的服务,将中间件作为轻量级资源部署在边缘设备上,使中间件可以执行指定的应用,并通过 API 的方式对任务卸载调度进行动态管理。
Takahashi 等人~\cite{DBLP:conf/mobilecloud/TakahashiTK15}将任务卸载应用于Web浏览加速服务。通过将 Web 应用程序卸载到边缘服务器上,边缘服务器收集大量的 Web 内容并根据用户的请求,快速将 Web 内容渲染成最终结果。移动设备可以直接拉去渲染结果,直接显示在屏幕上,节省渲染的时间和能耗。对于 Web 上的音视频数据,依然通过客户端硬件进行解码播放。
Liu等人~\cite{DBLP:conf/edge/LiuWB16}利用网络边缘中的计算资源和存储资源,让 WiFi 接入点可以获取并处理接入设备的上下文信息,在任务卸载的同时,为连入终端提供预先提供的自定义服务。

\subsection{边缘服务部署策略优化}

物联网能够让成千上万的设备进行交互和通信,产生并交换真实世界中各种对象数字信息。
这些设备产生并交换现实世界中的事物产生的数据。
在过去的几十年里,利用信息处理和抽象技术~\cite{DBLP:journals/iotj/GanzPBC15},物联网已经整合了网络世界和物理世界。
% 真实世界的信息化,提高了人类生活的智能程度。
目前,物联网网络空间~\cite{DBLP:journals/cm/WangYXJD17}中的设备的包含大量用户数据,使得物联网应用有了宽泛的应用场景,涉及人类生活的各个方面。

为了弥补物联网设备资源受限的缺陷,通常引入云计算来进行数据处理。
最近的研究~\cite{DBLP:journals/iotj/RazzaqueMPC16}指出了云计算对物联网应用的相对较大的延迟和不可预测的抖动干扰。
作为一种系统级架构,边缘计算旨在将存储、计算和网络的资源和服务分布到任何地方,以确保这些服务的服务质量~\cite{AI201877}。
因此,为了向海量的终端用户提供高效安全的服务,边缘计算成为未来的物联网建设的首选支撑技术~\cite{DBLP:journals/iotj/LinYZYZZ17}。
为了支持物联网应用的弹性资源需求,Brogi 等人~\cite{DBLP:Journals/IOTj/BroGif17}调查边缘计算增强物联网的开发能力。
文献~\cite{DBLP:Journals/IOTj/BasuDanlS17}在公共汽车上部署了边缘服务,通过车辆和乘客来感知并监控道路状况。
同时,车辆可以与边缘服务一起使用,以实现这些车辆通信和计算资源\cite{DBLP:journals/tvt/HouLCWJC16}的最佳利用。
对于紧急情况和医疗保健,\cite{7466912}在上传过程中使用边缘计算对传感数据进行预处理,以归档对延迟敏感和隐私敏感的要求。
文献~\cite{7835115}构建了一个基于边缘计算的人脸识别框架,以解决一些安全和隐私问题。
由于边缘服务的灵活部署,边缘计算可以提取有趣的模式或知识,以增强物联网应用中的数据挖掘,为物联网应用提供了实时决策、低延迟、安全改进的商业价值。
% 总之,边缘计算为受限网络中的物联网应用提供了实时决策、低延迟、安全改进的商业价值。

Satria 等人~\cite{DBLP:journals/fgcs/SatriaPJ17}研究了边缘服务器如何利用交叉覆盖和动态中继方法,来避免边缘服务器过载或故障时导致的服务失效。
T{\"{a}}rneberg 等人~\cite{DBLP:journals/fgcs/TarnebergMWTEKE17}对移动网络空间中的数据中心、网络、应用和用户进行建模,抽象出移动网络中的资源管理模型和应用部署模型,用以评估应用部署的成本,并针对最小部署成本做出合适的资源调度分配。
Chen 等人~\cite{DBLP:journals/ton/ChenJLF16}借助分布式的博弈论方法,设计了一种高效的计算卸载模型。由于无线网络中的信道干扰问题,多移动用户使用相同的无线信道卸载计算任务会引起带宽资源失衡。作者首先证明了无线网络中多用户计算卸载时的资源竞争是一个 NP 问题。然后借助纳什均衡和收敛时间的上限,实现了高效的分布式计算卸载算法。
Zhang 等人~\cite{DBLP:conf/rndm/ZhangMLV016}将边缘计算应用到车载网络中,设计了一种车联网中的计算卸载框架,利用边缘服务器帮助智能车完成大型计算任务。在该框架中,使用了契约理论方法进行任务卸载决策。通过优先匹配最优契约,最大化边缘服务提供商的收益,同时针对边缘服务器的有限资源进行计算资源分配,提高边缘服务器上的资源利用率。
Habak 等人~\cite{DBLP:conf/IEEEcloud/HabakAHZ15}设计了一种“微云”系统,将相互通信范围内的移动设备构建成一个可自配置的动态移动云。“微云”中的移动设备共享自身的容量、能耗等信息,动态协调并分配可以使用的计算能力、存储能力、网络带宽等。利用这种机制,可以减少动态行为所带给系统的波动。通过将计算任务卸载到云端,平摊移动云中的计算负载,最终实现“微云”系统中的群体利益最大化。


% 1)整合网络
% 首先,本文总结边缘计算的体系结构建立。
% 其次,结合移动群智感知,介绍目前边缘网络中的网络与通信管理手段。
% 然后,总结了移动群智感知下,边缘服务的卸载方法。
% 首先,整合边缘网络中各种无线通信手段,增加移动群智感知应用的参与者。
% 其次,利用边缘计算的服务卸载功能,减少系统时延,优化移动群智感知应用的用户体验。
% 最后,利用边缘计算的分布式特征,针对移动群智感知应用的特殊需求进行合理的资源分配。
% 本节主要从这3个方面来介绍边缘计算中的资源调度研究现状。

% \textbf{要引用的文章}
% \cite{DBLP:journals/jsac/LyuNTLWGP17}

\subsection{现有研究的不足}

% \textbf{总括:面向移动群智感知的边缘计算资源调度,现有研究有哪些不足(感知、通信、数据处理)}

上述国内外的主要研究工作,主要围绕边缘计算展开,并让其能够更好的支撑移动群智感知应用。
但是,针对移动群智感知的三个重要过程:感知、通信和数据处理,仍然有一些关键问题没有得到深入的研究。

% 综合上述国内外的主要研究工作,面向移动群智感知的边缘计算资源调度研究还存在如下问题。

\textbf{第一,移动设备的运动状态以及端到端通信对群智感知任务的影响。}
对于移动群智感知的感知过程,任务的发起者更关注的是感知任务的覆盖范围、最终收集的感知数据量和执行任务所需要的成本。
为了进一步扩大移动群智感知的覆盖范围并提高数据采集量,科研人员尝试在边缘网络中部署更多的边缘服务器来加快任务的传播;引入端到端通信来帮助感知任务的扩散;改进激励机制吸引更多的参与者。
文献~\cite{}也发现在边缘网络中利用边缘服务器分发感知任务,也能加快任务部署并提到任务覆盖率;文献~\cite{}已经验证终端设备之间的端到端通信可以提高感知任务的部署效率和感知数据的收集效率,并且在相同成本下可以增加感知任务的部署量。
尽管这些研究都使用了移动设备来作为数据采集的基础单位,但是它们并没有深入考虑这些移动设备的运动特征对移动群智感知应用带来的影响。
尤其是在边缘网络中,交叉覆盖着不同结构的无线网络,移动设备会随着位置的变化而改变网络接入点,导致网络状态的改变。
另一方面,鉴于端到端通信的引入,终端设备的移动行为让机会式通信网络中的数据交互更加的频繁,并且不同的运动模式对感知任务部署效率和感知数据收集效率都有不同的影响效果。
因此,探讨终端设备的移动特征和端到端通信对移动群智感知中任务覆盖和数据收集的影响,也是利用边缘计算更好地支撑移动群智感知的重要研究内容。
% 因此,本文针对边缘计算支撑下的移动群智感知应用,研究终端设备的移动特征对移动群智感知中任务覆盖和数据收集的影响,并以数学建模的方式来解释终端设备移动特征的对移动群智感知的作用原理,并提出相关的优化方案。


% \textbf{第一,边缘计算场景下的感知质量缺少系统性研究。}
% 目前现有的研究工作主要利用边缘计算中的网络资源调度、计算任务卸载,以达到减少移动群智感知应用网络时延、增加移动群智感知参与者的目的。
% 虽然这些研究都能在不同层面对移动群智感知应用做出优化,但是并没有可用的参照体系来比较不同方法的优化质量。
% 只能单一的比较时延增减、或者能耗增减。
% 因此,需要对移动群智感知的服务质量进行合理的定义,并纳入不同资源进行定量分析,作为不同优化策略好坏的评判标准,并指出不同场景下感知质量优化的关键资源类型。

\textbf{第二,端到端通信和计算任务卸载对移动群智感知的影响。}
『计算任务卸载』『SDN』
随着 5G 通信标准的完善,核心网服务正逐步从集中式部署向分布式部署转移,以实现低时延、高带宽的端到端连接。
同时,软件定义网络的不断积累和网络功能虚拟化技术的快速发展,移动群智感知可以在边缘网络中更加方便的管理移动设备的网络资源并将感知数据的计算卸载到边缘服务器上。
但是,在软件定义网络中,由于存储转发规则的三态内容可寻址存储器(Ternary Content Addressable Memory,TCAM)成本昂贵,因此无法避免规则空间的限制~\cite{}。
所以,利用软件定义网络管理大量的终端设备本身具有一定的难度。
另一方面,为了加速感知数据的计算工作,许多研究~\cite{}将感知数据的处理工作卸载到边缘服务器上,而计算卸载适用与否本身也存在着与网络时延相关的前置条件。
所以,在边缘计算中运用软件定义网络虽然能够更加方便的管理移动设备的网络结构,但是也会对移动群智感知中的计算任务卸载带来影响。
如何利用利用软件定义网络分配给移动设备『分配网络资源并规划计算任务卸载』,也是优化移动群智感知效率的重要研究内容。


% \textbf{第二,受限的网络资源会影响任务卸载决策。}
% 利用边缘计算支撑移动群智感知应用,可以将数据预处理服务部署在移动节点上,以此将移动设备上的数据处理工作卸载到边缘服务器上,从而节省移动设备的能耗,降低移动用户的成本。
% 无论是以移动设备的全体能耗最小化为目标,还是以边缘服务提供商的利益最大化为目标,这些任务卸载决策算法,都忽略了边缘网络中的时延限制和带宽限制,并且只考虑了单一的通信手段。
% 尽管很多研究将感知数据的预处理卸载到边缘服务器上,并以移动设备的全体能耗最小化或边缘服务提供商的利益最大化为目标进行任务卸载决策。
% 除此之外,移动设备可以使用多种网络通信方式进行数据传输。
% 所以任务卸载算法不仅考虑到收益问题,还要规划好通信方式并选择合适的通信链路。

% 无论是移动群智感知还是边缘计算场景中,都遍布着大量的移动用户。
% 这也意味着大量的智能设备都处于移动状态。
% 然而大部分的研究在考虑资源调度时,并没有考虑设备的运动特征所带来的影响。
% 例如移动设备从某个移动网络中进入另一个移动网络,网络抖动或者链路变化所带来的影响。
% 这些连锁反应也会影响任务卸载的收益。
% 另一方面,在移动群智感知应用中,设备的移动亦会改变感知数据的覆盖范围,影响感知质量。
% 这些都是现有研究缺乏的内容。

% 在软件定义网络( SDN )中,规则空间是由三进制内容可寻址存储器( TCAM )引起的不可避免的限制。当负载变得更受欢迎时,负载数量的急剧增加会影响链路调度和客户的服务质量( QoS )。因此,SDN中的链路调度必须考虑规则空间、带宽和QoS的约束。为了优化移动计算中的能耗,我们构建了一个模型来理解负载决策、延迟和能耗之间的关系。我们实现了一个两阶段算法,通过使用交换机中的规则空间和链路中的带宽来调度链路。通过对比评估,验证了算法的可行性。当SDN网络中规则空间的使用不充分时,与最佳解决方案相比,该算法可以将超过90 \%的能量效率存档。

\textbf{第三,边缘服务部署对移动群智感知的影响。}
在边缘计算中,通常租用基站作为边缘服务器来部署相应的数据处理服务来加速感知数据处理~\cite{},以提高移动群智感知的效率并减少主干网络中的感知数据流量。
在物理世界中,虽然已有大量的基站可以作为边缘服务器,但是对于涉及智慧城市应用的大规模移动群智感知而言,部署的边缘服务器过多,不仅会造成覆盖区域的重复,还会大大提高移动群智感知应用的执行成本。
因此,在大范围移动群智感知应用中,通常会利用人类的社交网络特性来选择边缘服务器。
例如文献~\cite{}\textbf{举例}。
然而,人类的社交网络通常会根据参与人自身行为发生变动,导致部署方案的效率发生变化。
因此,对于智慧城市类型的移动群智感知,迫切地需要一种稳定的边缘服务部署和调度方法,在不同需求下保障移动群智感知应用的效率,并减少移动群智感知的执行成本。

% 包含『关键技术』和『应用举例』

\section{研究意义与目的}

针对上述面向移动群智感知的边缘计算研究挑战和国内外研究现状分析,本文围绕移动群智感知的服务质量优化和边缘计算资源调度机制展开研究。如图~\ref{Figure_Re_Part},本文主要提出以下创新性理论和方法。

\begin{figure}[!h]
  \centering
  % \vspace{-1em}
  \includegraphics[width=380pt]{./figures/Sec_Intro/研究内容和各部分之间的联系.pdf}
  \vspace{-0.5em}
  \caption{研究内容和各部分之间的联系}
  % \vspace{-4em}
  \label{Figure_Re_Part}
\end{figure}

\textbf{1)边缘网络中基于感知质量的任务调度机制。}
为了让移动群智感知实现更大的范围覆盖并收集更多的传感数据,可以利用边缘计算将移动群智感知任务分发给更多、更远的边缘设备。
本文通过在边缘服务器上部署任务分发服务和数据收集服务,并引入端到端通信来提高任务分发和数据收集的效率。
通过量化边缘计算中的资源分配,本文建立了移动群智感知的感知质量分析模型。
基于这一模型,可以利用数学的方式来描述移动群智感知应用中的两个关键过程——任务分发过程和数据收集过程。
同时该模型可以反映出边缘计算中不同类型资源对感知质量的影响系数。
基于此研究,本文针对移动群智感知任务的执行过程提出了基于服务质量的资源调度机制。

\textbf{2)边缘网络中基于任务卸载的链路调度机制。}
移动群智感知应用中,移动设备可以将感知数据的处理任务卸载到已部署相关服务的边缘服务器上。
对于智能设备(例如智能手机)而言,感知数据可以在本地进行处理后再上传至边缘服务器。
但是对于摄像头或者可穿戴设备而言,往往需要将计算任务卸载至边缘服务器以减少数据处理时延。
但是在边缘计算中,边缘设备所处的网络环境往往处于异构且不稳定的状态,因此在计算任务卸载决策时,还需要考虑网络资源的限制和额外的能耗开销。
因此,本文研究了边缘网络中移动群智感知计算任务卸载和网络状态的关系,提出了计算任务卸载和网络链路的联合调度算法。
在最大化计算任务卸载收益的基础上,保障网络中各链路的负载均衡。

\textbf{3)边缘网络中基于社会网络的边缘服务调度机制。}
% 在万物互联的场景下,移动群智感知和信息物理系统(Cyber Physical Systems,CPS)的结合越来越紧密。
不少移动群智感知应用已经结合了移动用户的社交网络来优化移动群智感知的感知质量。
契合移动群智感知在智慧城市建设中的重要作用,本文在城市级别范围内利用社会网络和端到端通信来优化大规模移动群智感知的感知质量。
本文考虑公共交通行为中乘客的流动特征,结合端到端通信技术,旨在研究城市移动群智感知中边缘服务的部署决策和移动群智感知执行成本之间的关系。
基于该研究,提出了基于社会网络的边缘服务的部署决策算法,在感知质量最大化的前提下,求解成本最小的服务部署策略。

\section{论文组织结构}

% 论文组织结构如图~\ref{}所示。

本文主要研究内容为在边缘计算中利用资源分配和调度机制优化移动群智感知应用的感知质量,本文的内容组织结构如下。

第一章为绪论。首先给出面向移动群智感知的边缘计算技术的的研究背景和面临的挑战,然后介绍现阶段国内外研究现状,最后给出研究内容和主要贡献以及论文组织结构。


第二章为边缘网络中基于感知质量的任务调度机制。
首先基于随机移动模型建立移动群智感知任务分发和时间的关系,并依据传染病模型对移动群智感知任务分发的过程建立数学模型。
接着分析移动群智感知任务的持续时间和边缘节点的移动模型对移动群智感知任务分发过程以及数据回收过程的影响,同样也建立数学模型。
然后,提出移动群智感知任务的感知质量。
通过理论分析,总结了提高感知质量的途径和方法。
接着提出优化感知质量的资源调度算法。
最后结合仿真模拟验证算法对移动群智感知任务感知质量的优化效果。

第三章为边缘网络中基于任务卸载的链路调度机制。
在边缘计算模型中,移动群智感知的数据预处理服务(例如压缩、去冗余、校验)可以部署在边缘服务器上。
虽然卸载数据处理任务能够在一定程度上减少移动设备的能耗,但是未压缩的感知数据在资源受限的无线网络中传输可能会耗费更多的能耗。
为此,在保证移动群智感知感知质量的前提下,本文对任务卸载决策建立数据模型,并提出了让移动设备群体能耗最小的卸载决策和网络链路分配算法。

第四章为边缘网络中基于社会网络的边缘服务调度机制。
首先基于城市公交路线建立移动群智感知边缘服务部署框架。
通过分析城市居民在公共交通中的行为,建立移动群智感知数据回收和成本的关系模型。
基于此种模型,提出基于公交路线的边缘服务部署决策算法。
最后采用仿真模拟验证算法效率以及成本节省比例。

第五章总结全文,概括文中的主要贡献并展望未来的研究工作。
