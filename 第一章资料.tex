\chapter{绪论参考资料}


随着 5G 通信标准的确立,物联网时代正式拉开了序幕。
种类繁多的移动设备、高度集成的传感器、以及万物互联的网络空间,促使群智感知应用对存储资源、网络资源和计算资源的需求达到了更高的境界。
云计算作为集中式服务模型,在支撑覆盖区域广、时效性强的群智感知应用时已渐显乏力。
边缘计算作为一种与云计算互补的分布式计算模型,如何利用其特性为群智感知服务提供基础支撑,是一个亟待解决的研究课题。
%,种类繁多的非通信专用设备也可以接入互联网,并享受云计算服务带来的便利。
% 目前云计算中心的建设速度,已经无法满足网络中所有设备平等地享受云服务带来的便利。
% 因此,边缘计算作为一种新型计算模型,用来应对复杂网络空间中的海量服务需求。
% 为了保障各终端的服务质量,如何快速分配、调度网络空间中的计算资源、网络资源和『』,是一个亟待解决的研究课题。
% 这种新型计算模型不仅要保障这些终端设备的隐私、安全和服务质量,还需要平衡分配网络空间中的计算资源、存储资源和网络资源。
本章首先介绍群智感知场景下边缘计算中的资源调度问题;其次简要介绍当前国内外研究现状和现有工作的不足;再次阐述了本文研究的目的与意义;最后介绍论文的组织结构和层次关系。

\section{研究背景}

\subsection{边缘计算}
边缘计算是与云计算~\cite{DBLP:journals/cacm/ArmbrustFGJKKLPRSZ10}相辅相成的一种新型计算模型。
云计算作为集中式处理模型,需要将用户数据收集至部署有云平台的数据中心~\cite{DBLP:conf/icdcs/Montresor16},才能有效地服务于用户。
但在面临 万物互联的新型网络空间时,云计算服务将面临以下挑战:

1)\textbf{实时性不足}。
在物联网场景中,存在着许多强实时性的应用。在云计算中,数据需要不断在客户端和云端之间往返,导致响应时延增加、用户体验下降。
例如在智能驾驶应用中,云计算无法达到毫秒级的数据处理时延~\cite{DBLP:conf/cvpr/GeigerLU12}。
移动终端上的虚拟视觉(Virtual Reality,VR)框架 Furion~\cite{DBLP:conf/mobicom/LaiHCSD17}在探索过程中发现,仅靠云计算服务无法帮助移动终端获得高质量的实时 VR 服务。
而利用边缘计算将渲染服务卸载到边缘服务器中,MUVR~\cite{DBLP:conf/edge/LiG18}成功实现了一个低通信且稳定的多用户 VR 框架。

2)\textbf{带宽不足}。
在网络空间中,将边缘设备产生的数据传送至云计算中心,需要消耗大量的带宽资源。
Intel 在 2016 年的报告~\cite{DBLP:journals/micro/KatoTINTH15}指出,一辆智能驾驶汽车工作一天可以产生 4 TB 的数据。
而一架波音787飞机在飞行途中,其数据产生速率达到了 5 GB 每秒~\cite{JCRD/shi17}。
如此巨大的数据量,利用云服务进行保存和处理,不仅需要耗费大量的带宽,且网络传输导致的时延也将导致计算服务毫无意义。

3)\textbf{能耗过高}。
云计算进入实践以来,研究者们更多关注数据中心中的能耗问题。
据《中国数据中心能效研究报告》~\cite{whitebooks/cn15}显示『白皮书引用格式』,我国2015年度数据中心能源消耗已经超越1000亿千瓦时。
在美国,2013年度数据中心总能耗就已达到910亿千瓦时,Sverdlik 在调研~\footnote{https://www.datacenterknowledge.com/archives/2016/06/27/heres-how-much-energy-all-us-data-centers-consume/}中预测,2020年度全美数据中心能耗会到达1400亿千瓦时。
随着用户、应用数据的增加,全球数据中心的能耗仍在进一步上升。

4)\textbf{用户数据安全和隐私}。
在外物互联的网络空间中,边缘设备已经走入用户的私人生活空间。
例如家用摄像头、便携式移动设备、智能网联汽车,这些设备中都拥有大量的私人隐私数据。
一旦这些数据被上传至云端,用户隐私的泄露风险也会成倍增加。
目前,欧盟已经强制实施“通用数据保护条例”(General Data Protection Regulation,GDPR)来保护用户的隐私。
对于云计算公司,如何保护用户的数据安全和隐私安全,也更为重要。

为了避免云计算模型在物联网场景下的不足之处,2013年西北太平洋国家实验室的 Ryan LaMothe 提出一种新性计算模型概念——边缘计算(Edge Computing)。
2015年,欧洲电信标准协会~\footnote{https://www.etsi.org}(European Telecommunications Standards Institute,ETSI)在白皮书~\cite{hu2015mobile}中正式定义了边缘计算的概念。
边缘计算就是将应用任务在靠近数据源(例如移动设备、传感器、最终用户等)的资源上进行处理。
“边缘”主要相对于云计算中心而言,指代数据产生源到附近的任意计算、存储、网络资源。
同年,思科也推出了雾计算白皮书~\cite{computing2015internet}。
雾计算这一概念最早于2011年由Bonomi首次提出~\cite{bonomi2011connected}。
雾计算通过虚拟化架构将远端的云端服务迁移至本地节点,让高高在上的云服务更贴近用户,从而提高应用服务的访问效率和服务质量。
『虽然边缘计算和雾计算在指导思路上有相似处,都是将云端服务落地到更接近用户的网络边缘侧。
但是雾计算更多探讨的是实际应用的落地;边缘计算更倾向于研究边缘计算体系结构中的问题。
所以学术界更愿意探讨边缘计算下的本质问题。』

\textbf{『典型的边缘设备\\缺图』}

相较于云计算模型,边缘计算具有3个明显的优势:

1)\textbf{大量临时数据不需要上传至云端服务器}。
在边缘计算中,应用程序可以利用边缘节点中的资源完成数据的存储、计算工作。
避免用户数据上传至云端服务器,为主干网络节省了大量的带宽。

2)\textbf{计算任务不再需要云计算中心的响应}。
在边缘计算中,云端服务可以卸载至资源丰富的边缘节点上。
更短的网络路径让应用服务的响应延时大大减少,不仅增加了响应能力,也提高了用户的使用体验。

3)\textbf{隐私数据保存在边缘设备,无需上传}。
由于用户隐私数据可以存储在边缘设备中而不是云端服务器,减少了隐私数据的传输路径,在一定程度上规避了隐私泄露的风险。

得益于这些优势,近年来边缘计算得到了突飞猛进的发展。
利用边缘计算,不仅可以在广袤的无线网络接入范围内提供更好、更快、更准确的信息技术服务和云计算能力;还可以在数据的边缘,利用富余的资源快速完成应用服务。
目前,边缘计算已经被采用在以下典型应用场景中:公共安全中的实时数据处理、智能网联车和自动驾驶、虚拟现实、工业物联网、智能家居以及智慧城市。
这些真实的应用场景与人类的生活、生产、娱乐息息相关。
并且这些应用还有一个共同的特点,它们的需要从大量的边缘设备中收集大规模的传感数据作为『数据分析』的基础。这种应用范式,也叫群智感知。

\subsection{群智感知}

\textbf{无线传感网络+为什么要有群智感知+端到端通信}

群智感知是一种基于物联网的“以人为本”的感知模式。
其构想起源于2006年《连线》(Wired)杂志提出的众包一词,旨在利用分布式解决方案将感知工作分配出去来共同完成应用任务或提供服务。
利用这一特性,群智感知可以完成个体无法实现的复杂环境下大规模动态社会感知任务,例如交通拥堵状态、城市空气质量监测等。
在早期,这些社会感知任务可以利用无线传感网络~\cite{DBLP:journals/cn/AkyildizSSC02}完成。
由于不同类型的感知任务需要部署不同功能的传感器甚至不同架构的传感网络,随着社会感知任务需求的增多和变更,静态的传感网络在部署和维护上都会耗费大量的人力成本和物力成本~\cite{CNKI/2006/WSNRen}。

『\textbf{缺图}』

群智感知利用众包的方式,将感知任务众包给用户的移动设备(例如手机、平板、智能手表等)作为基本感知设备。
这些移动设备集成了丰富的传感器,可以获取大量与设备所处环境相关的数字信息,例如环境光(光学传感器)、噪声(麦克风)、地理位置(GPS 传感器)、移动状态(陀螺仪、加速计)等。
除此之外,这些移动设备还可以利用自身优秀的通信能力快速交付数据,甚至利用本地计算资源进行数据处理。
将对于无线传感网络而言,群智感知应用的部署成本少、灵活性高,更适合复杂网络环境下的大规模动态社会感知任务。

除了移动设备内置了大量传感器之外,大量的现代交通工具中,雷达、摄像头、GPS、陀螺仪、加速计等设备也已经成为保障安全驾驶的必备传感器。
而物联网的飞速普适,也让更多的传感设备具备了网络连接功能和简单数据处理逻辑。
因此,群智感知可以收集到种类更多、信息更全的感知数据,从而实现更多创新型研究与应用。
西北工业大学利用校内学生的智能手机,在校区内实现了基于群智感知的噪声监测系统~\cite{CNKI/2014/CSNSYu}。
通过众包方式收集不同位置和不同时间的噪声污染数据,该系统利用离散数据重建出高精度的城市噪声时空分布地图,为城市噪声治理提供了可视化的监控平台。
论文~\cite{DBLP:conf/wcnc/AliAEJH12}利用车内传感器和道路行人手机内的传感器,将多个信息源混合在一起。
在不借助其它传感器和通信网络的前提下,搭建智能交通拥堵检测系统。
Waze~\footnote{www.waze.com}是一款基于群智感知的导航服务。
在传统的允许用户帮助编辑图资的基础上,Waze 开创性的引入用户的 Facebook 和 Twitter 消息,实时更新周边商户信息和道路交通状态。
通过整合用户数据,Waze 比传统的导航服务内容更加丰富、信息也更精准。

在物联网环境下,人类社会和网络空间中已经包含了海量的、具备多元化传感功能的智能设备。
众多的潜在参与者、饱满的空间覆盖、便利的数字信息提取,让群智感知成为了环境研究~\cite{DBLP:conf/sensys/DuttaAKMMWW09}、人类社会关系研究~\cite{DBLP:conf/globecom/AslIAM13}、智慧城市建设研究~\cite{DBLP:journals/cm/WangZWCHM16}的基础工具。
% 相较于无线传感网络,群智感知不仅能够灵活部署,还能够节省成本,成为了许多科技公司青睐的数据收集方案。
% 同时,这些以人为本的海量感知数据,能够帮助应用程序提供更高品质的服务。

\subsection{边缘计算与群智感知应用}

\textbf{边缘计算在群智感知中的作用}

\textbf{资源管理分配与任务调度}

\textbf{传感(数据采集)、通信(数据传输)、计算(数据处理)}

\textbf{总括:面向群智感知的边缘计算资源调度,现有研究有哪些不足}

% 在物联网网络空间中,移动设备仍然处在爆炸式增长趋势中。
群智感知应用的工作流程可以简单划分为四步~\cite{zh_cn:shi}:任务分发,传感器感知,数据上传和数据处理。
其中,任务分发是指根据群智感知任务的覆盖区域、执行任务的用户数量、以及任务的持续时间,快速将感知任务发送到合适的移动设备上。
移动设备收到感知任务后,通知传感器工作并产生感知数据。
然后,移动设备利用自身的通信能力,将传感数据上传到等待数据处理的地方(例如云服务器)。
最后,对这些数据进行去重、映射、分析,从而推断出符合共同利益的结果~\cite{DBLP:journals/cm/GantiYL11}。

『插图』

早期,群智感知应用通常借助云上资源来进行感知数据的保存和处理。
随着群智感知应用的需求增加,覆盖范围扩大,所收集的感知数据『越来越多』,对云端资源的需求也成倍递增。
利用边缘计算可以将服务卸载到数据产生源附近的特征,可以有效缓解群智感知应用对云计算资源的压力。
\textbf{举例:MCS+Edge Computing}
CrowdITS Crowdsourcing in Intelligent Transportation Systems
阿里巴巴的城市大脑2.0~\footnote{https://damo.alibaba.com/labs/city-brain}

在边缘计算模型中,为了保障群智感知服务的稳定性和服务质量,必须针对群智感知的工作阶段对边缘网络中的各种资源进行调度和管理。
这也意味在群智感知的数据采集阶段、数据传输阶段、以及数据处理阶段,需要对边缘网络中的网络资源、计算资源、存储资源进行合理的划分和调度。


% 目前,智慧城市建设已经成为当先信息领域的前沿研究热点。
% 智慧城市的构建,需要对城市主体和物理世界对象进行大量的数字信息挖掘,在数据处理和学习之后,从中获得人类活动和城市运作规律之间的潜在联系,并将其应用到各种服务和创新中。
% 在智慧城市场景下,群智感知已经成为数据信息收集的主要手段。
% 杭州市作为我国智慧城市建设的典范,功不可没。
% 通过收集、耦合全市的交通灯信息、道路监控录像、环境信息、

前文已经阐明边缘计算旨在将云上服务迁移至靠近移动设备的网络边缘侧。
而群智感知需要汇聚大量移动设备收集海量的传感数据,为具体应用提供服务支撑。
这两种技术的有机结合,已经成为智慧城市建设的前沿研究热点。

\subsection{面向群智感知的边缘计算研究及其挑战}
目前,边缘网络正在快速发展阶段。


\textbf{编程模型}
应用、服务的功能划分。亚马逊的 Lambda 计算服务,可实现任务的自适应卸载。\\
虚拟化技术\\
代码卸载(code offloading)\\

\textbf{硬件标准、软件标准}

\textbf{动态调度}
数据可分布(任务分发+数据采集)\\
网络可分布(SDN 中网络管理和路径规划)\\
资源可分布(边缘服务节点弹性部署)\\

\section{国内外研究现状}

利用边缘计算优化群智感知应用改的研究有很多。
按照研究方法可以将这些研究划分为3类。
首先,整合边缘网络中各种无线通信手段,增加群智感知应用的参与者。
其次,利用边缘计算的服务卸载功能,减少系统时延,优化群智感知应用的用户体验。
最后,利用边缘计算的分布式特征,针对群智感知应用的特殊需求进行合理的资源分配。
本节主要从这3个方面来介绍边缘计算中的资源调度研究现状。

\textbf{要引用的文章}
\cite{DBLP:journals/jsac/LyuNTLWGP17}

\subsection{网络资源调度}

移动边缘网络的核心思想是利用SDN和NFV技术,将网络功能、内容和资源移动到更接近终端用户的地方,即网络边缘。网络资源主要包括计算、存储或缓存以及通信资源。

在早期的群智感知研究工作中,群智感知的参与设备多使用蜂窝通信~\cite{DBLP:conf/globecom/ZhangJLLC16,DBLP:conf/icdcs/XiaoWHHH16}来上传感知数据。
随后,Karaliopoulos 等人的研究~\cite{DBLP:conf/infocom/KaraliopoulosTK15}发现,在传感数据上传过程中利用端到端通信,可以减少边缘设备的能耗成本的网络通信成本,并以此来为群智感知服务雇佣更多的参与者。
同时,论文~\cite{DBLP:journals/tpds/ZhaoMTL15}通过研究机会式通信网络~\cite{}中的数据包转发规律,发现在边缘网络中利用数据包融合技术可以有效减少数据传输延时。
移动边缘网络的架构在提供邻近服务方面有很大的优势,因为边缘服务器更接近终端用户,端到端通信技术~\cite{nunna2015enabling}可以得到利用。
从而可以减少无线接入网上的流量负载。


由于处理和存储能力接近终端用户,通信延迟可以显著降低。从中受益的主要应用是下载和视频内容传送的计算。论文~\cite{DBLP:conf/infocom/RimalVM16}的工作表明,使用MEC和云相结合的解决方案,可以在不影响网络性能的情况下减少负载数据包延迟。胡等人~\cite{DBLP:conf/apsys/HuGHWACPS16}在WiFi和LTE网络中,通过边缘计算平台进行加载可以大大提高高度交互和计算密集型应用的延迟,例如增强现实和认知辅助。高等人~\cite{gao2015cloudlets}展示了Wi-Fi和4G LTE网络的实验结果,显示与云负载相比,云负载可以将响应时间提高51 \%。在移动边缘网络基础设施中部署边缘服务器可以为带宽需求高的应用和计算密集型应用节省高达67 \%的运营成本~\cite{DBLP:conf/saso/MehtaTKTKE16}。研究结果显示,利用主动缓存方案~\cite{DBLP:journals/cm/BastugBD14},回程节省高达22 \%。如果存储容量增加,可能会获得更高的收益。

在群智感知应用中,服务所需要的原始数据基本来源于边缘设备的传感数据。
在该场景下,用户和边缘设备多处于运动状态中~\cite{},例如智能车联网中的车辆、群智感知应用中的参与者。
这些设备和参与者随着地理位置和所处网络环境的改变,都会在网络中引起网络拓扑结构的变化。
然而,群智感知服务往往以连续的方式收集参与者的信息并提供反馈结果。
所以在边缘计算实际应用中,随着参与者或者边缘设备状态的改变,边缘网络需要时刻对边缘设备的网络状态进行维护并合理调度边缘网络中的数据流~\cite{}。


例如,论文~\cite{DBLP:journals/cm/GantiYL11} 在车辆上部署 GPS 和加速度计,可以在车辆行驶过程中定位城市中的坑洼路面;

\subsection{计算资源调度}

在计算机科学中,计算卸载是将计算任务迁移到外部源的过程,例如云、网格或集群~\cite{DBLP:journals/network/MaZZWP13}。计算卸载是通过将计算转移到位于不同位置的资源更丰富的服务器来增强移动设备容量的解决方案~\cite{DBLP:journals/monet/KumarLLB13}。资源密集型应用的出现,如3D游戏,将继续需要更多的移动资源。移动设备和网络的改进仍然无法跟上需求的趋势。因此,移动设备将永远不得不牺牲有限的资源,例如资源贫乏的硬件、不安全的连接和能源驱动的计算任务~\cite{DBLP:conf/ispa/KovachevYK12}。例如,在移动电话上编辑视频剪辑需要大量的能量和计算,这与台式机或笔记本电脑相比有一些局限性。为了应对这些限制,许多研究人员研究了将计算卸载到资源丰富的平台,如云~\cite{DBLP:conf/secon/HassanXWC15,DBLP:journals/pervasive/SatyanarayananBCD09,DBLP:journals/computer/KumarL10}。

2015年,高桥等人~\cite{DBLP:conf/mobilecloud/TakahashiTK15}提出了边缘加速Web浏览( EAB )原型,设计用于使用更好的卸载技术执行Web应用程序。EAB的目的是通过将应用卸载推到在RAN中实现的边缘服务器来改善用户体验。客户端EAB前端检索在EAB服务器中处理的呈现的Web内容,而音频和视频流通过EAB后端,并根据客户端硬件的能力进行解码。

2015年,Sardellitti等人~\cite{DBLP:journals/tsipn/SardellittiSB15}提出了一种基于算法的设计,称为逐次凸逼近( SCA )。该算法优化了密集部署的多个无线接入点之间的计算卸载。作者考虑了MIMO多小区通信系统,其中几个移动用户请求在中央云服务器上承载他们的计算任务。他们首先测试了一个用户在云服务器上卸载计算任务,这导致了非凸优化问题。在多用户场景中,基于SCA的算法获得了原始非凸问题的局部最优解。根据公式结果,作者声称他们的算法超过了不相交的优化方案。此外,他们表示,提议的SCA设计更适合于获得高计算任务的应用程序,并将能耗降至最低。

2016年,陈等人~\cite{DBLP:journals/ton/ChenJLF16}设计了一个高效的计算卸载模型,采用了一种分布式的博弈论方法。博弈论是一种有说服力的工具,可以帮助同时连接的用户在基于战略互动连接无线信道时做出正确的决定。如果所有用户设备使用相同的无线信道卸载计算活动,则可能会导致彼此之间的信号干扰和无线质量降低。具体来说,博弈论针对多用户计算卸载引起的NP-hard问题,并通过获得多用户计算卸载博弈的纳什均衡来提供解决方案。

2016年,张等人~\cite{DBLP:conf/rndm/ZhangMLV016}提出了基于合同的计算资源分配方案。该方案提高了车辆终端的效用,该终端在低计算条件下智能地利用MEC服务提供商提供的服务。MEC提供商根据车辆卸载到MEC服务器的计算任务,从车辆接收付款。使用无线通信服务,合同信息和支付信息被广播给路上的车辆。

2015年,Habak等人~\cite{DBLP:conf/IEEEcloud/HabakAHZ15}提出了毫微微云系统,该系统形成了一个由协同定位的移动设备组成的云,这些移动设备可以自我配置成相关的移动云系统。毫微微云客户端计算服务安装在每个移动设备上,以计算设备计算能力以及与其他移动设备共享的容量和能量信息。移动属性建立并维护在用户配置文件中,该配置文件在连接到WiFi网络中可用的云或控制设备的移动集群中共享。代码形式的密集计算任务被发送到云,以利用其他连接的移动设备的计算能力。毫微微云模型旨在减少集中位置的计算负荷,并将其带到移动网络的边缘。

\subsection{基于地理位置的资源划分}



\subsection{现有研究的不足}

针对群智感知应用,边缘计算的实现具有三大挑战:\\
1)利用边缘计算提高群智感知的覆盖率和感知质量\\
2)边缘计算下大规模群智感知数据的链路调度\\
3)边缘网络中的群智感知服务卸载调度\\

包含『关键技术』和『应用举例』

\section{研究意义与目的}

\section{论文组织结构}
