\chapter{总结与展望}

\section{全文总结}

% 本文对边缘计算和 D2D 通信支撑的移动群智感知中感知质量优化问题进行了深入研究,
% 针对移动群智感知的三个重要过程(感知、通信和数据处理)中所面临的感知任务调度问题、网络资源调度问题、边缘服务调度问题,

在移动群智感知应用中,众多的参与设备会产生大量的传感数据,并且这些感知数据分布在广泛的地理空间中。
这为感知数据的收集、查找与分析带来了极大的挑战。
如果没有高效、合理的解决方案,庞大的感知数据将对网络资源、计算资源、存储资源造成巨大的压力。
本文利用边端融合为移动群智感知的提供支撑,针对群智感知的执行过程优化,提出了三种基于边缘融合的移动群智感知执行优化机制。
% 以提高群智感知应用的覆盖范围、感知效率和感知质量。
% 然而,参与群智感知的移动设备众多、且个体能力存在差异,需要合理利用边缘网络中的各种资源来保障群智感知应用的执行效率。
% 另一方面,移动设备之间的 D2D 通信受运动状态影响会导致网络资源不稳定且难于管理。
% 为了保障群智感知应用的执行效率和感知质量,本文围绕群智感知的三个重要过程:感知、通信和数据处理,针对群智感知应用的执行过程和边缘网络中资源进行调度进行研究。
% 本文针对边缘计算和 D2D 通信支撑下的移动群智感知中
本文的主要创新成果如下:

(1)基于边端融合的群智感知任务调度机制

在移动群智感知应用的生命周期内,可以通过提高感知任务的分发效率和感知数据的收集效率,来达到感知质量优化的目的。
为了实现这一目标,本文结合边缘服务和 D2D 通信,构建了移动群智感知应用的随机过程分析模型。
首先,考虑了移动过程中感知设备基于 D2D 通信的随机特性,
分析了感知应用生命周期内,多个边缘服务器对任务分发效率以及数据接收效率的影响。
利用常微分方程组构建了移动群智感知的执行过程模型。
通过对模型的分析与求解,进一步推导出边缘网络中各类资源配额、移动设备运动特征和感知质量的量化关系。
同时,考虑群智感知应用的有限生命周期,提出了任务分发阶段和数据收集阶段的时间划分算法,优化并预测了边缘融合场景下的群智感知执行效率,并分析了执行效率的影响因素与其作用机理。

(2)面向计算任务卸载的群智感知网络流调度机制

针对群智感知应用中感知数据处理,研究了用户设备与边缘服务器之间计算任务卸载和网络资源调度问题。并特别考虑了基于 SDN 管理的边缘网络资源,构建了 SDN 流表容量限制下的高能效卸载决策与流调度 ILP 模型。
基于对模型的求解分析,确立了任务卸载决策和移动设备对网络资源需求之间的关系。
针对求解 ILP 模型的高计算复杂度,本文提出依据权重的数据流调度算法,确定任务卸载决策以及移动设备和边缘服务器的通信链路调度决策。在保障任务卸载收益最大化的同时,实现边缘网络中的流量负载均衡,并减小感知数据处理能耗成本。
经过对比试验,该算法可以达到最优解87.4\%以上的节能效率。

(3)基于用户移动特征的群智感知数据收集方法

针对移动群智感知应用在城市感知中的海量感知数据收集问题,
借助城市中的公共交通网络分析出居民的城市移动特征,
并基于这一特征部署边缘服务,以达到充分利用 D2D 通信,降低数据收集成本的目的。
本文围绕感知应用的生命周期、感知质量要求以及边缘服务部署策略,构建了 ILP 模型。
然后通过动态规划算法决定边缘服务的部署策略。
经仿真测试对比,该算法解决了交叉路径中感知数据的重复收集问题,提高 D2D 通信利用率并减少蜂窝网络通信的使用,降低了城市感知中感知数据收集的总成本。

\section{研究展望}

本文基于移动群智感知应用的执行过程,针对网络边缘的资源调度进行了研究,以提高感知效率、降低执行成本。
但是在基于边端融合的移动群智感知中,云端资源与边缘资源的协同管理问题、边缘网络的联合自治问题、以及感知数据的流式处理问题等,本文并没有逐一考虑。
因此,为进一步提高移动群智感知的感知质量,仍需更全面、深入的研究。

(1)云端资源与边缘资源的协同管理问题

移动群智感知借助边缘计算,虽然能够利用无处不在的边缘设备收集感知数据,并通过边缘服务在传输过程中进行简单的数据处理,但是最终的运算仍然需在云平台进行处理。
另一方面,边缘计算作为与云计算模型互补的分布式计算模型,现有的边缘计算平台通常与云计算平台组合成双层架构来为应用程序提供支撑。
因此,边缘计算平台和云平台也存在数据、任务的同步与调度问题。

(2)边缘网络的联合自治问题

在移动群智感知应用中,尤其是智慧城市感知,感知任务需要覆盖的范围过于庞大。
往往需要将目标区域划分构建成多个边缘云进行覆盖。
但是移动设备的运动特征,无法保障移动设备在感知任务的生命周期内一直处于同一边缘云中。
因此,面对边缘云内移动设备的加入和退出,边缘云之间需要对移动设备进行联合管理,确保资源分配的准确性。
同时,针对感知应用,边缘云还需要确保运行状态的一致性。

(3)感知数据的流式处理问题

边缘计算的引入,让感知数据得以在传输过程中借助边缘服务得到一定程度的处理。
但是,感知数据的收集具备一定的离散性,特别是当采用 D2D 通信传输感知数据时,这种机会式通信网络让边缘服务器收到感知数据的时间间隔和数据大小毫无规律可言。
因此,边缘服务器在处理感知数据时,可以借助缓存、打包等策略,提高边缘服务器上的资源利用率。
另外,面对监控类型的感知数据,由于时效性要求,边缘服务也应当根据感知任务的要求,构建合理的数据处理、传输流水线,确保感知任务能够稳定、流畅地运行。
