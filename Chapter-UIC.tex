% UIC 文章的翻译稿

\chapter{边缘网络中基于感知质量的任务调度机制}

% 作为一种新的云服务提供范例,边缘计算已经彻底改变了许多移动计算应用,包括移动众包(MCS)。在这项研究中,我们专注于边缘计算授权MCS与设备到设备(D2D)通信,其中任务分发和数据收集都部署为边缘节点(例如,基站)和移动节点中的服务通过流行病路由的D2D通信机会性地转发任务和传感数据。在这种情况下出现的一个自然问题是对服务部署对感知质量(例如,覆盖)的影响。为此,我们有动力对边缘计算支持的基于D2D的MCS进行随机分析。特别地,我们使用常微分方程(ODE)来描述任务分发阶段和数据收集阶段,并将可实现的覆盖范围推导为参数的函数,例如所利用的边缘节点数,移动节点数,遇到率。我们还应用我们的分析来找出截止时间受限的MCS应用的最佳时间分配。通过广泛的基于模拟的评估,我们验证了我们的分析的正确性,平均误差小于9.6%,以及我们的时间分配方案的最优性。

边缘计算是一种新的云端服务模型,它为许多移动计算应用提供了新的范式,尤其是群智感知应用。
在群智感知应用场景中,任务分发服务和数据收集服务均可以部署在边缘代理节点(例如基站等)上。
利用边缘节点之间的机会式通信进行数据交互,可以将群智感知任务分发到更多的边缘设备上,并避免过多的成本开销。
所以,边缘计算中的资源分配策略会对群智感知应用的服务部署和感知质量造成影响。
为此,本文针对群智感知应用在边缘网络中的数据传输行为进行了理论分析。
通过引入常微分方程组来描述任务的传播过程和数据收集过程,建立了感知质量和边缘计算中各种资源配额的函数关系。
『其中参数包含边缘代理节点数量,边缘设备数量,边缘设备的相遇率等』。
利用数学分析具有截止期限的群智感知应用,本文提出了可以找出群智感知应用中任务分发和数据收集的最佳时间分配算法。
借助模拟测试和评估,验证了本文分析方法的正确性,时间分配算法的可行性。
评估结果表明,本文分析方法的结果和模拟数据的平均误差小于 9.6\%。

首先基于随机移动模型建立群智感知任务分发和时间的关系,并依据传染病模型对群智感知任务分发的过程建立数学模型。
接着分析群智感知任务的持续时间和边缘节点的移动模型对群智感知任务分发过程以及数据回收过程的影响,同样也建立数学模型。
然后,提出群智感知任务的感知质量。
通过理论分析,总结了提高感知质量的途径和方法。
接着提出优化感知质量的资源调度算法。
最后结合仿真模拟验证算法对群智感知任务感知质量的优化效果。

\section{研究背景}

% 通过将传统云服务推向最终用户附近的边缘计算已经成为一种广泛关注的服务提供范例。 通过探索网络边缘的边缘节点(例如,蜂窝网络中的基站)的资源,人们普遍认为边缘计算可以显着提升服务质量,这要归功于其广泛部署,地理分布, 和用户接近[1]。 边缘计算的巨大潜力吸引了学术界和工业界的很多兴趣。 许多开创性的研究人员已经讨论了如何在不同的应用领域应用边缘计算[1]  -  [3]。

在传统的云计算模型中,大量的用户数据被收集并发送至云端服务器,由云端的计算资源对海量数据进行处理。
边缘计算模型和云计算模型相背而驰,通过探索网络边缘中的边缘节点(例如蜂窝网络中的基站)的可利用资源,将云端服务迁移到位于网络边缘侧的边缘服务器上,使计算、数据处理资源更加贴近数据生产源。
由于边缘网络中边缘设备繁多、地域分布广、且靠近用户,研究者们普遍认为边缘计算可以显著提升网络应用的服务质量~\cite{DBLP:conf/sigcomm/BonomiMZA12}。
近年来,学术界和工业界都在不断挖掘边缘计算的巨大潜力,研究者们都致力于将边缘计算应用到不同领域的开创性工作中~\cite{DBLP:journals/access/MarjanovicAZ18,DBLP:journals/iotj/ChiangZ16}。

% 一个值得注意的应用领域是移动计算,它从边缘计算的地理分布和用户接近中获益良多。 一个代表性的移动计算是移动众包(MCS)。 MCS是一项人为驱动的活动,利用无处不在的无线连接和众多移动设备的内置传感器来执行参与性和机会性感知。 与传统的专用无线传感器网络不同,MCS可以很好地收集有关我们日常生活中各种现象的不同信息[4]。 作为补充,MCS可以广泛应用于不同领域,如Wi-Fi性能测量[5],天气预报[6],空气质量传感[7],噪声映射[8],交通监测[9]。

目前,移动计算领域和边缘计算正日益紧密的结合在一起。
鉴于边缘计算的地域分布特性以及数据处理更靠近数据源,移动计算应用可以利用边缘计算实现高带宽、低延时的实时服务。
群智感知是移动计算领域中的一种典型应用。
它利用无处不在的无线网络和众多移动设备中的内置传感器来完成『环境』感知。
不同于传统的专用型无线传感器网络,群智感知可以更方便、快捷地收集与人类生活、自然环境息息相关的各种信息~\cite{DBLP:journals/cm/GuoCZYC16}。
不仅如此,群智感知还被广泛应用于其它不同领域,例如 WiFi 性能测量~\cite{DBLP:journals/cm/RosenLLCMB14}、天气预报~\cite{DBLP:journals/tpds/ZhaoMTL15}、空气质量监测~\cite{DBLP:conf/huc/ZhangXWC14}、城市噪声监测~\cite{DBLP:conf/huc/ZhengLWZLC14}和城市智能交通建设~\cite{DBLP:conf/icdcs/ZhouJL15}。


% 当应用云计算赋予MCS权力时,如[10],[11]中所讨论的,所有传感数据都应传输到云端。由于数据源(例如,移动节点)与远程数据中心中的云之间的遥远距离,MCS应用可能经历高网络延迟和不可预测的网络抖动。幸运的是,边缘计算允许MCS探索边缘节点中的资源,如基站或网络边缘上的接入点,以收集和处理来自移动节点的数据。为了扩大MCS的应用范围,我们认为数据收集和任务分发都可以作为边缘节点中的服务进行部署。因此,边缘节点负责MCS的任务分发和数据收集。在如图1所描述的场景中,每当移动节点(例如,移动电话,车辆等)从边缘节点接收到MCS任务时,它可以进行相应的感测操作并将感测数据报告给边缘节点。由于高蜂窝通信成本,有时移动节点不愿意在MCS任务上合作。为了解决这个问题,移动节点利用设备到设备(D2D)作为第五代(5G)网络中的重要通信方式,以机会方式交换关于任务或感测数据的消息。实际上,基于D2D通信的MCS也已在文献中广泛讨论[12]。

\begin{figure}[!b]
  \centering
  \vspace{-1em}
  \includegraphics[width=240pt]{./figures/Sec_UIC/群智感知应用场景.pdf}
  \vspace{-0.5em}
  \caption{群智感知应用场景}
  % \vspace{-4em}
  \label{Figure_UIC_MCS}
\end{figure}

当使用云计算技术支撑群智感知应用时,如论文~\cite{DBLP:journals/fgcs/AntonicMPZ16,DBLP:conf/ccnc/MessaoudRG16}中所讨论,所有的感知数据都会被上传到云端。
由于产生感知数据的移动设备(例如智能手机、智能手表)和位于数据中心的云端设备距离遥远,感知数据在传输过程中可能会遇到较高的网络延时和不可预测的网络抖动。
通过边缘计算的支援,群智感知应用可以探索边缘节点中的可利用资源(例如蜂窝网络的基站或者其它大型无线网络接入点),进行任务转发和数据收集工作。
『为了进一步扩大群智感知应用的适用范围,任务分发工作和数据收集工作都可以作为服务被部署到合适的边缘节点中』。
如图~\ref{Figure_UIC_MCS}所描述的场景中,当移动节点收到来自边缘节点的群智感知任务是,这些移动节点可以执行相应的传感操作并将感知数据反馈给边缘节点。
介于蜂窝通信的成本较高,减退了许多潜在群智感知参与者的热情。
为了解决这一问题,许多学者研究了群智感知中参与者的激励机制,也有部分学者发现将端到端通信机制引入群智感知应用,同样可以缓解这一问题~\cite{DBLP:journals/puc/WangLL17}。

% 在本文中,我们提出边缘计算授权MCS与D2D通信,其中整个MCS过程有两个主要阶段。在第一阶段(任务分发)中,边缘节点作为源节点执行以将任务分发到移动节点。在第二阶段(数据收集)中,移动节点成为源节点,边缘节点成为目标节点。任务和传感数据都以机会方式通过D2D通信传输。在这种情况下,第一个自然问题是服务部署如何影响感知质量(例如,覆盖)。直观地知道通过部署相应的服务来利用更多的边缘节点,将实现更高的传感质量。但是,在边缘节点中部署服务并不是免费的。因此,量化这种对服务部署决策的影响是非常重要的。现有的研究,例如[13]  -  [15],已经分析了D2D机会网络中的消息传递延迟。这些研究表明,消息传递性能受很多因素的影响很大,例如移动节点的数量和移动节点的遭遇率。正如我们注意到的,他们通常假设有一个数据源节点和一个目标节点。它们都不能用于分析边缘计算授权MCS的性能,因为可能有多个边缘节点分别在任务分发阶段和数据收集阶段中作为源节点和目的节点执行。

在本章节中,边缘计算被用以辅佐群智感知应用中的任务分发过程和数据收集过程,以提高群智感知应用的覆盖范围和服务质量。
在任务分发过程中,已部署任务分发服务的边缘节点作为任务分发的源头,不断地将群智感知任务分发至移动节点。
在数据收集过程中,移动节点作为感知数据的源头,将数据回传至已经部署数据收集服务的边缘节点。
任务的传播和感知数据的转发不仅可以通过蜂窝数据进行发送,也可以通过端到端通信的方式传输。
现有的研究~\cite{DBLP:conf/wcnc/QinF13,DBLP:journals/twc/LiW14,DBLP:journals/winet/ZhaoMLT18}已经分析了端到端机会式通信网络中的信息传递延时。
这些研究工作表明,端到端通信机制中消息传递延时的影响因素主要有移动节点的数量和移动节点的相遇率。
然而,这些研究工作中通常在分布式节点中选取一个源节点和一个目标节点。
但是在边缘网络强化的群智感知场景下,存在着复数个对等的源节点和目标节点。
这些工作并不适合用来分析边缘计算下的『群智感知』。
因此,为了提高群智感知服务的质量,必须用新的方法量化边缘网络中服务资源部署对群智感知应用的影响。
在该述求下,第一个挑战就是探索服务的部署方案影响群智感知应用的覆盖范围和服务质量的机理。
另外,群智感知应用中的设备随着时间的推移会改变自身的位置。
所以第二个挑战是探索移动节点的移动模型对群智感知应用服务质量的影响。

为了解决这两个问题,本章节在边缘计算下的群智感知中引入端到端通信,并对数据传输过程进行理论分析性能分析。本章节的主要贡献如下:

1)本章利用常微分方程组,以描述边缘计算环境下的群智感知中的任务分发阶段和数据收集阶段。通过解常微分方程组进行分析和求解,可以量化群智感知应用的覆盖范围并推导出可实现的感知质量。

2)由于群智感知服务的具有时效性,因此本章设计了一种算法时间划分算法,以找出任务分发阶段和数据收集阶段的最佳时间分配,帮助群智感知应用获得更好的感知质量。

3)本章经过大量的模拟实验分析,验证了模型的正确性和准确性。
同时,基于仿真的测试还验证了时间分配算法的效率。

\section{系统架构}

% 本章节中,我们重点研究了与论文[22]中描述的MCS过程。 在第一个任务分发阶段,任务分发者将MCS任务分发到移动节点。 此后,在第二数据收集阶段,已经接收到任务的移动节点进行感测并将感测数据报告回数据收集器。 我们进一步使雾计算参与上述程序。 在本节中,我们首先介绍一些预备,然后定义随机分析的系统模型。

本章节中,采用了和论文~\cite{DBLP:journals/tpds/ZhaoMTL15}相似的群智感知应用范式,主要研究对象为群智感知应用中的两个重要过程:任务分发过程和数据收集过程。
其中,任务分发过程是指已部署任务分发服务的边缘节点将任务分发至边缘网络中的移动节点,数据收集过程是指已经接受到任务的移动节点将感知数据反馈给已部署数据收集服务的边缘节点。
考虑到群智感知参与者本身具备移动性,本章也将移动节点的运动特征纳入考虑范围。
本小节重点介绍分析模型中的参数定义和模型建立。

\section{Preliminaries}

% 当结合雾计算时,任务分发者和数据收集器都被部署为雾节点中的服务。在不失一般性的情况下,我们假设基站是本文中的雾节点。图2描述了具有D2D通信的雾计算授权MCS的工作过程。首先,MCS服务提供商选择适当的基站来部署MCS服务。其次,在任务分发阶段,这些基站开始向移动节点传播MCS任务。为了避免蜂窝通信成本,基站仅将任务传递到进入其通信范围的移动节点。然后,已经完成任务的移动节点开始将任务转发到在其移动期间通过D2D通信遇到的其他移动节点。传播的时间消耗是T传播。在时间段T感测中,移动节点获得感测数据并将它们发送回数据收集器。设T集合表示数据收集的时间消耗。在整个阶段,我们假设所有移动节点都是MCS服务的志愿者。请注意,具有该任务的任何移动节点都可以帮助传播任务并通过D2D通信收集传感数据。显然,这形成了与流行病学路由的机会网络。

\begin{figure}[!b]
  \centering
  \vspace{-1em}
  \includegraphics[width=240pt]{./figures/Sec_UIC/群智感知应用的时间分布.pdf}
  \vspace{-0.5em}
  \caption{群智感知应用的时间分布}
  % \vspace{-4em}
  \label{Figure_MCS_Delay}
\end{figure}

在边缘计算场景下,群智感知中的任务分发服务和数据收集服务都可以部署在边缘节点中。
在不失普适性的情况下,本章假设蜂窝网络的基站作为能够承载这些服务的边缘节点。
此时群智感知服务的流程大体分为四步:
1)群智感知的服务发起者选择合适的基站将任务分发服务和数据收集服务部署到这些基站上;
2)在任务分发阶段,部署有任务分发服务的基站利用蜂窝网络和端到端通信将群智感知任务发送到附近的移动节点上;
3)移动节点在收到群智感知任务之后,一边处理任务,一边利用端到端通信将任务广播给附近的其它移动节点;
4)已经完成任务的移动节点,可以利用蜂窝网络或者端到端通信,将感知数据反馈给已经部署有数据收集服务的边缘节点。
由于本章节的重心并不是激励移动节点收集数据,所以在整个群智感知应用执行期间,论文假设所有的移动节点都是群智感知应用的志愿者。
图~\ref{Figure_MCS_Delay}描述了边缘计算环境中,群智感知应用中各阶段的时间开销。
在云端服务器将任务分发服务和数据收集服务部署至边缘节点之后,
$T_{dissemination}$ 表示群智感知任务从边缘服务节点发送到移动节点的时间开销;
$T_{sensing}$ 表示移动节点执行任务需要的时间;
$T_{collection}$ 表示移动节点将感知数据反馈回边缘服务节点的时间;
$T_{total\_time}$ 则为三者之和。

% 图3总结了雾计算授权MCS与D2D通信的任务分发的两种主要通信模式。 一旦移动节点完成任务,除了传播任务外,它还进行协作和机会感知以在移动期间获取感测数据。 然后,在数据收集阶段,移动节点开始将其感测数据发送回数据收集器,即,利用MCS服务部署的雾节点。 与任务分发阶段类似,传感数据可以直接传输到基站或通过基于D2D的流行路由到达基站,如图4所示。


\begin{figure}[!h]
  \centering
  % \vspace{-1em}
  \includegraphics[width=270pt]{./figures/Sec_UIC/任务分发过程.pdf}
  \vspace{-1em}
  \caption{任务分发过程}
  % \vspace{-0.5em}
  \label{Figure_PropagationProcedure}
\end{figure}

\begin{figure}[!h]
  \centering
  \vspace{-1em}
  \includegraphics[width=300pt]{./figures/Sec_UIC/数据收集过程.pdf}
  \vspace{-1em}
  \caption{数据收集过程}
  \vspace{-1.5em}
  \label{Figure_FeedbacksCollection}
\end{figure}

图~\ref{Figure_PropagationProcedure}展示了边缘计算下的群智感知服务利用蜂窝网络和端到端通信完成任务分发。
首先,部署有任务分发服务的基站利用蜂窝网络将任务发送至移动设备。
已经收到任务的移动设备,在其移动过程中,会有一定的概率遇到没有接收到任务的移动节点。
此时,已经收到任务的设备利用端到端通信,将任务分发至还没有收到任务的设备。
待任务执行完成后,移动设备可以利用蜂窝网络将感知数据传送给基站,也可以利用端到端通信将结果委托给在移动过程中相遇的其它设备,并传送给基站。
这一过程,如图~\ref{Figure_FeedbacksCollection}所示。

\subsection{系统建模}
% 按照上述工作流程,在本文中,我们认为在MCS服务中部署了N f个雾节点。它们既可以作为任务分发者,也可以作为数据收集者。在雾中,有N个移动节点随机移动并且可以自愿参与MCS过程。由于移动,移动节点可以机会性地与另一个移动节点相遇并且在接触持续时间期间与遇到的移动节点通信。我们假设接触持续时间足够长以完成任务或传感数据的传输。

本章将边缘网络中的移动设备分为两类:一类是『部署有任务分发服务和数据收集服务的边缘服务节点』;另一类是『执行群智感知任务的边缘设备』。
其中,边缘服务节点上同时部署任务分发服务和数据收集服务。
本章中,边缘服务节点的数量用 $N_f$ 表示。
在群智感知应用的目标覆盖区域内,边缘设备的数量记为 $N$,且这些设备都可认为是群智感知服务中的志愿者。
由于边缘设备在移动的过程中,会与其它边缘设备相遇,利用相遇的过程,可以利用端到端通信完成任务分发和数据转发。
这里假设在边缘设备的接触时间内数据交换工作能够顺利完成。

% 众所周知,移动节点应该以一定的速率机会性地遇到雾节点。设λf表示雾节点和移动节点之间的遇到率,λn是两个移动节点之间的成对相遇率。有许多不同的移动模型来描述网络中移动节点的移动行为。在本文中,我们关注直接给出遇到率的一般情况。然而,我们可以根据指定的移动模型计算λn和λf的值。例如,Groenevelt等。 [23]提出随机路点移动模型中的遇到率可以计算为方程。 (1),其中d代表通信半径,E [V *]是特定区域中节点的平均速度,A表示该区域的面积,w≈1.3683。

由于边缘终端的移动特性,每个移动设备都有一定的概率遇到相邻设备并进行端到端数据交互。
此处用 $\lambda_f$ 表示边缘服务节点和边缘设备之间的相遇概率;用 $\lambda_n$ 表示任意边缘设备之间的相遇概率。
在边缘网络中,边缘设备的相遇过程可以利用不同的移动模型进行推导。
在 Groenevelt 等人的论文~\cite{DBLP:journals/pe/GroeneveltNK05}中,随机路径移动模型中的相遇概率可以用式~\ref{Formula_EncounterRate}进行计算。
其中,$d$ 是设备的通信距离,
$\mathbb{E}[V^*]$ 指的是特定区域内所有节点的平均速度,
$A$ 表示所有节点通信半径覆盖的区域面积,
$w$ 是常数,其近似值为 1.3683。

\begin{equation}
  \label{Formula_EncounterRate}
  \begin{gathered}
  \lambda = \frac{2 w d \mathbb{E}[V^*]}{A}
  \end{gathered}
  % \vspace{-0.5em}
\end{equation}

% 由于大多数MCS应用程序都是有时间限制的,因此我们将图2中的T总时间视为截止日期。 T传播,T传感和T收集代表任务分发,数据传感和数据收集的时间。 实际上,数据生成过程非常快,因为数据生成器通过读取内置传感器来获取这些数据。 因此,我们忽略了T感知,并关注这项工作流程中的其他两个阶段:任务分发阶段和数据收集阶段。

鉴于群智感知应用的时效性,图~\ref{Figure_MCS_Delay}中的 $T_{total\_time}$ 可以被认为是群智感知服务的生命周期。
同时,$T_{dissemination}$ 代表了任务分发的周期;
$T_{sensing}$ 代表了数据感知的周期;
$T_{collection}$ 代表了数据收集的周期。
相较于任务分发周期和数据收集周期,数据感知的时间周期非常小,所以重点考虑任务分发周期和数据收集周期,对于数据感知周期则忽略不计。

\section{分析模型}
% 在本节中,我们给出了任务传播阶段,数据收集阶段的随机分析,以及在给定参数(如N,N f,λn,λ)的覆盖度量中可实现的感知质量的推导。 F 。 为方便读者,表I总结了本文中使用的主要符号。
本小节主要阐述任务分发过程和数据收集过程的理论分析方法。
表 ~\ref{table_notations} 给出了本章中使用的符号定义。

\begin{table}[h]
  \vspace{-1.5em}
  \caption{『符号和定义』}
  \vspace{-0.5em}
  \centering
  \label{table_notations}
  % \centering
  \begin{tabular}{|c|p{7cm}|}
  \hline
  \textbf{符号} & \textbf{定义}\\
  \hline
  $N$ & 移动设备的数量\\\hline
  $N_f$ & 边缘服务节点的数量\\\hline
  $\lambda_n$ & 移动设备之间的相遇概率\\\hline
  $\lambda_f$ & 边缘服务节点和移动设备的相遇概率\\\hline
  $I(t)$ & $t$ 时刻已经收到任务的移动设备的数量\\\hline
  $I'(t)$ & $t$ 时刻 $I(t)$ 的增量\\\hline
  $P_{rcv}(t)$ & $t$ 时刻边缘服务节点收到感知数据的概率\\\hline
  \end{tabular}
  \vspace{-1em}
\end{table}

\subsection{任务分发过程分析}

% 正如我们所知,无论何时发布任务,都会通过使用流行病路径的机会性D2D通信向移动参与者传播,该流行病路由模仿传染病在特定人群中向大量人群传播。已知易感染感染恢复(SIR)可用于描述流行病学路由过程。由于雾计算的参与,传统的SIR模型不能用于描述任务传播阶段。使用MCS服务部署的任何雾节点都可以直接“感染”其通信范围内的移动节点。因此,任务传播阶段与雾节点数N f高度相关。令I(t)是已经接收到任务的移动节点的数量,其能够通过经由D2D通信将任务转发给它们来“感染”其他易受影响的移动节点。通过总结这两部分,“受感染”移动节点的增长率如同等式1。 (2),其中I'(t)是I(t)相对于t的微分。

在任务分发过程中,当边缘设备从边缘服务节点收到任务之后,在其移动过程中利用端到端通信对群智感知任务进行二次分发。
这种基于端到端的任务分发过程和人群中传染病的扩散原理类似。
因此可以借助 SIR(Susceptible Infective Removal)模型来描述边缘设备之间的任务分发过程。
但是和传统 SIR 模型不同的是,边缘服务节点可以利用蜂窝网络直接将任务分发至移动设备。
所以任务的分发速度也和边缘服务节点的数量 $N_f$ 高度相关。
同时,$I(t)$ 作为已经收到任务的移动设备的数量,其大小也会直接已经任务分发的速度。
将这两种任务分发的途径结合考虑,任务分发的速率如式~\eqref{Formula_SIR_with_fog}所示。
其中,$I'(t)$ 是 $I(t)$ 的导数。

\begin{equation}
  \label{Formula_SIR_with_fog}
  I'(t) = N_f \lambda_f (N-I(t)) + I(t) \lambda_n (N-I(t))
\end{equation}

% 考虑到初始条件I(0)= 0,这意味着在开始时没有移动节点被感染,我们可以在时间t将“受感染”移动节点的数量的闭合形式表达式导出为等式1。(3)。 详细的推导过程类似于SIR模型的推导过程,为简洁起见,此处省略。
上式是黎卡提(Riccati)微分方程的。
考虑到初始条件 $I(0)=0$,即在群智感知开始时刻($t=0$)没有移动节点被感染,则时刻$t$已接收到任务的移动设备的数量可由式\eqref{Formula_It_with_fog}表示。

% \vspace{-1em}
\begin{equation}
\label{Formula_It_with_fog}
\begin{aligned}
I(t) = \frac{N (e^{(\lambda_n N + \lambda_f N_f) t} - 1)}{e^{(\lambda_n N + \lambda_f N_f) t} + \frac{\lambda_n N}{\lambda_f N_f}}, \forall\ \lambda_n \in [0, 1),\ \lambda_f N_f \in [0, 1)
\end{aligned}
\end{equation}

% 每当移动节点直接从雾节点或间接从其他移动节点接收任务时,它开始进行感测操作并将感测数据存储到其本地缓冲器中。 为了理解参与MCS的移动节点的比例,我们将受感染的移动节点的数量相对于移动节点的总数标准化为Eq。(4)。 当C T(t)= 100%时,表示所有移动节点已经接收到任务并参与了MCS。

当移动节点接收到群智感知任务之后,根据任务内容执行感知操作并将感知数据存储在其本地缓存中,然后等待机会将数据传送回边缘服务节点。
为了真实了解参与到群智感知任务的移动节点数量,式~\eqref{Formula_Coverage_of_Task} 中的 $C_T(t)$ 表示所有移动设备中已经收到群智感知任务的设备覆盖率。
当 $C_T(t)=1$ 时,表示区域内所有的移动节点都参与了群智感知服务。

\begin{equation}
  \label{Formula_Coverage_of_Task}
  C_T(t) = \frac{I(t)}{N} = \frac{e^{(\lambda_n N + \lambda_f N_f) t} - 1}{e^{(\lambda_n N + \lambda_f N_f) t} + \frac{\lambda_n N}{\lambda_f N_f}}
\end{equation}

\subsection{数据收集过程分析}

% 在任务传播阶段之后,移动节点开始将感测数据返回到数据收集阶段中的雾节点。 因为许多MCS应用程序都有截止日期来接收移动设备生成的数据。 对于MCSusers,雾节点接收的数据越多,覆盖范围越大。 因此,了解雾节点在特定时间点接收到多少数据是很重要的。
已经收到任务的移动节点按照要求获取感知数据后,需要及时地将感知数据传送回边缘服务节点。
而群智感知应用在其生命周期内,收集到的数据越多,数据覆盖范围越大,则群智感知服务能够达到的服务质量就越高。
这也意味着在规定的时间范围内,边缘服务节点必须尽可能收集来自所有移动设备上的感知数据。
因此,在时域上分析数据收集过程也是非常重要的工作。

% 众所周知,传感数据也通过流行路由路由回到雾节点。 在此阶段,参与MCS的移动节点是源节点,任何雾节点都可以是目标节点。 移动节点上的感测数据被打包到一个分组中,该分组可以在一对移动节点之间或移动节点和雾节点之间的一个接触中被传送。 感测数据独立地路由而不与雾节点冲突。 Whenevera雾节点接收到数据包,我们说传感数据已成功接收。 我们首先分析一个雾节点在timet接收到apacket的概率。 请注意,雾节点可以直接或间接接收数据包,两种方法都是独立的。 因此,雾时钟点接收的数据包的概率Prcv(t)可以计算为

和任务分发时数据流的途径一样,感知数据在收集过程中,移动设备也可以使用蜂窝网络或者端到端通信的方式将发送感知数据。
在这一过程中,任何边缘服务节点都可以作为感知数据的目的地。
在使用端到端通信时,移动设备可以相互交换感知数据,并将收到的感知数据进行打包处理并未下一次转发做好准备。
一个感知数据包一旦到达任何一个边缘服务节点,则该组感知数据被认为已经收集完成。
在时刻 $t$,一个数据包被边缘服务节点收到的概率记为 $P_{rcv}(t)$。
在边缘网络中,数据包可以利用蜂窝网络或者端到端通信传输,而这两种传播方式都是相互独立的。
因此,$P_{rcv}(t)$ 的计算方法如式~\eqref{Formula_ProbaRcvT}。

\begin{equation}
\label{Formula_ProbaRcvT}
\begin{aligned}
P_{rcv}(t) = 1 - P_{nD2D}(t) P_{nDirect}(t)
\end{aligned}
\end{equation}

上式中,$P_{nD2D}(t)$ 和 $P_{nDirect}(t)$ 都可以利用式~\eqref{Formula_SIR_with_fog} 推导得出。
其中,$P_{nD2D}(t)$ 指的是数据包以端到端通信未送达的概率,计算方法如式~\eqref{Formula_ProbaNP};
$P_{nDirect}(t)$ 指的是数据包以蜂窝网络通信未送达的概率,计算方法如式~\eqref{Formula_ProbaNA}。

\begin{equation}
  \label{Formula_ProbaNP}
  \begin{aligned}
  P_{nD2D}(t) = (\frac{N-S(t)}{N})^{\lambda_f N_f t}, \forall \ \lambda_f N_f \in [0,1)
  \end{aligned}
\end{equation}

\begin{equation}
  \label{Formula_ProbaNA}
  \begin{aligned}
  P_{nDirect}(t) = (\frac{N}{N + N_f})^{S(t)-1}
  \end{aligned}
\end{equation}

在上式中,$S(t)$ 表示一组感知数据包在传输交换之后,网络中该数据包的拷贝总数。
其计算方法如式~\eqref{Formula_St}。

\begin{equation}
\label{Formula_St}
  \begin{aligned}
    S(t) = \frac{N e^{\lambda_n N t}}{e^{\lambda_n N t} + N -1}, \forall \ \lambda_n \in [0,1)
  \end{aligned}
\end{equation}

\subsection{传感质量分析}

% 在MCS应用程序中,任务需要覆盖广泛的领域。该区域的覆盖范围意味着感应地图的感应质量。例如,十字路口的交通状况可以通过附近所有道路上的车辆数量来确定。为了准确计算覆盖率,我们将网络区域切割成一组网格单元= fg1G2;G3;: : : : :;gmg,如图5所示。这些单元的大小意味着空间上的感知粒度,这由MCS应用程序用户决定。假设地图中有移动参与者,letLij(t )表示网格中的移动参与者是否在某个时间( I ),即..
群智感知的应用,往往需要覆盖防范的地域范围。
其传感数据所涵盖的地域范围越广,则群智感知服务的普适性越强。
例如,一个十字路口的交通状况需要通过附近所有道路上的车辆数量来推理。
为了准确计算群智感知应用的覆盖率,群智感知应用的目标区域如图~\ref{Figure_CoverageofArea}被切割成一组网格单元:$G = \{g_1,g_2,g_3,\ldots,g_m\}$。
网格单元的大小意味着地域空间上的感知密度,该参数由群智感知应用的开发者所决定。
在分析传感质量是,这里用 $L_{ij}(t)$ 表示在时刻 $t$ 时,移动设备 $j$ 是否在网格$g(i)$中。
其定义如式~\eqref{Formula_LocationIJ}表示。

\begin{figure}[!h]
  \centering
  % \vspace{-1.5em}
  \includegraphics[width=200pt]{./figures/Sec_UIC/群智感知的空间分布.pdf}
  \vspace{-0.5em}
  \caption{群智感知的空间分布}
  \vspace{-0.5em}
  \label{Figure_CoverageofArea}
\end{figure}

\begin{equation}
  \label{Formula_LocationIJ}
  L_{ij}(t) = \left \{
  \begin{aligned}
  & 1,\ \text{mobile node $j$ is in $g(i)$ at time $t$.}\\
  & 0,\ \text{otherwise.}
  \end{aligned}
  \right.
\end{equation}

利用 $L_{ij}(t)$,移动设备 $j$ 在网格 $g(i)$ 中的总停留时间 $C_{ij}$ 可以用式~\eqref{Formula_UserTrace}表示。

\begin{equation}
  \label{Formula_UserTrace}
  C_{ij}(t) = \int_{0}^{t}L_{ij}(\varepsilon )d\varepsilon , \forall \ i \in [0,m], j \in [0,N]
\end{equation}

因此,在 $t$ 时刻,群智感知服务的感知质量 $C(t)$ 可以利用来式~\eqref{Formula_UserCoverageGrip}来计算。

\begin{equation}
  \label{Formula_UserCoverageGrip}
  \begin{gathered}
  C(t) = \sum^{m}_{i=0}\sum^{N}_{j=0} g_i \times C_{ij}(t), \forall \ t>0
  \end{gathered}
\end{equation}

\section{Applications of the ODE-based Analysis}

% 在本节中,我们在最后一节中介绍了一个基于随机分析的应用用例。尽管具有D2D通信的MCSwith应该是延迟容忍的,但是一些应用也受到截止时间的限制,因为感测数据和截止时间变得非常宝贵。由于MCS过程有两个阶段,因此研究如何在这两个阶段分配时间以最大限度地提高平均水平具有重要意义。根据上面的分析,假设任务分发阶段持续,即分配给任务分发的时间,我们可以得出deadlineTtotaltimeas接收到的数据包总数
基于上一小节的模型,本节利用『stochastic analysis』的方法对边缘计算下的群智感知应用进行理论分析。
由于群智感知应用对感知数据的时效性非常敏感,不具备时效性的感知数据可能会导致错误的计算结果。
因此在使用机会式端到端通信交换感知数据时,必须考虑群智感知应用的截止时间限制。
前文已经阐述群智感知应用中,足够多的参与设备和大量的反馈数据才能保证群智感知的时效性。
所以在固定生命周期内,合理分配这两个过程的时间配额才能最大程度的提高群智感知应用的感知质量。

通过前文的分析,当任务分发过程的时间配额为 $t$ 时,在周期 $T_{total\_time}$ 中群智感知应用可以收集到的感知数据总数 $D(t)$ 可以利用式~\eqref{Formula_DataAmount}计算。
该式中,实际参与群智感知应用的移动设备数量为 $N\times C_T(t)$。
$T_{total\_time}-t$ 是数据收集阶段的时间配额,在该时间段类可以收到的『数据比例为 $C_F(T_{total\_time}-t)$』。

\begin{equation}
  \label{Formula_DataAmount}
  \begin{gathered}
  D(t) = N \cdot C_T(t) \cdot C_F(T_{total\_time}-t),\forall \ t \in [0, T_{total\_time}]
  \end{gathered}
\end{equation}

% 根据等式。( 12 )这一价值对可实现的覆盖范围有着明显的深刻影响。很容易推导出0 ) =和Ttotaltime ) = 0。SinceCT(t)0is在光子递增函数oft中,andCF(Ttotaltimet)0is在单调递减函数oft中,必须存在一个可以最大化( t )的值,表示最高的感知质量,即覆盖范围。为了找出最佳的时间分配,我们通过枚举时间分配,通过数值计算设计了一种算法。算法1描述了算法。
在式~\eqref{Formula_DataAmount}中,$t$ 是唯一的自变量。
因此,不同阶段的时间配额会直接影响到群智感知应用的感知质量。
当 $t=0$ 或 $t=t_{total\_time}$ 时,$D(t)$ 恒为零。
且当 $0 \leq t \leq t_{total\_time}$ 时,$C_T(t)\geq 0$ 且 $C_T(t)$ 为单调递增函数;$C_F(T_{total\_time}-t) \geq 0$ 且 $C_F(T_{total\_time}-t)$ 为单调递减函数。
因此,当$0 \leq t \leq t_{total\_time}$时,$D(t) \geq 0$ 且对于 $t$ 必定存在一个值使得 $D(t)$ 取到最大值,即群智感知应用的感知质量达到最优。
为了找到最佳的时间划分,本文设计了一个枚举算法来找出 $t$ 的最优解。
该算法伪代码如『算法~\ref{algo_findbestt}』所示。

% \vspace{-0.5em}
% \begin{algorithm}[h]
% \caption{Find out the best allocation $t$ to maximize $D(t)$}
% \footnotesize
% \label{algo_findbestt}
% \textbf{Requires:\\}
% \text{\quad \quad input $[N, N_f, \lambda_n, \lambda_f]$ and the total time $T_{total\_time}$\\}
% \begin{algorithmic}[1]
%   \State \text{let $t = T_{total\_time}/N\_slots$}
%   \State \text{let $t\_list = [t, 2t, 3t, 4t,\ldots,N\_slots \times t]$}
%   \State \text{let $res\_list = [0] \times N\_slots$}
%   \State \text{let $index = 0;\ maximum = 0;\ best\_t = 0$}
%   \For{$item \in t\_list$}
%     \State \text{$res\_list[index] = D(item)$ \quad (Formula~\ref{Formula_DataAmount})}
%     \If {$D(item) > maximum$}
%       \State $maximum = D(item)$
%       \State $best\_t = t\_list(index)$
%     \EndIf
%     \State \text{$index$ ++}
%   \EndFor
%   \State \text{return $maximum, best\_t$}
% \end{algorithmic}
% \end{algorithm}
% \vspace{-0.5em}

\begin{algorithm}[h]
\setstretch{\algostretch}
\KwIn{$N$ : 移动设备的总数量}
\KwIn{$N_f$ : 边缘服务节点的数量}
\KwIn{$\lambda_n$: 移动设备之间的相遇率}
\KwIn{$\lambda_f$: 边缘服务节点和移动设备之间的相遇率}
\KwIn{$T_{total\_time}$ : 群智感知服务的生命周期}
\KwData{『输入数据』}
% \KwData{$Occ[1:T]$: 每个程序当前的NCP大小,初始为0,随着缓存路的分配增加,最终为每个程序在整个缓存上的NCP}
let $t = T_{total\_time}/N\_slots$\\
let $t\_list = [t, 2t, 3t, 4t,\ldots,N\_slots \times t]$\\
let $res\_list = [0] \times N\_slots$\\
let $index = 0;\ maximum = 0;\ best\_t = 0$\\
\For{$item \in t\_list$}{
  $res\_list[index] = D(item)$ \quad (Formula~\ref{Formula_DataAmount})\\
  \If {$D(item) > maximum$}{
    $maximum = D(item)$\\
    $best\_t = t\_list(index)$
  }
  $index$ ++
}
% return $maximum, best\_t$\\
\KwOut{$maximum, best\_t$ : $D(t)$ 的最大值,和对应 $t$ 的值}
\caption{找出让 $D(t)$ 最大的时间配额划分 $t$}
\label{algo_findbestt}
\end{algorithm}

% 在算法1中,总时间TotalTimes被分成intoNslotsslots。然后,我们枚举时隙时间时隙,以找到能够最大化( t )的最佳分配。这种算法的时间复杂性是isO ( Nslots )。
在算法~\ref{algo_findbestt}中,群智感知应用的生命周期被切分成 $N\_slots$ 份。
通过对不同的时间配额进行枚举计算,求出让 $D(t)$ 取值最大的时间划分。
该算法的时间复杂度为 $O(N\_slots)$。

\section{系统测试}

% 为了验证我们分析的正确性和准确性,我们在本节中报告了我们基于模拟的结果。 此外,我们的算法在寻找最优时间分配方面的效率得到了证明。
为了验证本文分析方法的正确性和准确度,本节使用模拟器和前文提出的理论分析方法进行对比。
同时,本文的算法在寻找最优时间分配方面的效率也得到了证明。

\subsection{模拟器设置}
% 我们进行跟踪驱动模拟,以广泛评估我们对雾计算中MCS应用的基于ODE的分析。遵循ONE模拟器[ 24 ]的原理,我们捕获雾环境中的网络行为,并获得参数,如N;国家森林机构;andf. n通过追踪数据包路径和所有移动节点的状态,可以获得有效MCS参与者的数量、雾节点接收到的数据包数量,以及在一定时间内达到的覆盖率。在我们的随机分析中,这些信息可以由场景参数sN导出;国家森林机构;andf. n将分析结果和仿真结果进行比较,我们能够验证我们分析的正确性和准确性。为了彻底验证随机分析的准确性,我们首先验证我们对每个阶段的分析,然后对整体分析进行比较。
在测试过程中,本文使用 ONE 模拟器~\cite{DBLP:conf/simutools/OK09}对边缘计算模型下的群智感知应用进行模拟仿真。
在仿真实验中,主要的输入参数有:$N$、$N_f$、$\lambda_n$、$\lambda_f$,其具体含义如表~\ref{table_notations}所示。
对仿真过程产生的日志信息进行处理,可以获得参与群智感知应用的移动设备数量以及每个数据包所经过的具体路径。
结合时间戳信息进一步分析,还可以获得边缘服务节点在任意时刻收到的感知数据数量以及已收集的感知数据所覆盖的地域范围。

利用本文中的随机分析方法,可以利用参数 $N$、$N_f$、$\lambda_n$和$\lambda_f$推导出同样的信息。
将模拟仿真和随机分析得到的结果进行对比,可以验证本文中『随机分析』的正确性。
在对比过程中,先分别验证任务分发过程和数据收集过程的覆盖率,在针对群智感知全过程,验证其范围覆盖和感知质量。

% 我们首先通过使用不同的参数asN来验证我们对任务传播阶段的分析;国家森林机构;andf,which n定义在表一中。结果在图6中报告,包括模拟结果和通过求解等式获得的分析结果。( 4 )。从这个图中,可以观察到我们获得了高精度,因为分析结果都非常接近于模拟结果。在5000组实验中,在任务传播阶段,平均误差为5.7 %。我们还注意到,有效的MCS参与者( EMP )的数量随着时间的增加而增加。最初,增长率低的原因是雾节点,很少有“受感染”的移动节点能够使其他“易受感染”的移动节点生效。后来,有效的MCS参与者的数量急剧增加,因为已经有很多有效的MCS参与者。然而,这种增长最终会收敛,因为大多数移动电话已经被“感染”。这种现象符合我们的常识。

\subsection{On the Accuracy of our Stochastic Analysis}

\begin{figure}[!b]
  \centering
  % \vspace{-2em}
  {\includegraphics[width=210pt]{./figures/Sec_UIC/Propagation/F3-1.pdf}}
  {\includegraphics[width=210pt]{./figures/Sec_UIC/Propagation/F3-2.pdf}}\\
  {\includegraphics[width=210pt]{./figures/Sec_UIC/Propagation/F3-3.pdf}}
  {\includegraphics[width=210pt]{./figures/Sec_UIC/Propagation/F3-4.pdf}}
  \vspace{-1em}
  \caption{任务分发过程分析结果和模拟结果对比}
  % \vspace{-0.5em}
  \label{Figure_PropagationTest}
\end{figure}

首先通过使用不同的参数 $N, N_f, \lambda_n$, and $\lambda_f$ 来验证本文对任务分发过程的分析。
图~\ref{Figure_PropagationTest}反应了不同参数取值对群智感知效率的影响。
图中横坐标为时间轴,纵坐标为群智感知应用收到的『感知数据的数量』,其结果可通过式~\eqref{Formula_Coverage_of_Task}计算得到。
通过曲线对比仿真结果和模拟结果,两者的趋势是非常接近的。
在5000组不同参数的实验对比中,任务分发分析的平均误差为 5.7\%。
对图像进行观察可以发现,群智感知应用执行初期,『』增长率偏低,这是因为任务只能通过边缘服务节点分发。
在获得任务的移动设备增多之后,区域内的任务分发源也越来越多,因此能够执行感知任务的移动设备数量呈指数级上涨。
当任务覆盖率接近饱和时,『』增长速度逐渐降低,获得任务的移动设备数量逐步趋近于目标区域内的移动设备总数。

% 同样的方法被用来验证我们对数据收集阶段的分析的准确性。首先,我们假设每个EMP都有一个包,由发送到雾节点的所有传感数据组成。这一过程类似于传统的流行病传播途径。然而,有多个雾节点作为目的节点,在传统流行病路由中只有一个目的节点。我们认为,一旦任何fog节点接收到数据包,该数据包就会被成功传递。我们计算在时间t接收到的数据包数量。性能评估结果如图7所示,其中分析结果通过求解等式获得。( 5 )。我们再次注意到,我们对数据收集的分析也非常准确,因为结果都接近分析结果。在5000组随机种子的模拟中,平均误差为9.6 %。与我们的一般情况一致,接收到的数据包数量随着时间的推移而增加。首先,接收的数据包数量缓慢增加,然后快速增加,最后收敛。

然后,用同样的方法来验证本文对数据收集过程的分析。
假设每个获得感知任务的移动设备在收集到所需数据后,用一个数据包将结果反馈给边缘服务节点。
此时,在目标区域内,已经部署有多个边缘服务节点。
当数据包抵达任何一个边缘服务节点时,则可以认为数据反馈成功。
利用式~\eqref{Figure_CollectionTest},可以计算被成功接收的数据包数量。
图~\ref{Figure_EntireTest}展示了模拟试验和理论分析的对比结果。
图中,两者的变化趋势依然保持一致。
在5000组不同参数的实验对比中,数据收集理论分析的平均误差为 9.6\%。
且变化趋势和实际情况相符。

\begin{figure}[!h]
  \centering
  % \vspace{-1em}
  {\includegraphics[width=210pt]{./figures/Sec_UIC/Collection/R1-0.pdf}}
  {\includegraphics[width=210pt]{./figures/Sec_UIC/Collection/R1-1.pdf}}\\
  {\includegraphics[width=210pt]{./figures/Sec_UIC/Collection/R1-2.pdf}}
  {\includegraphics[width=210pt]{./figures/Sec_UIC/Collection/R1-3.pdf}}
  \vspace{-1em}
  \caption{数据收集过程分析结果和模拟结果对比}
  \label{Figure_CollectionTest}
  \end{figure}

% 在对我们的雾计算中的两个主要阶段进行验证后,我们用D2D通信授权MCS,接下来我们考虑两个阶段,并调查我们的总体随机分析是如何进行的。在本实验中,我们设置总时隙数Ttotaltimeas 100s,并在不同的网络设置中从0到100改变时间分配。结果报告在图8中,其中通过求解等式获得分析结果。( 12 )。直观地说,在任何oft值下,分析结果总是接近模拟结果。这表明我们对这两个阶段的总体分析也是准确的。此外,在图8中观察到的一个有趣现象是,接收到的分组数量首先随着时间分配的分配增加而增加,然后随着达到最大值而减少。最初,随着分配给任务分发阶段的时间越来越多,越来越多的移动节点可以有效地参与MCS过程。因此,接收到的数据包数量会增加。达到最大值后,进一步增加将减少分配给数据收集阶段的时间。虽然MCS参与者更有效,但是在截止日期之前,传感数据几乎没有时间被传送到雾节点。因此,性能甚至会下降。这意味着存在一个最佳时间分配,可以最大限度地提高感知质量,找到这样一个最佳设置非常重要。幸运的是,我们提出的算法1能够处理这个问题

\begin{figure}[!b]
  \centering
  % \vspace{-1.5em}
  {\includegraphics[width=210pt]{./figures/Sec_UIC/RcvsSim/0.pdf}}
  {\includegraphics[width=210pt]{./figures/Sec_UIC/RcvsSim/1.pdf}}\\
  {\includegraphics[width=210pt]{./figures/Sec_UIC/RcvsSim/2.pdf}}
  {\includegraphics[width=210pt]{./figures/Sec_UIC/RcvsSim/3.pdf}}
  \vspace{-1em}
  \caption{不同时间配额下的分析结果和模拟结果}
  % \vspace{-2em}
  \label{Figure_EntireTest}
\end{figure}

最后,对边缘计算下的群智感知应用全过程进行分析验证。
结合任务分发和数据收集两个重要阶段,验证本文中时间划分的合理性。
在本次实验中,群智感知应用的生命周期$T_{total\_time}$被设置为100秒。
针对场景定义参数$N, N_f, \lambda_n$, and $\lambda_f$设计了四种不同的场景,分析群智感知应用可以收到的数据量和时间的关系。
测试结果如图~\ref{Figure_EntireTest}所示,理论测试值通过式~\eqref{Formula_DataAmount}计算得出。
图示结果直观的反映出本文的理论分析犯法和模拟实验结果非常吻合,在任何时间分配方案下,分析结果总是接近模拟结果。
这也验证了本文对任务分发和数据收集两个阶段的分析也是准确无误的。
此外,在图中还能观察到的一个有趣现象:接收到的感知数据总量首先随着任务分发的时间配额的增加而增加,然后随着达到最大值而减少。
最初,随着分配给任务分发阶段的时间越来越多,越来越多的移动节点可以有效地参与MCS过程。
因此,接收到的数据包数量会增加。
达到最大值后,进一步的增加将减少分配给数据收集阶段的时间配额。
而在固定的声明周期内,不仅需要任务顺利发送到移动设备,也需要感知数据能准确的传送回边缘服务节点。
因此,当数据收集时间减少时,群智感知应用的感知质量会随之下降。
这意味着存在一个最佳时间分配,可以最大限度地提高感知质量,找到这样一个最佳设置非常重要。
幸运的是,我们提出的算法1能够处理这个问题。

为了验证时间配额划分算法的准确性,测试中使用参数 $N=6000, N_f=4, \lambda_n = 0.000027, \lambda_f=0.00018$ 定义的模拟场景。
群智感知应用的生命周期 $T_{total\_time}$ 设置为 100 秒。
任务分发和数据收集的最佳时间分配,接收到的数据包数量应该是最大的。
假设每个群智感知参与者需要必须返回一个数据包,则应该收集的数据包总数和群智感应参与者的数量应当相同,即为 $N$。
因此可以使用被收集的数据包总数与群智感知参与者数量的比例来表示群智感知应用的效率。
图~\ref{Figure_BestT}展示了任务分发过程的时间配额分别为25秒、50秒、75秒以及58.6秒时的群智感知应用执行效率。
其中,58.6秒是利用时间配额划分算法找出的最佳任务分发时间配额。
对比实验结果可以看出,最佳分配的结果比其他分配方法导致的执行效率要高出20\%以上。
在算法实验中,$N\_slots$ 被设置为 $10^7$,算法执行时间不超过 2500 毫秒。

\begin{figure}[!h]
  \centering
  % \vspace{-1em}
  {\includegraphics[width=210pt]{./figures/Sec_UIC/BestT/0.pdf}}
  {\includegraphics[width=210pt]{./figures/Sec_UIC/BestT/1.pdf}}\\
  {\includegraphics[width=210pt]{./figures/Sec_UIC/BestT/2.pdf}}
  {\includegraphics[width=210pt]{./figures/Sec_UIC/BestT/3.pdf}}
  \vspace{-1em}
  \caption{不同时间配额的对感知质量的影响}
  % \vspace{-1em}
  \label{Figure_BestT}
\end{figure}

\subsection{The Coverage under Different Deployment Settings}

% 在每组实验中改变一个设置,并显示最终接收到的数据包数量。不同valuesofN,Nf,nandfare的比较都在图10中报道。我们注意到,可实现的传感质量显示为任一参数的递增函数,正如模拟结果和分析结果所表明的,它们总是彼此接近。这进一步广泛地验证了我们分析的正确性和准确性,因为它在任何设置下都能接近模拟结果。在每组测试中,只有一个场景参数作为变量。移动节点的数量N、相遇概率λ n和相遇概率λ f每一步增加10 %的基数。雾节点的数量每增加一步就增加一倍。通过比较,可以发现MCS应用中影响因素的优先级排序(降序)如下: N、λ n、N f和λ f。
该测试将时间配额划分算法应用到不同的网络场景下,获得在群智感知应用的生命周期内可以接受到的数据包总量。
其中,基准群智感知的网络环境定义为 $N=5000, N_f=2, \lambda_n = 0.00003, \lambda_f=0.0004$。
在每组实验中,四个环境变量值只改变其中一个,剩余三个环境变量保持不见。
以此可观察出一种环境变量的变化对群智感知应用的感知质量的影响。
也从侧面反映出边缘计算对群智感知应用的影响。
测试结果如图~\ref{Figure_SingleValTest}所示。
通过仿真数据和理论数据的对比,验证了本文的『Stochastic Analysis』方法准确性。
同时,测试结果也展示了不同类型的边缘网络环境变量对群智感知应用的感知质量的影响效果。
测试中,边缘服务节点的数量 $N_f$ 以倍数关系增长,移动设备的数量$N$、相遇率 $\lambda_n$ 和 $\lambda_f$ 均以 10\% 等比例增长。
对比发现,在『边缘计算下的群智感知应用』中,移动设备的数量$N$对感知效率的影响最大,其次是移动设备的相遇率 $\lambda_n$,然后是边缘服务节点的数量 $N_f$,最后是边缘服务节点和移动设备的相遇率 $\lambda_f$。

\begin{figure}[!h]
  \centering
	% \vspace{-1.5em}
	{\includegraphics[width=210pt]{./figures/Sec_UIC/SingleVar/1n.pdf}}
	{\includegraphics[width=210pt]{./figures/Sec_UIC/SingleVar/1ln.pdf}}\\
	{\includegraphics[width=210pt]{./figures/Sec_UIC/SingleVar/1nf.pdf}}
	{\includegraphics[width=210pt]{./figures/Sec_UIC/SingleVar/1lf.pdf}}
	\vspace{-1em}
	\caption{The number of packets received under different deployment settings}
	% \vspace{-1.5em}
	\label{Figure_SingleValTest}
\end{figure}

\section{本章小结}

本章节研究了一个实用的和潜在的用于具有D2D通信的MCS的雾计算应用,其中雾节点可以同时充当任务传播者和数据收集器。特别是,我们考虑机会性D2D通信来转发任务和传感数据。基于D2D的任务分发阶段和数据收集阶段都由ODS描述。通过求解这些方程,我们提出了有效MCS参与者和接收分组的封闭形式表达式,作为雾节点数量、移动节点数量、移动节点之间的相遇率以及雾节点和移动节点之间的相遇率的函数。然后,我们通过共同探索任务传播阶段和数据收集阶段的分析,进一步得出可实现的感知质量。我们还运用我们的分析来找出两个主要阶段的最佳时间分配,从而最大限度地提高截止日期受限的MCS应用的感知质量。通过广泛的基于仿真的研究,我们验证了我们分析的正确性和准确性。与此同时,我们的随机分析总是能够揭示使用D2D通信的雾计算授权MCS的真实行为。